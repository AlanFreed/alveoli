%% This is file `elsarticle-template-1-num.tex',
%%
%% Copyright 2009 Elsevier Ltd
%%
%% This file is part of the 'Elsarticle Bundle'.
%% ---------------------------------------------
%%
%% It may be distributed under the conditions of the LaTeX Project Public
%% License, either version 1.2 of this license or (at your option) any
%% later version.  The latest version of this license is in
%%    http://www.latex-project.org/lppl.txt
%% and version 1.2 or later is part of all distributions of LaTeX
%% version 1999/12/01 or later.
%%
%% The list of all files belonging to the 'Elsarticle Bundle' is
%% given in the file `manifest.txt'.
%%
%% Template article for Elsevier's document class `elsarticle'
%% with numbered style bibliographic references
%%
%% $Id: elsarticle-template-1-num.tex 149 2009-10-08 05:01:15Z rishi $
%% $URL: http://lenova.river-valley.com/svn/elsbst/trunk/elsarticle-template-1-num.tex $
%%
%% \documentclass[preprint,12pt]{elsarticle}
\documentclass[final,3p,12pt]{elsarticle}


%% Use the option review to obtain double line spacing
%% \documentclass[preprint,review,12pt]{elsarticle}

%% Use the options 1p,twocolumn; 3p; 3p,twocolumn; 5p; or 5p,twocolumn
%% for a journal layout:
%% \documentclass[final,1p,times]{elsarticle}
%% \documentclass[final,1p,times,twocolumn]{elsarticle}
%% \documentclass[final,3p,times]{elsarticle}
%% \documentclass[final,3p,times,twocolumn]{elsarticle}
%% \documentclass[final,5p,times]{elsarticle}
%% \documentclass[final,5p,times,twocolumn]{elsarticle}

%% if you use PostScript figures in your article
%% use the graphics package for simple commands
%% \usepackage{graphics}
%% or use the graphicx package for more complicated commands
%% \usepackage{graphicx}
%% or use the epsfig package if you prefer to use the old commands
%% \usepackage{epsfig}

%% The amssymb package provides various useful mathematical symbols
\usepackage{amssymb}
%% The amsthm package provides extended theorem environments
\usepackage{amsthm}
\newtheorem*{cor}{Corollary}
\newtheorem*{defn}{Definition}
% \newtheorem*{algorithm}[theorem]{Algorithm}

\renewcommand{\thepart}{\arabic{part}}
\renewcommand{\thesection}{\thepart.\arabic{section}}
\renewcommand{\thesubsection}{\thesection.\arabic{subsection}}
\renewcommand{\thesubsubsection}{\thesubsection.\arabic{subsubsection}}

\renewcommand{\theequation}{\thepart.\arabic{equation}}
\renewcommand{\thefigure}{\thepart.\arabic{figure}}
\renewcommand{\thetable}{\thepart.\arabic{table}}
%
%% The lineno packages adds line numbers. Start line numbering with
%% \begin{linenumbers}, end it with \end{linenumbers}. Or switch it on
%% for the whole article with \linenumbers after \end{frontmatter}.
\usepackage{lineno}

%% natbib.sty is loaded by default. However, natbib options can be
%% provided with \biboptions{...} command. Following options are
%% valid:

%%   round  -  round parentheses are used (default)
%%   square -  square brackets are used   [option]
%%   curly  -  curly braces are used      {option}
%%   angle  -  angle brackets are used    <option>
%%   semicolon  -  multiple citations separated by semi-colon
%%   colon  - same as semicolon, an earlier confusion
%%   comma  -  separated by comma
%%   numbers-  selects numerical citations
%%   super  -  numerical citations as superscripts
%%   sort   -  sorts multiple citations according to order in ref. list
%%   sort&compress   -  like sort, but also compresses numerical citations
%%   compress - compresses without sorting
%%
%% \biboptions{comma,round}

\biboptions{sort&compress}

\journal{Draft}


%%
%%  The following packages, definitions and macros were added by the authors.
%%

%
\usepackage{amsmath}
\usepackage{stmaryrd}    % source of \mapsfrom
%\usepackage{mathtools}
%
\DeclareMathAlphabet{\mathbfit}{\encodingdefault}{\rmdefault}{bx}{sl}
\DeclareMathAlphabet{\mathbfsf}{\encodingdefault}{\sfdefault}{bx}{n}
%
\usepackage{algorithm}
\usepackage{algpseudocode}
%
% \newdefinition{defn}{Definition}
% \newdefinition{rmk}{Remark}
\newdefinition{conjecture}{Conjecture}
% \newtheorem{thm}{Theorem}
% \newtheorem{cor}{Corollary}
% \newproof{pf}{Proof}
%
%  Macros for creating nice text and script fractions.
%  Do not break these long lines into multiple lines.
%  Spaces CANNOT be present; it can introduce unwanted `rubber' spacing.
%
\newcommand{\textfrac}[2]
   {\mbox{\leavevmode \kern.1em\raise.5ex\hbox{\ensuremath{\scriptstyle{#1}}}\kern-.1em\hspace{0em}/\hspace{0em}\kern-.1em\lower.25ex\hbox{\ensuremath{\scriptstyle{#2}}}\kern.1em
   }}
\newcommand{\scriptfrac}[2] 
   {\mbox{\leavevmode \kern.05em\raise.25ex\hbox{\ensuremath{\scriptscriptstyle{#1}}}\kern-.05em\hspace{0em}\hbox{\ensuremath{\scriptstyle{/}}}\hspace{0em}\kern-.05em\lower.125ex\hbox{\ensuremath{\scriptscriptstyle{#2}}}\kern.05em
   }}
%
%
% Macro for typesetting ":=" in a nice manner.
%
\newcommand*{\defeq}{\mathrel{\vcenter{\baselineskip0.5ex \lineskiplimit0pt
                     \hbox{\scriptsize.}\hbox{\scriptsize.}}}%
                     =}
%
%
%% Create a rectangle that is used to denote squeeze
%
\newcommand{\rectangle}[2]{{% #1 = width, #2 = height
  \fboxsep=-\fboxrule\sbox0{}\wd0=#1\ht0=#2\relax\fbox{\box0}}}
\newcommand*{\squeeze}{\rectangle{2.5pt}{5pt}}
%
% Create stacked accents that look right
%
\usepackage{mathabx}
\newsavebox{\accentbox}
\newcommand{\myaccent}[2]{%
  \sbox\accentbox{$#2$}#1{\usebox\accentbox}}
%
% Create a d and an integral sign with slashes throug it
%
\newcommand{\dbar}{\ensuremath{\mathchar'26\mkern-11mu \mathrm{d}}}
\newcommand{\ibar}{\mbox{
   \lower.35ex\hbox{\ensuremath{\mathchar'26}}\kern-1ex\raise.35ex\hbox{\ensuremath{\int}}\kern0ex
   }}
\newcommand{\tint}{\hbox{\ensuremath{\int}}\kern0ex}
%
%
% Creates equation numbers like (2.1) for 1st equation in section 2. 
% Requires amsmath
%
% \numberwithin{equation}{section}
%
% to rotate a figure or a table and their captions using
%
\usepackage{rotating}
\usepackage{graphicx}
%
% which is implemented via the command
% \begin{sidewaysfigure} or \begin{sidewaystable}
%
\usepackage{color}
%
% Use for creating labeled lists
%
\usepackage{blindtext}
\usepackage{scrextend}
\addtokomafont{labelinglabel}{\sffamily}
%
% Use the following lines to output the various parts as separate pdf files.
%
% \includeonly{partIntroduction}
% \includeonly{partDodecahedron}
% \includeonly{partKinematics}
% \includeonly{partConstitutiveTheory}
% \includeonly{partNumericalIntegration}
% \includeonly{partVariationalFormulation}
% \includeonly{appImplicitElasticity}
% \includeonly{appOverview}
% \includeonly{appDodecahedron}
% \includeonly{appVertices}
% \includeonly{appChords}
% \includeonly{appPentagons}
% \includeonly{appModels}
% \includeonly{appSolvers}

%%
% \usepackage{wasysym}
% Extracted out the font for a pentagon directly to avoid 'too many math alphabets'
\def\wasyfamily{\fontencoding{U}\fontfamily{wasy}\selectfont}
\def\pentagon   {\mbox{\wasyfamily\char68}}

\usepackage{geometry,mathtools}
\setcounter{MaxMatrixCols}{12}



\begin{document}

\begin{frontmatter}

%% Title, authors and addresses

%% use the tnoteref command within \title for footnotes;
%% use the tnotetext command for the associated footnote;
%% use the fnref command within \author or \address for footnotes;
%% use the fntext command for the associated footnote;
%% use the corref command within \author for corresponding author footnotes;
%% use the cortext command for the associated footnote;
%% use the ead command for the email address,
%% and the form \ead[url] for the home page:
%%
%% \title{Title\tnoteref{label1}}
%% \tnotetext[label1]{}
%% \author{Name\corref{cor1}\fnref{label2}}
%% \ead{email address}
%% \ead[url]{home page}
%% \fntext[label2]{}
%% \cortext[cor1]{}
%% \address{Address\fnref{label3}}
%% \fntext[label3]{}

\title{A Dodecahedral Model for Alveoli}

%% use optional labels to link authors explicitly to addresses:
%% \author[label1,label2]{<author name>}
%% \address[label1]{<address>}
%% \address[label2]{<address>}

\author[add1,add2]{Alan D.~Freed\corref{corr}}

\ead{afreed@tamu.edu, alan.d.freed.civ@mail.mil}

\cortext[corr]{Corresponding author}

\author[add1]{Shahla Zamani}

\author[add2]{John D.~Clayton}

% \author[add1]{J.~N.\ Reddy}

\date{Draft version: \today}

\address[add1]{Department of Mechanical Engineering, 
              Texas A\&M University,
              College Station, \\ TX 77843, 
              United States}
          
\address[add2]{Impact Physics Branch, 
               U.S.\ Army Research Laboratory,
               Aberdeen Proving Ground, 
               Aberdeen, MD 21005, 
               United States}
              
              
\begin{abstract}

write

\end{abstract}

\begin{keyword}
%% keywords here, in the form: keyword \sep keyword

write

%% MSC codes here, in the form: \MSC code \sep code
%% or \MSC[2008] code \sep code (2000 is the default)

% \MSC 15A06 \sep 65F05 \sep 74A05 \sep 74A10 \sep 74A15 \sep 74A20

\end{keyword}

\end{frontmatter}

% \linenumbers

%% main text

\newpage
\setcounter{part}{0}
\section{Introduction}
\label{partIntroduction}

Injuries that occur after a non-penetrating ballistic projectile impacts a Soldier wearing personal protective equipment (PPE) are referred to as behind armor blunt trauma (BABT).  The kinetic energy from such an impact is absorbed by the Soldier's PPE and the bony and soft tissues of the Soldier beneath \cite{Cannon01,grimal2004,cronin2015}.  Standards have been written to which PPE have been designed since 1972.  Verification is through experiments where, typically, a suit of body armor is placed over a ``body'' subjected to a ballistic impact from a projectile fired by a weapon, all in accordance with a standard.  Current practice is to use clay (usually Roma Plastilina No.~1 clay) as a surrogate for the human body in these tests \cite{HanlonGillich2012}.   Injuries that occur from a rapidly increasing overpressure (e.g., shockwaves from an explosion) are denoted as primary blast injuries (PBI) \cite{gibbons2015}.   Blast lung injury (BLI) refers to PBI experienced by the lung \cite{zhou2010}.  Dynamic blunt thoracic trauma can also occur in nonmilitary settings (e.g., automobile accidents), and PBI likewise may occur irrespective of the presence of PPE.

A principal objective of an internal US Army Combat Capabilities Development Command (DEVCOM) Army Research Laboratory (ARL)--Weapons and Materials Research Directorate (WMRD) project, \textit{Modeling Large Deformations and Stress Wave Mechanics in Soft Biological Tissue}, is to develop accurate material models for the human body that are also efficient in their finite element implementation, thereby facilitating the study of BABT and PBI in an effort to improve the designs of PPE.  This is a 6.1 research project whose hand-off to a 6.2 development team at project's end will aid Army engineers in their design of improved PPE by allowing them to run in-silico BABT tests to complement their actual in-field testing.

The DEVCOM Army Research Laboratory-WMRD \textit{Modeling Large Deformations and Stress Wave Mechanics in Soft Biological Tissue\/} project has three primary objectives: \textit{i\/}) new material models, \textit{ii\/}) new experiments, and \textit{iii\/}) new trauma metrics.  Lung has been selected as the soft tissue of interest for this study.  What are sought are models and metrics whose parameters are physical and unique, and whose numeric implementation will be efficient and stable.  Continuum thermo\-dynamic models for lung tissue and a trauma metric are being developed (Clayton, Freed, and co-authors \cite{Clayton2019TRL,ClaytonFreed19,ClaytonFreed20,ClaytonFreed20a,Clayton2019AIP,claytonBM20,clayton2020TRL} and this document).  The work done under this sub-project, \textit{A Dodecahedral Model for Alveoli}, complements its parent project, \textit{Modeling Large Deformations and Stress Wave Mechanics in Soft Biological Tissue}, with regard to the first and third objectives of this ARL-WMRD program.  The models being developed are expected to be improvements over those currently supplied by \texttt{LS-DYNA} in their materials library that, e.g., have been used to study shock waves traversing a human torso not wearing body armor, cf.\ Fig.~\ref{figShockWaveInLung}.  

\begin{figure}
    \centering
    \includegraphics[width=0.95\textwidth]{figures/shockWaveInTorso.png}
    \caption{Finite element analysis done using \texttt{LS-DYNA}  to model shock waves traversing a cross-sectional slice of a human torso.  Material models from the \texttt{LS-DYNA}  library of available models were used \cite{Josey10}.}
    \label{figShockWaveInLung}
\end{figure}

BABT and BLI manifest at the microscopic level of alveoli, which make up the parenchyma, i.e., the spongy tissue of lung that composes around 90\% of lung by volume, cf.\ Fig.~\ref{figLungDrawing}, there being some 500 million alveoli in a typical human lung.  Most damage occurs just beneath the visceral pleural, as seen in Fig.~\ref{figDamagedLung}, and is thought to be a consequence of the large disparity in wave speeds between solid tissues ($\sim$1,500~m/s) and the spongy parenchyma ($\sim$30--40~m/s) \cite{Stuhmiller08}.  The objective of this work is to develop a mechanistic multi-scale model that is capable of describing the deformation and damage that occur at an alveolar level, caused by a shock wave traveling through the parenchyma, induced through either a blast or a ballistic impact to a Soldier's PPE.  In-silico experiments done using this microscopic model are to be used to ``inform'' our macroscopic model in those areas where actual lung experiments are difficult, if not impossible, to perform.

\begin{figure}
    \centering
    \includegraphics[width=0.7\textwidth]{figures/theRespiratorySystem.png}
    \caption{A medical drawing of the respiratory system \cite{Josey10}}
    \label{figLungDrawing}
\end{figure}

\begin{figure}
    \centering
    \includegraphics[width=0.7\textwidth]{figures/lungInjuryResultingFromBlast.png}
    \caption{Lungs excised from animals (most likely ovine) who expired from blast injury \cite{Stuhmiller08}}
    \label{figDamagedLung}
\end{figure} 

This is the first full-length report for the project \textit{A Dodecahedral Model for Alveoli}.  This first report discusses theoretical foundations and numerical techniques for interrogating the dodecahedral response.  A second report describing results of numerical calculations is anticipated in the next 12 months.

\subsection{$\,$Problem Statement}

Pulmonary contusion is one of the most common thoracic soft-tissue injuries caused by blunt trauma, with a mortality rate of 10\%--25\% \cite{Stitzeletal05}.  Damage to lungs is the main cause of morbidity following high-level blast exposures \cite{Stuhmilleretal88}.  Lung laceration is also common and debilitating \cite{VlessisTrunkey97}.  Existing constitutive models for lung tissue have been developed from limited static test data, e.g., Fung, Vawter \textit{et~al}.\ \cite{Fungetal78,Vawteretal79,Vawter80} These models, and others developed since then, omit relevant physics pertinent to blast and ballistic impacts required to assess BLI and BABT, respectively.  They also require cumbersome optimization protocols to fit non-unique parameter sets \cite{Gayziketal07,Gayziketal11}, and\slash or are not validated against independent data \cite{Yuenetal08}.  Better lung models suitable for dynamic analysis are needed so that the Army can design improved PPE to better protect Soldiers.

The primary objective of the ARL-WMRD project \textit{Modeling Large Deformations and Stress Wave Mechanics in Soft Biological Tissue\/} is to develop such models for deformation and damage\slash injury assessment.  These are continuum models derived from thermo\-dynamics that utilize internal state variables to account for the irreversible aspects of response \cite{ClaytonFreed19,ClaytonFreed20}.  Models (both macro\-scopic and micro\-scopic) are specifically sought whose parameters are physical and whose parameterization is straightforward.  Characterization of the parameters in a model requires experimental data.  This presents an enormous challenge, one that is being addressed in this ARL-WMRD project through other university collaborators.  

Performing experiments for the purpose of model characterization is extremely difficult when it comes to modeling lung.  Lung is a structure; parenchyma is a material.  Therefore, one would normally choose to test the parenchyma, and from these data extract one's model parameters but, because of its spongy nature, we are challenged to do so in a physically meaningful way.  Consequently, one typically tests whole lungs, or lobes thereof, and from these structural experiments we are tasked to extract material parameters through an inverse analysis.  An alternative approach whereby one could, in principle, acquire parameters for the continuum models being developed at ARL-WMRD would be to homo\-genize a microscopic structural response for the alveoli of the parenchyma.  The work presented here addresses this approach in our modeling of deformation, damage, and injury in alveolar structures.

Our approach is also advantageous for understanding the influence of microstructure on the higher-scale continuum properties.  Curve fitting to macroscopic data alone does not provide such insight.  This multi-scale approach can also be used to determine properties for regimes (stress\slash strain\slash strain-rate histories) that cannot be reached experimentally, due to limitations in testing facilities, capabilities, or sufficient animal\slash human tissue availability. 

The narrative that follows seeks to develop two material models for lung: one for mechanical deformation and the other for damage\slash injury\slash trauma.  Models are sought whose parameters have physical interpretation.  Ideally, they will enhance our understanding of the deformation and damage mechanisms at play during BABT and BLI.  Specifically, they will describe how alveoli respond to pressure-waves and\slash or shear-waves as these wave fronts pass through them.  This modeling will be accomplished by constructing a multi-scale model connecting the parenchyma (macro) and alveolar (micro) levels.  In-silico experiments can then be done on the alveolar structural model, whose homogenized response can serve as an aid in the characterization of ARL's continuum models.  These ARL-WMRD continuum models are being designed to perform efficiently in their implementation in finite element codes.  This will allow for BABT and BLI analyses to be done during the design of future PPE with an ultimate goal of saving Soldiers' lives.

The primary purpose of this work is to provide a micro\-scopic model for lung tissue that can be used as an aid in the parameterization of a macro\-scopic model for lung that will be reasonably accurate yet efficient to run in full torso finite element analyses \cite{clayton2020TRL} to study BABT and BLI for the purpose of improving PPE.  Secondarily, the alveolar-level model will provide stand-alone information that will increase our fundamental understanding of the thermomechanical response of lung parenchyma to dynamic loading.  

\subsection{$\,$Approach}

Figure~\ref{figRatLung} shows micrographs from a rat lung taken at different magnifications. In the lower-resolution image, one sees numerous alveoli that became exposed because of the sectioning process.  Also present are several alveolar ducts that connect individual alveoli to a bronchial tree.  In the higher-resolution image we observe the faceted structure of these alveoli, wherein one can see the septal chords and membranes, the latter being traversed by capillaries through which gas exchange occurs.  Gas exchange is not modeled here.

\begin{figure}
    \centering
    \subfigure[Magnification at 100X. This is Fig.~5 in Freed \textit{et~al}.\ \cite{Freedetal12}.]{
        \includegraphics[width=0.45\textwidth]{figures/ratLung100X.jpg}
        \label{figRatLung100}
    }
    \hfill
    \subfigure[Magnification at 750X. This is Fig.~7 in Freed \textit{et~al}.\ \cite{Freedetal12}.]{
        \includegraphics[width=0.45\textwidth]{figures/ratLung750X.jpg}
        \label{figRatLung750}
    }
    \caption{Scanning electron microscopy photographs from a sectioned rat lung.  The alveolar diameter in rat lung is about one quarter the alveolar diameter in human lung.}
    \label{figRatLung}
\end{figure}

Alveolar geometry is modeled here as a dodecahedron, i.e., a soccer-ball like structure comprising 12 pentagonal facets bordered by 30 septal cords that are connected at 20 vertices.  Each vertex links three neighboring cords of the alveolus with a fourth chord from a neighboring alveolus.  BABT and BLI can occur through multiple mechanisms, e.g., tearing of septal cords and\slash or alveolar membranes, and in more severe cases, rupturing of capillaries can also happen causing interstitial fluids and blood to leak into neighboring alveoli, all illustrated in Fig.~\ref{figAlveolarDamage}.  Our dodecahedral model for alveoli is capable of capturing these trauma events.

\begin{figure}
    \centering
    \subfigure[Immunohistochemical staining for hemoglobin showing edema fluid buildups (arrows) caused by blast injury.]{
        \includegraphics[width=0.3\textwidth]{figures/edemaDamage.png}
        \label{figAlveolarDamageA}
    }
    \hfill
    \subfigure[Histopathology image showing tearing of septal membranes (arrows) caused by blast injury.]{
        \includegraphics[width=0.3\textwidth]{figures/septalDamage.png}
        \label{figAlveolarDamageB}
    }
    \hfill
    \subfigure[Electron microscope image showing perforations (arrows) of the alveolar wall caused by blast injury.]{
    \includegraphics[width=0.3\textwidth]{figures/alveolarDamage.png}
    \label{figAlveolarDamageC}
    }
    \caption{Local injury mechanisms in blast lung. All images from Tsokos \textit{et~al}.\ \cite{Tsokosetal03}.}
    \label{figAlveolarDamage}
\end{figure}

\begin{conj}
\textit{A micro\-scopic strain field, measured at the scale of alveoli, is the same as its macro\-scopic strain field in which it resides, measured at the scale of parenchyma.  The motion is affine, and the local motion is homogeneous.}  
\label{conjecture}
\end{conj}

This hypothesis was tested and confirmed in an experimental study done by Butler \textit{et~al}.\ \cite{Butleretal96} where they used light scattering to study changes in geometry of the septal planes in alveoli, from which they concluded: ``the micro\-scopic strain field does not differ significantly from the macro\-scopic field.''  We employ this hypothesis by taking the deformation gradient from, say, a Gauss point in a finite element model of lung, e.g., from a Gauss point associated with Fig.~\ref{figShockWaveInLung}, and imposing it as a far-field deformation onto our dodecahedral model of an alveolus.  From this kinematic input we arrive at an upper bound on the macro\-scopic stress\slash stiffness response, akin to a Voigt approximation \cite{nemat1999,clayton2011}, through a homogenization of the micro\-scopic forces created within our structural model for an alveolus.

The authors of a recent review article on alveolar strain finished by writing:
\begin{quotation}
    \noindent\small ``In general, computational mechanics approaches to determine function in a healthy or diseased lung have proven to be useful in explaining or measuring observations that are not captured by imaging modalities. However, for these models to fully explain complex physiological mechanical events, appropriate mechanical properties, boundary conditions, and mechanical loads must be identified. Moreover, validation of such computational models, which is an essential component of any computational mechanics approach, remains to be a challenge in the analysis of soft tissue mechanics.''
    
    \nopagebreak
    \mbox{} \hfill Roan \& Waters \cite{RoanWaters11} (p. L633) \normalsize
\end{quotation}
In this research we set out to develop a constitutive framework for alveolar mechanics, fully cognizant of the aforementioned challenges.  Our objectives are different from those of prior studies in alveolar mechanics in that we seek to describe the response\slash injury of a human lung that has been subjected to a stress wave propagating across the thorax region caused by an impact from either a blunt object or a blast wave.  Consequently, some important aspects in the modeling of a breathing lung are thought to be less impactful here, e.g., the effect of surfactant in keeping alveoli from collapsing at the end of expiration.

As a foundation, we adopt the guideline:
\begin{quotation}
	\noindent\small ``Constitutive equations are phenomenological. They are regarded as empirical by experimenters, and axiomatic by mathematicians.  In biomechanics, we often try to derive them on the basis of micro\-structure $\ldots$ in order to gain a better understanding, or to get some guidance to the mathematical form.''
	
	\nopagebreak
	\mbox{} \hfill Y.-C.~Fung \cite{Fung90} (p.~431) \normalsize
\end{quotation}
The approach adopted here is to use the geometry of a dodecahedron as a \textit{micro}\-scopic mechanical model for alveoli, whose far-field response to mechanical stimuli, in accordance with our Conjecture on p.~\pageref{conjecture}, will be used to inform the development of a \textit{macro\/}scopic mechanical model for parenchyma, \cite{ClaytonFreed20} the predominant tissue in lung.  This is deemed necessary because of the complex porous structure of parenchyma, as compared with the homo\-geneous structure of rubbery elastic solids whose theories have historically been employed to model parenchyma \cite{Fung75,Fungetal78,Vawteretal79,Fung88}.  The complementary continuum (macroscopic) model for parenchyma \cite{Clayton2019TRL,ClaytonFreed20,Clayton2019AIP,claytonBM20,clayton2020TRL} is implemented into finite element codes with an end objective of providing a numerical tool that can be used by Army engineers in their efforts to develop improved and more effective designs for a Soldier's PPE. 

\subsection{$\,$Organization}

This document is organized in the following manner.  Section~\ref{partDodecahedron} introduces the dodecahedron as a model for alveoli.  Its geometric properties are derived in detail with regards to its three geometric features: 1D septal chords, 2D septal membranes, and 3D alveolar sacs.  Section~\ref{partKinematics} develops the kinematics required for us to model a deforming dodecahedron, again focusing on the 1D~chords, 2D~membranes, and the 3D~volume within, including the shape functions needed for interpolating each geometry.  Section~\ref{partConstitutive} derives constitutive models suitable for describing the thermomechanical response for the structural constituents of an alveolus: its septal chords, its permeable membranes, and its volume.  Section~\ref{partNumericalMethods} presents numerical methods used for solving first- and second-order, ordinary, differential equations (ODEs) and spatial integrations along a bar, across a pentagon, and throughout a tetrahedron using Gaussian quadrature schemes designed for each geometry.  Section~\ref{partVariational} describes a variational formulation used to create our structural modeling of an alveolus, which consists of three separate models: one consisting of septal chords, another consisting of septal membranes, and the third consisting of alveolar volume.  Forces at the vertices are summed and homogenized for return to the macroscopic solver.  Constitutive equations suitable for modeling biological tissues are derived from thermo\-dynamics using the theory of implicit elasticity, and are presented in the Appendix.


\newpage
\setcounter{equation}{0}
\setcounter{figure}{0}
\setcounter{section}{0}
\setcounter{table}{0}
\part{Dodecahedra: A Model for Alveoli}
\label{partDodecahedron}

Typical alveoli are fourteen sided polyhedra with one face normally being open as a mouth to an alveolar duct, and whose septal membranes are likely to become flat at transpulmonary pressures of about 2~cm~$\text{H}_2\text{O}$ \cite{HoppinHildebrandt77}.  To capture the microstructural features of lung, researchers have modeled both alveoli and alveolar ducts, as seen in Fig.~\ref{figRatLung}; we only address alveoli here.  Three different geometric shapes are typically employed when modeling an alveolus: a dodecahedron introduced by Frankus \& Lee \cite{FrankusLee74} in 1974, a rhombic dodecahedron introduced by de~Ryk, Thiesse, Namati \& McLennan \cite{Ryketal07} in 2007, and a truncated octahedron, i.e., a tetrakaidecahedron, introduced by Dale, Matthews \& Schroter \cite{Daleetal80} in 1980.  The dodecahedron and rhombic dodecahedron are both twelve sided polyhedra with faces being pentagons and rhombuses, respectively.  A tetrakaidecahedron is a pair of pyramids stacked bottom to bottom, forming an octahedron, whose six points are then removed.  The end result is a fourteen sided polyhedron with six faces that are squares and eight faces that are hexagons, where like shapes have like dimensions.

The tetrakaidecahedron and rhombic dodecahedron are both volume filling.  This property is preferred whenever one sets out to construct assemblages of alveoli to build a micro\-structural model that is to be solved numerically via a finite element method.  The purpose of such an exercise is to homogenize the response of an alveolar assembly up to the macroscopic level, i.e., the level of a continuum mass point, a.k.a., the parenchyma \cite{Daleetal80,DennySchroter95,DennySchroter97,DennySchroter00,Koweetal86,Ryketal07,Chenetal14}.  Such a finite element model can serve as a representative volume element (RVE) for parenchyma.

The dodecahedron is an isotropic structure, or very nearly so as we shall show, and is nearly volume filling \cite{Kimmeletal87}.  It becomes a preferred geometry whenever a single alveolus is to be used as the RVE of homo\-genization, and from which closed form solutions have been derived \cite{BudianskyKimmel87,KimmelBudiansky90,Kimmeletal87,Freedetal12}.  Here isotropy of the microstructure ensures an isotropic macro response.  Parenchyma, as a tissue, is isotropic \cite{Weedetal15,Fung88,Hughesetal72}; whereas, lung, as an organ, is a complex, heterogeneous structure \cite{Mead73,West07}.  This distinction has, from time-to-time, gotten lost \cite{DennySchroter06}.  

For the reasons stated above, a dodecahedron, with vertices labeled according to Fig.~\ref{figDodecahedron}, is the geometric structure selected for use in this study.
A dodecahedron contains twelve pentagons.  As such, we address the geometry of a pentagon first.  

\begin{figure}
	{\par\centering
		\resizebox*{0.45\textwidth}{0.3\textheight}
		{\includegraphics{figures/dodecahedron.jpg}} 
		\resizebox*{0.45\textwidth}{0.3\textheight}
		{\includegraphics{figures/dodecahedronVertices.jpg}}
		\par}
	\caption{Geometric representations for a dodecahedron.  Left image:  A cube is contained within a dodecahedron, with one of its five possible orientations being displayed.  Atop each face of the cube resides four pentagonal sub-areas that form the shape of a hipped roof line.  Right image:  Vertices 1 through 8 are located at the corners of such a cube.  The centroid for the cube is also the centroid for the dodecahedron.  Vertices 9 through 20 are corners of the hipped roof lines residing above each face of the cube.}
	\label{figDodecahedron}
\end{figure}

\section{Properties of a Regular Pentagon}

Figure~\ref{figRegPentagon} presents a regular pentagon drawn in its natural co-ordinate system with co\-ordinates designated as $(\xi, \eta)$.  Vertices of such a pentagon are placed at
\begin{equation}
	\xi = \cos \left( \frac{2(k-1)\pi}{5} + \frac{\pi}{2} \right) \quad
	\eta = \sin \left( \frac{2(k-1)\pi}{5} + \frac{\pi}{2} \right) \quad
	k = 1, 2, \ldots, 5
	\label{regPentagon}
\end{equation}
wherein $k$ denotes the vertex number, as assigned in Fig.~\ref{figRegPentagon}.  These vertices inscribe a pentagon within the unit circle.

\begin{figure}
	\centering
	\includegraphics[width=6cm]{figures/regPentagon.pdf}
	\caption{A regular pentagon inscribed within the unit circle establishes its natural co\-ordinate system with co-ordinates $(\xi, \eta)$ described in Eqn.~(\ref{regPentagon}), and whose origin is located at its centroid.  Vertices are numbered counterclockwise with the uppermost vertex being labeled~1.}
	\label{figRegPentagon}
\end{figure}

Lengths of the five chords in a regular pentagon, when measured in its natural co-ordinate system, are all
\begin{equation}
	L^{\!p} = 2 \cos (\omega) \approx 1.176 
	\label{regPentagonLength}
\end{equation}
while the area of this pentagon is
\begin{equation}
	A^p = \tfrac{5}{4} \tan ( \omega ) \, L^{\!p^2} = 
	5 \sin (\omega) \cos (\omega) \approx 2.378
	\label{regPentagonArea}
\end{equation}
where the area of an unit circle is $\pi r^2 \approx 3.142$ because $r=1$.  The
inside angles of a regular pentagon all measure $2\omega = 108^{\circ}$.  All approximations are truncated at four significant figures.

\subsection{Properties of a Regular Dodecahedron}

Like the pentagon considered above, which inscribes the unit circle, here we consider a dodecahedron that inscribes the unit sphere.  Let this geometry be described in its natural co-ordinate system with co-ordinates $(\xi , \eta , \zeta )$ whose origin is located at its centroid, the center of the sphere.  The twenty vertices of this dodecahedron, all of which touch the unit sphere, are placed at
\begin{equation}
\begin{tabular}{ccc}
	$\xi$ & $\eta$ & $\zeta$ \\ \hline
	$\pm 1 / \sqrt{3}$ & $\pm 1 / \sqrt{3}$ & $\pm 1 / \sqrt{3}^{\vphantom{|}}$ \\
	$\pm \phi / \sqrt{3}$ & $\pm 1 / \sqrt{3} \phi$ & 0 \\
	0 & $\pm \phi / \sqrt{3}$ & $\pm 1 / \sqrt{3} \phi$ \\
	$\pm 1 / \sqrt{3} \phi$ & 0 & $\pm \phi / \sqrt{3}$
\end{tabular}
\label{regDodecahedron}
\end{equation}
where $\phi = (1 + \sqrt{5})/2 \approx 1.618$ is also known as the golden ratio.

Lengths of the thirty chords in a regular dodecahedron, when measured in its natural co-ordinate system, are all
\begin{equation}
	L^{\!d} = \frac{2}{\sqrt{3} \phi} \approx 0.7136
	\label{regDodecahedronLength}
\end{equation}
while the volume of such a dodecahedron is
\begin{equation}
	V^d = \frac{40}{3 \sqrt{3} \phi^3} \tan^2 ( \omega ) \sin ( \omega ) \approx 2.785
\label{regDodecahedronVolume}
\end{equation}
where volume of the unit sphere is $\tfrac{4}{3} \pi r^3 \approx 4.189$ because $r=1$.

The scale factor to map between the natural co-ordinates of a pentagon, defined in Eq.~(\ref{regPentagon}), with those of a dodecahedron, defined in Eq.~(\ref{regDodecahedron}), is
\begin{equation}
	\frac{L^{\!p}}{R^p} = \frac{L^{\!d}}{R^p_d} 
	\quad \text{or} \quad
	R^p_d = \frac{R^p L^{\!d}}{L^{\!p}} = \frac{L^{\!d}}{L^{\!p}} = 
	\frac{1}{\sqrt{3} \phi \cos (\omega)} \approx 0.6071
	\label{scaleFactor}
\end{equation}
because $R^p = 1$, with scale factor $R^p_d$ being the radius that inscribes a pentagon on the surface of a dodecaheron that itself inscribes the unit sphere.

\subsection{Dimensions of Human Alveoli}
\label{alveolarSize}

Septal chord length $L(D)$, expressed as a function of alveolar diameter $D$, can be estimated by considering the areal projection of a dodecahedron onto a plane that contains one of its pentagonal faces, which leads to
\begin{equation}
	L = \frac{D}{\tan ( \omega ) ( 1 + \cos ( \alpha ) )} 
	\approx  \frac{D}{2.685} ,
\label{dodecahedralHeight}
\end{equation} 
where $\alpha = \textfrac{\pi}{10} = 18^{\circ}$.  (There are twenty, equal, pie-shaped wedges that comprise this projected area.)  This is an average of the shortest and longest distances across this plane of projection.  Alveolar diameter $D$ is a property that can be measured during an histological study of parenchyma.  

To dimension the alveoli of human lung, Sobin, Fung \& Tremer \cite{Sobinetal88} measured the mean diameter across an individual alveolus, viz., $D$ of Eq.~(\ref{dodecahedralHeight}),  sectioned from human lungs that were fixed at three different pressures.  Samples were taken from nine lungs extracted postmortem from individuals between 16 to 89 years of age.\footnote{%
	Sobin \textit{et~al}.\ \cite{Sobinetal88} documented an age effect in these data that has been averaged over here, i.e., ignored.
}
At a transpulmonary pressure of 4~cm~$\text{H}_2$O, the mean alveolar diameter was $D = 191$ $\pm$ 86~$\mu$m determined from a sampling size of 1423; at a pressure of 10~cm~$\text{H}_2$O, $D = 202$ $\pm$ 88~$\mu$m determined from a sampling size of 1296; and at a pressure of 14~cm~$\text{H}_2$O, $D = 235$ $\pm$ 99~$\mu$m determined from a sampling size of 1083.  These data are plotted in Fig.~\ref{septalLengthFig}.  All reported and drawn error bounds pertain to plus\slash minus one standard deviation in error.

\begin{figure}
	\centering
	\includegraphics[width=0.4\textwidth]{figures/septalLength.jpg}
    \includegraphics[width=0.44\textwidth]{figures/alveolarDiaHistogram.jpg}
	\caption{Mean and standard deviations for alveolar diameter in human lung illustrated in the left graphic, as reported by Sobin \textit{et~al}.\ \cite{Sobinetal88}, with a typical histogram for these statistics illustrated in the right graphic.  Note that this distribution has been truncated at an alveolar diameter of 24~$\mu$m.}
	\label{septalLengthFig}
\end{figure}

\subsection{Geometries of an Irregular Dodecahedron}
\label{sec:geometries}

Formul\ae\ (\ref{regPentagonArea} \& \ref{regDodecahedronVolume}) only apply for regular pentagons and dodecahedra evaluated in their respective natural co-ordinate systems.  For irregular dodecahedra, the areas of its irregular pentagons are calculated via\footnote{
	Bourke, P., ``Polygons, Meshes.'' \texttt{http://paulbourke.net/geometry}.
}
\begin{equation}
	A = \frac{1}{2} \sum_{i=1}^5 ( x_i y_{i+1} - x_{i+1} y_i)
	\label{irregularPentagonArea}
\end{equation}
where $x_6 \Leftarrow x_1$ and $y_6 \Leftarrow y_1$.  In order for the predicted area to be positive when using this formula, it is necessary that the vertices $(x_i , y_i)$ index counterclockwise, as drawn in Fig.~\ref{figRegPentagon}.  The centroid of this pentagon has co-ordinates\footnotemark[\value{footnote}]
\begin{subequations}
	\label{centroidPentagon}
	\begin{align}
		c_x & = \frac{1}{6 A} \sum_{i=1}^5 (x_i + x_{i+1})
			( x_i y_{i+1} - x_{i+1} y_i) \\
		c_y & = \frac{1}{6 A} \sum_{i=1}^5 (y_i + y_{i+1})
		( x_i y_{i+1} - x_{i+1} y_i)
	\end{align}
\end{subequations}
wherein the vertex co-ordinates $x_i$ and $y_i$ are quantified in a 2D pentagonal frame of reference, e.g., as established later in Fig.~\ref{figPentagonCoord}.

To compute the volume of an irregular dodecahedron, use the formula\footnote{
	Colins, K.~D., ``Cayley-Menger Determinant.'' From MathWorld--A Wolfram Web Resource, created by Eric W.\ Weisstein. \texttt{http://mathworld.wolfram.com/Cayley\-MengerDeterminant.html}.
}
\begin{equation}
	288 V^2_{tet} = \left| \begin{matrix}
	0 & 1 & 1 & 1 & 1 \\
	1 & 0 & \ell_{12}^{\,2} & \ell_{13}^{\,2} & \ell_{14}^{\,2} \\
	1 & \ell_{21}^{\,2} & 0 & \ell_{23}^{\,2} & \ell_{24}^{\,2} \\
	1 & \ell_{31}^{\,2} & \ell_{32}^{\,2} & 0 & \ell_{34}^{\,2} \\
	1 & \ell_{41}^{\,2} & \ell_{42}^{\,2} & \ell_{43}^{\,2} & 0
	\end{matrix} \right|
	\label{tetrahedralVolume}
\end{equation}
to calculate each of the 60 individual tetrahedral volumes that collectively fill the volume of an irregular dodecahedron.  Here $\ell_{ij}$ is the length of that tetrahedral edge with vertices $i$ and $j$; $i, j = 1, 2, 3, 4$; $i \neq j$; with $\ell_{ij} = \ell_{ji}$.

\subsection{Indexing Scheme for Dodecahedra}

In order to implement the dodecahedron as a geometric model for an alveolar sac, as suggested by the images in Fig.~\ref{figRatLung}, it first becomes necessary to introduce a labeling strategy. Such a scheme is arbitrary, but once chosen it enables an analysis to be put forward.  The labeling scheme adopted in this work is illustrated in the right image of Fig.~\ref{figDodecahedron}. 
  
The co-ordinates positioning the twenty vertices of a regular dodecahedron in its natural frame of reference are presented in Table~\ref{TableDodecahedron}.  According to the labeling scheme of Fig.~\ref{figDodecahedron}, the thirty chords of a dodecahedron are given vertex assignments according to Table~\ref{Tablechordae}, while its twelve pentagons are given vertex assignments according to Table~\ref{TablePentagons}.

\begin{table}
	\begin{center}
	\begin{tabular}{|c|ccc||c|ccc|}
		\hline 
		Vertex & $\xi$ & $\eta$ & $\zeta$ & Vertex & $\xi$ & $\eta$ & $\zeta$ \\ \hline
		1 & $1 / \sqrt{3}$ & $1 / \sqrt{3}$ & $1 / \sqrt{3}^{\vphantom{|}}$ & 
		   11 & $\phi / \sqrt{3}$ & $1 / \sqrt{3} \phi$ & 0 \\
		2 & $1 / \sqrt{3}$ & $1 / \sqrt{3}$ & -$1 / \sqrt{3}$ & 
		   12 & $\phi / \sqrt{3}$ & -$1 / \sqrt{3}\phi$ & 0 \\
		3 & -$1 / \sqrt{3}$ & $1 / \sqrt{3}$ & -$1 / \sqrt{3}$ & 
		   13 & -$\phi / \sqrt{3}$ & $1/ \sqrt{3}\phi$ & 0 \\
		4 & -$1 / \sqrt{3}$ & $1 / \sqrt{3}$ & $1 / \sqrt{3}$ & 
		   14 & -$\phi / \sqrt{3}$ & -$1 / \sqrt{3}\phi$ & 0 \\
		5 & $1 / \sqrt{3}$ & -$1 / \sqrt{3}$ & $1 / \sqrt{3}$ & 
		   15 & $1 / \sqrt{3} \phi$ & 0 & $\phi / \sqrt{3}$ \\
		6 & $1 / \sqrt{3}$ & -$1 / \sqrt{3}$ & -$1 / \sqrt{3}$ & 
		   16 & -$1 / \sqrt{3}\phi$ & 0 & $\phi / \sqrt{3}$ \\
		7 & -$1 / \sqrt{3}$ & -$1 / \sqrt{3}$ & -$1 / \sqrt{3}$ & 
		   17 & $1 / \sqrt{3}\phi$ & 0 & -$\phi / \sqrt{3}$ \\
		8 & -$1 / \sqrt{3}$ & -$1 / \sqrt{3}$ & $1 / \sqrt{3}$ & 
		   18 & -$1 / \sqrt{3}\phi$ & 0 & -$\phi / \sqrt{3}$ \\
		9 & 0 & $\phi / \sqrt{3}$ & $1 / \sqrt{3}\phi$ & 
		   19 & 0 & -$\phi / \sqrt{3}$ & $1 / \sqrt{3}\phi$ \\
		10 & 0 & $\phi / \sqrt{3}$ & -$1 / \sqrt{3}\phi$ & 
		   20 & 0 & -$\phi / \sqrt{3}$ & -$1 / \sqrt{3}\phi$ \\
		\hline
	\end{tabular}
	\end{center}
	\caption{Natural co-ordinates for the vertices of a regular dodecahedron, as labeled in Fig.~\ref{figDodecahedron} according to Eq.~(\ref{regDodecahedron}).}
	\label{TableDodecahedron}
\end{table}

The sixty tetrahedra that fill the volume of the dodecahedron contain vertices according to the following strategy.  Beginning with pentagon~1 and sequencing to pentagon~12, two of the four vertices come from a side of the pentagon in question with the remaining two vertices being the centroid for the associated pentagon and the centroid for the dodecahedron, i.e., the co-ordinate origin.  From Table~\ref{TablePentagons}, tetrahedron~1 contains vertices 11 and 2 of pentagon~1, tetrahedron~2 contains vertices 2 and 10, tetrahedron~3 contains vertices 10 and 9, tetrahedron~4 contains vertices 9 and 1, tetrahedron~5 contains vertices 1 and 11, tetrahedron~6 contains vertices 10 and 2 from pentagon~2, etc.

\begin{table}
	\begin{center}
	\begin{tabular}{|c|c||c|c||c|c|}
		\hline
		Chord & Vertices & Chord & Vertices & Chord & Vertices \\ \hline
		1 & 9, 10 & 11 & 17, 18 & 21 & 7, 18 \\
		2 & 1, 9 & 12 & 3, 18 & 22 & 7, 14 \\
		3 & 2, 10 & 13 & 4, 16 & 23 & 13, 14 \\
		4 & 3, 10 & 14 & 15, 16 & 24 & 8, 14 \\
		5 & 4, 9 & 15 & 1, 15 & 25 & 8, 16 \\
		6 & 1, 11 & 16 & 5, 15 & 26 & 5, 19 \\
		7 & 2, 11 & 17 & 5, 12 & 27 & 6, 20 \\
		8 & 3, 13 & 18 & 11, 12 & 28 & 7, 20 \\
		9 & 4, 13 & 19 & 6, 12 & 29 & 8, 19 \\
		10 & 2, 17 & 20 & 6, 17 & 30 & 19, 20 \\
		\hline
	\end{tabular}
	\end{center}
	\caption{Vertices that locate the endpoints of septal chords in a dodecahedron, as labeled in Fig.~\ref{figDodecahedron}.}
	\label{Tablechordae}
\end{table}

\begin{table}
	\begin{center}
	\begin{tabular}{|c|c|c|}
		\hline
		Pentagon & Vertices & Chords \\ \hline
		1 & 11, 2, 10, 9, 1 & 6, 7, 3, 1, 2 \\
		2 & 10, 2, 17, 18, 3 & 4, 3, 10, 11, 12 \\
		3 & 13, 4, 9, 10, 3 & 8, 9, 5, 1, 4 \\
		4 & 9, 4, 16, 15, 1 & 2, 5, 13, 14, 15 \\
		5 & 15, 5, 12, 11, 1 & 15, 16, 17, 18, 6 \\
		6 & 17, 2, 11, 12, 6 & 20, 10, 7, 18, 19 \\
		7 & 18, 7, 14, 13, 3 & 12, 21, 22, 23, 8 \\
		8 & 16, 4, 13, 14, 8 & 25, 13, 9, 23, 24 \\
		9 & 12, 5, 19, 20, 6 & 19, 17, 26, 30, 27 \\
		10 & 14, 7, 20, 19, 8 & 24, 22, 28, 30, 29 \\
		11 & 20, 7, 18, 17, 6 & 27, 28, 21, 11, 20 \\
		12 & 19, 5, 15, 16, 8 & 29, 26, 16, 14, 25 \\
		\hline
	\end{tabular}
	\end{center}
	\caption{Vertices that locate the corners of regular pentagonal surfaces in a regular dodecahedron, and the chords that connect them.  They are indexed counterclockwise when viewed looking from the outside in, and labeled according to Fig.~\ref{figDodecahedron}.  The apex for each pentagon resides at the peak of the hipped roof-line for that pentagon.  This turns out to be important.} 
	\label{TablePentagons}
\end{table}

\subsection{Co-ordinate Systems for Chordal Fibers and Pentagonal Membranes}

The dodecahedron used to model an alveolus is considered to be regular in its `natural' configuration, with the capability of being irregular in its reference configuration, and certainly becoming irregular after deformation.  The co-ordinate frame of its natural state is taken to have its origin positioned at the centroid of this regular dodecahedron, i.e., at the centroid of its enclosed cube (cf.\ Fig.~\ref{figDodecahedron}) or, equivalently, at the origin of the unit sphere the dodecahedron inscribes, as presented in Table~\ref{TableDodecahedron}.  We denote the base vectors associated with this frame of reference as $( \vec{\boldsymbol{\imath}} , \vec{\boldsymbol{\jmath}} , \vec{\mathbfit{k}} \, )$.  There are two other co-ordinate systems with relevance to our analysis: those for the chordal fibers, and those for the pentagonal membranes.

The local co-ordinate system of a chordal fiber is presented in Fig.~\ref{figchord}, while the local co-ordinate system of a pentagonal membrane is presented in Fig.~\ref{figPentagonCoord}.  The software described in \ref{appchords} and \ref{appPentagons} utilize these co-ordinate systems, whose origins are located at their centroids, along with their rotation and spin matrices, which are measured relative to the co-ordinate frame of the dodecahedron, i.e., relative to $( \vec{\boldsymbol{\imath}} , \vec{\boldsymbol{\jmath}} , \vec{\mathbfit{k}} \, )$.

\begin{figure}
    \centering
    \includegraphics[width=8cm]{figures/chord.pdf}
    \caption{The co-ordinate system of a chord $( \vec{\mathbf{e}}_{\,1} , \vec{\mathbf{e}}_{\,2} , \vec{\mathbf{e}}_{\,3} )$ relative to the co-ordinate system of its dodecahedron $( \vec{\boldsymbol{\imath}} , \vec{\boldsymbol{\jmath}} , \vec{\mathbfit{k}} \,)$ with origins located at their respective centroids that are offset by a translation $\boldsymbol{\chi}$.  These describe a mapping $[ \{ \vec{\mathbf{e}}_{\,1} \} \{ \vec{\mathbf{e}}_{\,2} \} \{ \vec{\mathbf{e}}_{\,3} \} ] = [ \{ \vec{\boldsymbol{\imath}} \, \}\{ \vec{\boldsymbol{\jmath}} \, \}\{ \vec{\mathbfit{k}} \, \} ] \mathbfsf{P}$ where $\mathbfsf{P}$ is an orthogonal rotation.  The tangent base vector $\vec{\mathbf{e}}_{\,1}$ aligns with the axis of this chord. The normal base vector $\vec{\mathbf{e}}_{\,2}$ is coaxial with a line segment drawn from the origin out to the chordal axis such that $\vec{\mathbf{e}}_{\,1} \cdot \vec{\mathbf{e}}_{\,2} = 0$. While the binormal base vector is given by the cross product $\vec{\mathbf{e}}_{\,3} = \vec{\mathbf{e}}_{\,1} \times \vec{\mathbf{e}}_{\,2}$.}
    \label{figchord}
\end{figure}

\begin{figure}
    \centering
    \includegraphics[width=8cm]{figures/pentagonCoord.pdf}
    \caption{The co-ordinate system of a pentagon $( \vec{\mathbf{e}}_{\,1} , \vec{\mathbf{e}}_{\,2} , \vec{\mathbf{e}}_{\,3} )$ relative to the co-ordinate system of its dodecahedron $( \vec{\boldsymbol{\imath}} , \vec{\boldsymbol{\jmath}} , \vec{\mathbfit{k}} \,)$ with origins located at their respective centroids that are offset by a translation $\boldsymbol{\chi}$.  These describe a mapping $[ \{ \vec{\mathbf{e}}_{\,1} \} \{ \vec{\mathbf{e}}_{\,2} \} \{ \vec{\mathbf{e}}_{\,3} \} ] = [ \{ \vec{\boldsymbol{\imath}} \, \}\{ \vec{\boldsymbol{\jmath}} \, \}\{ \vec{\mathbfit{k}} \, \} ] \mathbfsf{P}$ where $\mathbfsf{P}$ is an orthogonal rotation.  Base vector $\vec{\mathbf{e}}_{\,1}$ is coaxial to a line segment that connects two vertices which locate a pair of shoulders in a pentagon, viz., vertices 2 and 5 in Fig.~\ref{figRegPentagon}.  Base vector $\vec{\mathbf{e}}_{\,2}$ is coaxial with a line segment drawn from the head of this pentagon, i.e., vertex 1 in Fig.~\ref{figRegPentagon}, down to its base such that $\vec{\mathbf{e}}_{\,1} \cdot \vec{\mathbf{e}}_{\,2} = 0$.  Base vector $\vec{\mathbf{e}}_{\,3} = \vec{\mathbf{e}}_{\,1} \times \vec{\mathbf{e}}_{\,2}$ is the outward normal to this surface.}
    \label{figPentagonCoord}
\end{figure}



\newpage
\setcounter{equation}{0}
\setcounter{figure}{0}
\setcounter{section}{0}
\setcounter{table}{0}
\part{Kinematics}
\label{partKinematics}

The irregular dodecahedron used here as a model for alveoli describes a 3D structure composed of thirty 1D rods (the septal chords) joined at twenty nodes (the vertices) that collectively circumscribe twelve 2D pentagonal membranes (the alveolar septa) that in turn envelop an alveolar sac whose volume is represented using sixty tetrahedra.  To be able to describe the overall mechanical response of this 3D dodecahedral structure, it is conjectured to be sufficient to know the individual mechanical responses of its 1D septal chords, its 2D septal membranes, and the 3D void within.  Their relevant kinematics are presented here, along with the shape functions used for interpolation and their descriptions of deformation via stretch, using Laplace stretch \cite{Freedetal20} as our chosen kinematic field.

\section{1D Chords}

The stretch of a rod under extension is a ratio of its lengths.  Specifically, $\lambda \defeq L / L_0$ where $L$ and $L_0$ are its current and reference lengths, respectively, whose strain and strain rate are taken to be $e = \ln \lambda$ and $\mathrm{d} e = \lambda^{-1} \mathrm{d} \lambda$.  This is often referred to as a logarithmic, natural or true strain.  Consequently, the kinematic analysis of a chord is trivial.

\subsection{Shape Functions for Interpolating a Rod}

A two-noded alveolar chord has shape functions $N_i$, $i=1,2$, that, when evaluated in its natural co-ordinate system where $-1 \leq \xi \leq 1$, describe a matrix with elements
\begin{subequations}
    \label{shapeFnsChord}
    \begin{align}
\mathbf{N} & = \begin{bmatrix} N_1 & N_2 \end{bmatrix} =
\begin{bmatrix}
\frac{1}{2} \, (1 - \xi) &  \frac{1}{2} \, (1 + \xi)
\end{bmatrix} \\
\intertext{that interpolate vector fields according to}
\mathbfit{x} ( \xi ) & = \sum_{i=1}^2 N_i ( \xi ) \, x_i , \quad
\mathbfit{u} ( \xi ) = \sum_{i=1}^2 N_i ( \xi ) \, u_i  \\
\intertext{etc., and whose spatial gradients are} 
N_{1,\xi} & = -\tfrac{1}{2} 
\quad \text{and} \quad
N_{2,\xi} = \tfrac{1}{2}
\end{align}
\end{subequations}
wherein $\xi$ is the natural co-ordinate.  Components $x_i$ and $u_i \defeq x_i - x_{0i}$, $i=1,2$, are their global co-ordinates and displacements, respectively, located at the two nodes of a chord evaluated in the co-ordinate frame $( \vec{\mathbfsf{e}}_1 , \vec{\mathbfsf{e}}_2 , \vec{\mathbfsf{e}}_3 )$ of Fig.~\ref{figchord} with the chordal axis lying in the $\vec{\mathbfsf{e}}_1$ direction.

\subsection{Deformation Gradient for a Rod}

The deformation gradient in this case is simply
\begin{multline}
    \mathbfsf{F} ( \xi ) = 1 + \frac{\partial \mathbfit{u}}{\partial \xi} 
    \left( \frac{\partial \mathbfit{x}_0 }{\partial \xi} \right)^{-1} = 
    1 + \sum_{i=1}^2 N_{i,\xi} u_i \left( \sum_{i=1}^2 N_{i,\xi} x_{0i} \right)^{-1} \\
    = 1 + \frac{u_2 - u_1}{x_{02} - x_{01}} \, \vec{\mathbfsf{e}}_1 \otimes \vec{\mathbfsf{e}}_1 = \frac{x_2 - x_1}{x_{02} - x_{01}} \, \vec{\mathbfsf{e}}_1 \otimes \vec{\mathbfsf{e}}_1
\end{multline}
which is uniform over the length of a chord, i.e., it is independent of $\xi$.

\section{2D Triangles}

Triangular elements are needed in a support capacity in order to construct our alevolar model; specifically, the four surfaces of a tetrahedron are triangles.  What is required of them is a capability to compute the traction acting across such a surface through integration.  This requires knowledge of their shape functions and quadrature rules, the latter topic being discussed in Part~\ref{partNumericalMethods}. 

\subsection{Shape Functions for Interpolating a Triangle}

The shape functions for a triangle expressed in terms of its natural co-ordinates $( \xi , \eta )$, where $0 \leq \xi \leq 1$ and $0 \leq \eta \leq 1-\xi$, are given by
\begin{subequations}
    \label{triangleShapeFns}
    \begin{align}
    N_1 & = 1 - \xi - \eta &
    N_2 & = \xi &
    N_3 & = \eta 
    \intertext{with gradients of}
    N_{1,\xi} & = -1 & N_{2,\xi} & = 1 & N_{3,\xi} & = 0 \\
    N_{2,\eta} & = -1 & N_{2,\eta} & = 0 & N_{3,\eta} & = 1
    \end{align}
\end{subequations}
so that the area of a triangle in its natural co-ordinates is \textfrac{1}{2}.  

No further kinematics are required from triangular elements in our analysis.



\section{2D Irregular Pentagons}

The kinematics of an irregular pentagon, on the other hand, are not trivial.  Shape functions are required from which deformation gradients can then be constructed.  Once a deformation gradient is in hand, the state of stretch occurring within a pentagon at its Gauss points can finally be derived.  Several possible decompositions of the deformation gradient are possible, i.e., the notion of stretch is not unique in 2D (nor in 3D).  Here we employ the Laplace stretch \cite{Freedetal19}.

\subsection{Wachspress' Shape Functions for Interpolating an Irregular Pentagon}
\label{secShapeFns}

The idea here is to model each pentagonal face of a dodecahedron with one pentagonal finite element.  Five constant-strain triangles were originally considered, but their accuracy was found to be wanting when compared with that of a single pentagonal element whenever the deformation becomes non-uniform.  There was no difference between them whenever the deformation was just uniform dilation.

In 1975, Wachspress \cite{Wachspress75,Wachspress16} derived a set of shape functions $N_i$ that are capable of interpolating convex polyhedra.  His shape functions take on the form of rational polynomials, viz., $N_i = A_i / B$ where $A_i$ and $B$ are polynomials.  In contrast, classic isoparametric elements are constructed from polynomial shape functions \cite{Hughes87}.  For the Wachspress shape functions of a pentagon, the $A_i$ are cubic polynomials, while $B$ is a quadratic polynomial.

Let us consider a convex pentagonal domain $\Omega$ defined over $\mathbb{R}^2$ whose vertices have global co-ordinates of
\begin{displaymath}
(x_1, y_1) , \; (x_2, y_2) , \; (x_3, y_3) , \; (x_4, y_4), \; (x_5, y_5)
\end{displaymath}
when evaluated in the pentagonal co-ordinate system $( \vec{\mathbfsf{e}}_1 , \vec{\mathbfsf{e}}_2 )$ of Fig.~\ref{figPentagonCoord}, with $\vec{\mathbfsf{e}}_3$ being an outward normal to the pentagon.  Associated with this set of global co-ordinates is a set of local or natural co-ordinates
\begin{displaymath}
(\xi_1 , \eta_1) , \; (\xi_2 , \eta_2) , \; (\xi_3 , \eta_3) , \; (\xi_4 , \eta_4) , \; (\xi_5 , \eta_5)
\end{displaymath}
that describe a mapping of interpolation where
\begin{equation}
\begin{aligned}
x(\xi, \eta) & = \sum\nolimits_{i=1}^5 N_i (\xi, \eta) \, x_i \\
y(\xi, \eta) & = \sum\nolimits_{i=1}^5 N_i (\xi, \eta) \, y_i
\end{aligned} 
\qquad \text{or} \qquad
\mathbfit{x}(\boldsymbol{\xi}) = \sum_{i=1}^5 N_i (\boldsymbol{\xi}) \, 
\mathbfit{x}_{\,i}
\end{equation}
which relate natural co-ordinates $\boldsymbol{\xi} \equiv (\xi, \eta)$ to global co-ordinates $\mathbfit{x} \equiv (x, y)$, where $\mathbfit{x}_{\,i} \equiv (x_i, y_i)$ are nodal co-ordinates at the $i^{\mathrm{th}}$ vertex, with $i$ indexing counter\-clockwise around a pentagon according to Fig.~\ref{figRegPentagon}.  Displacement $\mathbfit{u} (\mathbfit{x}) \defeq \mathbfit{x} - \mathbfit{x}_0$, with reference co-ordinates $\mathbfit{x}_0 \equiv (x_0, y_0)$, also obeys this mapping
\begin{equation}
\begin{aligned}
u(\xi, \eta) & = \sum\nolimits_{i=1}^5 N_i (\xi, \eta) \, u_i \\
v(\xi, \eta) & = \sum\nolimits_{i=1}^5 N_i (\xi, \eta) \, v_i
\end{aligned} 
\qquad \text{or} \qquad
\mathbfit{u}(\boldsymbol{\xi}) = \sum_{i=1}^5 N_i (\boldsymbol{\xi}) \, 
\mathbfit{u}_{\,i}
\end{equation}
whose components $\mathbfit{u}_{\,i} \equiv (u_i , v_i)$ designate the nodal displacements.

Shape functions $N_i (\boldsymbol{\xi}) \equiv N_i (\xi, \eta)$ are interpolation functions that place any position $P$ with local co-ordinates $\boldsymbol{\xi} \equiv (\xi, \eta) \in \widebar{\Omega}$, where $\widebar{\Omega} \defeq \Omega \cup \partial\Omega$, into their global co-ordinates $\mathbfit{x} \equiv (x,y)$.  The shape functions of Wachspress \cite{Wachspress75,Wachspress16} possess the following properties \cite{SukumarMalsch06}:
\begin{enumerate}
	\item Partition of unity: $\sum\nolimits_{i=1}^5 N_i (\boldsymbol{\xi}) = 1$, \; $0 \leq N_i (\boldsymbol{\xi}) \leq 1$.
	\item Interpolate nodal data: $N_i (\boldsymbol{\xi}_{\,j}) = \Xi_{ij}$.
	\item Linear completeness: $\sum\nolimits_{i=1}^5 N_i (\boldsymbol{\xi}) \, \mathbfit{x}_{\,i} = \mathbfit{x}$.
	\item For $\boldsymbol{\xi} \in \Omega$, $N_i (\boldsymbol{\xi})$ is $C^{\infty}$, but for $\boldsymbol{\xi} \in \partial \Omega$, $N_i (\boldsymbol{\xi})$ is $C^0$, i.e., interpolation is linear along an edge (or alveolar chord) connecting two neighboring vertices.
\end{enumerate}
Item 4 is often considered a disadvantage of Wachspress shape functions, viz., the linear interpolation along their boundaries.  However, this is appropriate for our modeling of alveoli, because the septal boundaries are alveolar chords that are taken to interpolate linearly.

For interpolating a convex, planar, pentagonal shape, the shape functions of Wachspress have polynomials of order three in their numerators, and another polynomial of order two in their denominators; specifically, we write them here as
\begin{subequations}
	\label{shapeFunctions}
	\begin{align}
	N_{i+1} (\xi, \eta) & = \kappa_i \, A_i (\xi, \eta) / B(\xi, \eta) , 
	\qquad i = 1, 2 , \dots , 5 \\ 
	\intertext{using a scaling factor of $\kappa_i$, where $N_1 \Leftarrow N_6$.  The numerators and denominator for interpolating a pentagon take on the general form of}
	A_i (\xi, \eta) & = \alpha_{0i} + \alpha_{1i} \xi + \alpha_{2i} \eta + 
	\alpha_{3i} \xi^2 + \alpha_{4i} \xi\eta + \alpha_{5i} \eta^2 \notag \\ 
	\mbox{} & \phantom{\mbox{} = \alpha_{0i}} + \alpha_{6i} \xi^3 + 
	\alpha_{7i} \xi^2 \eta + \alpha_{8i} \xi \eta^2 + \alpha_{9i} \eta^3
	\label{shapeFnNum} \\
	B (\xi, \eta) & = \beta_0 + \beta_1 \xi + \beta_2 \eta + \beta_3 \xi^2 + 
	\beta_4 \xi\eta + \beta_5 \eta^2 
	\label{shapeFnDenom}
	\end{align}
\end{subequations}
where coefficients in the numerator, i.e., the $A_i$, differ with index $i$, while those in the denominator, viz., the $B \defeq \sum_{i=1}^5 A_i$, are the same for all five shape functions.  

We apply the construction technique of Dasgupta \cite{Dasgupta03} to compute the shape functions of Wachspress for an irregular convex pentagon.  Consider a chord $c_i$ that connects vertex $\mathbfit{\xi}_{\,i-1} = (\xi_{i-1} , \eta_{i-1})$ with vertex $\mathbfit{\xi}_{\,i} = (\xi_i , \eta_i)$ via a straight line segment such that $\ell_i = 0$ with $\ell_i \defeq 1 - a_{i} \xi - b_i \eta$ wherein
\begin{subequations}
	\begin{align}
	a_i & = \frac{\eta_i - \eta_{i-1}}{\xi_{i-1} \eta_i - \xi_i \eta_{i-1}} \\
	b_i & = \frac{\xi_{i-1} - \xi_i}{\xi_{i-1} \eta_i - \xi_i \eta_{i-1}} \\
	\intertext{for which Dasgupta derived the following set of constraints}
	\kappa_i & = \kappa_{i-1} \left( 
	\frac{a_{i+1} (\xi_{i-1} - \xi_i) + b_{i+1} (\eta_{i-1} - \eta_i)}
	{a_{i-1} (\xi_i - \xi_{i-1}) + b_{i-1} (\eta_i - \eta_{i-1})} \right) 
	\end{align}
\end{subequations}
with recursion starting at $\kappa_1 \defeq 1$.  Coefficients $\kappa_i$ enforce property 4 listed above.

With this information in hand, we then derived rational polynomials describing Wachspress' shape functions for a pentagon specified in Eqn.~(\ref{shapeFunctions}) in terms of the parameters $a_i$, $b_i$ and $\kappa_i$.  The polynomial coefficients for the $A_i$ in Eqn.~(\ref{shapeFnNum}) have values of
\begin{subequations}
	\label{shapeFnCoefs}
	\begin{align}
	\alpha_{0i} & = 1 \\
	\alpha_{1i} & = -( a_{i+1} + a_{i+2} + a_{i+3} ) \\
	\alpha_{2i} & = -( b_{i+1} + b_{i+2} + b_{i+3} ) \\
	\alpha_{3i} & = a_{i+1} a_{i+2} + a_{i+2} a_{i+3} + a_{i+3} a_{i+1} \\
	\alpha_{4i} & = a_{i+1} ( b_{i+2} + b_{i+3} ) + a_{i+2} ( b_{i+1} + 
	b_{i+3} ) + a_{i+3} ( b_{i+1} + b_{i+2} ) \\
	\alpha_{5i} & = b_{i+1} b_{i+2} + b_{i+2} b_{i+3} + b_{i+3} b_{i+1} \\
	\alpha_{6i} & = -a_{i+1} a_{i+2} a_{i+3} \\
	\alpha_{7i} & = -( a_{i+1} a_{i+2} b_{i+3} + a_{i+1} b_{i+2} a_{i+3} + 
	b_{i+1} a_{i+2} a_{i+3} ) \\
	\alpha_{8i} & = -( a_{i+1} b_{i+2} b_{i+3} + b_{i+1} a_{i+2} b_{i+3} + 
	b_{i+1} b_{i+2} a_{i+3} ) \\
	\alpha_{9i} & = -b_{i+1} b_{i+2} b_{i+3}
	\end{align}
\end{subequations}
which differ for each shape function via index $i = 1,2,\dots,5$, while the polynomial coefficients for $B$ in Eqn.~(\ref{shapeFnDenom}) have values of
\begin{equation}
\beta_i = \sum_{j=1}^5 \alpha_{ij} \kappa_j, \qquad i = 0, 1, \dots, 5
\end{equation}
which are the same for all five shape functions.  Sums over the four cubic terms in Eqn.~(\ref{shapeFnCoefs}) all vanish---a byproduct of Wachpress' formulation.  In the above formul\ae, an index count of $i \equiv 0 \implies i = 5$, while index counts of $i \equiv 6 \implies i = 1$, $i \equiv 7 \implies i = 2$ and $i \equiv 8 \implies i = 3$.  Shape function $N_1$ is illustrated in Fig.~\ref{figShapeFuntion}, with like images applying for the other four shape functions.

\begin{figure}
	\centering
	\includegraphics[width=0.9\textwidth]{figures/shapeFunction.jpg}
	\caption{Wachspress shape functions for a pentagon, in this case, shape function $N_1$.}
	\label{figShapeFuntion}
\end{figure}

\subsection{First Derivatives of the Shape Functions}

The first derivatives of Wachspress' shape functions for a pentagon are
\begin{subequations}
    \label{shapeFunctionGradients}
	\begin{align}
	N_{i+1,\xi} (\xi, \eta) & = \kappa_i \, \mathcal{N}_{i,\xi}(\xi, \eta) /
	B^2(\xi, \eta) \\
	N_{i+1,\eta} (\xi, \eta) & = \kappa_i \, \mathcal{N}_{i,\eta}(\xi, \eta) /
	B^2(\xi, \eta) \\
	\intertext{where $N_{i+1,\xi}  (\xi, \eta) = \partial N_{i+1} (\xi, \eta) / \partial \xi$ and $N_{i+1,\eta} (\xi, \eta) = \partial N_{i+1} (\xi, \eta) / \partial \eta$ with}
	\mathcal{N}_{i,\xi} (\xi,\eta) & = B(\xi, \eta) A_{i,\xi} (\xi, \eta) -
	B_{,\xi} (\xi, \eta) A_i (\xi, \eta) \\
	\mathcal{N}_{i,\eta} (\xi,\eta) & = B(\xi, \eta) A_{i,\eta} (\xi, \eta) - 
	B_{,\eta} (\xi, \eta) A_i (\xi, \eta) \\
	\intertext{which contain the polynomials}
	A_{i,\xi} (\xi, \eta) & = \alpha_{1i} + 2 \alpha_{3i} \xi + \alpha_{4i} \eta +
	3 \alpha_{6i} \xi^2 + 2 \alpha_{7i} \xi \eta + \alpha_{8i} \eta^2
	\label{shapeFnGradXiNum} \\
	A_{i,\eta} (\xi, \eta) & = \alpha_{2i} + \alpha_{4i} \xi + 2 \alpha_{5i} \eta + 
	\alpha_{7i} \xi^2 + 2 \alpha_{8i} \xi\eta + 3 \alpha_{9i} \eta^2 
	\label{shapeFnGradEtaNum} \\
	B_{,\xi} (\xi, \eta) & = \beta_1 + 2 \beta_3 \xi + \beta_4 \eta
	\label{shapeFnGradXiDenom} \\
	B_{,\eta} (\xi, \eta) & = \beta_2 + \beta_4 \xi + 2 \beta_5 \eta 
	\label{shapeFnGradEtaDenom}
	\end{align}
\end{subequations}
from which the deformation and displacement gradients are constructed.

\subsection{Second Derivatives of the Shape Functions}

The second derivatives of these shape functions, which we used to test the compatibility conditions of this element, are described by
\begin{subequations}
	\begin{align}
	N_{i+1,\xi\xi} & = \kappa_i \, \mathfrak{N}_{i,\xi\xi}(\xi,\eta) /
	B^3 (\xi,\eta) \\
	N_{i+1,\xi\eta} & = \kappa_i \, \mathfrak{N}_{i,\xi\eta}(\xi,\eta) /
	B^3(\xi,\eta) \\
	N_{i+1,\eta\xi} & = \kappa_i \, \mathfrak{N}_{i,\eta\xi}(\xi,\eta) /
	B^3(\xi,\eta) \\
	N_{i+1,\eta\eta} & = \kappa_i \, \mathfrak{N}_{i,\eta\eta}(\xi,\eta) /
	B^3(\xi,\eta) \\
	\intertext{where $N_{i+1,\xi\eta}  (\xi, \eta) = \partial^2 N_{i+1} (\xi, \eta) / \partial \xi \partial \eta$, etc., and  where}
	\mathfrak{N}_{i,\xi\xi} (\xi,\eta) & = B(\xi,\eta)
	\mathcal{N}_{i,\xi\xi} (\xi,\eta) - 2 B_{,\xi} (\xi,\eta) 
	\mathcal{N}_{i,\xi} (\xi,\eta) \\
	\mathfrak{N}_{i,\xi\eta}  (\xi,\eta)& = B(\xi,\eta) 
	\mathcal{N}_{i,\xi\eta} (\xi,\eta) - 2 B_{,\xi} (\xi,\eta) 
	\mathcal{N}_{i,\eta} (\xi,\eta) \\
	\mathfrak{N}_{i,\eta\xi}  (\xi,\eta)& = B(\xi,\eta)
	\mathcal{N}_{i,\eta\xi} (\xi,\eta) - 2 B_{,\eta} (\xi,\eta) 
	\mathcal{N}_{i,\xi} (\xi,\eta) \\
	\mathfrak{N}_{i,\eta\eta} (\xi,\eta) & = B(\xi,\eta) 
	\mathcal{N}_{i,\eta\eta} (\xi,\eta) - 2 B_{,\eta} (\xi,\eta) 
	\mathcal{N}_{i,\eta} (\xi,\eta) \\
	\intertext{wherein}
	\mathcal{N}_{i,\xi\xi} (\xi,\eta) & = B(\xi,\eta) A_{i,\xi\xi} (\xi,\eta) -
	B_{,\xi\xi} (\xi\eta) A_{i} (\xi\eta) \\
	\mathcal{N}_{i,\xi\eta} (\xi,\eta) & = B (\xi,\eta) A_{i,\xi\eta} (\xi,\eta) +
	B_{,\xi} (\xi,\eta) A_{i,\eta} (\xi,\eta) \notag \\
	\mbox{} & - B_{,\eta} (\xi,\eta) A_{i,\xi} (\xi,\eta) - 
	B_{,\xi\eta} (\xi,\eta) A_i (\xi,\eta) \\
	\mathcal{N}_{i,\eta\xi} (\xi,\eta) & = B (\xi,\eta) A_{i,\eta\xi} (\xi,\eta) +
	B_{,\eta} (\xi,\eta) A_{i,\xi} (\xi,\eta) \notag \\
	\mbox{} & - B_{,\xi} (\xi,\eta) A_{i,\eta} (\xi,\eta) - 
	B_{,\eta\xi} (\xi,\eta) A_i (\xi,\eta)  \\
	\mathcal{N}_{i,\eta\eta} (\xi,\eta) & = B(\xi,\eta) A_{i,\eta\eta} (\xi,\eta) -
	B_{,\eta\eta} (\xi,\eta) A_{i} (\xi,\eta) \\
	\intertext{which contain polynomials}
	A_{i,\xi\xi} (\xi,\eta) & = 2 \alpha_{3i} + 6 \alpha_{6i} \xi + 2 \alpha_{7i} \eta \\
	A_{i,\xi\eta} (\xi,\eta) & = \alpha_{4i} + 2 \alpha_{7i} \xi + 2 \alpha_{8i} \eta \\
	A_{i,\eta\eta} (\xi,\eta) & = 2 \alpha_{5i} + 2 \alpha_{8i} \xi + 6 \alpha_{9i} \eta \\
	B_{,\xi\xi} (\xi,\eta) & = 2 \beta_3 \\
	B_{,\xi\eta} (\xi,\eta) & = \beta_4 \\
	B_{,\eta\eta} (\xi,\eta) & = 2 \beta_5
	\end{align}
\end{subequations}
with $A_{i,\xi\eta} (\xi,\eta) = A_{i,\eta\xi} (\xi,\eta)$ and $B_{,\xi\eta} (\xi,\eta) = B_{,\eta\xi} (\xi,\eta)$. 

\subsection{Deformation Gradient for an Irregular Pentagon}

Derivatives of displacement $(u, v)$ taken with respect to the local co-ordinates $(\xi, \eta)$ described in terms of gradients of the shape functions $N_{i,\xi} (\xi, \eta)$ and $N_{i,\eta}(\xi, \eta)$ of a pentagon have components
\begin{subequations}
	\label{gradient}
	\begin{align}
	\begin{bmatrix}
	\partial u / \partial\xi & \partial u / \partial\eta \\
	\partial v / \partial\xi & \partial v / \partial\eta
	\end{bmatrix} & = 
	\begin{bmatrix}
	\sum\nolimits_{i=1}^5 N_{i,\xi} (\xi,\eta) \, u_i & \sum\nolimits_{i=1}^5 N_{i,\eta} (\xi,\eta) \, u_i \\
	\sum\nolimits_{i=1}^5 N_{i,\xi} (\xi,\eta) \, v_i & \sum\nolimits_{i=1}^5 N_{i,\eta} (\xi,\eta) \, v_i
	\end{bmatrix} 
	\label{displacementGradients} \\
	\intertext{where $u \defeq x - x_0$ and $v \defeq y - y_0$.  Gradients of the global co-ordinates $(x_0,y_0)$ evaluated in a reference state taken with respect to the local co-ordinates $(\xi, \eta)$ have components} 
	\begin{bmatrix}
	\partial x_0 / \partial\xi & \partial x_0 / \partial\eta \\
	\partial y_0 / \partial\xi & \partial y_0 / \partial\eta
	\end{bmatrix} & = 
	\begin{bmatrix}
	\sum\nolimits_{i=1}^5 N_{i,\xi} (\xi,\eta) \, x_{0i} & \sum\nolimits_{i=1}^5 N_{i,\eta} (\xi,\eta) \, x_{0i} \\
	\sum\nolimits_{i=1}^5 N_{i,\xi} (\xi,\eta) \, y_{0i} & \sum\nolimits_{i=1}^5 N_{i,\eta} (\xi,\eta) \, y_{0i}
	\end{bmatrix}
	\label{co-ordinateGradients} \\
	\intertext{wherein $(x_{0i}, y_{0i})$ are the reference global co-ordinates at the $i^{\mathrm{th}}$ vertex, while gradients of the global co-ordinates $(x,y)$ evaluated in the current state taken with respect to the local co-ordinates $(\xi, \eta)$ have components}
	\begin{bmatrix}
	\partial x / \partial\xi & \partial x / \partial\eta \\
	\partial y / \partial\xi & \partial y / \partial\eta
	\end{bmatrix} & = 
	\begin{bmatrix}
	\sum\nolimits_{i=1}^5 N_{i,\xi} (\xi,\eta) \, x_i & \sum\nolimits_{i=1}^5 N_{i,\eta} (\xi,\eta) \, x_i \\
	\sum\nolimits_{i=1}^5 N_{i,\xi} (\xi,\eta) \, y_i & \sum\nolimits_{i=1}^5 N_{i,\eta} (\xi,\eta) \, y_i
	\end{bmatrix}
	\label{currentGradients} \\
    \intertext{whose transpose establishes the Jacobian matrix}
    \mathbf{J} \defeq \begin{bmatrix}
    \partial x / \partial\xi & \partial y / \partial\xi \\
    \partial x / \partial\eta & \partial y / \partial\eta
    \end{bmatrix} & = 
    \begin{bmatrix}
    \sum\nolimits_{i=1}^5 N_{i,\xi} (\xi,\eta) \, x_i & \sum\nolimits_{i=1}^5 N_{i,\xi} (\xi,\eta) \, y_i \\
    \sum\nolimits_{i=1}^5 N_{i,\eta} (\xi,\eta) \, x_i & \sum\nolimits_{i=1}^5 N_{i,\eta} (\xi,\eta) \, y_i
    \end{bmatrix}
    \label{JacobianMtx2D}
	\end{align}
\end{subequations}
wherein $(x_i, y_i)$ denote the current global co-ordinates at the $i^{\mathrm{th}}$ vertex.

From the above matrices, one can construct the deformation gradient $\mathbfsf{F} = \partial \mathbfit{x} / \partial \mathbf{x}_0 = \mathbfsf{I} + \partial \mathbfit{u} / \partial \mathbfit{x}_0$ for an irregular pentagon via
\begin{subequations}
    \label{deformationGradient}
    \begin{align}
\mathbfsf{F} (\xi, \eta) & = 
\begin{bmatrix}
F_{11}(\xi, \eta) & F_{12}(\xi, \eta) \\
F_{21}(\xi, \eta) & F_{22}(\xi, \eta)
\end{bmatrix} \notag \\ & = 
\begin{bmatrix}
1 & 0 \\
0 & 1
\end{bmatrix} + 
\begin{bmatrix}
\partial u / \partial \xi & \partial u / \partial \eta \\
\partial v / \partial \xi & \partial v / \partial \eta
\end{bmatrix}
\begin{bmatrix}
\partial x_0 / \partial \xi & \partial x_0 / \partial \eta \\
\partial y_0 / \partial \xi & \partial y_0 / \partial \eta
\end{bmatrix}^{-1} \\
\intertext{whose inverse is}
\mathbfsf{F}^{-1} (\xi, \eta) & =
\frac{1}{F_{11} (\xi, \eta) F_{22} (\xi, \eta) - 
    F_{21} (\xi, \eta) F_{12} (\xi, \eta)}
\begin{bmatrix}
F_{22} (\xi, \eta) & -F_{12} (\xi, \eta) \\
-F_{21} (\xi, \eta) & F_{11} (\xi, \eta)
\end{bmatrix}
\end{align}
\end{subequations}
while its associated displacement gradient $\mathbfsf{G} = \partial \mathbfit{u} / \partial \mathbfit{x}$ is given by
\begin{equation}
\mathbfsf{G} (\xi, \eta) = 
\begin{bmatrix}
G_{11}(\xi, \eta) & G_{12}(\xi, \eta) \\
G_{21}(\xi, \eta) & G_{22}(\xi, \eta)
\end{bmatrix} 
\mbox{} = 
\begin{bmatrix}
\partial u / \partial \xi & \partial u / \partial \eta \\
\partial v / \partial \xi & \partial v / \partial \eta
\end{bmatrix}
\begin{bmatrix}
\partial x / \partial \xi & \partial x / \partial \eta \\
\partial y / \partial \xi & \partial y / \partial \eta
\end{bmatrix}^{-1} 
\label{displacementGradient}
\end{equation}
which is not invertible, in general.  All are evaluated in the 12 plane belonging to a co-ordinate system $( \vec{\mathbfsf{e}}_2 , \vec{\mathbfsf{e}}_2 , \vec{\mathbfsf{e}}_3 )$ that orients this pentagon, with $\vec{\mathbfsf{e}}_3$ being normal to its surface, as illustrated in Fig.~\ref{figPentagonCoord}.  The deformation and displacement gradients are two, fundamental, kinematic fields commonly used in the construction of constitutive equations.

\subsection{Compatibility Conditions}

To ensure that a deformation is compatible, and therefore integrable, it follows that the curl of its deformation gradient must be zero \cite{Clayton15}.  This condition is trivially satisfied for the shape functions that we use for 1D chords, 2D triangles, and 3D tetrahedra.  However, for the Wachspress shape function used to interpolate pentagons, this needs to be verified.  Vanishing of the curl of $\mathbfsf{F}$ results in two constraint equations for the planar case, they being
\begin{equation}
\label{compatibility}
F_{11,2} = F_{12,1} 
\qquad \text{and} \qquad
F_{22,1} = F_{21,2}
\end{equation}
whose spatial derivatives associate with the $( \vec{\mathbfsf{e}}_1 , \vec{\mathbfsf{e}}_2 )$ co-ordinate frame.

From Eqn.~\eqref{deformationGradient}, it follows that the spatial derivatives of the deformation gradient are
\begin{multline}
\mathbfsf{F}_{,1} (\xi, \eta) = \frac{\partial}{\partial x_0}
\begin{bmatrix}
F_{11}(\xi, \eta) & F_{12}(\xi, \eta) \\
F_{21}(\xi, \eta) & F_{22}(\xi, \eta)
\end{bmatrix} \\ 
\mbox{} = \frac{\partial \xi}{\partial x_0} \left( \frac{\partial}{\partial \xi} \left(
\begin{bmatrix}
\partial u / \partial \xi & \partial u / \partial \eta \\
\partial v / \partial \xi & \partial v / \partial \eta
\end{bmatrix} \right)
\begin{bmatrix}
\partial x_0 / \partial \xi & \partial x_0 / \partial \eta \\
\partial y_0 / \partial \xi & \partial y_0 / \partial \eta
\end{bmatrix}^{-1} -
\begin{bmatrix}
\partial u / \partial \xi & \partial u / \partial \eta \\
\partial v / \partial \xi & \partial v / \partial \eta
\end{bmatrix} \right. \\ \times \left.
\begin{bmatrix}
\partial x_0 / \partial \xi & \partial x_0 / \partial \eta \\
\partial y_0 / \partial \xi & \partial y_0 / \partial \eta
\end{bmatrix}^{-1}
\frac{\partial}{\partial \xi} \left(
\begin{bmatrix}
\partial x_0 / \partial \xi & \partial x_0 / \partial \eta \\
\partial y_0 / \partial \xi & \partial y_0 / \partial \eta
\end{bmatrix} \right)
\begin{bmatrix}
\partial x_0 / \partial \xi & \partial x_0 / \partial \eta \\
\partial y_0 / \partial \xi & \partial y_0 / \partial \eta
\end{bmatrix}^{-1} \right)
\addtocounter{equation}{1}
\tag{\theequation a}
\end{multline}
and
\begin{multline}
\mathbfsf{F}_{,2} (\xi, \eta) = \frac{\partial}{\partial y_0}
\begin{bmatrix}
F_{11}(\xi, \eta) & F_{12}(\xi, \eta) \\
F_{21}(\xi, \eta) & F_{22}(\xi, \eta)
\end{bmatrix} \\ 
\mbox{} = \frac{\partial \eta}{\partial y_0} \left(
\frac{\partial}{\partial \eta} \left(
\begin{bmatrix}
\partial u / \partial \xi & \partial u / \partial \eta \\
\partial v / \partial \xi & \partial v / \partial \eta
\end{bmatrix} \right)
\begin{bmatrix}
\partial x_0 / \partial \xi & \partial x_0 / \partial \eta \\
\partial y_0 / \partial \xi & \partial y_0 / \partial \eta
\end{bmatrix}^{-1} -
\begin{bmatrix}
\partial u / \partial \xi & \partial u / \partial \eta \\
\partial v / \partial \xi & \partial v / \partial \eta
\end{bmatrix} \right. \\ \times \left.
\begin{bmatrix}
\partial x_0 / \partial \xi & \partial x_0 / \partial \eta \\
\partial y_0 / \partial \xi & \partial y_0 / \partial \eta
\end{bmatrix}^{-1}
\frac{\partial}{\partial \eta} \left(
\begin{bmatrix}
\partial x_0 / \partial \xi & \partial x_0 / \partial \eta \\
\partial y_0 / \partial \xi & \partial y_0 / \partial \eta
\end{bmatrix} \right)
\begin{bmatrix}
\partial x_0 / \partial \xi & \partial x_0 / \partial \eta \\
\partial y_0 / \partial \xi & \partial y_0 / \partial \eta
\end{bmatrix}^{-1} \right)
\tag{\theequation b}
\end{multline}
wherein
\begin{subequations}
	\begin{align}
	\frac{\partial}{\partial \xi}
	\begin{bmatrix}
	\partial u / \partial\xi & \partial u / \partial\eta \\
	\partial v / \partial\xi & \partial v / \partial\eta
	\end{bmatrix} & = 
	\begin{bmatrix}
	\sum\nolimits_{i=1}^5 N_{i,\xi\xi} (\xi,\eta) \, u_i & \sum\nolimits_{i=1}^5 N_{i,\xi\eta} (\xi,\eta) \, u_i \\
	\sum\nolimits_{i=1}^5 N_{i,\xi\xi} (\xi,\eta) \, v_i & \sum\nolimits_{i=1}^5 N_{i,\xi\eta} (\xi,\eta) \, v_i
	\end{bmatrix} \\
	\frac{\partial}{\partial \eta}
	\begin{bmatrix}
	\partial u / \partial\xi & \partial u / \partial\eta \\
	\partial v / \partial\xi & \partial v / \partial\eta
	\end{bmatrix} & = 
	\begin{bmatrix}
	\sum\nolimits_{i=1}^5 N_{i,\eta\xi} (\xi,\eta) \, u_i & \sum\nolimits_{i=1}^5 N_{i,\eta\eta} (\xi,\eta) \, u_i \\
	\sum\nolimits_{i=1}^5 N_{i,\eta\xi} (\xi,\eta) \, v_i & \sum\nolimits_{i=1}^5 N_{i,\eta\eta} (\xi,\eta) \, v_i 
	\end{bmatrix} \\
	\intertext{and}
	\frac{\partial}{\partial \xi}
	\begin{bmatrix}
	\partial x_0 / \partial\xi & \partial x_0 / \partial\eta \\
	\partial y_0 / \partial\xi & \partial y_0 / \partial\eta
	\end{bmatrix} & = 
	\begin{bmatrix}
	\sum\nolimits_{i=1}^5 N_{i,\xi\xi} (\xi,\eta) \, x_{0i} & \sum\nolimits_{i=1}^5 N_{i,\xi\eta} (\xi,\eta) \, x_{0i} \\
	\sum\nolimits_{i=1}^5 N_{i,\xi\xi} (\xi,\eta) \, y_{0i} & \sum\nolimits_{i=1}^5 N_{i,\xi\eta} (\xi,\eta) \, y_{0i}
	\end{bmatrix} \\
	\frac{\partial}{\partial \eta}
	\begin{bmatrix}
	\partial x_0 / \partial\xi & \partial x_0 / \partial\eta \\
	\partial y_0 / \partial\xi & \partial y_0 / \partial\eta
	\end{bmatrix} & = 
	\begin{bmatrix}
	\sum\nolimits_{i=1}^5 N_{i,\eta\xi} (\xi,\eta) \, x_{0i} & \sum\nolimits_{i=1}^5 N_{i,\eta\eta} (\xi,\eta) \, x_{0i} \\
	\sum\nolimits_{i=1}^5 N_{i,\eta\xi} (\xi,\eta) \, y_{0i} & \sum\nolimits_{i=1}^5 N_{i,\eta\eta} (\xi,\eta) \, y_{0i}
	\end{bmatrix}
	\end{align}
\end{subequations}
with $\partial \xi / \partial x_0$ and $\partial \eta / \partial y_0$ effectively being scaling factors that we take to be described as a ratio of septal chord lengths; specifically, let
\begin{equation}
\frac{\partial \xi}{\partial x_0} \simeq
\frac{\partial \eta}{\partial y_0} \approx 
\frac{L(\xi, \eta)}{L_0 (x, y)} = 
\frac{\cos (\omega)}{\sqrt{A_0 / 5 \tan (\omega)}}
\end{equation}
where $L(\xi,\eta)$ is the septal length of a pentagonal edge in its natural configuration, as drawn in Fig.~\ref{figRegPentagon}, while $L_0(x,y)$ is the actual, alveolar, septal length with $A_0(x,y)$ being the area of an alveolar septum in its reference state.  This formula follows from Eqns.~(\ref{regPentagonLength} \& \ref{regPentagonArea}).

\textbf{Note}: We study compatibility only for the purpose of assessing applicability in our choice of selecting Wachspress shape functions.  Otherwise, it is not required in our modeling of an alveolus via a dodecahedron. 

\subsection{Gram-Schmidt Decomposition of the Deformation Gradient}
\label{secQR}

To describe kinematics of a planar membrane, an upper-triangular Gram-Schmidt decomposition of the deformation gradient $\mathbfsf{F}$ is used in lieu of the symmetric polar decomposition that is commonly adopted \cite{Srinivasa12,FreedSrinivasa15,Freedetal17,FreedZamani19,Freedetal19}.  McLellan \cite{McLellan76,McLellan80} was the first to propose a triangular decomposition of $\mathbfsf{F}$, to prove its uniqueness and existence, and to establish many of its physical properties.  This idea has been rediscovered several times since then. \cite{Rosakis90,Souchet93,Srinivasa12}  A thorough history of the \textbf{QR} (Gram-Schmidt) decomposition has been written by Leon, Bj\"orck \& Gander \cite{Leonetal13}, with a brief history regarding its application to kinematics being given in Freed \textit{et~al}. \cite{Freedetal19}

A Lagrangian Gram-Schmidt factorization of the deformation gradient $\mathbfsf{F}$ is written here as $\mathbfsf{F} = \boldsymbol{\mathcal{RU}}$, where the rotation $\boldsymbol{\mathcal{R}}$ is orthogonal, and where the Laplace stretch $\boldsymbol{\mathcal{U}}$ is upper-triangular \cite{Freedetal19}.\footnote{
	The \textbf{QR} rotation $\boldsymbol{\mathcal{R}}$ and stretch $\boldsymbol{\mathcal{U}}$ tensors are distinct from those that arise from a polar decomposition of a deformation gradient, typically denoted as $\mathbfsf{R}$ and $\mathbfsf{U}$, as found in any, modern, continuum mechanics text.  McLellan \cite{McLellan76,McLellan80} introduced the Laplace stretch in 1976, which he denoted as $\mathbfsf{H}$, while Srinivasa \cite{Srinivasa12} denoted it as $\tilde{\mathbfsf{F}}$ in his 2012 paper.
} 
(An Eulerian Gram-Schmidt factorization has just been derived, \cite{Freedetal20} but it came along too late to adopt in this study.  Its application is a topic for future study.)  This triangular measure of stretch possesses an inherent property in two space: the direction aligned with the rotated 1-axis, denoted as $\vec{\mathbfit{g}}_{\hspace{0.5pt}1}$, remains invariant under transformation $\boldsymbol{\mathcal{U}}$ \cite{McLellan80}, i.e., it is a material vector in a neighborhood surrounding that particle whereat $\mathbfsf{F}$ is evaluated \cite{FreedZamani18}.  This property has some interesting ramifications addressed in \S\ref{secDilemma}.

\subsubsection{\textbf{QR} Factorization of\/ $\mathbfsf{F}$}
\label{secQR2D}

The $2 \times 2$ deformation gradient associated with a planar membrane has a Gram-Schmidt decomposition expressed in terms of four physical attributes.  Three of these attributes describe deformation.  They are defined as \cite{Freedetal17}
\begin{equation}
a = \sqrt{F_{11}^{\,2} + F_{21}^{\,2}} , \quad
b = \frac{F_{11} F_{22} - F_{12} F_{21}}
{\sqrt{F_{11}^{\,2} + F_{21}^{\,2}}} , \quad
g = \frac{F_{11} F_{12} +  F_{22} F_{21}}
{F_{11}^{\,2} + F_{21}^{\,2}} 
\label{physicalVariables}
\end{equation}
thereby populating Laplace stretch $\boldsymbol{\mathcal{U}}$ and its inverse $\boldsymbol{\mathcal{U}}^{-1}$ with components
\begin{equation}
\boldsymbol{\mathcal{U}} = \begin{bmatrix}
a & a g \\ 0 & b
\end{bmatrix} \qquad \text{and} \qquad
\boldsymbol{\mathcal{U}}^{-1} = \begin{bmatrix} 
1 / a & -g / b \\ 0 & 1 / b
\end{bmatrix}
\label{LaplaceStretch2D}
\end{equation}
where $a$ and $b$ are the principal elongations (ratios of current lengths to reference lengths) and $g$ is the extent of in-plane shear, as measured in a co-ordinate frame $(  \vec{\mathbfsf{g}}_1 , \vec{\mathbfsf{g}}_2 )$ illustrated in Fig.~\ref{figKinematics}.  It is worth pointing out that the components of Laplace stretch, viz., $\mathcal{U}_{ij}$, are evaluated in the reference co-ordinate system $( \vec{\mathbfsf{e}}_1 , \vec{\mathbfsf{e}}_2 )$ of the pentagon, as $\mathbfsf{F} = F_{ij} \, \vec{\mathbfsf{e}}_i \otimes \vec{\mathbfsf{e}}_j$, but their physical interpretations arise in the Gram rotated co-ordinate system $( \vec{\mathbfsf{g}}_1 , \vec{\mathbfsf{g}}_2 )$.

\begin{figure}
	\centering
	\includegraphics[width=8cm]{figures/deformation.png}
	\caption{Physical attributes of a planar deformation: $a$ and $b$ represent elongations, while $g = \tan \phi$ denotes the magnitude of shear.  They are measured in a physical frame of reference with unit base vectors $( \vec{\mathbfsf{g}}_1 , \vec{\mathbfsf{g}}_2 )$ where $\vec{\mathbfsf{g}}_1$ embeds in the material.}
	\label{figKinematics}
\end{figure}

Orthogonal tensor $\boldsymbol{\mathcal{R}} = \bigl[ \vec{\mathbfsf{g}}_1 \bigm| \vec{\mathbfsf{g}}_2 \bigr] = \delta_{ij} \, \vec{\mathbfsf{g}}_i \otimes \vec{\mathbfsf{e}}_j = \mathcal{R}_{ij} \, \vec{\mathbfsf{e}}_i \otimes \vec{\mathbfsf{e}}_j$ rotates the reference co-ordinate axes $( \vec{\mathbfsf{e}}_1 , \vec{\mathbfsf{e}}_2 )$ into a physical co-ordinate system $( \vec{\mathbfsf{g}}_1 , \vec{\mathbfsf{g}}_2 )$ through an angle $\theta$, which is the fourth physical attribute arising from a \textbf{QR} factorization of $\mathbfsf{F}$.  This angle of rotation describes a proper orthogonal matrix, specifically
\begin{equation}
\boldsymbol{\mathcal{R}} = \begin{bmatrix}
\cos \theta & -\sin \theta \\
\sin \theta & \cos \theta
\end{bmatrix} 
\label{rotation}
\end{equation}  
with
\begin{equation}
\sin \theta = \frac{F_{21}}
{\sqrt{F_{11}^{\,2} + F_{21}^{\,2}}} , \quad
\cos \theta = \frac{F_{11}}
{\sqrt{F_{11}^{\,2} + F_{21}^{\,2}}} 
\quad \therefore \quad
\theta = \tan^{-1} \left( \frac{F_{21}}{F_{11}} \right)
\label{trigFns}
\end{equation}  
where a positive angle $\theta$ corresponds with a counter\-clockwise rotation of physical axes $( \vec{\mathbfsf{g}}_1 , \vec{\mathbfsf{g}}_2 )$ about reference axes $( \vec{\mathbfsf{e}}_1 , \vec{\mathbfsf{e}}_2 )$.  

From the four independent components of a planar deformation gradient $F_{ij}$ come three deformation attributes, i.e., $a$, $b$ and $g$, and one rotational attribute, i.e., $\theta$.


\subsubsection{Dilemma}
\label{secDilemma} 

Until recently, \cite{Pauletal20} there has been a tacit assumption in prior applications of Gram-Schmidt factorizations of $\mathbfsf{F}$: Specifically, the physical base vectors $( \vec{\mathbfsf{g}}_1 , \vec{\mathbfsf{g}}_2 )$ satisfy a geometric condition whereby the physical 1-direction $\vec{\mathbfsf{g}}_1$ rotates out of the reference 1-direction $\vec{\mathbfsf{e}}_1$, but this need not always be the case.  Physical vector $\vec{\mathbfsf{g}}_1$ could equally likely rotate out of the 2-direction $\vec{\mathbfsf{e}}_2$ of the reference frame.  At issue is not: How the physical base vectors orient in space?  That is managed by Gram's procedure.  Rather, at issue is: How do the physical base vectors index with respect to the reference base vectors?  This topic is addressed in \S\ref{reindexing3D} for the 3D case; below, we address this topic for the 2D case.

To illustrate the concern, consider two deformation histories, as drawn in Fig.~\ref{figDilemma}, each of which describes a simple shear taking place in the plane of a membrane.  In one case shear occurs in the 1-direction, while in the other case shear occurs in the 2-direction.  There are no elongations in either deformation considered.  These motions lead to different Gram-Schmidt factorizations of the deformation gradient.  When following the protocol of Eqns.~(\ref{physicalVariables}--\ref{trigFns}), these factorizations are found to be
\begin{subequations}
	\label{shears}
	\begin{align}
	\mathbfsf{F} = 
	\begin{bmatrix} 1 & \gamma \\ 0 & 1 \end{bmatrix} & \implies 
	\boldsymbol{\mathcal{R}} = 
	\begin{bmatrix} 1 & 0 \\ 0 & 1 \end{bmatrix} , \quad
	\boldsymbol{\mathcal{U}} = 
	\begin{bmatrix} 1 & \gamma \\ 0 & 1  \end{bmatrix} 
	\label{shear1} \\
	\intertext{and}
	\mathbfsf{F} = 
	\begin{bmatrix} 1 & 0 \\ \gamma & 1 \end{bmatrix} & \implies \left\{
	\begin{aligned} \mbox{}
	\boldsymbol{\mathcal{R}} & = \frac{1}{\sqrt{1 + \gamma^2}}
	\begin{bmatrix} 1 & -\gamma \\ \gamma & 1 \end{bmatrix} \\
	\boldsymbol{\mathcal{U}} & = 
	\begin{bmatrix} \sqrt{1 + \gamma^2} & \gamma \\ 
	0 & 1 \bigm/ \sqrt{1 + \gamma^2} \end{bmatrix}
	\end{aligned} \right.
	\label{shear2}
	\end{align}
\end{subequations}
respectively, where we see that shear $\mathcal{U}_{12}$ has the same physical interpretation in both cases, viz., $\gamma$, but elongations $\mathcal{U}_{11}$ and $\mathcal{U}_{22}$ do not, viz., $\mathcal{U}_{11}=1$ and $\mathcal{U}_{22}=1$ in Eqn.~(\ref{shear1}), whereas $\mathcal{U}_{11} = \sqrt{1 + \gamma^2}$ and $\mathcal{U}_{22} = 1 / \sqrt{1 + \gamma^2}$ for the motion described in Eqn.~(\ref{shear2}).  Consequently, two geometric interpretations are produced for just one physical mode of deformation.  This cannot be!

\begin{figure}
	\centering
	\includegraphics[width=0.5\textwidth]{figures/figDilemma.png}
	\caption{The left graphic designates a reference configuration while the right two graphics designate deformed configurations, both in basis $( \vec{\mathbfsf{g}}_1 , \vec{\mathbfsf{g}}_2 )$.  The top graphic associates with the motion of Eqn.~(\ref{shear1}), while the bottom graphic associates with the motion of Eqn.~(\ref{shear2}).}
	\label{figDilemma}
\end{figure}

The only difference between the motions that lead to the two deformation gradients presented in Eqn.~(\ref{shears}) is one's choice for labeling the co-ordinate directions.  Matrix operations of row and column pivoting, taken from linear algebra, allow one to transform the lower-triangular form of Eqn.~(\ref{shear2}) into an upper-triangular form like Eqn.~(\ref{shear1}); hence, producing an unified physical interpretation for both shearing motions, and thereby providing a means for establishing a remedy to this dilemma. 


\subsubsection{Remedy}
\label{secRemedy}

For 2D membranes, there are only two co-ordinate re-indexings that are possible (for 3D solids there are six, cf.\ \S\ref{reindexing3D}).  The default is no re-indexing at all, in which case 
\begin{subequations}
	\label{membraneRelabling}
	\begin{align}
	[ \mathbfsf{P} ] = [ \mathbfsf{P}_0 ] & \defeq 
	\begin{bmatrix} 1 & 0 \\ 0 & 1 \end{bmatrix} & 
	\implies & & \begin{bmatrix}
	\mathcal{F}_{11} & \mathcal{F}_{12} \\
	\mathcal{F}_{21} & \mathcal{F}_{22}
	\end{bmatrix} & \defeq \begin{bmatrix}
	F_{11} & F_{12} \\
	F_{21} & F_{22}
	\end{bmatrix} \label{Q0} \\
	\intertext{while in the second case there is a re-indexing specified by}
	[ \mathbfsf{P} ] = [ \mathbfsf{P}_1 ] & \defeq 
	\begin{bmatrix} 0 & 1 \\ 1 & 0 \end{bmatrix} & 
	\implies & & \begin{bmatrix}
	\mathcal{F}_{11} & \mathcal{F}_{12} \\
	\mathcal{F}_{21} & \mathcal{F}_{22}
	\end{bmatrix} & \defeq \begin{bmatrix}
	F_{22} & F_{21} \\
	F_{12} & F_{11}
	\end{bmatrix}
	\label{Q1}
	\end{align}
\end{subequations}
where components $\mathcal{F}_{ij} = P_{ki} F_{k\ell} P_{\ell j}$ are the components to be used in the Gram-Schmidt factorization presented in \S\ref{secQR2D}, see also \S\ref{reindexing3D}, and where $\mathbfsf{P} \in \{ \mathbfsf{P}_0 , \mathbfsf{P}_1 \}$ is orthogonal, i.e., $\mathbfsf{P} \mathbfsf{P}^{\mathsf{T}} = \mathbfsf{P}^{\hspace{-1pt}\mathsf{T}} \mathbfsf{P} = \mathbfsf{I}$ with $\det \mathbfsf{P} = \pm 1$; specifically, $\det \mathbfsf{P}_0 = +1$ while $\det \mathbfsf{P}_1 = -1$.

The challenge in implementing such a strategy is to determine when to switch from $\mathbfsf{P}_0$ (case 1) to $\mathbfsf{P}_1$ (case 2), or back again, viz., from $\mathbfsf{P}_1$ to $\mathbfsf{P}_0$.  Continuity in the physical fields of deformation $(a , b , g )$ must be satisfied in order for such a change in co-ordinate frame to be physically meaningful.  To this end, it is useful to represent the components of a planar deformation gradient as
\begin{equation}
\begin{bmatrix}
\mathcal{F}_{11} & \mathcal{F}_{12} \\
\mathcal{F}_{21} & \mathcal{F}_{22}
\end{bmatrix} =
\begin{cases}
\mathrm{case} \; 1: & \begin{bmatrix}
F_{11} & F_{12} \\
F_{21} & F_{22}
\end{bmatrix}_{\vphantom{|}} = \begin{bmatrix}
x & \beta y \\ \alpha x & y
\end{bmatrix} \\
\mathrm{case} \; 2: & \begin{bmatrix}
F_{22} & F_{21} \\
F_{12} & F_{11}
\end{bmatrix} = \begin{bmatrix}
y & \alpha x \\ \beta y & x
\end{bmatrix}
\end{cases}
\label{deformationGradientRelabelling}
\end{equation}
where $x = F_{11}$ and $y = F_{22}$ are elongations, while ratios $\alpha = F_{21} / F_{11}$ and $\beta = F_{12} / F_{22}$ are magnitudes of shear, as illustrated in Fig.~\ref{figF}.  

\begin{figure}
	\centering
	\includegraphics[width=0.3\textwidth]{figures/figF.png}
	\caption{A general description for homogeneous planar deformation, where $x , y \in \mathbb{R}_+$ and $\alpha , \beta \in \mathbb{R}$.  Shears $\alpha$ and $\beta$ are drawn in their positive sense.}
	\label{figF}
\end{figure}

The physical attributes for Laplace stretch, as they pertain to the two cases in Eqn.~(\ref{membraneRelabling}), written in terms of components $F_{ij}$ from $\mathbfsf{F} = F_{ij} \, \vec{\mathbfsf{e}}_i \otimes \vec{\mathbfsf{e}}_j$ as defined in Eqn.~(\ref{deformationGradientRelabelling}), are respectively given by
\begin{subequations}
	\label{physicalAttributes}
	\begin{align}
	\tilde{a} & = x \sqrt{1 + \alpha^2} & 
	\hat{a} & = y \sqrt{1 + \beta^2} 
	\label{aAttribute} \\
	\tilde{b} & = y ( 1 - \alpha \beta ) \bigm/ \sqrt{1 + \alpha^2} &
	\hat{b} & = x ( 1 - \alpha \beta ) \bigm/ \sqrt{1 + \beta^2} 
	\label{bAttribute} \\ 
	\tilde{g} & = y ( \alpha + \beta ) \bigm/ x (1 + \alpha^2) &
	\hat{g} & = x ( \alpha + \beta ) \bigm/ y (1 + \beta^2)
	\label{gAttribute} \\
	\tilde{\theta} & = \tan^{-1} ( -\alpha ) & 
	\hat{\theta} & = \tan^{-1} ( -\beta )
	\label{thetaAttribute}
	\end{align}
\end{subequations}
where attributes in the left column apply to case~1 (i.e., Eqn.~\ref{Q0}) while those in the right column apply to case~2 (viz., Eqn.~\ref{Q1}).  The actual set of physical attributes $\{ a, b, g, \theta \}$ that are to be used when quantifying Laplace stretch and its inverse, according to Eqn.~(\ref{LaplaceStretch2D}), are then selected via the strategy  
\begin{subequations}
	\label{attributeMaps}
	\begin{align}
	\mathrm{if} \; | \tilde{g} | \geq | \hat{g} | : & &
	\{ \tilde{a} , \tilde{b} , \tilde{g} , \tilde{\theta} \} &
	\mapsto \{ a , b , g , \theta \}  \\
	\mathrm{else} \; | \tilde{g} | \leq | \hat{g} | : & &
	\{ \hat{a} , \hat{b} , \hat{g} , \hat{\theta} \} & 
	\mapsto \{ a , b , g , \theta \}
	\end{align}
\end{subequations}
where it is easily verified that $\tilde{a} = \hat{a}$ and $\tilde{b} = \hat{b}$ whenever $\tilde{g} = \hat{g}$; consequently, the physical attributes of deformation $a , b , g$ remain continuous across a co-ordinate switch, however, the angle of co-ordinate rotation $\theta$ will not be continuous across such a switch between co-ordinate frames, as they represent rotations out of different co-ordinate directions.  A like statement applies in the 3D case whenever one uses the re-indexing scheme presented in \S\ref{reindexing3D}, i.e., the physical attributes of Laplace stretch remain continuous across a re-indexing of one's co-ordinate frame.

The above strategy returns matrices for the rotation and Laplace stretch described in Eqn.~(\ref{shear1}) for both deformation gradients presented in Eqn.~(\ref{shears}). The dilemma is remedied.  Laplace stretch, as remedied, therefore has an unique physical interpretation.    Co-ordinate re-indexing ensures that the invariant properties of Laplace stretch \cite{McLellan80} are adhered to.

The above protocol is the 2D version of the 3D version \cite{Pauletal20} presented in \S\ref{reindexing3D}.  It is easier to understand what is happening in the 2D case, which is why more detail is presented here.  It may certainly happen that even when the 3D co-ordinates are re-indexed, there may be one or more of the twelve pentagons whose 2D co-ordinates need to be re-indexed, too.

There are three kinematic variables that describe deformation in a planar membrane: elongation ratios $a$ and $b$ and simple shear $g$.  These variables will vary both temporally and spatially throughout a pentagon whenever Wachspress' shape functions are used.

\subsection{Thermodynamic Strains and Strain Rates}
\label{strainsAndStrainRates2D}

In terms of the above physical attributes for stretch, i.e., $a$, $b$ and $g$, and their reference values, viz., $a_0$, $b_0$ and $g_0$, one can define a set of strain attributes derived from thermo\-dynamics, specifically \cite{Freed17}
\begin{subequations}
    \label{thermodynamicStrains2D}
    \begin{align}
    \xi & \defeq \ln \left( \sqrt{\frac{a}{a_0} \frac{b}{b_0}} \right) & 
    \mathrm{d} \xi & = \frac{1}{2} \left( \frac{\mathrm{d}a}{a} + 
    \frac{\mathrm{d}b}{b} \right) \\
    \varepsilon & \defeq \ln \left( \sqrt{\frac{a}{a_0} \frac{b_0}{b}} \right) &
    \mathrm{d} \varepsilon & = \frac{1}{2} \left( \frac{\mathrm{d}a}{a} - 
    \frac{\mathrm{d}b}{b} \right) \\
    \gamma & \defeq g - g_0 & 
    \mathrm{d} \gamma & = \mathrm{d} g
    \end{align}
\end{subequations}
whose rates are exact differentials, i.e., they are independent of path---a tacit requirement from thermo\-dynamics \cite{Caratheodory09}.  Here $\xi$ denotes dilation (uniform areal stretch), $\varepsilon$ denotes squeeze (pure shear), and $\gamma$ denotes (simple) shear. 

\subsubsection{Stretch Rates}

The following approximations for stretch rates were derived by Freed \&\ Zamani \cite{FreedZamani18}.  From these, the various strain rates listed in Eqn.~(\ref{thermodynamicStrains2D}) can be established.  

A forward difference formula is used to approximate rates in the reference configuration for the various stretch attributes, as obtained from $\mathrm{d}\boldsymbol{\mathcal{U}}_0 = ( \boldsymbol{\mathcal{U}}_1 -  \boldsymbol{\mathcal{U}}_0 ) / \mathrm{d}t + \mathcal{O}(\mathrm{d}t)$ that, neglecting higher-order terms, produces
\begin{equation}
\mathrm{d} a_0 = \frac {a_1 - a_0}{\mathrm{d}t} , \quad 
\mathrm{d} b_0 = \frac {b_1 - b_0}{\mathrm{d}t} , \quad 
\mathrm{d} g_0 = \frac{a_1}{a_0} 
\left( \frac{g_1 - g_0}{\mathrm{d}t} \right) 
\label{forwardDifference1stOrder2D}
\end{equation}
where $\mathrm{d} t = t_1 - t_0$ is the applied time step.  A backward difference formula $\mathrm{d} \boldsymbol{\mathcal{U}}_1 = ( \boldsymbol{\mathcal{U}}_1 - \boldsymbol{\mathcal{U}}_0 ) / \mathrm{d}t + \mathcal{O}(\mathrm{d}t)$ is used to estimate rates for the various stretch attributes at the end of its first integration step that, neglecting higher-order terms, give
\begin{equation}
\mathrm{d} a_1 = \frac {a_1 - a_0}{\mathrm{d}t} , \quad
\mathrm{d} b_1 = \frac {b_1 - b_0}{\mathrm{d}t} , \quad
\mathrm{d} g_1 = \frac{a_0}{a_1} 
\left(\frac{g_1 - g_0}{\mathrm{d}t} \right) .
\label{backwardDifference1stOrder2D}
\end{equation}
Curiously, there is a distinction in how the shear rates are approximated at the two nodes for this first interval of integration.

Equations (\ref{forwardDifference1stOrder2D} \& \ref{backwardDifference1stOrder2D}) are first-order approximations for these derivatives.  Second-order approximations can be established whenever $i > 0$ provided the stepsize for step $[i, i+1]$ equals the stepsize for step $[i-1, i]$, where state $i=0$ associates with an initial condition.  The backward difference formula  $\mathrm{d} \boldsymbol{\mathcal{U}}_{i+1} = ( 3 \, \boldsymbol{\mathcal{U}}_{i+1} -  4 \, \boldsymbol{\mathcal{U}}_{i} + \boldsymbol{\mathcal{U}}_{i-1} ) / 2\mathrm{d}t + \mathcal{O} \bigl( (\mathrm{d}t)^2 \bigr)$ then produces rates for the stretch attributes of
\begin{equation}
\begin{aligned}
\mathrm{d} a_{i+1} & 
= \frac {3a_{i+1} - 4a_i +  a_{i-1}}{2\mathrm{d}t} \\ 
\mathrm{d} b_{i+1} & 
= \frac {3b_{i+1} - 4b_i +  b_{i-1}}{2\mathrm{d}t} \\
\mathrm{d} g_{i+1} & 
= \frac{2a_i} {a_{i+1}} \left(\frac{g_{i+1} - g_{i}}{\mathrm{d}t} \right) - \frac{a_{i-1}}{a_{i+1}} \left( \frac{g_{i+1} - g_{i-1}}{2\mathrm{d}t} \right) 
\end{aligned}
\label{backwardDifference2ndOrder2D}
\end{equation}
which require stretch attributes $a_{i-1}$, $b_{i-1}$ and $g_{i-1}$ to be stored in a finite element setting.


\section{3D Irregular Dodecahedra}

The primary kinematic variables needed to describe the deformation of an irregular dodecahedron used as a model for an alveolar sac are its volume $V$ (see \S\ref{sec:geometries}) and the differential change in volume $\mathrm{d}V$, with the former following from Eq.~(\ref{tetrahedralVolume}) and the latter coming from a suitable finite difference formula.  Whenever the material filling an alveolar sac is air (its normal healthy condition), no further breakdown of these kinematics is required.  

However, whenever an alveolar sac is filled with fluid (blood, interstitial fluids, pflem, etc.) this fluid can be expected to behave solid-like in the face of a passing shock wave.  In this situation, the non-uniform measures for strain (i.e., shears) can be expected to produce non-uniform responses in stress.

\subsection{Shape Functions for Interpolating an Irregular Tetrahedron}

The shape functions associated with the four vertices of a tetrahedron $N_i$, $i = 1, 2, 3, 4,$ are defined as follows
\begin{subequations}
    \begin{align}
    N_1 & = 1 - \xi - \eta - \zeta , \quad
    N_2 = \xi , \quad
    N_3 = \eta , \quad
    N_4 = \zeta \\
    \intertext{where $\xi$, $\eta$ and $\zeta$ represent natural co-ordinates with $0 \leq \xi \leq 1$, $0 \leq \eta \leq 1-\xi$ and $0 \leq \zeta \leq 1-\xi-\eta$.  Gradients of these shape functions are} 
    N_{1,\xi} & = -1 , \quad N_{1,\eta} = -1 , \quad N_{1,\zeta} = -1 \notag \\
    N_{2,\xi} & = 1 , \quad \phantom{-} N_{2,\eta} = 0 , \quad \phantom{-} N_{2,\zeta} = 0 \notag \\
    N_{3,\xi} & = 0 , \quad \phantom{-} N_{3,\eta} = 1 , \quad \phantom{-} N_{3,\zeta} = 0 \notag \\
    N_{4,\xi} & = 0 , \quad \phantom{-} N_{4,\eta} = 0 , \quad \phantom{-} N_{4,\zeta} = 1 
    \end{align}
\end{subequations}
and consequently the deformation gradient will be constant throughout its volume, like the deformation gradients used for chords and triangles.

\subsubsection{Deformation Gradient for an Irregular Tetrahedron}

The deformation gradient for a volume element is constructed from
\small
\begin{equation}
\mathbfsf{F} ( \xi , \eta , \zeta ) = \begin{bmatrix} 1 & 0 & 0 \\
0 & 1 & 0 \\ 0 & 0 & 1 \end{bmatrix} + \begin{bmatrix}
\partial u / \partial \xi & \partial u / \partial \eta & \partial u / \partial \zeta \\
\partial v / \partial \xi & \partial v / \partial \eta & \partial v / \partial \zeta \\
\partial w / \partial \xi & \partial w / \partial \eta & \partial w / \partial \zeta
\end{bmatrix} \begin{bmatrix}
\partial x_0 / \partial \xi & \partial x_0 / \partial \eta & \partial x_0 / \partial \zeta \\
\partial y_0 / \partial \xi & \partial y_0 / \partial \eta & \partial y_0 / \partial \zeta \\
\partial z_0 / \partial \xi & \partial z_0 / \partial \eta & \partial z_0 / \partial \zeta
\end{bmatrix}^{-1}
\end{equation}
\normalsize
such that, for the four-node tetrahedron considered here, one has
\begin{subequations}
    \begin{align}
    \begin{bmatrix}
    \partial u / \partial \xi & \partial u / \partial \eta & \partial u / \partial \zeta \\
    \partial v / \partial \xi & \partial v / \partial \eta & \partial v / \partial \zeta \\
    \partial w / \partial \xi & \partial w / \partial \eta & \partial w / \partial \zeta
    \end{bmatrix} & = \begin{bmatrix}
    \sum_{i=1}^4 N_{i,\xi} u_i & \sum_{i=1}^4 N_{i,\eta} u_i & \sum_{i=1}^4 N_{i,\zeta} u_i \\
    \sum_{i=1}^4 N_{i,\xi} v_i & \sum_{i=1}^4 N_{i,\eta} v_i & \sum_{i=1}^4 N_{i,\zeta} v_i \\
    \sum_{i=1}^4 N_{i,\xi} w_i & \sum_{i=1}^4 N_{i,\eta} w_i & \sum_{i=1}^4 N_{i,\zeta} w_i 
    \end{bmatrix} \notag \\
    & = \begin{bmatrix}
    u_2 - u_1 & u_3 - u_1 & u_4 - u_1 \\
    v_2 - v_1 & v_3 - v_1 & v_4 - v_1 \\
    w_2 - w_1 & w_3 - w_1 & w_4 - w_1
    \end{bmatrix} \\
    \intertext{whose nodal displacements $\mathbfit{u}_i \defeq \mathbfit{x}_i - \mathbfit{x}_{0i}$, $i=1,2,3,4$, have components of $\mathbfit{u}_i = u_i \, \vec{\mathbfsf{E}}_1 + v_i \, \vec{\mathbfsf{E}}_2 + w_i \, \vec{\mathbfsf{E}}_3$  with $u_i \defeq x_i - x_{0i}$, $v_i \defeq y_i - y_{0i}$ and $w_i \defeq z_i - z_{0i}$, evaluated in the reference co-ordinate frame $( \vec{\mathbfsf{E}}_1 , \vec{\mathbfsf{E}}_2 , \vec{\mathbfsf{E}}_3 )$ of the dodecahedron, and}
    \begin{bmatrix}
    \partial x_0 / \partial \xi & \partial x_0 / \partial \eta & \partial x_0 / \partial \zeta \\
    \partial y_0 / \partial \xi & \partial y_0 / \partial \eta & \partial y_0 / \partial \zeta \\
    \partial z_0 / \partial \xi & \partial z_0 / \partial \eta & \partial z_0 / \partial \zeta
    \end{bmatrix} & = \begin{bmatrix}
    \sum_{i=1}^4 N_{i,\xi} x_{0i} & \sum_{i=1}^4 N_{i,\eta} x_{0i} & \sum_{i=1}^4 N_{i,\zeta} x_{0i} \\
    \sum_{i=1}^4 N_{i,\xi} y_{0i} & \sum_{i=1}^4 N_{i,\eta} y_{0i} & \sum_{i=1}^4 N_{i,\zeta} y_{0i} \\
    \sum_{i=1}^4 N_{i,\xi} z_{0i} & \sum_{i=1}^4 N_{i,\eta} z_{0i} & \sum_{i=1}^4 N_{i,\zeta} z_{0i}
    \end{bmatrix} \notag \\
    & = \begin{bmatrix}
    x_{02} - x_{01} & x_{03} - x_{01} & x_{04} - x_{01} \\
    y_{02} - y_{01} & y_{03} - y_{01} & y_{04} - y_{01} \\
    z_{02} - z_{01} & z_{03} - z_{01} & z_{04} - z_{01}
    \end{bmatrix} \\
    \intertext{whose initial nodal positions are $\mathbfit{x}_{0i} = x_{0i} \, \vec{\mathbfsf{E}}_1 + y_{0i} \, \vec{\mathbfsf{E}}_2 + z_{0i} \, \vec{\mathbfsf{E}}_3$ at vertex $i$.  This matrix is invertible, becuase the four vertices of a tetrahedron are distinct.  The Jacobian matrix is therefore given by}
    \mathbf{J} \defeq \begin{bmatrix}
    \partial x / \partial \xi & \partial y / \partial \xi & \partial z / \partial \xi \\
    \partial x / \partial \eta & \partial y / \partial \eta & \partial z / \partial \eta \\
    \partial x / \partial \zeta & \partial y / \partial \zeta & \partial z / \partial \zeta
    \end{bmatrix} & = \begin{bmatrix}
    \sum_{i=1}^4 N_{i,\xi} x_{i} & \sum_{i=1}^4 N_{i,\xi} y_{i} & \sum_{i=1}^4 N_{i,\xi} z_{i} \\
    \sum_{i=1}^4 N_{i,\eta} x_{i} & \sum_{i=1}^4 N_{i,\eta} y_{i} & \sum_{i=1}^4 N_{i,\eta} z_{i} \\
    \sum_{i=1}^4 N_{i,\zeta} x_{i} & \sum_{i=1}^4 N_{i,\zeta} y_{i} & \sum_{i=1}^4 N_{i,\zeta} z_{i}
    \end{bmatrix} \notag \\
    & = \begin{bmatrix}
    x_{2} - x_{1} & y_{2} - y_{1} & z_{2} - z_{1} \\
    x_{3} - x_{1} & y_{3} - y_{1} & z_{3} - z_{1} \\
    x_{4} - x_{1} & y_{4} - y_{1} & z_{4} - z_{1}
    \end{bmatrix}
    \label{tetJacobian}
    \end{align}
\end{subequations}
whose determinant is used in integrations.  The current nodal positions have components $\mathbfit{x}_{i} = x_i \, \vec{\mathbfsf{E}}_1 + y_i \, \vec{\mathbfsf{E}}_2 + z_i \, \vec{\mathbfsf{E}}_3$, $i=1,2,3,4$, in the dodecahedral frame $( \vec{\mathbfsf{E}}_1 , \vec{\mathbfsf{E}}_2 , \vec{\mathbfsf{E}}_3 )$.  The Jacobian matrix remains invertible provided that the four vertices of a tetrahedron remain distinct.

\subsection{\textbf{QR} Factorization of $\mathbfsf{F}$}
\label{secQR3D}

The re-indexed deformation gradient presented in \S\ref{reindexing3D} has a Gram-Schmidt decomposition that we denote as $\mathbfsf{F} = \boldsymbol{\mathcal{RU}}$ whose components are an orthogonal rotation matrix $\boldsymbol{\mathcal{R}} = \bigl[ \vec{\mathbfsf{g}}_1 \bigm| \vec{\mathbfsf{g}}_2 \bigm| \vec{\mathbfsf{g}}_3 \bigr] = \delta_{ij} \, \vec{\mathbfsf{g}}_i \otimes \vec{\mathbfsf{E}}_j = \mathcal{R}_{ij} \, \vec{\mathbfsf{E}}_i \otimes \vec{\mathbfsf{E}}_j$ and an upper-triangular matrix $\boldsymbol{\mathcal{U}} = \mathcal{U}_{ij} \, \vec{\mathbfsf{E}}_i \otimes \vec{\mathbfsf{E}}_j$ called Laplace stretch, \cite{Freedetal19} both evaluated in the reference co-ordinate frame $( \vec{\mathbfsf{E}}_1 , \vec{\mathbfsf{E}}_2 , \vec{\mathbfsf{E}}_3 )$, so that $\mathbfsf{F} = \mathcal{F}_{ij} \, \vec{\mathbfsf{E}}_i \otimes \vec{\mathbfsf{E}}_j = \mathcal{R}_{ik\,} \mathcal{U}_{kj} \, \vec{\mathbfsf{E}}_i \otimes \vec{\mathbfsf{E}}_j$, and therefore $\mathcal{F}_{ij} = \mathcal{R}_{ik\,} \mathcal{U}_{kj}$.

The components of Laplace stretch $\mathcal{U}_{ij}$ are readily gotten through a Cholesky factorization of the right Cauchy-Green deformation tensor $\mathbfsf{C} = \mathcal{C}_{ij} \, \vec{\mathbfsf{E}}_i \otimes \vec{\mathbfsf{E}}_j$ with tensor components $\mathcal{C}_{ij} = \mathcal{F}_{ki\,} \mathcal{F}_{kj}$ that relate to their physical attributes via \cite{Freed17}
\begin{equation} 
\boldsymbol{\mathcal{U}} = 
\begin{bmatrix}
a & a \gamma & a \beta \\ 0 & b & b \alpha \\ 0 & 0 & c
\end{bmatrix} 
\quad \text{with inverse} \quad
\boldsymbol{\mathcal{U}}^{-1} = \begin{bmatrix}
1/a & -\gamma / b & -( \beta - \alpha\gamma ) / c \\
0 & 1/b & -\alpha / c \\
0 & 0 & 1/c
\end{bmatrix}
\label{LaplaceStretch3D}
\end{equation}
with tensor components $\mathcal{U}_{ij}$ being evaluated according to formul\ae\ \cite{Srinivasa12}
\begin{equation}
\begin{aligned}
\mathcal{U}_{11} & = \sqrt{\mathcal{C}_{11}} & 
\mathcal{U}_{12} & = \mathcal{C}_{12} / \mathcal{U}_{11} &
\mathcal{U}_{13} & = \mathcal{C}_{13} / \mathcal{U}_{11} \\
\mathcal{U}_{21} & = 0 &
\mathcal{U}_{22} & = \sqrt{\mathcal{C}_{22} - \mathcal{U}_{12}^{\,2}} &
\mathcal{U}_{23} & = \bigl( \mathcal{C}_{23} - 
\mathcal{U}_{12\,} \mathcal{U}_{13} \bigr) / \mathcal{U}_{22} \\
\mathcal{U}_{31} & = 0 &
\mathcal{U}_{32} & = 0 & 
\mathcal{U}_{33} & = \sqrt{\mathcal{C}_{33} - \mathcal{U}_{13}^{\,2} - 
    \mathcal{U}_{23}^{\,2}}
\end{aligned}
\label{LagrangianLaplaceStretch3D}
\end{equation}
implying that the physical attributes for Laplace stretch can be evaluated via
\begin{equation}
a \defeq \mathcal{U}_{11} , \quad
b \defeq \mathcal{U}_{22} , \quad
c \defeq \mathcal{U}_{33} , \quad
\alpha \defeq \frac{\mathcal{U}_{23}}{\mathcal{U}_{22}} , \quad
\beta \defeq \frac{\mathcal{U}_{13}}{\mathcal{U}_{11}} , \quad
\gamma \defeq \frac{\mathcal{U}_{12}}{\mathcal{U}_{11}}
\label{LagrangianPhysicalAttributes3D}
\end{equation}
where $a$, $b$ and $c$ are three, orthogonal, elongation ratios, and where $\alpha$, $\beta$ and $\gamma$ are three, orthogonal, simple shears, with $a_0$, $b_0$, $c_0$, $\alpha_0$, $\beta_0$ and $\gamma_0$ denoting their values in some reference state. The elongations must be positive, whereas the shears may be of either sign. Collectively, they constitute a complete set of physical attributes for describing stretch from which constitutive equations can then be constructed. 

No eigen\-value\slash eigen\-vector analysis is required to acquire either the stretch components or their attributes when using this technique. \cite{Srinivasa12} The eigen\-values and eigen\-vectors of the triangular Laplace stretch equate with the eigen\-values and eigen\-vectors of the symmetric polar stretch \textit{only\/} in an absence of shear. \cite{Rosakis90}  Laplace stretch associates with the geometric description of a cube deforming into a parallelepiped; whereas, polar stretch associates with the geometric description of a sphere deforming into an ellipsoid.  They are distinct geometric measures for stretch.

\subsection{Thermodynamic Strains and Strain Rates}
\label{strainsAndStrainRates3D}

In terms of the above physical attributes for stretch, one can define an useful set of strain attributes derived from thermo\-dynamics, specifically \cite{Freed17}
\begin{subequations}
    \label{thermodynamicStrains3D}
    \begin{align}
    \Xi & \defeq \ln \left( \sqrt[3]{\frac{a}{a_0} \frac{b}{b_0} \frac{c}{c_0}} \right) & 
    \mathrm{d} \Xi & = \frac{1}{3} \left( \frac{\mathrm{d}a}{a} + 
    \frac{\mathrm{d}b}{b} + \frac{\mathrm{d}c}{c} \right) \\
    \varepsilon_1 & \defeq \ln \left( \sqrt[3]{\frac{a}{a_0} \frac{b_0}{b}} \right) &
    \mathrm{d} \varepsilon_1 & = \frac{1}{3} \left( \frac{\mathrm{d}a}{a} - 
    \frac{\mathrm{d}b}{b} \right) \\
    \varepsilon_2 & \defeq \ln \left( \sqrt[3]{\frac{b}{b_0} \frac{c_0}{c}} \right) &
    \mathrm{d} \varepsilon_2 & = \frac{1}{3} \left( \frac{\mathrm{d}b}{b} - 
    \frac{\mathrm{d}c}{c} \right) \\
    \gamma_1 & \defeq \alpha - \alpha_0 & 
    \mathrm{d} \gamma_1 & = \mathrm{d} \alpha \\
    \gamma_2 & \defeq \beta - \beta_0 & 
    \mathrm{d} \gamma_2 & = \mathrm{d} \beta \\
    \gamma_3 & \defeq \gamma - \gamma_0 & 
    \mathrm{d} \gamma_3 & = \mathrm{d} \gamma
    \end{align}
\end{subequations}
whose rates are exact differentials, i.e., they are independent of path---a tacit requirement from thermo\-dynamics \cite{Caratheodory09}.  Here $\Xi$ represents dilatation, $\varepsilon_1$ is a squeeze in the 12~plane, and $\varepsilon_2$ is a squeeze in the 23-plane, while $\gamma_1$ is a shear in the 23~plane, $\gamma_2$ is a shear in the 13~plane, and $\gamma_3$ is a shear in the 12~plane, which are three, orthogonal, simple shearing motions.  There is a third squeeze, too, viz., $\varepsilon_3 = -\varepsilon_1 - \varepsilon_2$, but it is not an independent descriptor of strain.

\subsubsection{Stretch Rates}

The following approximations for stretch rates were derived by Freed \&\ Zamani \cite{FreedZamani18}.  From these, the various strain rates listed in Eqn.~(\ref{thermodynamicStrains3D}) can be established.  

A forward difference formula is used to approximate rates in the reference configuration for the various stretch attributes, as obtained from $\mathrm{d} \boldsymbol{\mathcal{U}}_0 = ( \boldsymbol{\mathcal{U}}_1 -  \boldsymbol{\mathcal{U}}_0 ) / \mathrm{d}t + \mathcal{O}(\mathrm{d}t)$.  Neglecting higher-order terms, this produces
\begin{equation}
\begin{aligned}
\mathrm{d} a_0 &
= \frac {a_1 - a_0}{\mathrm{d}t} \quad &
\mathrm{d} \alpha_0 & 
= \frac{b_1}{b_0} \left(\frac{\alpha_1 - \alpha_0}{\mathrm{d}t} \right) \\
\mathrm{d} b_0 & 
= \frac {b_1 - b_0}{\mathrm{d}t} \quad & 
\mathrm{d} \beta_0 & 
= \frac{a_1}{a_0} \left( \frac{\beta_1 - \beta_0}{\mathrm{d}t} \right) \\
\mathrm{d} c_0 & 
= \frac {c_1 - c_0}{\mathrm{d}t} \quad & 
\mathrm{d} \gamma_0 & = \frac{a_1}{a_0} \left( \frac{\gamma_1 - \gamma_0}{\mathrm{d}t}\right) .
\end{aligned}
\label{forwardDifference1stOrder3D}
\end{equation}
A backward difference formula $\mathrm{d} \boldsymbol{\mathcal{U}}_1 = ( \boldsymbol{\mathcal{U}}_1 -  \boldsymbol{\mathcal{U}}_0 ) / \mathrm{d}t + \mathcal{O}(\mathrm{d}t)$ is used to estimate rates for the various stretch attributes at the end of its first integration step, from which it follows that
\begin{equation}
\begin{aligned}
\mathrm{d} a_1 & 
= \frac {a_1 - a_0}{\mathrm{d}t} \;\; & 
\mathrm{d} \alpha_1 & 
= \frac {b_0}{b_1} \left( \frac{\alpha_1 - \alpha_0}{\mathrm{d}t} \right) \\
\mathrm{d} b_1 & 
= \frac {b_1 - b_0}{\mathrm{d}t} \;\; & 
\mathrm{d} \beta_1 & 
= \frac {a_0} {a_1} \left( \frac{\beta_1 - \beta_0}{\mathrm{d}t} \right) \\
\mathrm{d} c_1 & 
= \frac {c_1 - c_0}{\mathrm{d}t} \;\; & 
\mathrm{d} \gamma_1 & 
= \frac{a_0}{a_1} \left(\frac{\gamma_1 - \gamma_0}{\mathrm{d}t} \right) .
\end{aligned}
\label{backwardDifference1stOrder3D}
\end{equation}
Curiously, there is a distinction in how the shear rates are approximated at the two nodes belonging to this first interval of integration.

Equations (\ref{forwardDifference1stOrder3D} \& \ref{backwardDifference1stOrder3D}) are first-order approximations for these derivatives.  Second-order approximations can be established whenever $i > 0$ provided the stepsize for step $[i, i+1]$ equals the stepsize for step $[i-1, i]$, where state $i=0$ associates with an initial condition.  The backward difference formula  $\mathrm{d} \boldsymbol{\mathcal{U}}_{i+1} = ( 3 \, \boldsymbol{\mathcal{U}}_{i+1} -  4 \, \boldsymbol{\mathcal{U}}_i + \boldsymbol{\mathcal{U}}_{i-1} ) / 2\mathrm{d}t + \mathcal{O} \bigl( ( \mathrm{d}t )^2 \bigr)$ produces differential stretch rates of
\begin{equation}
\begin{aligned}
\mathrm{d} a_{i+1} & 
= \frac {3a_{i+1} - 4a_i +  a_{i-1}}{2\mathrm{d}t} \\ 
\mathrm{d} b_{i+1} & 
= \frac {3b_{i+1} - 4b_i +  b_{i-1}}{2\mathrm{d}t} \\
\mathrm{d} c_{i+1} & 
= \frac {3c_{i+1} - 4c_i +  c_{i-1}}{2\mathrm{d}t} \\
\mathrm{d} \alpha_{i+1} & 
= 2 \frac{b_i} {b_{i+1}} \left( \frac{\alpha_{i+1} - \alpha_i}{\mathrm{d}t} \right) - \frac{b_{i-1}} {b_{i+1}} \left( \frac{\alpha_{i+1} - \alpha_{i-1}}{2\mathrm{d}t} \right) \\
\mathrm{d} \beta_{i+1} & 
= 2 \frac{a_i}{a_{i+1}} \left( \frac{\beta_{i+1} - \beta_i }{\mathrm{d}t} \right) - \frac{a_{i-1}} {a_{i+1}} \left( \frac{\beta_{i+1} - \beta_{i-1}}{2\mathrm{d}t} \right) \\ 
\mathrm{d} \gamma_{i+1} & 
= 2 \frac{a_i} {a_{i+1}} \left(\frac{\gamma_{i+1} - \gamma_i}{\mathrm{d}t} \right) - \frac{a_{i-1}}{a_{i+1}} \left( \frac{\gamma_{i+1} - \gamma_{i-1}}{2\mathrm{d}t} \right) 
\end{aligned}
\label{backwardDifference2ndOrder3D}
\end{equation}
which require data to be stored for the previous state associated with step $i-1$.


\section{Code Verification: Kinematics}
\label{sec:verification}

The thermodynamic conjugate pairs of Freed \textit{et~al}.\ \cite{Freed17,Freedetal17,FreedZamani19} result in the following geometric/thermo\-dynamic strain measures for our dodecahedral model.  For 1D rods: an axial strain $e = \ln ( L / L_0 )$.  For 2D membranes: a dilation $\xi = \ln \sqrt{ab/a_0b_0}$ $= \ln \sqrt{A/A_0}$, a squeeze (or pure shear) $\varepsilon = \ln \sqrt{ab_0/a_0b} = \ln \sqrt{\Gamma / \Gamma_0}$, and a (simple) shear $\gamma = g - g_0$.  And for 3D dodecahedra: a dilatation $\Xi = \ln \sqrt[3]{V \! / V_0}$ and, for those cases where the medium within an alveolar sac can support non-uniform stresses, two squeezes $\varepsilon_1 = \ln \sqrt[3]{a b_0 / a_0 b}$ and $\varepsilon_2 = \ln \sqrt[3]{b c_0 / b_0 c}$ plus three shears $\gamma_1 = \alpha - \alpha_0$, $\gamma_2 = \beta - \beta_0$ and $\gamma_3 = \gamma - \gamma_0$. 

\subsection{Isotropic Motions}

Imposing an uniform far-field motion of a volumetric expansion onto our dodecahedral model results in a dodecahedral dilatation ($\Xi \defeq \ln \sqrt[3]{ V \! / V_0 }$) that equals its pentagonal dilation ($\xi \defeq \ln \sqrt{ A / A_0 }$) that equals its chordal strain ($e \defeq \ln ( L / L_0 )$).  These three strain measures follow from the 3-mode thermo\-dynamic theory of Freed \textit{et~al}., \cite{Freedetal17,FreedZamani19} as presented above.  Other choices for strain measures do not result in one-to-one relationships when exposed to an isotropic motion like those observed here.  This is a particularly useful result in that it establishes a meaningful scaling in terms of strains between the three dimensions, cf.\ Fig.~\ref{figDilatation}.  It also provides for a verification of the numerical implementation of our dodecahedral model.  

\begin{figure}
	\centering
	\includegraphics[width=\textwidth]{figures/dilatation.jpg}
	\caption{Response of a dodecahedron exposed to an isotropic motion of dilatation.  The abscissa is the control variable and the ordinates are response variables. The right graphic plots the areal response of the pentagons $\xi = \ln \sqrt{A / A_0}$, while the left graphic plots the axial response of the chords $e = \ln ( L / L_0)$. Both are plotted against the volumetric response of the dodecahedron $\Xi = \ln \sqrt[3]{V \! / V_0}$.  Here $V$ denotes dodecahedral volume, $A$ denotes pentagonal area, and $L$ denotes chordal length, all being evaluated in the current state, whose reference values are $V_0$, $A_0$ and $L_0$.}
	\label{figDilatation}
\end{figure}

\subsubsection{Geometric vs.\ Thermodynamic Strains}

There are two types of strain measures that one can use to quantify deformation within a pentagon of a dodecahedron: geometric and thermo\-dynamic.  For the uniform far-field motion of volumetric expansion, only a thermo\-dynamic strain known as dilation, i.e., $\xi = \ln \sqrt{ab/a_0b_0}$, varies with the motion, and its response equals that of the geometric strain $\ln \sqrt{A / A_0}$, see Fig.~\ref{figDilatation2}.  Also present in this graph is an observation that the thermo\-dynamic strains for squeeze $\varepsilon$ and shear $\gamma$ do not contribute under motions of pure dilatation, as expected.  This further verifies the numerical implementation of our dodecahedral model.

\begin{figure}
	\centering
	\includegraphics[width=\textwidth]{figures/dilatationGeoVsThermo.jpg}
	\caption{Response of a dodecahedron exposed to a far-field isotropic motion of dilatation.  The abscissa is the control variable and the ordinates are response variables. The right graphic plots the three thermo\-dynamic strains, as they apply to a pentagon, while the left graphic plots the geometric strain of a pentagon.}
	\label{figDilatation2}
\end{figure}

To put this into perspective, we compare with studies done by multiple investigators where ratios of alveolar surface area, viz., $A/A_0$, have been measured in rat, rabbit, guinea pig, and cat, cf.\ Roan \& Waters \cite[Table~1]{RoanWaters11}.  These experiments considered ranges that went as low as 25\% and as high as 100\% of total lung capacity.  Taking statistics of their tabulation produced results of: $A/A_0 = 1.47 \pm 0.44$ during inflation and $A/A_0 = 1.18 \pm 0.14$ during deflation, which correspond to a $\xi = \ln\sqrt{A/A_0} = 0.19 \pm 0.18$ for inflation and a $\xi = \ln\sqrt{A/A_0} = 0.08 \pm 0.07$ for deflation.  These areal strain values coincide with chordal strains of $e=\ln(L/L_0) = 0.13$ measured \textit{in~vivo\/} around the periphery of an alveolus in rat lung, as reported by Perlman \& Bhattachary \cite{PerlmanBhattacharya07}.  Our kinematics have been verified well past these physiologic ranges, viz., for dilatations up to 100\% logarithmic strain. 

\subsection{Isochoric Motions}

The motions of pure and simple shears are volume preserving.  Imposing these shears as far-field motions onto our dodecahedral model produced the results displayed in Fig.~\ref{figIsochoric}.  For a simple shear, the numerical model is in error by about machine precision, i.e., $\epsilon_m \approx 2.2 \times 10^{-16}$, for strains up to 100\%, while for pure shear (a special case of squeeze in 3D) the model is in error by about machine precision for strains up to of about 60\%, after which the error increases up to about $10\epsilon_m$ at strains around~100\%.  This further verifies the numerical implementation of our dodecahedral model.

\begin{figure}
	\centering
	\includegraphics[width=10cm]{figures/isochoric.jpg}
	\caption{Response of a dodecahedron exposed to far-field motions of pure and simple shears.  Note that the ordinate is $\times 10^{-15}$ and machine precision is $\sim 2.2 \times 10^{-16}$.}
	\label{figIsochoric}
\end{figure}

\subsubsection{Geometric Strains}

How the thirty chords and the twelve irregular pentagons deform under far-field motions of pure shear is displayed in Fig.~\ref{figPureShears}.  Figure~\ref{figIsochoric} demonstrates that the overall response of a dodecahedron is isochoric during pure shear.  Regardless, Fig.~\ref{figPureShears} demonstrates that the individual chordal and pentagonal constituents deform in a non-homogeneous manner, where the strains have been calculated as geometric changes in dodecahedral shape.  This result agrees with \textit{in~vivo\/} observations made by Perlman \& Bhattacharya \cite{PerlmanBhattacharya07} where confocal microscopy was used to image a breathing rat lung.

\begin{figure}
	\centering
	\includegraphics[width=\textwidth]{figures/squeeze12.jpg} \\
	\includegraphics[width=\textwidth]{figures/squeeze23.jpg} \\
	\includegraphics[width=\textwidth]{figures/squeeze13.jpg} 
	\caption{Response of a dodecahedron exposed to far-field pure-shear motions in the sense of Treloar \cite{Treloar75}: $a = \ell$, $b = 1/\ell$ and $c = 1$ in the top images; $a = 1$, $b = \ell$ and $c = 1/\ell$ in the middle images; and $a = 1/\ell$, $b = 1$ and $c = \ell$ in the bottom images, with $\ell$ denoting an elongation of extrusion.  In all six graphic images, the relevant (controlled) motion of the far-field pure shear is plotted along the abscissa.  In each image pair, the right graphic presents pentagonal dilations, while the left graphic presents chordal elongations. Only unique responses are plotted; repetitions are not.}
	\label{figPureShears}
\end{figure}

For the chords, there are six independent responses for dodecahedral motions of pure shear: two chords each for three of these lines, and eight chords each for the remaining three curves present in the left images of Fig.~\ref{figPureShears}.  For pentagons, there are three independent responses with four pentagons responding according to each curve shown in the right images.  Although different chords and pentagons deform differently when sheared in different directions, their collective responses are the same regardless of the far-field direction being sheared.  Consequently, the local geometric response of a dodecahedron is isotropic under the far-field motions of pure shear.  

How the thirty chords and the twelve irregular pentagons deform under far-field motions of simple shear is displayed in Fig.~\ref{figSimpleShears}.  Figure~\ref{figIsochoric} demonstrates that the overall response of a dodecahedron is isochoric during a far-field simple shear. Figure~\ref{figSimpleShears} demonstrates that the individual chordal and pentagonal constituents deform in a non-homogeneous manner during simple shears, like they do for pure shears.  However, unlike pure shears whose collective chordal and pentagonal responses remain isotropic, here they diverge slightly from isotropy under motions of simple shear.  Simple shears in the 12 and 23 planes have the same collective response; whereas, simple shear in the 13 plane has a slightly different response with respect to changes in the shearing direction.  

\begin{figure}
	\centering
	\includegraphics[width=\textwidth]{figures/shear12.jpg} \\
	\includegraphics[width=\textwidth]{figures/shear23.jpg} \\
	\includegraphics[width=\textwidth]{figures/shear13.jpg} 
	\caption{Response of a dodecahedron exposed to far-field simple-shear motions.  In all six graphic images, the relevant (controlled) motion of simple shear is plotted along the abscissa.  In each image pair, the right graphic presents pentagonal dilations, while the left graphic presents chordal elongations. Only unique responses are plotted; repetitions are not. Responses in the 13 plane differ from those of the 12 and 23 planes.}
	\label{figSimpleShears}
\end{figure}

Figures \ref{figDilatation}--\ref{figSimpleShears} show that a dodecahedron is (nearly, but not completely) isotropic in its kinematic response, as measured by the geometric strains $e = \ln (L / L_0)$, $\xi = \ln \sqrt{A / A_0}$ and $\Xi = \ln \sqrt[3]{V / V_0}$.  Furthermore, even though a far-field deformation is homogeneous, in accordance with our Conjecture on pg.~\pageref{conjecture}, the local deformations within the individual constituents of an alveolus will typically be heterogeneous, which agrees with imaging data \cite{PerlmanBhattacharya07}.

\subsubsection{Thermodynamic Strains}

Addressing the septal response, modeled here as a set of twelve irregular pentagons per alveolus, we desire to come to a determination regarding how to best model the deformation occurring within these alveolar septa.  In the section above we investigated the geometric response of alveolar septa via the strain measure $\ln \sqrt{A/A_0}$, which quantifies dilation.  

The thermo\-dynamic strains arising from a Gram-Schmidt factorization of the deformation gradient put forward in \S\ref{secQR} specify three strain measures pertinent to a membrane: dilation $\xi = \ln \sqrt{ab/a_0 b_0}$, squeeze $\varepsilon = \ln \sqrt{ab_0 / a_0 b}$ and shear $\gamma = g - g_0$, where elongations $a$ and $b$ and magnitude of shear $g$ are illustrated in Fig.~\ref{figKinematics}.  Of these, dilation is an uniform response, while squeeze and shear describe isochoric non-uniform responses.  To acquire them requires knowing the deformation gradient.

The curves in Figs.~\ref{figPureShears} \& \ref{figSimpleShears} were obtained from geometric measures for chordal strain $\ln (L/L_0)$ and areal dilation $\ln \sqrt{A/A_0}$.  They were computed under separate far-field conditions of pure and simple shears.  The curves in Figs.~\ref{figPureShearsPentagons} \& \ref{figSimpleShearsPentagons} were obtained from thermo\-dynamic measures for membrane strain under the same far-field deformations.  The strains of dilation $\xi$, squeeze $\varepsilon$, and shear $\gamma$ were computed in accordance with \S\ref{secQR} using deformation gradients gotten from the pentagonal shape functions of Wachspress \cite{Wachspress75} discussed in \S\ref{secShapeFns}.\footnote{
	Five constant-strain triangles were also used to quantify the deformation gradient for each pentagonal surface at its centroid---the common vertex to all five triangles.  This approach provided accurate descriptions for uniform strain, i.e., dilation $\xi$, but not for the two non-uniform strains, viz., squeeze $\varepsilon$ and shear $\gamma$; hence, our preference to use Wachspress shape functions for alveolar planes.
} 

\begin{figure}
	\centering
	\includegraphics[width=\textwidth]{figures/pentagonalPureShear12.jpg} \\
	\includegraphics[width=\textwidth]{figures/pentagonalPureShear23.jpg} \\
	\includegraphics[width=\textwidth]{figures/pentagonalPureShear13.jpg} \\
	\caption{Same boundary conditions as in Fig.~\ref{figPureShears}.  Pentagonal areas were used to compute dilation in Fig.~\ref{figPureShears}.  The shape functions of Wachspress were used to compute dilation here.  The uniform response in the right column of Fig.~\ref{figPureShears} and in the left column above are the same, providing additional assurance that the code has been correctly implemented.  The squeeze response shown in the center column is the same for all three orientations of far-field pure shear, i.e., this response is isotropic.  The right column has ordinates scaled by $10^{-14}$ implicating that there is no effective simple shear response occurring within any pentagonal surface of the dodecahedron whenever it is subjected to a far-field motion of pure shear.}
	\label{figPureShearsPentagons}
\end{figure}

\begin{figure}
	\centering
	\includegraphics[width=\textwidth]{figures/pentagonalSimpleShear12.jpg} \\
	\includegraphics[width=\textwidth]{figures/pentagonalSimpleShear23.jpg} \\
	\includegraphics[width=\textwidth]{figures/pentagonalSimpleShear13.jpg} \\
	\caption{Same boundary conditions as in Fig.~\ref{figSimpleShears}.  Pentagonal areas were used to compute dilation in Fig.~\ref{figSimpleShears}.  The shape functions of Wachspress were used to compute dilation here. The uniform response in the right column of Fig.~\ref{figSimpleShears} and in the left column above are the same, providing additional assurance that the code has been correctly implemented.  Like the dilational responses of the left column, the squeeze responses of the center column are the same in the 12 and 23 planes, but differ in the 13 plane.  In all cases, the simple shear response of any pentagonal plane is proportional to that of the far-field shear imposed, further substantiating the code's implementation.  The shear response of the septal membranes is isotropic.}
	\label{figSimpleShearsPentagons}
\end{figure}

Figures~\ref{figPureShears}--\ref{figSimpleShearsPentagons} allow us to conclude that if septal dilation were the only mode of planar deformation thought to cause a mechanical response, then knowledge of the geometric strain $\xi = \ln \sqrt{A/A_0}$ would be adequate; there would be no need to introduce a separate finite-element discretization of the septal planes for acquiring their deformation gradients.  However, if the non-uniform responses of squeeze $\varepsilon$ and shear $\gamma$ are thought to contribute to the overall mechanical response of these membranes, then the shape functions of Wachspress \cite{Wachspress75,Wachspress16} ought to be used for acquiring the deformation gradient within a septal plane.  We found, but do not present figures to support this observation, that constant-strain triangles are not accurate enough for our application whenever non-uniform deformations are considered.  Strains derived from Wachspress shape functions are inhomogeneous; consequently, the deformation gradient will need to be evaluated at each Gauss point of integration within a pentagon, cf.\ \S\ref{secPentagonGaussPts}.

\subsection{Co-ordinate Pivoting}

The pivoting strategy of \S\ref{secRemedy} used to address the physical dilemma of \S\ref{secDilemma} did not engage often during our assessment of the code, but it did arise at least twice with effects illustrated in Figs.~\ref{figPivoting1} \& \ref{figPivoting3}.  Here one can see that there is a clear effect on the shear response within four pentagonal planes; however, no change is observed to have occurred in either the dilation or squeeze responses, as expected.  It is not always possible to know when or where a co-ordinate relabeling ought to occur; consequently, the algorithm put forward in \S\ref{secRemedy} is deemed necessary.

\begin{figure}
	\centering
	\includegraphics[width=\textwidth]{figures/continuousShearWithPivot1.jpg}
	\includegraphics[width=\textwidth]{figures/continuousShearNoPivot1.jpg}
	\caption{A far-field shear of $\gamma_{23}$ is imposed on the dodecahedron.  Pentagons 1 and 8 exhibit the plotted response.  The top set of figures result whenever the pivoting strategy of \S\ref{secRemedy} is used, while the bottom set of figures result whenever no pivoting strategy is employed.  The dilation (left graphs) and squeeze (center graphs) responses are not effected by pivoting, only shear (right graphs) is effected.  Pivoting maintains a linear shear response under a far-field shearing of the dodecahedron, as desired.}
	\label{figPivoting1}
\end{figure}

\begin{figure}
	\centering
	\includegraphics[width=\textwidth]{figures/continuousShearWithPivot3.jpg}
	\includegraphics[width=\textwidth]{figures/continuousShearNoPivot3.jpg}
	\caption{A far-field shear of $\gamma_{23}$ is imposed on the dodecahedron.  Pentagons 3 and 10 exhibit the plotted response.  The top set of figures result whenever the pivoting strategy of \S\ref{secRemedy} is used, while the bottom set of figures result whenever no pivoting strategy is employed.  The dilation (left graphs) and squeeze (center graphs) responses are not effected by pivoting, only shear (right graphs) is effected.  Pivoting maintains a linear shear response under a far-field shearing of the dodecahedron, as desired.}
	\label{figPivoting3}
\end{figure}

\subsection{Compatible Membrane Deformations}

For a deformation to be compatible, and therefore integrable, the curl of its deformation gradient must vanish, viz., $\textrm{curl} (\mathbfsf{F}) = \textbf{0}$ \cite{Clayton15}. Equation~(\ref{compatibility}) provides constraint equations for the compatibility of planar motions, e.g., septal planes of an alveolus.  Here we test to make sure that these conditions are satisfied within the pentagonal planes of our alveolar dodecahedron, assuming that the shape functions of Wachspress apply.

Figure \ref{figCompatDilatation} presents the compatibility response at the centroid of a typical pentagonal plane during the uniform expansion of a regular dodecahedron out to 100\% strain.   Theoretically, all four derivatives should be zero for this motion. Actually, their values are on the order of machine precision.  Most importantly, whenever they are not zero, they lie along the $45^{\circ}$ diagonal, thereby verifying compatibility in the case of a dilatation.

\begin{figure}
	\centering
	\includegraphics[width=\textwidth]{figures/compatibilityDilatation.jpg}
	\caption{Planar compatibility requires $F_{11,2} = F_{12,1}$ and $F_{22,1} = F_{21,2}$ where the left-hand sides of these formul\ae\ are plotted as the absciss\ae\ and the right-hand sides are plotted as the ordinates.  For compatibility, the response ought to lie along the $45^{\circ}$ diagonal, which is drawn in red over the range of $\pm 10^{-15}$ where machine precision is about $2.2 \times 10^{-16}$.  Here the motion is one of uniform dilatation out to 100\% strain.}
	\label{figCompatDilatation}
\end{figure}

Similarly, Figs.~\ref{figCompatPureShearP5} \& \ref{figCompatSimpleShearP5} present typical responses for testing compatibility during far-field pure shear (Fig.~\ref{figCompatPureShearP5}) and simple shear (Fig.~\ref{figCompatSimpleShearP5}) deformations.  In both cases, one of the four pentagons around the girth of the dodecahedron (viz., \#5) has been selected, as both modes of deformation are activated in this pentagon.  In both cases, errors are typically less than ten times machine precision, thereby verifying compatibility in the cases of squeeze and shear.

\begin{figure}
	\centering
	\includegraphics[width=\textwidth]{figures/compatibilityPureShearP5G7.jpg}
	\includegraphics[width=\textwidth]{figures/compatibilityPureShearP5G5.jpg}
	\caption{Planar compatibility requires $F_{11,2} = F_{12,1}$ and $F_{22,1} = F_{21,2}$ where the left-hand sides of these formul\ae\ are plotted as the absciss\ae\ and the right-hand sides are plotted as the ordinates.  For compatibility, the response ought to lie along the $45^{\circ}$ diagonal, which is drawn in red over the range of $\pm 10^{-15}$ where machine precision is about $2.2 \times 10^{-16}$.  Here the motion is one of pure shear out to 100\% strain with elongation occurring in the 1-direction, contraction occurring in the 2~direction, while the 3-direction is held fixed.  These results pertain to pentagon~5: nodes 15, 5, 12, 11, 1, cf.\ Fig.~\ref{figDodecahedron} and Table~\ref{TablePentagons}.  The top row of figures is the best response among the Gauss points, while the bottom row of figures is the worst response.}
	\label{figCompatPureShearP5}
\end{figure}

\begin{figure}
	\centering
	\includegraphics[width=\textwidth]{figures/compatibilitySimpleShearP5G7.jpg}
	\includegraphics[width=\textwidth]{figures/compatibilitySimpleShearP5G5.jpg}
	\caption{Planar compatibility requires $F_{11,2} = F_{12,1}$ and $F_{22,1} = F_{21,2}$ where the left-hand sides of these formul\ae\ are plotted as the absciss\ae\ and the right-hand sides are plotted as the ordinates.  For compatibility, the response ought to lie along the $45^{\circ}$ diagonal, which is drawn in red over the range of $\pm 10^{-15}$ where machine precision is about $2.2 \times 10^{-16}$.  Here the motion is one of simple shear out to 100\% strain, shearing along 1-2~planes in the 1-direction.  These results pertain to pentagon~5: nodes 15, 5, 12, 11, 1, cf.\ Fig.~\ref{figDodecahedron} and Table~\ref{TablePentagons}.  The top row of figures is the best response among the Gauss points, while the bottom row of figures is the worst response.}
	\label{figCompatSimpleShearP5}
\end{figure}

This collective set of graphs, Figs.~\ref{figCompatDilatation}--\ref{figCompatSimpleShearP5}, investigate the constraint of compatibility in terms of the three fundamental modes of deformation: dilatation, squeeze and shear.  These figures verify that the constraint of compatibility is satisfied when using the pentagonal shape functions of Wachspress \cite{Wachspress75,Wachspress16} in our dodecahedral model, as errors are typically less than ten times machine precision.  This has been verified out to deformations that are at least three times those of their normal physiologic range.

\textit{Our kinematic analysis of a dodecaheron has been verified, both theoretically and numerically.}


\newpage
\setcounter{equation}{0}
\setcounter{figure}{0}
\setcounter{section}{0}
\setcounter{table}{0}
\part{Constitutive Theory}
\label{partConstitutive}

Roan \& Waters \cite{RoanWaters11} and Suki \textit{et~al}.\ \cite{Sukietal05,Sukietal11} have both written extensive review articles on the mechanics of parenchyma.  They have provided detailed information about the structural constituents of alveoli.  And they have discussed their influence on the overall mechanical response of parenchyma.  Of particular relevance, from a mechanics perspective, are the constituent building blocks of alveolar tissue: collagen (types I and III, predominantly), elastin, proteoglycans and other proteins, surfactant and cells (epithelial and endothelial, predominantly).  These constituents are assembled in such a manner so as to produce a variety of alveolar sub-structures that are essentially 1D (alveolar chords), 2D (alveolar septa) and 3D (alveolar sacs) in their geometric construction.  

Birzle \textit{et~al}.\ \cite{Birzleetal19} performed a set of uniaxial experiments on rat parenchyma in an effort to delineate the separate effects of elastin, collagen and the ground substance (everything else) on the collective mechanical response of these tissues.  This information should be useful, especially in the construction of a mixture theory for alveolar septa, which is a topic of research of another member in our group.  Results from their research will also be a useful aid in our modeling efforts of septal planes.

A dodecahedron is used here as a geometric model for an alveolus \cite{FrankusLee74}, cf.\ Figs.~\ref{figRatLung} \& \ref{figDodecahedron}.  It is comprised of: thirty 1D rods that represent alveolar chords, twelve 2D membranes that represent alveolar septa, considered here to be pentagonal in shape, and one 3D cavity filled with air (or fluid in the case of a contusion caused by injury, or an edema caused by disease) whose geometry is considered to be dodecahedral in shape.  The thermo\-elastic constitutive equations presented in this chapter for spatial chords and membranes are derived in Appendix~\ref{appImplicitElasticity}.  Elastic behavior is sufficient for our intended application of studying alveoli subjected to traveling waves.

We recall from our kinematic study of a dodecahedron that the geometric strains (i.e., $e \defeq \ln ( L / L_0 )$ for the elongation of septal chords, $\xi \defeq \ln \sqrt{A / A_0}$ for the dilation of septal membranes, and $\Xi \defeq \ln \sqrt[3]{V / V_0}$ for the dilatation of alveolar volume) are equivalent to one another under motions of uniform compression\slash expansion.  These three, geometric, strain measures also exist as thermo\-dynamic strains, each associating with a distinct and unique conjugate stress. \cite{Freed17,FreedZamani19}

Constitutive equations are a derived consequence from physical laws governing thermo\-dynamic processes.  Here we derive constitutive equations applicable for modeling 1D thermo\-elastic fibers (alveolar chords), 2D thermo\-elastic membranes (alveolar septa), and 3D thermo\-elastic volumes (alveolar sacs).  In \S\ref{secUniformCE}, we assume that the motions are uniform in their spatial dimension.  Later, in Sections~\ref{secNonuniform2D} \& \ref{secNonuniform3D}, the non-uniform motions of squeeze and shear are included into our thermo\-dynamic framework for membranes and volumes, respectively.  Section~\ref{secAlveolus} pulls these results together, sufficient for the intended purpose of modeling the three structural facets that comprise an alveolous.  Specifically, all geometric entities (alveolar chords, alveolar septa, and alveolar sacs) are now described in terms of stresses ($\text{dyne/cm}^2$) instead of their intensive thermo\-dynamic forces (force, surface tension, and stress).  This is done to facilitate implementation of these models into code, and to facilitate interpretations of their results by engineers and scientists.  The chapter closes with a discussion of their implementation into finite elements in \S\ref{secFE_CE} along with a set of examples created to verify our code in \S\ref{secCE_verifyCode}.

\section{Green Thermoelastic Solids: Uniform Motions in 1D, 2D \& 3D}
\label{secUniformCE}

Combining the First and Second Laws of Thermo\-dynamics governing uniform, reversible, adiabatic processes results in the following three formul\ae, one per dimension; they are
\begin{subequations}
    \label{thermoelasticLaws}
    \begin{align}
    \mbox{} & \text{In 1 Dimension:} & 
    \mathrm{d}U & = \theta \, \mathrm{d} \eta +
    \tfrac{1}{\rho_{1D}} F \, \mathrm{d}L / L
    \label{thermoelastic1Dlaw} \\
    \mbox{} & \text{In 2 Dimensions:} &
    \mathrm{d}U & = \theta \, \mathrm{d} \eta + 
    \tfrac{1}{\rho_{2D}} T \, \mathrm{d}A / \! A
    \label{thermoelastic2Dlaw} \\
    \mbox{} & \text{In 3 Dimensions:} &
    \mathrm{d}U & = \theta \, \mathrm{d} \eta - 
    \tfrac{1}{\rho_{3D}} P \, \mathrm{d}V \! / V \!
    \label{thermoelastic3Dlaw}
    \end{align}
\end{subequations}
wherein $U$ is an internal energy density (erg/g = dyne.cm/g), which is a function of state, $\theta$ is a temperature in Kelvin ($273 + \mbox{}^{\circ}$C), $\eta$ is an entropy density (erg/g.K), $L$ is a length of line (cm), $A$ is an area of surface ($\text{cm}^2$), $V$ is a volume of space ($\text{cm}^3$), $F$ is a force (dyne), $T$ is a surface tension (dyne/cm), and $P$ is a pressure (dyne/$\text{cm}^2$ = barye), whereas the mass densities $\rho_{1D}$ ($\text{g/cm}$), $\rho_{2D}$ ($\text{g/cm}^2$) and $\rho_{3D}$ ($\text{g/cm}^3$) associate with a reference state of per unit length, or per unit area, or per unit volume, as appropriate.  Pressure $P$ is assigned to be positive whenever a body undergoes hydro\-static compression.  Per accepted continuum mechanics practice, the sign of pressure gets flipped back and forth depending on what pressure we are talking about in lung mechanics.

\subsection{Constitutive Equations}

Because the internal energy density $U$ is a state function, its differential rate of change describes a Pfaffian form \cite{Caratheodory09} out of which the following constitutive formul\ae\ are readily obtained
\begin{subequations}
    \label{GreenElasticCEs}
    \begin{align}
    \mbox{} & \text{In 1D:} & 
    \theta & = \partial_{\eta} U ( \eta , \ln (L/L_0)) &
    F & = \rho_{1D} \, \partial_{\ln(L/L_0)} U ( \eta , \ln (L/L_0) ) \\
    \mbox{} & \text{In 2D:} &
    \theta & = \partial_{\eta} U ( \eta , \ln (A / \! A_0) ) &
    T & = \rho_{2D} \, \partial_{\ln (A / \! A_0)} U ( \eta , \ln (A / A_0) ) \\
    \mbox{} & \text{In 3D:} &
    \theta & = \partial_{\eta} U ( \eta , \ln (V \! / V_0) ) &
    -P & = \rho_{3D} \, \partial_{\ln (V \! / V_0)} U ( \eta , \ln (V \! / V_0) )
    \end{align}
\end{subequations}
where strains are logarithms of dimension-appropriate stretches.  As a matter of convenience, we adopt the notation $\partial_{\eta} U \defeq \partial U / \partial \eta$, etc.  Employing the geometric strains of Part~\ref{partKinematics}, viz., $e \defeq \ln ( L / L_0 )$, $\xi \defeq \ln \sqrt{ A / \! A_0 }$ and $\Xi \defeq \ln \sqrt[3]{V \! / V_0}$ with differential rates of $\mathrm{d} e = L^{-1} \, \mathrm{d}L$, $\mathrm{d} \xi = \tfrac{1}{2} A^{-1} \, \mathrm{d}A$ and $\mathrm{d} \Xi = \tfrac{1}{3} V^{-1} \, \mathrm{d}V$, these constitutive equations take on the following form
\begin{subequations}
    \label{uniformCEs}
    \begin{align}
    \mbox{} & \text{In 1D:} & 
    \theta & = \partial_{\eta} U ( \eta , e) &
    F & = \rho_{1D} \, \partial_e U ( \eta , e ) \\
    \mbox{} & \text{In 2D:} &
    \theta & = \partial_{\eta} U ( \eta , \xi ) &
    \pi & = \rho_{2D} \, \partial_{\xi} U ( \eta , \xi ) \\
    \mbox{} & \text{In 3D:} &
    \theta & = \partial_{\eta} U ( \eta , \Xi ) &
    \Pi & = \rho_{3D} \, \partial_{\Xi} U ( \eta , \Xi )
    \end{align}
\end{subequations}
wherein $\pi \defeq 2T$ and $\Pi \defeq -3P$ are the measures for surface tension and pressure that we use in this work.  We find it useful to use this negative measure for pressure because the transpulmonary pressure in lung, under normal physiologic conditions, is typically negative; hence, $\Pi$ would be positive in its specification of transpulmonary pressure.  The above constitutive equations describe Green thermo\-elastic solids of specified dimension undergoing uniform motions in adiabatic enclosures.

We consider response variables for temperature and force\slash surface-tension\slash pressure to be $C^1$ functions of state; therefore, the internal energy $U$ is taken to be a $C^2$ function of state in a Green thermo\-elastic solid undergoing uniform adiabatic motions (cf.\ Weinhold \cite{Weinhold75c} and Gilmore \cite{Gilmore84}).  Under these conditions of smoothness, one can differentiate Eqn.~(\ref{uniformCEs}), thereby producing the following collection of coupled, partial, differential equations
\begin{subequations}
    \label{GreenElasticODEs}
    \begin{align}
    \mbox{} & \text{In 1 Dimension:} &
    \left\{ \begin{matrix} \mathrm{d} \theta \\ 
    \mathrm{d} F \end{matrix} \right\} & = \begin{bmatrix}
    \partial_{\eta\eta} U & \partial_{\eta e} U \\
    \rho_{1D} \, \partial_{e \eta} U & \rho_{1D} \, \partial_{ee} U \end{bmatrix} 
    \left\{ \begin{matrix} \mathrm{d} \eta \\
    \mathrm{d} e \end{matrix} \right\} \\
    % second formula
    \mbox{} & \text{In 2 Dimensions:} &
    \left\{ \begin{matrix} \mathrm{d} \theta \\ 
    \mathrm{d} \pi \end{matrix} \right\} & = \begin{bmatrix}
    \partial_{\eta\eta} U & \partial_{\eta \xi} U \\
    \rho_{2D} \, \partial_{\xi\eta} U & \rho_{2D} \, \partial_{\xi\xi} U \end{bmatrix} \left\{ \begin{matrix} \mathrm{d} \eta \\
    \mathrm{d} \xi \end{matrix} \right\} \label{GreenMembrane} \\
    % third formula
    \mbox{} & \text{In 3 Dimensions:} &
    \left\{ \begin{matrix} \mathrm{d} \theta \\ 
    \mathrm{d} \Pi \end{matrix} \right\} & = \begin{bmatrix}
    \partial_{\eta\eta} U & \partial_{\eta\Xi} U \\
    \rho_{3D} \, \partial_{\Xi \eta} U & \rho_{3D} \, \partial_{\Xi\Xi} U \end{bmatrix} \left\{ \begin{matrix} \mathrm{d} \eta \\
    \mathrm{d} \Xi \end{matrix} \right\} \label{GreenSolid}
    \end{align}
\end{subequations}
where mixed partial derivatives obey $\partial_{e \eta} U = \partial^2 U / \partial e \partial \eta = \partial^2 U / \partial \eta \partial e = \partial_{\eta e} U$, etc., that in the thermo\-dynamics literature are referred to as Maxwell's relations; they are also known as Silvester's criteria for the integrability of a Pfaffian form.

Exchanging cause and effect between entropy and temperature in Eqn.~(\ref{GreenElasticODEs}) gives
\small
\begin{subequations}
    \label{HelmholtzElasticODEs}
    \begin{align}
    \mbox{} & \text{In 1D:} &
    \left\{ \begin{matrix} \mathrm{d} \eta \\ 
    \mathrm{d} F \end{matrix} \right\} & = \begin{bmatrix}
    1/\partial_{\eta\eta} U & -\partial_{\eta e} U / 
    \partial_{\eta\eta} U \\
    \rho_{1D} \, \partial_{e\eta} U / \partial_{\eta\eta} U & \rho_{1D} ( \partial_{ee} U - \partial_{e\eta} U \!\cdot\! \partial_{\eta e} U / \partial_{\eta\eta} U ) \end{bmatrix} 
    \left\{ \begin{matrix} \mathrm{d} \theta \\
    \mathrm{d} e \end{matrix} \right\} \\
    % second formula
    \mbox{} & \text{In 2D:} &
    \left\{ \begin{matrix} \mathrm{d} \eta \\ 
    \mathrm{d} \pi \end{matrix} \right\} & = \begin{bmatrix}
    1/\partial_{\eta\eta} U & -\partial_{\eta \xi} U / \partial_{\eta\eta} U \\
    \rho_{2D} \, \partial_{\xi\eta} U / \partial_{\eta\eta} U & \rho_{2D} ( \partial_{\xi\xi} U - \partial_{\xi\eta} U \!\cdot\! \partial_{\eta \xi} U / \partial_{\eta\eta} U ) \end{bmatrix} \left\{ \begin{matrix} \mathrm{d} \theta \\
    \mathrm{d} \xi \end{matrix} \right\} \label{HelmholtzMembrane} \\
    % thrid formula
    \mbox{} & \text{In 3D:} &
    \left\{ \begin{matrix} \mathrm{d} \eta \\ 
    \mathrm{d} \Pi \end{matrix} \right\} & = \begin{bmatrix}
    1/\partial_{\eta\eta} U & -\partial_{\eta \Xi} U / \partial_{\eta\eta} U \\
    \rho_{3D} \, \partial_{\Xi\eta} U / \partial_{\eta\eta} U & \rho_{3D} ( \partial_{\Xi\Xi} U - \partial_{\Xi\eta} U \!\cdot\! \partial_{\eta \Xi} U / \partial_{\eta\eta} U ) \end{bmatrix} \left\{ \begin{matrix} \mathrm{d} \theta \\
    \mathrm{d} \Xi \end{matrix} \right\}
    \end{align}
\end{subequations}
\normalsize
where we employ the independent variables of a Helmholtz free energy, namely temperature and strain, but we do not adopt his potential, preferring to retain the internal energy potential so as to ensure a proper incorporation of Maxwell's constraint. 

Constitutive equations (\ref{GreenElasticODEs} \& \ref{HelmholtzElasticODEs}) take on the form of a hypo-elastic material model \cite{Truesdell55}, which is ideal for numerical implementation whenever one uses solution techniques like those presented in Part~\ref{partNumericalMethods}.  

\subsection{Material Response Functions}

Experiments are performed for the purpose of characterizing material behavior.  In mechanics, we relate measured material constants to gradients and curvatures of thermo\-dynamic potentials out of which material models are created.  Experiments are typically done to quantify the following material properties, selected per a material's physical dimension:
\begin{subequations}
\label{materialConstants}
\begin{align}
C_f & \defeq \theta \, \partial_{\theta} \eta |_F & 
\alpha_f & \defeq L^{-1} \, \partial_{\theta} L |_F \eqdef \alpha \:\: =
\partial_{\theta} e |_F &
E_{\theta} & \defeq L \, \partial_L F |_{\theta} = 
\partial_e F |_{\theta} \\
C_t & \defeq \theta \, \partial_{\theta} \eta |_T & 
\alpha_t & \defeq A^{-1} \, \partial_{\theta} A |_T = 2 \alpha \hspace{1pt} =
2 \, \partial_{\theta} \xi |_T &
M_{\theta} & \defeq A \, \partial_A T |_{\theta} = 
\tfrac{1}{4} \, \partial_{\xi} \pi |_{\theta} \\
C_p & \defeq \theta \, \partial_{\theta} \eta |_P & 
\alpha_p & \defeq V^{-1} \, \partial_{\theta} V |_P = 3 \alpha =
3 \, \partial_{\theta} \Xi |_P &
K_{\theta} & \defeq -V \, \partial_V P |_{\theta} = 
\tfrac{1}{9} \, \partial_{\Xi} \Pi |_{\theta}
\end{align}
\end{subequations}
where we employ the commonly used notation $\partial_{\theta} \eta |_F \defeq ( \partial \eta / \partial \theta ) |_F$, etc.  The various specific heats $C_f$, $C_t$, $C_p$ (erg/g.K) introduced here are, essentially, all equivalent as they are all defined per unit mass, insensitive to dimension.  They are evaluated at a fixed thermo\-dynamic force, which does depend on dimension, but this does not impact the value for specific heat.  Hereafter they will be denoted simply as $C$.  The various thermal expansions $\alpha_f$, $\alpha_t$, $\alpha_p$ (1/K) are, however, all distinct as they are each defined with respect to their physical dimension, viz., $\alpha_f \defeq L^{-1} \, \partial L / \partial \theta |_F$, $\alpha_t \defeq A^{-1} \, \partial A / \partial \theta |_T$ and $\alpha_p \defeq V^{-1} \, \partial V / \partial \theta |_P$.  Nevertheless, they relate to one another because $\ln \sqrt{A / \! A_0} = \ln (L / L_0) \implies \alpha_t = 2 \alpha_f$ and $\ln \sqrt[3]{V \! / V_0} = \ln (L / L_0) \implies \alpha_p = 3 \alpha_f$ for uniform motions.  Hereafter, only the linear coefficient for thermal expansion, denoted as $\alpha \defeq \alpha_f$, will be used.  Parameter $E_{\theta}$ is a modulus of extension (dyne), parameter $M_{\theta}$ is a modulus of dilation (dyne/cm), and parameter $K_{\theta}$ is a modulus of dilatation ($\mathrm{dyne/cm}^2$), a.k.a.\ the bulk modulus, with each modulus being measured at fixed temperature.  Shear moduli are discussed later in Sections~\ref{secNonuniform2D} \& \ref{secNonuniform3D}.  The above material constants are gradients.  They constitute tangents to their associated physical response curves.  Consequently, they need not be of constant value throughout state space, like a Hookean material would suppose them to be.  This is an important characteristic of the hypo-elastic constructions of Eqns.~(\ref{GreenElasticODEs} \& \ref{HelmholtzElasticODEs}), as they pertain to our application. 

In terms of the material properties given in Eqn.~(\ref{materialConstants}), of which there are three per dimension, the internal energy density has three curvatures that associate with it.  For 1D materials:
\begin{subequations}
    \label{internalEnergies}
    \begin{align}
    % for 1D materials
    \partial_{\eta\eta} U & = 
    \frac{\rho_{1D} \, \theta}
    {\rho_{1D} C - \alpha^2 \theta E_{\theta}} \\
    \partial_{ee} U & = \frac{C E_{\theta}}
    {\rho_{1D} C - \alpha^2 \theta E_{\theta}} \\
    \partial_{\eta e} U \equiv \partial_{e \eta} U & = 
    \frac{-\alpha \theta E_{\theta}}
    {\rho_{1D} C - \alpha^2 \theta E_{\theta}} \\
    \intertext{For 2D materials:}
    \partial_{\eta\eta} U & = 
    \frac{\rho_{2D} \, \theta}
    {\rho_{2D} C - 4 \alpha^2 \theta M_{\theta}} \\
    \partial_{\xi\xi} U & = \frac{4 C M_{\theta}}
    {\rho_{2D} C - 4 \alpha^2 \theta M_{\theta}} \\
    \partial_{\eta \xi} U \equiv \partial_{\xi \eta} U & = 
    \frac{-4 \alpha \theta M_{\theta}}
    {\rho_{2D} C - 4 \alpha^2 \theta M_{\theta}} \\
    \intertext{For 3D materials (cf.\ Weinhold \cite{Weinhold75c} and Gilmore \cite{Gilmore84}):}
    \partial_{\eta\eta} U & = 
    \frac{\rho_{3D} \, \theta}
    {\rho_{3D} C - 9 \alpha^2 \theta K_{\theta}} \\
    \partial_{\Xi\Xi} U & = \frac{9 C K_{\theta}}
    {\rho_{3D} C - 9 \alpha^2 \theta K_{\theta}} \\
    \partial_{\eta\Xi} U \equiv 
    \partial_{\Xi\eta} U & = 
    \frac{-9 \alpha \theta K_{\theta}}
    {\rho_{3D} C - 9 \alpha^2 \theta K_{\theta}}
    \end{align}
\end{subequations}
These materials constants are constrained by thermo\-dynamics in that
\begin{equation}
    \label{thermodynamicConstraints}
    0 < E_{\theta} < \frac{\rho_{1D} C}{\alpha^2 \theta} , \quad
    0 < M_{\theta} < \frac{\rho_{2D} C}{4 \alpha^2 \theta} , \quad
    0 < K_{\theta} < \frac{\rho_{3D} C}{9 \alpha^2 \theta} 
\end{equation} 
which ensure that their respective thermo\-dynamic Jacobians cannot become singular. Singularities can and do occur, e.g., during a phase change in a crystal \cite{McLellan76,Gilmore84}, but such processes are not expected to arise in our application.

\subsection{Thermoelastic Models for Modeling Alveoli: Uniform Motions}

We now write down our constitutive formul\ae\ for quantifying uniform responses in thermo\-elastic solids of 1, 2 and 3 dimensions.  They are thermo\-elastic constitutive equations (\ref{HelmholtzElasticODEs}) with Helmholtz variables expressed in terms of the material properties defined in Eqn.~(\ref{materialConstants}) assigned to the internal energy density $U$ according to Eqn.~(\ref{internalEnergies}), with outcomes of:
\begin{subequations}
    \label{HelmholtzCEs}
    \begin{align}\
    \text{For 1D Materials:} & &
    \left\{ \begin{matrix}
    \mathrm{d} \eta \\ \mathrm{d} F
    \end{matrix} \right\} & = \begin{bmatrix}
    C / \theta - \alpha^2 E / \rho& 
    \alpha E / \rho \\
    -\alpha E & E
    \end{bmatrix} \left\{ \begin{matrix}
    \mathrm{d} \theta \\ \mathrm{d} e
    \end{matrix} \right\} \label{Helmholtz1D} \\
    % the second equation
    \text{For 2D Materials:} & &
    \left\{ \begin{matrix}
    \mathrm{d} \eta \\ \mathrm{d} \pi
    \end{matrix} \right\} & = \begin{bmatrix}
    C / \theta - 4 \alpha^2 M / \rho & 
    4 \alpha M / \rho \\
    -4 \alpha M & 4 M
    \end{bmatrix} \left\{ \begin{matrix}
    \mathrm{d} \theta \\ \mathrm{d} \xi
    \end{matrix} \right\} \label{Helmholtz2D} \\
    % the third equation
    \text{For 3D Materials:} & &
    \left\{ \begin{matrix}
    \mathrm{d} \eta \\ \mathrm{d} \Pi
    \end{matrix} \right\} & = \begin{bmatrix}
    C / \theta - 9 \alpha^2 K / \rho & 
    9 \alpha K / \rho \\
    -9 \alpha K & 9 K
    \end{bmatrix} \left\{ \begin{matrix}
    \mathrm{d} \theta \\ \mathrm{d} \Xi
    \end{matrix} \right\} \label{Helmholtz3D}
    \end{align}
\end{subequations}
where we simplify our expressions by suppressing the notation specifying that the moduli are evaluated at constant temperature, and by suppressing the dimension for which mass density applies.  These are considered to be understood.  There are four material constants for each dimension (e.g., for 1D materials they are: $\rho$, $C$, $\alpha$ and $E$) with the latter three constants being defined according to Eqn.~(\ref{materialConstants}).

The upper-left element in each matrix of Eqn.~(\ref{HelmholtzCEs}) represents a specific heat evaluated at constant strain, divided by temperature---a material property not easily measured.  Whereas, the specific heat evaluated at constant pressure (viz., the $C$ found in the 11 matrix component of these tangent moduli) is more amenable to experiments, and is the property that one typically finds in published data tables.  

Constitutive equations (\ref{HelmholtzCEs}), derived from the First and Second Laws of Thermo\-dynamics, describe thermo\-elastic materials undergoing uniform motions through adiabatic processes.  They present themselves as hypo-elastic material models \cite{Truesdell55}, which are typically preferred for incorporating constitutive equations into finite element packages.

Equation (\ref{HelmholtzCEs}) has cause and effect variables that are appropriate for our multi\-scale application.  In this process, a localization procedure pulls the temperature $\theta$ and deformation gradient $\mathbfsf{F}$ taken from the parenchyma scale (e.g., Gauss points in a finite element grid of lung) down to the level of an alveolar scale (in our modeling, a dodecahedron).  Differential strain rates $\mathrm{d} \boldsymbol{\mathcal{U}} \cdot \boldsymbol{\mathcal{U}}^{-1}$ are then constructed through appropriate finite difference formul\ae, where $\boldsymbol{\mathcal{U}}$ denotes the Laplace stretch \cite{Freedetal19}. These continuum rates are then mapped into our local thermo\-dynamic rates, with alveolar entropy and stress following from a numerical integration of the above constitutive equations.  These constitutive equations apply to the various facets of our dodecahedral model for an alveolar sac through a finite element discretization.  Afterwords, an homogenization procedure takes the updated alveolar entropy and nodal tractions, and pushes them up to the continuum level as parenchymal entropy and stress. 

\section{Green Thermoelastic Membranes: Non-Uniform Motions}
\label{secNonuniform2D}

The First and Second Laws of Thermo\-dynamics governing a reversible adiabatic process are described by the formula $\mathrm{d}\hspace{1pt}U = \theta \, \mathrm{d} \eta + \tfrac{1}{\rho} \, \mathrm{d}W$, where $\mathrm{d}W$ is the mechanical power expended by stressing a material element of mass density $\rho$.  For the case of a 2D planar membrane, a mass density of $\rho \Leftarrow \rho_{2D}$ applies, with its change in mechanical work being expressed as \cite{Freedetal17,FreedZamani19,Freedetal20}
\begin{subequations}
\begin{align}
\mathrm{d} W & = \mathrm{tr} \left( 
\begin{bmatrix}
\mathcal{S}_{11} & \mathcal{S}_{12} \\
\mathcal{S}_{21} & \mathcal{S}_{22}
\end{bmatrix} \begin{bmatrix}
a^{-1} \, \mathrm{d}a & (a/b) \, \mathrm{d} g \\
0 & b^{-1} \, \mathrm{d}b
\end{bmatrix} \right) =  
\pi \, \mathrm{d} \xi + \sigma \, \mathrm{d} \varepsilon + 
\tau \, \mathrm{d} \gamma
\label{convectedWorkRate} \\
\intertext{wherein $\mathcal{S}_{ij}$ are the components of a surface tension evaluated in the co-ordinate frame of a membrane.  (In \S\ref{secFE_CE} they will be converted into components of the Eulerian Kirchhoff stress and the Lagrangian, second, Piola-Kirchhoff stress.)  
\medskip\newline
Equation (\ref{convectedWorkRate}) conjectures that the First and Second Laws of Thermo\-dynamics can be expressed as a differential equation known as a Pfaffian form that, in this case, is considered to look like}
\mathrm{d} \hspace{1pt} U & = \theta \, \mathrm{d} \eta + \tfrac{1}{\rho} 
\bigl( \pi \, \mathrm{d} \xi + \sigma \, \mathrm{d} \varepsilon + 
\tau \, \mathrm{d} \gamma \bigr)
\label{membraneThermo}
\end{align}
\end{subequations} 
where $\{ \pi , \sigma , \tau  \}$ describes a set of intensive scalar-valued stresses whose thermo\-dynamic conjugates $\{ \xi , \varepsilon , \gamma \}$ describe a set of extensive scalar-valued strains.  This contrasts with the classic approach, where the work done is decomposed into a scalar-valued isotropic part and a tensor-valued deviatoric part.  The above thermo\-dynamic strains are defined in \S\ref{secQR2D}, while their conjugate stresses, and how they relate to the tensor components of stress, are discussed in \S\ref{secFE_CE}. 

Conjugate pair $( \xi , \pi )$ describes a dilation $2 \, \mathrm{d} \xi \Leftarrow A^{-1} \, \mathrm{d} A$ caused by a surface tension $\pi \Leftarrow 2T$ where $\xi \defeq \ln \sqrt{A / \! A_0}$ and $\pi \defeq \mathcal{S}_{11} + \mathcal{S}_{22}$.  This pair describes the uniform contribution to stress power discussed in \S\ref{secUniformCE}.  Pair $( \varepsilon , \sigma )$ describes a squeeze $\varepsilon$ (or pure shear) caused by a normal-stress difference $\sigma \defeq \mathcal{S}_{11} - \mathcal{S}_{22}$.  And pair $( \gamma , \tau )$ describes an in-plane shear $\gamma$ caused by a shear stress $\tau$. Collectively, pairs $( \varepsilon , \sigma )$ and $( \gamma , \tau )$ account for any non-uniform contributions to stress power, i.e., contributions from other than uniform dilation.  These pairs are quantified in \S\ref{secConjugatePairs}.

\subsection{General Constitutive Equations}

Because a change in the internal energy $\mathrm{d} U$ governing a reversible adiabatic process is described by an exact differential \cite{Caratheodory09}, with $U( \eta, \xi, \varepsilon, \gamma )$ in the case of a planar membrane, it follows that a constitutive response for a Green thermo\-elastic membrane is described by
\begin{equation}
    \begin{aligned}
    \theta & = \partial_{\eta} U(\eta, \xi, \varepsilon, \gamma) &
    \phantom{\rho}
    \pi & = \rho \, \partial_{\xi} U(\eta, \xi, \varepsilon, \gamma)  \\
    \sigma & = \rho \, \partial_{\varepsilon} U(\eta, \xi, \varepsilon, \gamma) &
    \tau & = \rho \, \partial_{\gamma} U(\eta, \xi, \varepsilon, \gamma) .
    \end{aligned}
    \label{GreenThermoelasticMembrane}
\end{equation}
Considering each intensive variable, viz., $\theta$, $\pi$, $\sigma$ and $\tau$, to be at least a $C^1$ function of the set of extensive variables ($\eta , \xi , \varepsilon , \gamma$), thereby implies that the internal energy $U$ is at least a $C^2$ function of state, and therefore the constitutive expressions in Eqn.~(\ref{GreenThermoelasticMembrane}) can be recast into the following system of differential equations
\begin{equation}
\label{energies2D}
\left\{ \begin{matrix}
\mathrm{d} \theta \\ \mathrm{d} \pi \\
\mathrm{d} \sigma \\ \mathrm{d} \tau
\end{matrix} \right\} = \begin{bmatrix}
\partial_{\eta\eta} U & 
\partial_{\eta\xi} U & 
\partial_{\eta\varepsilon} U & 
\partial_{\eta\gamma} U \\ 
\rho \, \partial_{\xi\eta} U & 
\rho \, \partial_{\xi\xi} U & 
\rho \, \partial_{\xi\varepsilon} U &
\rho \, \partial_{\xi\gamma} U \\
\rho \, \partial_{\varepsilon\eta} U & 
\rho \, \partial_{\varepsilon\xi} U & 
\rho \, \partial_{\varepsilon\varepsilon} U & 
\rho \, \partial_{\varepsilon\gamma} U \\
\rho \, \partial_{\gamma\eta} U & 
\rho \, \partial_{\gamma\xi} U & 
\rho \, \partial_{\gamma\varepsilon} U & 
\rho \, \partial_{\gamma\gamma} U 
\end{bmatrix} 
\left\{ \begin{matrix}
\mathrm{d}\eta \\ \mathrm{d} \xi \\
\mathrm{d} \varepsilon \\ \mathrm{d} \gamma
\end{matrix} \right\}  
\end{equation}
whose upper-left $2\times 2$ sub-matrix also appears in Eqn.~(\ref{GreenMembrane}), which governs the uniform contribution of a response.  The above $4 \times 4$ matrix describes the full non-uniform response permissible by a Green thermo\-elastic membrane undergoing an adiabatic process.

For our application, it is reasonable to assume that the presence of a non-uniform planar motion will not cause an uniform planar response.  Said differently, it is reasonable to assume that pure $\varepsilon$ and simple $\gamma$ shears will not effect a change in either temperature $\theta$ or surface tension $\pi$.  As such, $\partial_{\eta\varepsilon} U = \partial_{\eta\gamma} U = \partial_{\xi\varepsilon} U = \partial_{\xi\gamma} U = 0$, and Eqn.~(\ref{energies2D}) simplifies to
\begin{displaymath}
\left\{ \begin{matrix}
\mathrm{d} \theta \\ \mathrm{d} \pi \\
\mathrm{d} \sigma \\ \mathrm{d} \tau
\end{matrix} \right\} = \begin{bmatrix}
\partial_{\eta\eta} U & 
\partial_{\eta\xi} U & 
0 & 0 \\ 
\rho \, \partial_{\xi\eta} U & 
\rho \, \partial_{\xi\xi} U & 
0 & 0 \\
0 & 0 & 
\rho \, \partial_{\varepsilon\varepsilon} U & 
\rho \, \partial_{\varepsilon\gamma} U \\
0 & 0 & 
\rho \, \partial_{\gamma\varepsilon} U & 
\rho \, \partial_{\gamma\gamma} U 
\end{bmatrix} 
\left\{ \begin{matrix}
\mathrm{d}\eta \\ \mathrm{d} \xi \\
\mathrm{d} \varepsilon \\ \mathrm{d} \gamma
\end{matrix} \right\} 
\end{displaymath}
with $\partial_{\varepsilon\eta} U = \partial_{\gamma\eta} U = \partial_{\varepsilon\xi} U = \partial_{\gamma\xi} U = 0$ because of Maxwell's relationships.  Furthermore, it is considered that the pure and simple shear responses act independently, too, so that $\partial_{\gamma\varepsilon} U = \partial_{\varepsilon\gamma} U = 0$.\footnote{
    There is a second-order coupling that can exist between the modes of squeeze and shear in a 3D solid.  It is the Poynting effect \cite{FreedZamani19}, but this effect is thought not arise in a 2D membrane.
}
Converting the above internal energy formulation into its Helmholtz equivalent produces two uncoupled matrix equations; they are,
\begin{subequations}
\label{Helmholtz2Duncoupled}
\begin{align}
\left\{ \begin{matrix}
\mathrm{d} \eta \\ \mathrm{d} \pi 
\end{matrix} \right\} & = \begin{bmatrix}
1 / \partial_{\eta\eta} U & 
-\partial_{\eta\xi} U / \partial_{\eta\eta} U \\ 
\rho \, \partial_{\xi\eta} U / \partial_{\eta\eta} U & 
\rho \bigl( \partial_{\xi\xi} U - \partial_{\xi\eta} U \!\cdot\! \partial_{\eta\xi} U / \partial_{\eta\eta} U \bigr)  
\end{bmatrix} 
\left\{ \begin{matrix}
\mathrm{d}\theta \\ \mathrm{d} \xi 
\end{matrix} \right\}
\label{Helmholtz2Duniform} \\
\intertext{and}
\left\{ \begin{matrix}
    \mathrm{d} \sigma \\ \mathrm{d} \tau
\end{matrix} \right\} & = \rho \begin{bmatrix}
    \partial_{\varepsilon\varepsilon} U & 0 \\
    0 & \partial_{\gamma\gamma} U 
\end{bmatrix} 
\left\{ \begin{matrix}
    \mathrm{d} \varepsilon \\ \mathrm{d} \gamma
\end{matrix} \right\}
\label{Helmholtz2Dnonuniform}
\end{align} 
\end{subequations}
that, collectively, provide a general theoretical structure for a Green thermo\-elastic membrane appropriate for our application.

\medskip\noindent
\textbf{Note}:  The uniform response (Eqn.~\ref{Helmholtz2Duniform}) and the non-uniform response (Eqn.~\ref{Helmholtz2Dnonuniform}) are, by supposition, decoupled in this constitutive construction.  There is experimental evidence that the bulk and shear moduli of parenchyma both depend upon transpulmonary pressure \cite{LaiFook79,Jahedetal90}.  It is conjectured that this is a structural effect of alveolar geometry; it is not a characteristic of the constituents that comprise an alveolus.  As such, we do not couple the uniform and non-uniform responses in the constitutive framework of Eqn.~(\ref{Helmholtz2Duncoupled}) at this time in order that we may test this conjecture.

\subsection{Material Response Functions}
\label{secMaterialConstants}

The material model put forward here for a thermo\-elastic membrane has six material properties\slash functions: a mass density $\rho$, a specific heat $C$ at constant tension, a coefficient for linear thermal expansion $\alpha$ at constant tension, an areal modulus $M$ at constant temperature (2D version of a bulk modulus), a squeeze modulus $N$, and a shear modulus $G$ at constant squeeze.  The specific heat $C$ is defined as
\begin{subequations}
    \label{defineMaterialConstants}
    \begin{align}
    C & \defeq \theta \, \partial_{\theta} \eta |_T = \theta \, \partial_{\theta} \eta |_{\frac{1}{2} ( \mathcal{S}_{11} + \mathcal{S}_{22} )} \! = \theta \, \partial_{\theta} \eta |_{\pi}
    \label{specificHeat2D} \\
    \intertext{where $\theta$ is temperature, $\eta$ is entropy, and $\pi = \mathcal{S}_{11} + \mathcal{S}_{22} = 2 T$ is the surface tension acting in a membrane. $C$ is commonly referred to in the literature as the specific heat at constant pressure.  The coefficient for linear thermal expansion $\alpha$ is}
    \alpha & \defeq L^{-1} \, \partial_{\theta} L |_T = \tfrac{1}{2} A^{-1} \, \partial_{\theta} A |_T  = \partial_{\theta} \xi |_{\pi}
    \label{thermalExpansionCoef2D} \\
    \intertext{where $A = ab$ denotes a relative area with $\xi = \ln \sqrt{A / \! A_0}$ being the areal strain, a.k.a.\ dilation, whose associated areal modulus $M$ is defined as}
    M & \defeq A \, \partial_A T |_{\theta} = A \, \partial_A \tfrac{1}{2} (\mathcal{S}_{11} + \mathcal{S}_{22}) |_{\theta} = \tfrac{1}{4} \, \partial_{\xi} \pi |_{\theta} 
    \label{arealCompliance2D} \\
    \intertext{which is a 2D version of the bulk modulus.  The in-plane squeeze modulus $N$ is defined as}
    N & \defeq \Gamma \, \partial_{\Gamma} N_1 = \Gamma \, \partial_{\Gamma} (\mathcal{S}_{11} - \mathcal{S}_{22}) = \tfrac{1}{2} \, \partial_{\varepsilon} \sigma 
    \label{arealSqueezeCompliance2D} \\
    \intertext{where $\sigma = \mathcal{S}_{11} - \mathcal{S}_{22}$ is the normal-stress difference, often denoted as $N_1$ in the literature, and where $\Gamma = a/b$ is the stretch of squeeze with $\varepsilon = \ln \sqrt{\Gamma \! / \Gamma_0}$ being the strain of squeeze.  Finally, an in-plane shear modulus $G$ is defined as}
    G & \defeq \Gamma \, \partial_{g\,} \mathcal{S}_{21} |_{\Gamma} = \partial_{\gamma} \tau |_{\Gamma} = \partial_{\gamma} \tau |_{\varepsilon} 
    \label{arealShearModulus2D}
    \end{align}
\end{subequations}
where $\tau = \Gamma \mathcal{S}_{21}$ determines the shear stress, with $\gamma = g - g_0$ establishing the shear strain. 

\subsection{Constitutive Equations Governing a Thermoelastic Membrane}

It is the Gibbs free-energy potential (viz., $\mathcal{G} ( \theta , \pi , \sigma , \tau ) = U - \theta \eta - \pi \xi - \sigma \varepsilon - \tau \gamma$, which exchanges cause and effect from that of the internal energy $U ( \eta , \xi , \varepsilon , \gamma )$), that is most easily expressed in terms of the above material properties, cf.\ Appendix~\ref{appImplicitElasticity}; specifically,
\begin{displaymath}
\left\{ \begin{matrix}
\mathrm{d}\eta \\ \mathrm{d} \xi \\
\mathrm{d} \varepsilon \\ \mathrm{d} \gamma
\end{matrix} \right\} = -\begin{bmatrix}
\partial_{\theta\theta} \mathcal{G} & 
\partial_{\theta\pi} \mathcal{G} & 0 & 0 \\ 
\rho \, \partial_{\pi\theta} \mathcal{G} & 
\rho \, \partial_{\pi\pi} \mathcal{G} & 0 & 0 \\
0 & 0 & \rho \, \partial_{\sigma\sigma} \mathcal{G} & 0 \\
0 & 0 & 0 & \rho \, \partial_{\tau\tau} \mathcal{G}
\end{bmatrix} 
\left\{ \begin{matrix}
\mathrm{d} \theta \\ \mathrm{d} \pi \\
\mathrm{d} \sigma \\ \mathrm{d} \tau
\end{matrix} \right\} 
\end{displaymath}
where $\partial_{\theta\pi} \mathcal{G} = \partial_{\pi\theta} \mathcal{G}$ from Maxwell's constraint.  Incorporating the material property definitions put forward in Eqn.~(\ref{defineMaterialConstants}) into the above differential equation leads to
\begin{displaymath}
\label{GibbsMembrane}
\left\{ \begin{matrix}
\mathrm{d}\eta \\ \mathrm{d} \xi \\
\mathrm{d} \varepsilon \\ \mathrm{d} \gamma
\end{matrix} \right\} = \begin{bmatrix}
C / \theta & \alpha / \rho & 0 & 0 \\ 
\alpha & 1 / 4 M & 0 & 0 \\
0 & 0 & 1 / 2 N & 0 \\
0 & 0 & 0 & 1 / G
\end{bmatrix} 
\left\{ \begin{matrix}
\mathrm{d} \theta \\ \mathrm{d} \pi \\
\mathrm{d} \sigma \\ \mathrm{d} \tau
\end{matrix} \right\} 
\end{displaymath}
where gradients $\partial \eta / \partial \theta$, $\partial \xi / \partial \theta$ and $\partial \pi / \partial \xi$ relate to the material properties through $\partial_{\theta\theta} \mathcal{G} = \partial \eta / \partial \theta$, $\rho \, \partial_{\pi\theta} \mathcal{G} = \partial \xi / \partial \theta = \rho \, \partial_{\theta\pi} \mathcal{G}$ and $\rho \, \partial_{\pi\pi} \mathcal{G} = \partial \xi / \partial \pi = ( \partial \pi / \partial \xi )^{-1}$.   The upper-left $2 \! \times \! 2$ sub-matrix, which describes the uniform contribution to a response, can be rearranged to read as
\begin{subequations}
    \label{HelmholtzMembraneODEs}
    \begin{align}
    \left\{ \begin{matrix}
    \mathrm{d} \eta \\ \mathrm{d} \pi
    \end{matrix} \right\} & = \begin{bmatrix}
    C / \theta - 4 \alpha^2 M / \rho & 
    4 \alpha M / \rho \\
    -4 \alpha M & 4 M
    \end{bmatrix} \left\{ \begin{matrix}
    \mathrm{d} \theta \\ \mathrm{d} \xi
    \end{matrix} \right\} , & & M = M ( \theta , \xi , \pi )
    \label{HelmholtzMembraneODEsUniform}
    \intertext{while the non-uniform or shear response of Eqn.~(\ref{Helmholtz2Dnonuniform}) is given quite simply by}
    \left\{ \begin{matrix}
    \mathrm{d} \sigma \\ \mathrm{d} \tau
    \end{matrix} \right\} & = \begin{bmatrix}
    2 N & 0 \\
    0 & G
    \end{bmatrix} \left\{ \begin{matrix}
    \mathrm{d} \varepsilon \\ \mathrm{d} \gamma
    \end{matrix} \right\} , & & \begin{aligned}
    N & = N( \varepsilon , \sigma ) \\
    G & = G ( \gamma , \tau )
    \end{aligned}
    \label{HelmholtzMembraneODEsNonUniform}
    \end{align}
\end{subequations}
that, collectively, are used to describe a thermo\-elastic membrane.  These moduli can depend upon both stress and strain, in accordance with the implicit theory of elasticity presented in Appendix~\ref{appImplicitElasticity}.

\subsubsection{The Poisson Effect}
\label{PoissonRatio}

The areal modulus $M$ is ideally determined from an equibiaxial experiment.  Assuming knowledge of its value, then given the following definition for Poisson's ratio
\begin{displaymath}
\nu \defeq - \frac{\mathrm{d}b / b}{\mathrm{d}a / a}
\end{displaymath}
it immediately follows that the squeeze modulus $N$ can be determined from an uniaxial experiment where traction is applied along that axis from which elongation $a$ is measured; specifically,
\begin{displaymath}
N = 2M \, \frac{1 - \nu}{1 + \nu} 
\quad \text{given that} \quad
\mathcal{S}_{11} \neq 0 
\quad \text{and} \quad
\mathcal{S}_{21} = \mathcal{S}_{22} = 0 
\end{displaymath}
provided that the temperature $\theta$ is held constant.  Consequently, $\tfrac{2}{3} M \leq N \leq 2M$ follows provided that $0 \leq \nu \leq \textfrac{1}{2}$, so the squeeze modulus $N$ plays an analogous role to the shear modulus $\mu$ found in the classical theory of elasticity.  

If one were to consider such a membrane as having uniform thickness $h$ that varies with deformation so as to preserve its volume, then $\nu = \textfrac{1}{2}$ and Eqn.~(\ref{HelmholtzMembraneODEsNonUniform}) would become
\begin{equation}
\left\{ \begin{matrix}
\mathrm{d} \sigma \\ \mathrm{d} \tau
\end{matrix} \right\} = \begin{bmatrix}
4 M / 3 & 0 \\
0 & G
\end{bmatrix} \left\{ \begin{matrix}
\mathrm{d} \varepsilon \\ \mathrm{d} \gamma
\end{matrix} \right\}
\label{HelmholtzMembraneODEsNonUniformCV}
    \tag{\ref{HelmholtzMembraneODEs}c}
\end{equation}
which is a useful result.

\medskip\noindent
\textbf{Note}: 
The conjugate pair approach presented here allows for a distinct shear modulus $G$ that can take on any positive value.  This is important because shear experiments done on soft tissues, which unfortunately are few in number, tend to produce shear moduli that are many orders in magnitude smaller than their bulk moduli, e.g., in parenchyma their ratio is $K/G \approx 10^{4}$ (150~MPa vs.\ 10--54~kPa).  \cite{Sarafetal07}  Classically, such a result has been used to argue that a material can be modeled, to a reasonable approximation, as being ideally incompressible---a 3D notion.  Such an assumption is no longer necessary in our formulation.

\section{Green Thermoelastic Solids: Non-Uniform Motions}
\label{secNonuniform3D}

The First and Second Laws of Thermo\-dynamics governing a reversible adiabatic process done on a 3D body result in the formula $\mathrm{d}\hspace{1pt}U = \theta \, \mathrm{d} \eta + \tfrac{1}{\rho} \, \mathrm{d}W$, where $\mathrm{d}W$ is the mechanical power expended by stressing a body with mass density $\rho$, specifically \cite{Freedetal17,FreedZamani19,Freedetal20}
\begin{subequations}
    \begin{align}
    \mathrm{d} W & = \mathrm{tr} \left( 
    \begin{bmatrix}
    \mathcal{S}_{11} & \mathcal{S}_{12} & \mathcal{S}_{13} \\
    \mathcal{S}_{21} & \mathcal{S}_{22} & \mathcal{S}_{23} \\
    \mathcal{S}_{31} & \mathcal{S}_{32} & \mathcal{S}_{33}
    \end{bmatrix} \begin{bmatrix}
    a^{-1} \, \mathrm{d}a & (a/b) \, \mathrm{d} \gamma & 
       (a/c) ( \mathrm{d} \beta - \alpha \, \mathrm{d} \gamma ) \\
    0 & b^{-1} \, \mathrm{d}b & (b/c) \, \mathrm{d} \alpha \\
    0 & 0 & c^{-1} \, \mathrm{d} c
    \end{bmatrix} \right) \notag \\ 
    & =  \Pi \, \mathrm{d} \Xi + \sum_{i=1}^3 \left( 
    \sigma_i \, \mathrm{d} \varepsilon_i + \tau_i \, \mathrm{d} \gamma_i \right)
    \label{convectedWorkRate3D} \\
    \intertext{which is subject to constraints $\sigma_3 = -(\sigma_1 + \sigma_2)$ and $\mathrm{d} \varepsilon_3 = -(\mathrm{d} \varepsilon_1 + \mathrm{d}\varepsilon_2)$.  Stress components $\mathcal{S}_{ij}$ can be either rotated into the Kirchhoff stress of an Eulerian frame, or they can be pulled back into the second Piola-Kirchhoff stress of a Lagrangian frame, as established in \S\ref{secFE_CE}.  
    \medskip\newline
    The above expression conjectures that the thermo\-dynamics of an elastic solid within an adiabatic enclosure can be described by a Pfaffian equation of the form}
    \mathrm{d} \hspace{1pt} U & = \theta \, \mathrm{d} \eta + \frac{1}{\rho} 
    \left( \Pi \, \mathrm{d} \Xi + \sum_{i=1}^2 \sigma_i \, \mathrm{d} \varepsilon_i + ( \sigma_1 + \sigma_2 ) ( \mathrm{d} \varepsilon_1 + 
    \mathrm{d} \varepsilon_2 ) + \sum_{i=1}^3 \tau_i \, \mathrm{d} \gamma_i \right)
    \label{solidThermo}
    \end{align}
\end{subequations} 
where stresses $\{ \Pi , \sigma_1 , \sigma_2 , \tau_1 , \tau_2 , \tau_3  \}$ describe a set of independent, scalar-valued, intensive variables, and where strains $\{ \Xi , \varepsilon_1 , \varepsilon_2 , \gamma_1 , \gamma_2 , \gamma_3 \}$ describe a set of independent, scalar-valued, extensive variables.  This contrasts with the classic approach where the work done decomposes into a scalar-valued isotropic part and a tensor-valued deviatoric part.  A direct consequence of adopting a triangular construction for strain rate is that the pure- and simple-shear contributions of a deviatoric response can be separated into independent scalar contributions; whereas, they remain coupled as a tensor contribution whenever a symmetric construction for strain rate is adopted, which is standard practice today.  The above thermo\-dynamic strains are defined in \S\ref{secQR3D}, while their conjugate stresses and how they relate to commonly used stress tensors are discussed later in \S\ref{secFE_CE}. 

\subsection{Constitutive Equations}

Because a change in the internal energy $\mathrm{d} U$ governing a reversible adiabatic process is described by an exact differential \cite{Caratheodory09}, with $U( \eta, \Xi, \varepsilon_1 , \varepsilon_2 , \gamma_1 , \gamma_2 , \gamma_3 )$ in three space, it necessarily follows that a constitutive response for a Green thermo\-elastic solid is governed by a constitutive equation for temperature \cite{Freed17}
\begin{subequations}
    \label{GreenThermoelasticSolid}
\begin{align}
\theta & = \partial_{\eta} U( \eta, \Xi, \varepsilon_1 , \varepsilon_2 , \gamma_1 , \gamma_2 , \gamma_3 ) \\
\intertext{a constitutive equation for pressure}
\Pi & = \rho \, \partial_{\Xi} U( \eta, \Xi, \varepsilon_1 , \varepsilon_2 , \gamma_1 , \gamma_2 , \gamma_3 )  \\
\intertext{constitutive equations for the normal-stress differences}
\left\{ \begin{matrix}
\sigma_1 \\ \sigma_2
\end{matrix} \right\} & = \frac{1}{3} \begin{bmatrix}
2 & -1 \\ -1 & 2
\end{bmatrix} \left\{ \begin{matrix}
\rho \, \partial_{\varepsilon_1} U( \eta, \Xi, \varepsilon_1 , \varepsilon_2 , \gamma_1 , \gamma_2 , \gamma_3 ) \\
\rho \, \partial_{\varepsilon_2} U( \eta, \Xi, \varepsilon_1 , \varepsilon_2 , \gamma_1 , \gamma_2 , \gamma_3 )
\end{matrix} \right\} \label{GreenThermoelasticSqueeze} \\
\intertext{and constitutive equations for the shear stresses}
\tau_1 & = \rho \, \partial_{\gamma_1} U( \eta, \Xi, \varepsilon_1 , \varepsilon_2 , \gamma_1 , \gamma_2 , \gamma_3 ) \\
\tau_2 & = \rho \, \partial_{\gamma_2} U( \eta, \Xi, \varepsilon_1 , \varepsilon_2 , \gamma_1 , \gamma_2 , \gamma_3 ) \\
\tau_3 & = \rho \, \partial_{\gamma_3} U( \eta, \Xi, \varepsilon_1 , \varepsilon_2 , \gamma_1 , \gamma_2 , \gamma_3 ) 
\end{align}
\end{subequations}
where the coupled expressions for the two squeeze stresses in Eqn.~(\ref{GreenThermoelasticSqueeze}) arise from the contribution
\begin{displaymath}
    \partial_{\varepsilon_1} U \, \mathrm{d} \varepsilon_1 +
    \partial_{\varepsilon_2} U \, \mathrm{d} \varepsilon_2 = 
    \sigma_1 \, \mathrm{d} \varepsilon_1 +
    \sigma_2 \, \mathrm{d} \varepsilon_2 + 
    (\sigma_1 + \sigma_2) (\mathrm{d} \varepsilon_1 + \mathrm{d} \varepsilon_2)
\end{displaymath}
that incorporates constraints $\sigma_3 = -(\sigma_1 + \sigma_2)$ and $\mathrm{d} \varepsilon_3 = -( \mathrm{d} \varepsilon_1 + \mathrm{d} \varepsilon_2 )$ into the work done, viz., $\sigma_3 \, \mathrm{d} \varepsilon_3$ does work, and as such, conjugate pair $( \sigma_3 , \varepsilon_3 )$ has an influence on constitutive response even though they are not independent variables.

Considering each independent intensive variable, viz., $\theta$, $\Pi$, $\sigma_1$, $\sigma_2$, $\tau_1$, $\tau_2$, $\tau_3$, to be a $C^1$ function of the set of independent extensive variables $\eta , \Xi , \varepsilon_1 , \varepsilon_2 , \gamma_1 , \gamma_2 , \gamma_3 $, then the internal energy $U$ must be a $C^2$ function of state and the constitutive expressions in Eqn.~(\ref{GreenThermoelasticSolid}) can be recast into the following system of differential equations
\footnotesize
\begin{equation}
\left\{ \begin{matrix}
\mathrm{d} \theta \\ \mathrm{d} \Pi \\
\mathrm{d} \sigma_1 \\ \mathrm{d} \sigma_2 \\ 
\mathrm{d} \tau_1 \\ \mathrm{d} \tau_2 \\ \mathrm{d} \tau_3
\end{matrix} \right\} = \begin{bmatrix}
\partial_{\eta\eta} U & 
\partial_{\eta\Xi} U & 
\partial_{\eta\varepsilon_1} U & 
\partial_{\eta\varepsilon_2} U &
\partial_{\eta\gamma_1} U &
\partial_{\eta\gamma_2} U &
\partial_{\eta\gamma_3} U \\ 
\rho \, \partial_{\Xi\eta} U & 
\rho \, \partial_{\Xi\Xi} U & 
\rho \, \partial_{\Xi\varepsilon1} U &
\rho \, \partial_{\Xi\varepsilon2} U &
\rho \, \partial_{\Xi\gamma_1} U &
\rho \, \partial_{\Xi\gamma_2} U &
\rho \, \partial_{\Xi\gamma_3} U \\
\rho \, M_{1\eta} & 
\rho \, M_{1\Xi} & 
\rho \, M_{1\varepsilon_1} & 
\rho \, M_{1\varepsilon_2} &
\rho \, M_{1\gamma_1} &
\rho \, M_{1\gamma_2} &
\rho \, M_{1\gamma_3} \\
\rho \, M_{2\eta} & 
\rho \, M_{2\Xi} & 
\rho \, M_{2\varepsilon_1} & 
\rho \, M_{2\varepsilon_2} &
\rho \, M_{2\gamma_1} &
\rho \, M_{2\gamma_2} &
\rho \, M_{2\gamma_3} \\
\rho \, \partial_{\gamma_1\eta} U & 
\rho \, \partial_{\gamma_1\Xi} U & 
\rho \, \partial_{\gamma_1\varepsilon_1} U & 
\rho \, \partial_{\gamma_1\varepsilon_2} U &
\rho \, \partial_{\gamma_1\gamma_1} U  &
\rho \, \partial_{\gamma_1\gamma_2} U &
\rho \, \partial_{\gamma_1\gamma_3} U \\
\rho \, \partial_{\gamma_2\eta} U & 
\rho \, \partial_{\gamma_2\Xi} U & 
\rho \, \partial_{\gamma_2\varepsilon_1} U & 
\rho \, \partial_{\gamma_2\varepsilon_2} U &
\rho \, \partial_{\gamma_2\gamma_1} U  &
\rho \, \partial_{\gamma_2\gamma_2} U &
\rho \, \partial_{\gamma_2\gamma_3} U \\
\rho \, \partial_{\gamma_3\eta} U & 
\rho \, \partial_{\gamma_3\Xi} U & 
\rho \, \partial_{\gamma_3\varepsilon_1} U & 
\rho \, \partial_{\gamma_3\varepsilon_2} U &
\rho \, \partial_{\gamma_3\gamma_1} U  &
\rho \, \partial_{\gamma_3\gamma_2} U &
\rho \, \partial_{\gamma_3\gamma_3} U
\end{bmatrix}
\left\{ \begin{matrix}
\mathrm{d}\eta \\ \mathrm{d} \Xi \\
\mathrm{d} \varepsilon_1 \\ \mathrm{d} \varepsilon_2 \\
\mathrm{d} \gamma_1 \\ \mathrm{d} \gamma_2 \\ \mathrm{d} \gamma_3
\end{matrix} \right\}
\label{energies3D}
\end{equation}
\normalsize
whose upper-left $2\times 2$ sub-matrix also appears in Eqn.~(\ref{GreenSolid}), which governs the uniform contribution of a response.  The squeeze response of Eqn.~(\ref{GreenThermoelasticSqueeze}) associates with tangent moduli that are defined accordingly
\begin{subequations}  
    \label{shearEnergies}
    \begin{align}
    M_{1\eta} & = \tfrac{1}{3} \bigl( 2 \partial_{\varepsilon_1 \eta} U -
        \partial_{\varepsilon_2 \eta} U \bigr) &
    M_{2\eta} & = \tfrac{1}{3} \bigl( 2 \partial_{\varepsilon_2 \eta} U -
    \partial_{\varepsilon_1 \eta} U \bigr) \\
    M_{1\Xi} & = \tfrac{1}{3} \bigl( 2 \partial_{\varepsilon_1 \Xi} U -
    \partial_{\varepsilon_2 \Xi} U \bigr) &
    M_{2\Xi} & = \tfrac{1}{3} \bigl( 2 \partial_{\varepsilon_2 \Xi} U -
    \partial_{\varepsilon_1 \Xi} U \bigr) \\
    M_{1\varepsilon_1} & = \tfrac{1}{3} \bigl( 2 \partial_{\varepsilon_1 \varepsilon_1} U -
    \partial_{\varepsilon_2 \varepsilon_1} U \bigr) & 
    M_{2\varepsilon_1} & = \tfrac{1}{3} \bigl( 2 \partial_{\varepsilon_2 \varepsilon_1} U -
    \partial_{\varepsilon_1 \varepsilon_1} U \bigr) \\
    M_{1\varepsilon_2} & = \tfrac{1}{3} \bigl( 2 \partial_{\varepsilon_1 \varepsilon_2} U -
    \partial_{\varepsilon_2 \varepsilon_2} U \bigr) & 
    M_{2\varepsilon_2} & = \tfrac{1}{3} \bigl( 2 \partial_{\varepsilon_2 \varepsilon_2} U -
    \partial_{\varepsilon_1 \varepsilon_2} U \bigr) \\
    M_{1\gamma_1} & = \tfrac{1}{3} \bigl( 2 \partial_{\varepsilon_1 \gamma_1} U -
    \partial_{\varepsilon_2 \gamma_1} U \bigr) &
    M_{2\gamma_1} & = \tfrac{1}{3} \bigl( 2 \partial_{\varepsilon_2 \gamma_1} U -
    \partial_{\varepsilon_1 \gamma_1} U \bigr) \\
    M_{1\gamma_2} & = \tfrac{1}{3} \bigl( 2 \partial_{\varepsilon_1 \gamma_2} U -
    \partial_{\varepsilon_2 \gamma_2} U \bigr) &
    M_{2\gamma_2} & = \tfrac{1}{3} \bigl( 2 \partial_{\varepsilon_2 \gamma_2} U -
    \partial_{\varepsilon_1 \gamma_2} U \bigr) \\
    M_{1\gamma_3} & = \tfrac{1}{3} \bigl( 2 \partial_{\varepsilon_1 \gamma_3} U -
    \partial_{\varepsilon_2 \gamma_3} U \bigr) &
    M_{2\gamma_3} & = \tfrac{1}{3} \bigl( 2 \partial_{\varepsilon_2 \gamma_3} U -
    \partial_{\varepsilon_1 \gamma_3} U \bigr)
    \end{align}
\end{subequations}
so that, collectively, Eqns.~(\ref{energies3D} \& \ref{shearEnergies}) describe the full non-uniform response permissible by a Green thermo\-elastic solid expressed as a hypo-elastic material undergoing an adiabatic process.

As in the case of membranes, it is reasonable to assume that the presence of a non-uniform motion will not cause an uniform response.  For our application, it is also reasonable to assume that there is no coupling between the modes of squeeze and shear.\footnote{
   The Poynting effect is a second-order effect that couples squeeze and shear \cite{FreedZamani19}.  It is assumed that such a coupling does not play a contributing role in the current application, and can therefore be neglected.
}
Furthermore, it is assumed that there is no coupling betwixt the two independent squeeze modes, nor between the three independent shear modes.  Consequently, all mixed partial derivatives that associate with a non-uniform response are taken to be zero, and therefore Eqns.~(\ref{energies3D} \& \ref{shearEnergies}) simplify to
\footnotesize
\begin{displaymath}
\left\{ \begin{matrix}
\mathrm{d} \theta \\ \mathrm{d} \Pi \\
\mathrm{d} \sigma_1 \\ \mathrm{d} \sigma_2 \\ 
\mathrm{d} \tau_1 \\ \mathrm{d} \tau_2 \\ \mathrm{d} \tau_3
\end{matrix} \right\} = \begin{bmatrix}
\partial_{\eta\eta} U & \partial_{\eta\Xi} U & 0 & 0 & 0 & 0 & 0 \\
\rho \, \partial_{\Xi\eta} U & \rho \, \partial_{\Xi\Xi} U & 0 & 0 & 0 & 0 & 0 \\
0 & 0 & \rho \tfrac{2}{3} \partial_{\varepsilon_1 \varepsilon_1} U & -\rho \tfrac{1}{3} \partial_{\varepsilon_2 \varepsilon_2} U & 0 & 0 & 0 \\
0 & 0 & -\rho \tfrac{1}{3} \partial_{\varepsilon_1 \varepsilon_1} U & \rho \tfrac{2}{3} \partial_{\varepsilon_2 \varepsilon_2} U & 0 & 0 & 0 \\
0 & 0 & 0 & 0 & \rho \, \partial_{\gamma_1\gamma_1} U & 0 & 0 \\
0 & 0 & 0 & 0 & 0 & \rho \, \partial_{\gamma_2\gamma_2} U & 0 \\
0 & 0 & 0 & 0 & 0 & 0 & \rho \, \partial_{\gamma_3\gamma_3} U
\end{bmatrix}
\left\{ \begin{matrix}
\mathrm{d}\eta \\ \mathrm{d} \Xi \\
\mathrm{d} \varepsilon_1 \\ \mathrm{d} \varepsilon_2 \\
\mathrm{d} \gamma_1 \\ \mathrm{d} \gamma_2 \\ \mathrm{d} \gamma_3
\end{matrix} \right\}
\end{displaymath}
\normalsize
where what may appear as being a coupling between $\mathrm{d} \sigma_1$ and $\mathrm{d} \sigma_2$ is actually a consequence arising from the two constraint equations $\mathrm{d} \sigma_3 = -( \mathrm{d} \sigma_1 + \mathrm{d} \sigma_2 )$ and $\mathrm{d} \varepsilon_3 = -( \mathrm{d} \varepsilon_1 + \mathrm{d} \varepsilon_2 )$.

The above system of equations can be rewritten as three independent systems of differential equations; specifically, the first differential matrix equation is
\begin{subequations}
    \label{hypoelastic3D}
    \begin{align}
    \left\{ \begin{matrix}
    \mathrm{d} \theta \\ \mathrm{d} \Pi
    \end{matrix} \right\} & = \begin{bmatrix}
    \partial_{\eta\eta} U & \partial_{\eta\Xi} U \\
    \rho \, \partial_{\Xi\eta} U & \rho \, \partial_{\Xi\Xi} U  
    \end{bmatrix}
    \left\{ \begin{matrix}
    \mathrm{d}\eta \\ \mathrm{d} \Xi 
    \end{matrix} \right\} \notag \\
    \intertext{that when rewritten in terms of Helmholz state variables becomes}
    \left\{ \begin{matrix}
    \mathrm{d} \eta \\ \mathrm{d} \Pi 
    \end{matrix} \right\} & = \begin{bmatrix}
    1 / \partial_{\eta\eta} U & -\partial_{\eta\Xi} U / \partial_{\eta\eta} U \\ 
    \rho \, \partial_{\Xi\eta} U / \partial_{\eta\eta} U & 
    \rho \bigl( \partial_{\Xi\Xi} U - \partial_{\Xi\eta} U \!\cdot\! \partial_{\eta\Xi} U / \partial_{\eta\eta} U \bigr)
    \end{bmatrix}
    \left\{ \begin{matrix}
    \mathrm{d} \theta \\ \mathrm{d} \Xi
    \end{matrix} \right\}
    \label{Helmholtz3Disotropic} \\
    \intertext{plus a full matrix equation that governs the squeeze response}
    \left\{ \begin{matrix}
    \mathrm{d} \sigma_1 \\ \mathrm{d} \sigma_2 
    \end{matrix} \right\} & = \frac{\rho}{3} \begin{bmatrix}
    2 \, \partial_{\varepsilon_1 \varepsilon_1} U & - \partial_{\varepsilon_2 \varepsilon_2} U \\
    - \partial_{\varepsilon_1 \varepsilon_1} U & 2 \, \partial_{\varepsilon_2 \varepsilon_2} U 
    \end{bmatrix}
    \left\{ \begin{matrix}
    \mathrm{d} \varepsilon_1 \\ \mathrm{d} \varepsilon_2 
    \end{matrix} \right\} \\
    \intertext{and a diagonal matrix equation that governs the shear response}
    \left\{ \begin{matrix} 
    \mathrm{d} \tau_1 \\ \mathrm{d} \tau_2 \\ \mathrm{d} \tau_3
    \end{matrix} \right\} & = \rho \begin{bmatrix}
    \partial_{\gamma_1\gamma_1} U & 0 & 0 \\
    0 & \partial_{\gamma_2\gamma_2} U & 0 \\
    0 & 0 & \partial_{\gamma_3\gamma_3} U
    \end{bmatrix}
    \left\{ \begin{matrix}
    \mathrm{d} \gamma_1 \\ \mathrm{d} \gamma_2 \\ \mathrm{d} \gamma_3
    \end{matrix} \right\}
    \end{align}
\end{subequations}
for which we now seek a description in terms of specified material properties.

\subsection{Material Properties}
\label{secMaterialConstants3D}

The material model put forward here is for a general thermo\-elastic solid that has, at most, nine material constants\slash functions: a mass density $\rho$, a specific heat $C$ and a coefficient for linear thermal expansion $\alpha$, both evaluated at constant pressure, a bulk modulus $K$ evaluated at constant temperature, two squeeze moduli $N_1$ and $N_2$, and three shear moduli $G_1$, $G_2$ and $G_3$ evaluated at constant squeeze.  The specific heat $C$ is defined as
\begin{subequations}
    \label{defineMaterialConstants3D}
    \begin{align}
    C & \defeq \theta \, \partial_{\theta} \eta |_P \! = \theta \, \partial_{\theta} \eta |_{- \frac{1}{3} ( \mathcal{S}_{11} + \mathcal{S}_{22} + \mathcal{S}_{33} )} \! = \theta \, \partial_{\theta} \eta |_{\Pi}
    \label{specificHeat3D} \\
    \intertext{where $\theta$ is temperature, $\eta$ is entropy, and $\Pi =  \mathcal{S}_{11} + \mathcal{S}_{22} + \mathcal{S}_{33} = -3P$ is negative pressure. (Transpulmonary pressure is negative---mean stress is positive---under normal physiologic conditions.)  The coefficient for thermal expansion $\alpha$ is defined as}
    \alpha & \defeq L^{-1} \, \partial_{\theta} L |_P = \tfrac{1}{3} V^{-1} \, \partial_{\theta} V |_P \! = \partial_{\theta} \Xi |_{\Pi}
    \label{thermalExpansionCoef3D} \\
    \intertext{where $V = abc$ denotes a relative volume with $\Xi = \ln \sqrt[3]{V \! / V_0}$ being volumetric strain, a.k.a.\ dilatation, whose associated bulk modulus $K$ is defined as}
    K & \defeq -V \, \partial_V P |_{\theta} = V \, \partial_V \tfrac{1}{3} (\mathcal{S}_{11} + \mathcal{S}_{22} + \mathcal{S}_{33}) |_{\theta} = \tfrac{1}{9} \, \partial_{\Xi} \Pi |_{\theta} 
    \label{bulkCompliance3D} \\
    \intertext{that together describe the uniform response.  The non-uniform response is described in terms of two in-plane squeeze moduli $N_1$ and $N_2$ that are defined as}
    N_1 & \defeq \Gamma_1 \, \partial_{\Gamma_1} (\mathcal{S}_{11} - \mathcal{S}_{22}) |_{\Gamma_2} = \tfrac{1}{3} \, \partial_{\varepsilon_1} \sigma_1 |_{\Gamma_2}
    \label{squeezeCompliance3D1} \\
    N_2 & \defeq \Gamma_2 \, \partial_{\Gamma_2} (\mathcal{S}_{22} - \mathcal{S}_{33}) |_{\Gamma_1} = \tfrac{1}{3} \, \partial_{\varepsilon_2} \sigma_2 |_{\Gamma_1}
    \label{squeezeCompliance3D2} \\
    \intertext{where $\sigma_1 = \mathcal{S}_{11} - \mathcal{S}_{22}$ and $\sigma_2 = \mathcal{S}_{22} - \mathcal{S}_{33}$ are the first and second normal-stress differences, with $\Gamma_1 = a/b$ and $\Gamma_2 = b/c$ being their respective stretches of squeeze with $\varepsilon_1 = \ln \sqrt[3]{\Gamma_1 \! / \Gamma_{1\,0}}$ and $\varepsilon_2 = \ln \sqrt[3]{\Gamma_2 / \Gamma_{2\,0}}$ being their strains of squeeze.    Finally, there are three in-plane shear moduli $G_{\varepsilon_1}$, $G_{\varepsilon_2}$ and $G_{\varepsilon_3}$ that are defined as}
    G_1 & \defeq \Gamma_2 \, \partial_{\gamma_1} \mathcal{S}_{32} |_{\Gamma_2} \\ 
    G_2 & \defeq \Gamma_1 \Gamma_2 \, \partial_{\gamma_2} \mathcal{S}_{31} |_{\Gamma_1 \Gamma_2} 
    \label{shearCompliance3D} \\
    G_3 & \defeq \Gamma_1 \, \partial_{\gamma_3} \mathcal{S}_{21} |_{\Gamma_1 , \gamma_1 , \tau_2} 
    \end{align}
\end{subequations}
where $\tau_1 = \Gamma_2 \mathcal{S}_{32}$, $\tau_2 = \Gamma_1 \Gamma_2 \mathcal{S}_{31}$ and $\tau_3 = \Gamma_1 \mathcal{S}_{21} - \alpha \tau_2$ quantify the three shear stresses, with $\gamma_1 = \alpha - \alpha_0$, $\gamma_2 = \beta - \beta_0$ and $\gamma_3 = \gamma - \gamma_0$ being their respective shear strains.

A material is said to be `isotropic' in our constitutive framework, like the materials considered in our application, if its squeeze moduli can be described via a single material function, i.e., $N_1 = N (\sigma_1 , \varepsilon_1 )$ and $N_2 = N ( \sigma_2 , \varepsilon_2 )$, and if its shear moduli can be described via a single material function, viz., $G_1 = G ( \tau_1 , \gamma_1 )$, $G_2 = G ( \tau_2 , \gamma_2 )$ and $G_3 = G ( \tau_3 , \gamma_3 )$.  In other words, the two squeeze response curves may have different tangents at any given moment, but these tangents are evaluated from the same material function for squeeze.  A like statement applies to shear.  An isotropic thermo\-elastic solid therefore has six material constants\slash functions: $\rho$, $C$, $\alpha$, $K$, $N$ and $G$.  This notion of isotropy is different from that of classical theory.

\subsection{Constitutive Equations Governing a Thermoelastic Solid}

In terms of the material properties put forward in Eqn.~(\ref{defineMaterialConstants3D}), the uniform response of the thermo\-elastic solid given in Eqn.~(\ref{Helmholtz3Disotropic}) takes on the form of 
\begin{subequations}
    \label{HelmholtzODEs}
    \begin{align}
    \left\{ \begin{matrix}
    \mathrm{d} \eta \\ \mathrm{d} \Pi 
    \end{matrix} \right\} & = \begin{bmatrix}
    C / \theta - 9 \alpha^2 K / \rho & 9 \alpha K / \rho \\
    -9 \alpha K & 9K
    \end{bmatrix} \left\{ \begin{matrix}
    \mathrm{d} \theta \\ \mathrm{d} \Xi 
    \end{matrix} \right\} , & &
    K = K ( \theta , \Pi , \Xi )
    \label{HelmholtzODEsUniform} \\
    \intertext{while the non-uniform squeeze response is described by}
    \left\{ \begin{matrix}
    \mathrm{d} \sigma_1 \\ \mathrm{d} \sigma_2
    \end{matrix} \right\} & = \frac{3}{2} \begin{bmatrix}
    2 N_1 & -N_2 \\
    -N_1 & 2N_2
    \end{bmatrix} \left\{ \begin{matrix}
    \mathrm{d} \varepsilon_1 \\ \mathrm{d} \varepsilon_2
    \end{matrix} \right\} , 
    & & \begin{aligned}
    N_1 & = N ( \sigma_1 , \varepsilon_1 ) \\
    N_2 & = N ( \sigma_2 , \varepsilon_2 )
    \end{aligned}
    \label{HelmholtzODEsSqueeze} \\
    \intertext{and the non-uniform shear response is described by}
    \left\{ \begin{matrix}
    \mathrm{d} \tau_1 \\ \mathrm{d} \tau_2 \\ \mathrm{d} \tau_3
    \end{matrix} \right\} & = \begin{bmatrix}
    G_1 & 0 & 0 \\ 0 & G_2 & 0 \\ 0 & 0 & G_3
    \end{bmatrix} \left\{ \begin{matrix}
    \mathrm{d} \gamma_1 \\ \mathrm{d} \gamma_2 \\ \mathrm{d} \gamma_3
    \end{matrix} \right\}, & & 
    \begin{aligned}
    G_1 & = G ( \tau_1 , \gamma_1 ) \\
    G_2 & = G ( \tau_2 , \gamma_2 ) \\
    G_3 & = G ( \tau_3 , \gamma_3 )
    \end{aligned}
    \label{HelmholtzODEsShear}
    \end{align}
\end{subequations}
which is the general form for a thermo\-elastic solid that we shall use going forward. These moduli can depend upon both stress and strain, in accordance with the implicit theory presented in Appendix~\ref{appImplicitElasticity}.

\subsubsection{The Poisson Effect}

Assuming that the bulk modulus $K$ is known, then the squeeze modulus $N$ for an isotropic material can be determined from a single uniaxial experiment by measuring its Poisson response via
\begin{displaymath}
\nu \defeq - \frac{\mathrm{d}b / b}{\mathrm{d}a / a} = 
- \frac{\mathrm{d} c / c}{\mathrm{d}a / a}
\end{displaymath}
from which it follows that
\begin{displaymath}
N = 3K \, \frac{1 - 2\nu}{1 + \nu}
\quad \text{provided that} \quad
\mathcal{S}_{11} \neq 0 
\quad \text{and} \quad
\mathcal{S}_{22} = \mathcal{S}_{33} = 0 
\end{displaymath}
where temperature $\theta$ has been held constant.  Consequently, $N = 2\mu$ where $\mu$ is the shear modulus from the classical theory of elasticity.  On the other hand, our shear modulus $G$ is distinct from its classical interpretation $\mu$.

\section{Modeling an Alveolus}
\label{secAlveolus}

\textit{To facilitate the numeric implementation of our models, and to facilitate interpretations of their results by engineers and scientists whom will use our framework, this section converts all fields defined in 1D and 2D into their 3D analogs; specifically, forces and surface tensions are converted into stresses, all moduli will now have units of stress, all coefficients for thermal expansion associate with linear expansions, and all mass densities relate mass to volume.}

Only one-third of the cross-sectional area of an alveolar chord, and only one-half of the wall thickness of an alveolar septum associate with any given dodecahedron \cite{Kimmeletal87}.  Specifically, a third of the total force carried by a septal fiber belongs with the  given alveolus, with the remaining two-thirds of the transmitted force belonging to its two adjoining alveoli.  Likewise, only half of the surface traction carried along a septal membrane belongs with the given alveolus, with the other half of its surface traction belonging to its adjacent alveoli.  Like statements apply for their entropies.

About 75\%\ of the acting transpulmonary pressure (the difference between pleural and alveolar pressures) is carried by the alveolar structure, with the remaining 25\%\ being carried by the pleural membrane encasing the lung \cite{Hajjietal79}.

\subsection{Constraints\slash Assumptions for Alveoli Subjected to Shock Waves}

Because the primary purpose for the alveolar model being constructed here is to better understand alveolar behavior as a shock wave passes over it, there are certain assumptions that we impose upon our model that under normal or different physiologic conditions might otherwise not apply.  

\textit{First\/}: An alveolus is considered to be an adiabatic pressure vessel in which air and heat cannot move into or out of as a shock wave passes over it, simply because the wave speed is too fast.  There is insufficient time for these transport phenomena to occur.  This is the `closed-cell' approximation used by Clayton \& Freed \cite{ClaytonFreed20} in the dynamic loading of their continuum model for parenchyma.

\textit{Second\/}: The tissues that comprise lung are visco\-elastic \cite{Hildebrandt69,HoppinHildebrandt77} mixtures of collagen, elastin, cells, and ground substance. \cite{RoanWaters11,Sukietal11}  Whenever a lung is subjected to a shock wave there is insufficient time for the viscous characteristics in a visco\-elastic response to manifest themselves; therefore, the overall response remains glassy elastic.  Furthermore, even though we could construct a mixture theory for modeling an alveolar membrane, it would be challenging to establish the boundary conditions, nor would we be able to construct the necessary experiments to paramterize it; consequently, an isotropic, elastic, homogeneous continuum is assumed for modeling the planar septa.

\textit{Third\/}: Temperature remains continuous in a jump across the kinematic discontinuity caused by a shock wave traveling through a compressible gas. \cite{AmesStaff53} As such, temperature is taken to be continuous across the spatial discontinuity of a shock wave traveling through parenchyma, too.  However, temperature can change both in front of and behind a traveling wave, where the material first compresses and then exapnds.  Throughout this excursion, the overall process is taken to be adiabatic, in accordance with the first assumption.

\textit{Fourth\/}: The air\slash membrane interface of an alveolus is lined with a surfactant, which is a thin bi-lipid film that plays a substantial role during normal lung function.  This film reduces alveolar surface tension to help advert total lung collapse at maximum exhale \cite{Stamenovic90}.  Even so, some alveoli still collapse, getting re-recruited during a later breath.  Models have been proposed for both surfactant \cite{Hills99} and alveolar recruitment \cite{Bates07}, but these effects are not included here as they are not thought to play a significant role in lung mechanics when a lung becomes subjected to a shock wave. 

\textit{Fifth\/}: Matsuda \textit{et~al}.\ \cite{Matsudaetal87} found the diameters of collagen and elastin fibers that circumscribe an alveolar mouth to be about 5-7 times larger than those of their septal chords.  The alveolar mouth, with its thicker fibers and open face that attach an alveolus to an alveolar duct, is modeled here as a phantom face, viz., with fibers sized like any of the other eleven pentagonal elements comprising a dodecahedron, and a twelfth phantom face placed where the alveolar mouth resides \cite{Freedetal12}.  Kimmel \& Budiansky supported this conjecture via a private communication they had with Prof.\ T.\ A.\ Wilson.  Kimmel \& Budiansky wrote: \cite{KimmelBudiansky90}
\small
\begin{quote}
    ``Professor T.\ A.\ Wilson notes that the present model does not take explicit account of either alveolar openings or their fibrous boundaries.  Wilson suggests that the elastic resistance of the ring boundaries tends to make up for the missing surface tension in the holes, so that neglect of both effects may be self-compensating.''
\end{quote}
\normalsize
This conjecture of Kimmel \& Budiansky, along with the experimental findings of Matsuda \textit{et~al}., provides a pathway by which we can scale the surface traction carried by a single alveolar membrane with that of the chords the envelope it.  In other words, this provides an avenue for parameterizing the membrane model in an other\-wise void of relevant experimental data needed to estimate its parameters.

\textit{Sixth\/}: Alveolar surfaces are modeled as membranes, not plates, and therefore are assumed to have no out-of-plane bending stiffness.  This is in concert with our assumption that the septal chords are modeled as rods, not beams, because of their slenderness ratio.  Furthermore, these septa tend to be flat because there are roughly equal pressures acting on both sides of these membranes, thereby eliminating curvature which is the driving force behind out-of-plane bending \cite{HoppinHildebrandt77} and, we surmise, also helps to suppress wrinkling.

\subsection{Modeling Septal Chords Subjected to Shock Waves}

\begin{table}
    \centering
    \begin{tabular}{|r|ccc|} 
        \hline 
        transpulmonary pressure &
        \multicolumn{3}{|c|}{$4 \, \mathrm{cm} \, \mathrm{H}_2^{\vphantom{|}} \mathrm{O}$} \\ 
        \hline
        Age & 15--35 & 36--45 & $>$ 65 $\vphantom{|}^{\vphantom{|}}$ \\ \hline 
        collagen: $\sqrt{D}^{\vphantom{|}}, \; ( \mu \textrm{m} )^{1/2}$ & 
        0.952 $\pm$ 0.242 & 0.958 $\pm$ 0.255 & 1.045 $\pm$ 0.270 \\
        elastin: $\sqrt{D}, \; ( \mu \textrm{m} )^{1/2}$ & 
        0.957 $\pm$ 0.239 & 0.970 $\pm$ 0.213 & 1.093 $\pm$ 0.274 \\
        \hline\hline       
        transpulmonary pressure &
        \multicolumn{3}{|c|}{$14 \, \mathrm{cm} \, \mathrm{H}_2^{\vphantom{|}} \mathrm{O}$} \\ 
        \hline
        Age & 15--35 & 36--45 & $>$ 65 $\vphantom{|}^{\vphantom{|}}$ \\ \hline 
        collagen: $\sqrt{D}^{\vphantom{|}}, \; ( \mu \textrm{m} )^{1/2}$ & 
        0.955 $\pm$ 0.246 & 0.994 $\pm$ 0.237 & 1.054 $\pm$ 0.279 \\
        elastin: $\sqrt{D}, \; ( \mu \textrm{m} )^{1/2}$ & 
        0.956 $\pm$ 0.237 & 0.988 $\pm$ 0.263 & 1.079 $\pm$ 0.281 \\
        \hline
    \end{tabular}
    \caption{\label{tab:alveolarProp}
        Mean and standard deviations in variance for the square root of septal chord diameters $\sqrt{D}$ reported by Sobin \textit{et~al}.\ \cite{Sobinetal88}.  These septal chords are comprised of collagen and elastin fibers that act independent of one another, and therefore, they are considered to be loaded in parallel with one another.}
\end{table}

Alveoli are biologic structures constructed of septal chords that circumscribe alveolar membranes that envelope an alveolar sac whereat gas exchange occurs.  These chords are comprised of individual collagen and elastin fibers loaded in parallel \cite{Matsudaetal87,Sobinetal88}.  The extent of elastic energy stored within a chord will depend upon the diameters $D^c$ and $D^e$ and length $L$ of these individual fibers.\footnote{
	Sobin \textit{et~al}.\ \cite{Sobinetal88} considered that the stored energy of chords also depends upon their curvature, which they measured and quantified, i.e., they considered these chords to be beams.  However, with a slenderness ratio of $\bar{L}/\bar{D} = 102 \pm 12$, which we obtained from analyzing their data, it is reasonable to model them as rods, not beams.  Consequently, the dodecahedral truss used as an alveolar model is considered to be a pinned truss, not a rigid truss, thereby greatly simplifying the boundary value problem.
}
Let superscript `$\mbox{}^c$' denote collagen, and superscript `$\mbox{}^e$' denote elastin.  Sobin \textit{et~al}.\ \cite{Sobinetal88} determined that the square root of their diameters $\sqrt{D}$ distribute normally, with a mean $\bar{D}^{1/2}$ and standard deviation $\sigma_{\sqrt{D}}$ that also depend upon age and transpulmonary pressure, as presented in Table~\ref{tab:alveolarProp} and illustrated in Fig.~\ref{fig:septalChordStats}. 

\begin{figure}
    \centering
    \subfigure[Histogram for collagen chord diameters.]{
        \includegraphics[width=0.475\textwidth]{figures/collagenFiberDiaHistogram.jpg}
        \label{fig:septalChordStatsC}
    }
    \hfill
    \subfigure[Histogram for elastin chord diameters.]{
        \includegraphics[width=0.475\textwidth]{figures/elastinFiberDiaHistogram.jpg}
        \label{fig:septalChordStatsE}
    }
    \caption{Typical histograms for alveolar chord diameters constructed using the statistics reported in Table~\ref{tab:alveolarProp}.  Their tails weigh heavy at the larger diameters, because their distributions are normal in the square root of their diameters.  These histograms are virtually identical.}
    \label{fig:septalChordStats}
\end{figure}

The collagen and elastin fibers that make up a septal chord have the same length $L$, they experience the same strain $e$, and they exist at the same temperature $\theta$; therefore, we employ Eqn.~(\ref{Helmholtz1D}) as the governing constitutive equation to describe their mechanical behaviors; specifically, for the collagen fiber in an alveolar chord
\begin{subequations}
    \label{alveolarChord}
    \begin{align}
    \left\{ \begin{matrix} 
    \mathrm{d} \eta^c \\ \mathrm{d} s^c
    \end{matrix} \right\} = \begin{bmatrix}
    C^c / \theta - ( \alpha^c )^2 E^c / \rho^c & \alpha^c E^c / \rho^c \\
    -\alpha^c E^c & E^c
    \end{bmatrix} \left\{ \begin{matrix}
    \mathrm{d} \theta \\ \mathrm{d} e
    \end{matrix} \right\} , & & 
    E^c = E^c ( \theta , e , s^c ) \\
    \intertext{and for the elastin fiber in an alveolar chord}
    \left\{ \begin{matrix} 
    \mathrm{d} \eta^e \\ \mathrm{d} s^e
    \end{matrix} \right\} = \begin{bmatrix}
    C^e / \theta - ( \alpha^e )^2 E^e / \rho^e & \alpha^e E^e / \rho^e \\
    -\alpha^e E^e & E^e
    \end{bmatrix} \left\{ \begin{matrix}
    \mathrm{d} \theta \\ \mathrm{d} e
    \end{matrix} \right\} , & &
    E^e = E^e ( \theta , e , s^e )
    \end{align}
\end{subequations}
where $\eta^c$ and $\eta^e$ are the entropy densities (erg/g.K) for collagen and elastin; $s^c \defeq \lambda F^c / A^c_0$ and $s^e \defeq \lambda F^e / A_0^e$ are the chordal stresses (barye = $\text{dyne/cm}^2$) carried by the collagen and elastin fibers, wherein $\lambda = L/L_0$ is the fiber stretch, $A^c_0$ and $A^e_0$ are their traction-free cross-sectional areas ($\text{cm}^2$), and $F^c$ and $F^e$ are the forces (dyne) they transmit.  Parameters $C^c$ and $C^e$ are their specific heats at constant pressure (erg/g.K), $\alpha^c$ and $\alpha^e$ are their linear coefficients of thermal expansion (1/K), $E^e$ and $E^e$ are their elastic moduli ($\text{dyne/cm}^2$ = $\text{erg/cm}^3$), and $\rho^c$ and $\rho^e$ are their mass densities ($\text{g/cm}^3$).  These differential equations are subject to initial conditions considered to be $s^c_0 = s^c |_{L = L_0} = 0$, $s^e_0 = s^e |_{L = L_0} = 0$, $\eta^c = \eta^c_0$ and $\eta^e = \eta^e_0$, where  $\eta_0^c$ and $\eta_0^e$ are their respective entropy densities at rest.

The actual force and entropy of an individual septal chord in our alveolar model is taken to be one third of a fiber's calculated values, as determined by Eqn.~(\ref{alveolarChord}), because each alveolar chord is typically shared between three adjoining alveoli; consequently, 
\begin{equation}
    \label{septalChordCEs}
    F^f = ( A_0^c s^c + A_0^e s^e ) / 3 \lambda 
    \quad \text{and} \quad
    S^f = ( \rho^c V_0^c \eta^c + \rho^e V_0^e \eta^e ) / 3 
\end{equation}  
where $F^f$ (dyne) is a third of the fiber's force carried by a septal chord, and $S^f$ (erg/K) is a third of the fiber's entropy.

Collagen is a fiber comprised of numerous, long, slender, wavy filaments whose waviness, known as crimp, straightens under sufficient deformation \cite{Kastelicetal'78,FreedDoehring05}.  Elastin is a linked fiber network, much like an elastomer, whose filaments between crosslinks rotate to align with an axis of loading under sufficient deformation \cite{AaronGosline81,Urry89}.  Consequently, collagen and elastin both recruit constituent filaments with increasing deformation into an overall, load-bearing, fiber response.  The internal energies of collagen and elastin may therefore be thought of as being comprised of a configurational energy and a strain energy.  As such, both collagen and elastin are modeled as Freed-Rajagopal biologic fibers, which are described in terms of two such internal energies.  Their model is derived from the theory of implicit elasticity, cf.\ App.~\ref{appImplicitElasticity}.  According to their model, tangent compliances for collagen and elastin, pertinent to the hypo-elastic constitutive formulation of Eqn.~(\ref{alveolarChord}), are described by
\begin{subequations}
    \label{septalChordModuli}
    \begin{align}
	\frac{1}{E^c (\theta , s^c , e )} & = \frac{e_t^c - e_1^c}{E_1^c e_t^c + 2s^c} + \frac{1}{E_2^c} &
	e_1^c & = e - \alpha^c (\theta - \theta_0) - \frac{s^c}{E_2^c} \\
    \frac{1}{E^e (\theta , s^e , e )} & = \frac{e_t^e - e_1^e}{E_1^e e_t^e + 2s^e} + \frac{1}{E_2^e} &
    e_1^e & = e - \alpha^e (\theta - \theta_0) - \frac{s^e}{E_2^e} 
    \end{align}
\end{subequations}
where $\theta_0$ is body temperature, i.e., 310~K.  Material constants $E_1^c$ and $E_2^c$ are the two asymptotic moduli for collagen that bound its response, i.e., $E_1^c \leq E^c \leq E^c_2$, while $E_1^e$ and $E_2^e$ are the two asymptotic moduli for elastin that bound its response, viz., $E^e_1 \leq E^e \leq E^e_2$, both having units of stress (barye = dyne/$\text{cm}^2$), with $e_t^c$ and $e_t^e$ being their respective transition strains (see their derivation in App.~\ref{appImplicitElasticity}).  

In compression, it is assumed $E^c = E^c_1 E^c_2 / ( E^c_1 + E^c_2 )$ and $E^e = E^e_1 E^e_2 / ( E^e_1 + E^e_2 )$, which are the elastic moduli for collagen and elastin at zero fiber stress.  This is done to help ensure a stable numerical implementation, as long slender rods readily buckle under compressive loads---a phenomenon not modeled here.

The material properties needed to model septal chords are listed in Tables~\ref{tab:alveolarProp} \& \ref{tableCollagenElastin}.  From Eqn.~(\ref{thermodynamicConstraints}), these moduli are bound from above by $E^c_{\max} = 4.9 \times 10^9$~barye ($\text{dyne/cm}^2$) and $E^e_{\max} = 1.7 \times 10^9$~barye.  We therefore observe that $E^c_2$ and $E^e_2$ are about 100 times smaller than $E^c_{\max}$ and $E^e_{\max}$.

\begin{table}
    \centering
    \begin{tabular}{|l|l|l|}
        \hline
        \multicolumn{3}{|c|}{Collagen$\vphantom{|^{|^|}}$} \\ \hline
        $\rho^c$ \hfill [$\textrm{g/cm}^{3^{\phantom{|}}}$] & $1.34$ & 
        Fels \cite{Fels64} \\
        $\eta_0^c$ \hfill [erg/g.K] & $3.7 \times 10^7$ &  \\
        $C^c$ \hfill [erg/g.K] & $1.7 \times 10^7$ & 
        Kanagy \cite{Kanagy55} \\
        $\alpha^c$ \hfill [1/C] & $1.8 \times 10^{-4}$ & 
        Weir \cite{Weir48}  \\
        $e^c_t$ & $0.09$ & estimated from TLC $\approx$ 30\% \\
        $E_1^c$ \hfill [barye] & $5.0 \times 10^5$ & authors experience \\
        $E_2^c$ \hfill [barye] & $3.0 \times 10^7$ & authors experience \\ \hline
        \multicolumn{3}{|c|}{Elastin$\vphantom{|^{|^|}}$} \\ \hline 
        Parameter & Value & Reference \\ \hline
        $\rho^e$ \hfill [$\textrm{g/cm}^{3^{\phantom{|}}}$] & $1.31$ & 
        Lillie \& Gosline \cite{LillieGosline02a} \\
        $\eta_0^e$ \hfill [erg/g.K] & $3.4 \times 10^7$ & 
        Shadwick \& Gosline \cite{ShadwickGosline85} \\
        $C^e$ \hfill [$\textrm{erg/g.K}$] & $4.2 \times 10^7$  & 
        Kakivaya \& Hoeve \cite{KakivayaHoeve75} \\
        $\alpha^e$ \hfill [1/C] & $3.2\times 10^{-4}$ & 
        Lillie \& Gosline \cite{LillieGosline02a} \\ 
        $e^e_t$ & 0.4 & Shadwick \& Gosline \cite{ShadwickGosline85} \\
        $E^e_1$ \hfill [barye] & $2.3 \times 10^6$ & Urry \cite[Fig.~18]{Urry89} \\ 
        $E^e_2$ \hfill [barye] & $1.0 \times 10^7$ & 
        Lillie \& Gosline \cite[Fig.~5]{LillieGosline07} \\ \hline
    \end{tabular}
    \caption{Physical properties for hydrated collagen and elastin fibers.  Collagen denatures at around $60^\circ$C \cite{HoermannSchlebusch71}, i.e., above this temperature collagen will shrink rapidly---an effect not modeled here.}
    \label{tableCollagenElastin}
\end{table}

\subsection{Modeling Alveolar Septa Subjected to Shock Waves}
\label{secConjugatePairs}

The thermo\-elastic response of a planar membrane used to model alveolar septa, as described in Eqn.~(\ref{HelmholtzMembraneODEs}), is governed by the following pair of differential equations.  The first set of ODEs establishes the uniform response of a membrane, as described in Eqn.~(\ref{HelmholtzMembraneODEsUniform}), viz.,
\begin{displaymath}
    \left\{ \begin{matrix}
    \mathrm{d} \eta \\ \mathrm{d} s^{\pi}
    \end{matrix} \right\} = \begin{bmatrix}
    C / \theta - 4 \alpha^2 M / \rho & 
    4 \alpha M / \rho \\
    -4 \alpha M & 4 M
    \end{bmatrix} \left\{ \begin{matrix}
    \mathrm{d} \theta \\ \mathrm{d} \xi
    \end{matrix} \right\} , \quad
    M = M ( \theta , s^{\pi} , \xi )
\end{displaymath}
where $s^{\pi} \defeq \pi / h$ has units of stress (dyne/$\text{cm}^2$) with $h$ denoting height or thickness of the spetal membrane.  Assuming the volume of a septal membrane remains constant, thickness would obey $h = h_0 \exp ( -2 \xi )$ with $h_0$ being its traction-free thickness.  Tangent modulus $M$ is an areal equivalent of the bulk modulus.  The second set of ODEs establishes the non-uniform response of a membrane, as described in Eqn.~(\ref{HelmholtzMembraneODEsNonUniform}), viz.,
\begin{displaymath}
    \left\{ \begin{matrix}
    \mathrm{d} s^{\sigma} \\ \mathrm{d} s^{\tau}
    \end{matrix} \right\} = \begin{bmatrix}
    2N & 0 \\
    0 & G
    \end{bmatrix} \left\{ \begin{matrix}
    \mathrm{d} \varepsilon \\ \mathrm{d} \gamma
    \end{matrix} \right\} , \quad
    \begin{aligned}
        N & = N ( s^{\sigma} , \varepsilon ) \\
        G & = G ( s^{\tau} , \gamma )
    \end{aligned}
\end{displaymath}
where $s^{\sigma} \defeq \sigma / h$ and $s^{\tau} \defeq \tau / h$ also have units of stress (dyne/$\text{cm}^2$).  Modulus $N$ is the squeeze (pure shear) tangent modulus, while modulus $G$ is the (simple) shear tangent modulus.  These moduli are distinct; however, moduli $M$ and $N$ relate to one another through Poisson's ratio, cf.\ Eqn.~(\ref{HelmholtzMembraneODEsNonUniformCV}).  How this is handled for biologic tissues is addressed below.

From a mechanics perspective, we know a great deal more about alveolar chords than we know about alveolar septa.  More judgment will therefore be required in our construction and parameterization of a material model for alveolar membranes.  

A typical alveolar septum is 4-5 $\mu$m thick \cite{Sukietal11}.  They are comprised of an outside layer of epithelial cells that encase capillaries made of endothelial cells along with a basement membrane that is composed of unorganized collagen and elastin filaments, plus proteoglycans and other extracellular proteins.  This basement membrane, roughly at mid-plane in an alveolar septum, has a width of about $0.5 \, \mu$m \cite{RoanWaters11}.  Inertial forces generated by these membranes are to be based upon a membrane thickness of $\sim\!\!5$~$\mu$m with an approximate density of water, while the structural forces that they carry are to be based upon a basement membrane thickness of $\sim\!\! 0.5$~$\mu$m.  

It is not known how much of the mechanical load is actually carried by the cells in an alveolar septum vs.\ the extracellular basement membrane they encase, but it is generally thought that this basement membrane carries the majority of the load \cite{Sukietal11}.  Therefore, by diminishing the moduli that are appropriate for describing a basement membrane with thickness $\sim\!\! 0.5$~$\mu$m by a factor of 10, we get estimates for effective septal moduli that are applicable when modeling a whole septal membrane with thickness $\sim\!\! 5$~$\mu$m.  We employ the model parameters specified in Table~\ref{tableVisceralPleura}, which are based upon this assumption.

\begin{table}
    \centering
    \begin{tabular}{|l|l|}
        \hline
        Property & Value \\ \hline
        $\rho$ \hfill [$\textrm{g/cm}^{3^{\phantom{|}}}$] & $1.1$ \\
        $\eta_0$ \hfill [erg/g.K] & $5.0 \times 10^6$ \\
        $C$ \hfill [erg/g.K] & $2.1 \times 10^7$ \\
        $\alpha$ \hfill [1/C] & $1.2 \times 10^{-4}$ \\ \hline
        $\xi_t$ & $0.24$ \\
        $M_1$ \hfill [barye] & $8.0 \times 10^2$ \\
        $M_2$ \hfill [barye] & $2.2 \times 10^5$ \\ \hline
        $\varepsilon_t$ & $0.06$ \\
        $N_1$ \hfill [barye] & $2M_1/3$ \\
        $N_2$ \hfill [barye] & $2M_2/3$ \\ \hline
        $\gamma_t$ & $0.4$ \\
        $G_1$ \hfill [barye] & $5.0 \times 10^2$ \\
        $G_2$ \hfill [barye] & $5.0 \times 10^3$ \\ \hline
    \end{tabular}
    \caption{The elastic properties reported here are for visceral pleura taken from Freed \textit{et~al}.\ \cite{Freedetal17} and parenchyma taken from Saraf \text{et~al}., \cite{Sarafetal07} divided by 10 to adjust for septal thickness vs.\ basement membrane thickness.  The thermo\-physical properties lie between that of water and collagen, weighted towards that of water, and evaluated at body temperature.}
    \label{tableVisceralPleura}
\end{table}

Collagen and elastin appear as thin filaments randomly oriented and somewhat uniformly dispersed throughout a basement membrane, unlike the strongly aligned fibers that appear in septal chords.  Consequently, for our purposes, we model this collective ensemble of tissue and structure types as a homo\-geneous isotropic membrane modeled after the Freed-Rajagopal biologic fiber \cite{FreedRajagopal16} that we have extended to membranes in App.~\ref{appImplicitElasticity}, specifically
\begin{subequations}
    \label{septalCompliances}
    \begin{align}
    \frac{1}{M(\theta, \xi, s^{\pi})} & = 
    \frac{\xi_t - \xi_1}{M_1 \xi_t + s^{\pi} / 2} + \frac{1}{M_2} 
    & \xi_1 & = \xi - \alpha (\theta - \theta_0) - \frac{s^{\pi}}{4M_2}
    \label{septalDilationCompliance} \\
    \frac{1}{N(\varepsilon , s^{\sigma})} & = \frac{ \mathrm{sgn} (\varepsilon_1) \, \varepsilon_t - \varepsilon_1}{N_1 \, \mathrm{sgn} (\varepsilon_1) \, \varepsilon_t + s^{\sigma}} + \frac{1}{N_2} &
    \varepsilon_1 & = \varepsilon - \frac{s^{\sigma}}{2N_2}
    \label{septalSqueezeCompliance} \\
    \frac{1}{G(\gamma , s^{\tau})} & = \frac{ \mathrm{sgn} (\gamma_1) \, \gamma_t - \gamma_1}{G_1 \, \mathrm{sgn} (\gamma_1) \, \gamma_t + 2 s^{\tau}} + \frac{1}{G_2} & 
    \gamma_1 & = \gamma - \frac{s^{\tau}}{G_2}
    \label{septalShearCompliance}
    \end{align}
\end{subequations}
where compliant, initial, tangent moduli $M_1$, $N_1$ and $G_1$ and stiff, terminal, tangent moduli $M_2$, $N_2$ and $G_2$ bound their respective responses so that $M_1 \leq M \leq M_2$, $N_1 \leq N \leq N_2$ and $G_1 \leq G \leq G_2$, with a gradual transition between their asymp\-totic bounds occurring around strains $\xi_t$, $\varepsilon_t$ and $\gamma_t$.

In compression, areal modulus $M$ is assigned a value of $M = M_1 M_2 / ( M_1 + M_2 )$ which is the tangent modulus at zero surface tension.  This extends into compression in an effort to maintain numeric stability.  Normally, a negative surface tension would cause wrinkling of the membrane surface, which is not addressed here.  In contrast, the squeeze $N$ and shear $G$ moduli maintain applicability whenever their arguments are negative valued.  This is handled via the sign function present in Eqns.~(\ref{septalSqueezeCompliance} \& \ref{septalShearCompliance}).

In \S\ref{PoissonRatio} it was determined that $N = \tfrac{2}{3} M$ whenever Poisson's ratio $\nu$ is a half.  Imposing this constriant of incompressibility is straightforward for Hookean materials, but it is not as obvious for biologic materials.  Examining the data of Freed \textit{et~al}.\ \cite{Freedetal17} for the pleural membrane, it seems reasonable to apply this proportionality factor to the limiting moduli, viz., it was found that $N_1 \approx \tfrac{2}{3} M_1$ and $N_2 \approx \tfrac{2}{3} M_2$.  However, the transition strains were not found to obey this scaling; rather, $\varepsilon_t \approx \tfrac{1}{4} \xi_t$.  Regarding the shear response of parenchyma, we use the bounds reported by Saraf \text{et~al}.\ \cite{Sarafetal07} for setting $G_1$ and $G_2$, while using the transition strain $\gamma_t$ reported in Freed \textit{et~al}.\ \cite{Freedetal17} for muscle tissue.  Clearly, there is a need for experiments to provide better estimates for these material properties.

Finite element technology is used to interpolate the entropy and these stresses, integrated at the Gauss points, to entropies and forces at the vertices of a pentagon, cf.\ Part~\ref{partVariational}.  The actual entropies and forces interpolated to these nodes are halved, because each septal plane belongs to two adjoining alveoli. 

\subsection{Modeling an Alveolar Volume Subjected to Shock Waves}
\label{sec:IdealGasLaw}

Alveoli are connected to bronchial trees via alveolar ducts.  Under normal conditions, air moves in and out of the alveoli via these ducts.  However, when subjected to a stress wave passing over an alveolus, there is no time for the transport of air to take place.  Hence, we can consider the air (and heat) within an alveolus to become `trapped', and the pressure to be uniform therein.  The governing thermo\-dynamic process is therefore adiabatic.  It is under this condition that we model the volumetric response of an alveolar sac.

\subsubsection{Alveoli Filled with Air}

Considering the saturated air within an alveolus to be an ideal gas, then \cite{Davison08}
\begin{equation}
P V = n \! R \theta
\quad \text{or} \quad
\frac{P V}{\theta} = \frac{P_0 V_0}{\theta_0} = n \! R = \mathrm{constant}
\label{idealGas}
\end{equation}
where, in our case, $P_0$ is taken to be atmospheric pressure at sea level (1~bar or $10^5$~Pa or $10^6$~barye), with $V_0$ being that alveolar volume whereat alveolar pressure and plural pressure are both atmospheric, while $\theta_0 = 37^{\circ}$C = 310~K is assigned as body temperature.  Parameter $n$ is the molar content of gas within an alveolus, and $R$ is the universal gas constant.  

The material properties associated with an ideal gas contained within an adiabatic enclosure are
\begin{subequations}
    \label{idealGasConstants}
    \begin{align}
\alpha & \defeq \frac{1}{L} \left. \frac{\partial L}{\partial \theta} \right|_P =
\frac{1}{3V} \left. \frac{\partial V}{\partial \theta} \right|_P = 
\frac{1}{3\theta_0} \, \frac{P_0 V_0}{P V} \\
\intertext{and}
K & \defeq -V \left. \frac{\partial P}{\partial V} \right|_{\theta} = 
P_0 \, \frac{V_0 \hspace{0.5pt} \theta}{\theta_0 V}
\end{align}
\end{subequations}
with the other two material properties pertaining to moist air at body temperature\footnote{
    Physical properties for air were taken from the website \texttt{www.peacesoftware.de} hosted by Berndt Wischnewski.
}
being its mass density $\rho$ of $1.125 \times 10^{-3} \; \text{g/cm}^3$ and its specific heat $C$ of $1.007 \times 10^7$~erg/g.K at constant pressure, constrained by $K < K_{\max} = \rho C / \alpha^2 \theta \approx \rho C \theta$.  An alveolar sac, when modeled as an adiabatic pressure vessel filled with an ideal gas, is described by
\begin{equation}
\left\{ \begin{matrix}
    \mathrm{d} \eta \\ -3 \, \mathrm{d} P
\end{matrix} \right\} = \begin{bmatrix}
    C / \theta - 9 \alpha^2 K / \rho & 
    9 \alpha K / \rho \\
    -9 \alpha K & 9 K
\end{bmatrix} \left\{ \begin{matrix}
    \mathrm{d} \theta \\ \mathrm{d} \Xi
\end{matrix} \right\}
\tag{\ref{Helmholtz3D}}
\end{equation}
where the entropy within an alveolar sac is given by $S^a = \rho V \eta$ whose initial condition is $S^a_0 = \rho V_0 \eta_0$ with $\rho \eta_0$ being the entropy per unit volume of humid air at body temperature and atmospheric pressure, viz., $\rho \eta_0 = 7.770 \times 10^4 \: \text{erg/cm}^3\text{.K}$.  Equation~(\ref{Helmholtz3D}) in conjunction with the physical properties describing an ideal gas (\ref{idealGasConstants}) result in the following equation governing pressure
\begin{displaymath}
\frac{\mathrm{d} P}{P} = \frac{P_0 V_0 \theta}{P V \theta_0} \left( 
\frac{P_0 V_0 \theta}{P V \theta_0} \, \frac{\mathrm{d} \theta}{\theta} - 
\frac{\mathrm{d} V}{V} \right)
\end{displaymath}
where pressure, volume and temperature all appear as logarithmic rates.

Pressure $P$ is mapped to nodal forces at the vertices of a dodecahedron in our alveolar model.  This requires finite element technology, which is discussed in Part~\ref{partVariational}.

\subsubsection{Alveoli Filled with Fluid}

In lung tissues that are not healthy, fluids may fill alveolar volumes at various regions throughout a lung, e.g., as could have been caused by injury, pneumonia, etc.  In such localities the mechanical response of the local parenchyma will be vastly stiffer than that of healthy tissue, and as such, it will respond very differently to a traveling shock wave.  For example, the speed of a wave moving over alveoli filled with fluid will be substantially faster than the speed of the same wave moving over healthy alveoli filled with air.

In the presence of a passing shock wave, we suppose that an unhealthy alveolar sac, like a healthy one, can be modeled as an adiabatic enclosure, but now the fluid within such an alveolus is considered to behave, momentarily, like an elastic solid.

The thermo\-elastic response of an alveolar volume, as described in Eqn.~(\ref{HelmholtzODEs}), is governed by three sets of uncoupled differential equations.  The first set of ODEs establishes the uniform response of Eqn.~(\ref{HelmholtzODEsUniform}) described by
\begin{displaymath}
\left\{ \begin{matrix}
\mathrm{d} \eta \\ \mathrm{d} \Pi 
\end{matrix} \right\} = \begin{bmatrix}
C / \theta - 3 \alpha^2 K / \rho & 3 \alpha K / \rho \\
-3 \alpha K & 9K
\end{bmatrix} \left\{ \begin{matrix}
\mathrm{d} \theta \\ \mathrm{d} \Xi 
\end{matrix} \right\} , \quad
K = K ( \theta , \Pi , \Xi )
\end{displaymath}
with the second set of ODEs in Eqn.~(\ref{HelmholtzODEsSqueeze}) governing the squeeze response
\begin{displaymath}
\left\{ \begin{matrix}
\mathrm{d} \sigma_1 \\ \mathrm{d} \sigma_2
\end{matrix} \right\} = \frac{3}{2} \begin{bmatrix}
2 N_1 & -N_2 \\
-N_1 & 2N_2
\end{bmatrix} \left\{ \begin{matrix}
\mathrm{d} \varepsilon_1 \\ \mathrm{d} \varepsilon_2
\end{matrix} \right\} , \quad
\begin{aligned}
N_1 & = N ( \sigma_1 , \varepsilon_1 ) \\
N_2 & = N ( \sigma_2 , \varepsilon_2 )
\end{aligned}
\end{displaymath}
while the third set of ODEs in Eqn.~(\ref{HelmholtzODEsShear}) governs the shear response
\begin{displaymath}
    \left\{ \begin{matrix}
    \mathrm{d} \tau_1 \\ \mathrm{d} \tau_2 \\ \mathrm{d} \tau_3
    \end{matrix} \right\} = \begin{bmatrix}
    G_1 & 0 & 0 \\ 0 & G_2 & 0 \\ 0 & 0 & G_3
    \end{bmatrix} \left\{ \begin{matrix}
    \mathrm{d} \gamma_1 \\ \mathrm{d} \gamma_2 \\ \mathrm{d} \gamma_3
    \end{matrix} \right\} , \quad
    \begin{aligned}
    G_1 & = G ( \tau_1 , \gamma_1 ) \\
    G_2 & = G ( \tau_2 , \gamma_2 ) \\
    G_3 & = G ( \tau_3 , \gamma_3 )
    \end{aligned}
\end{displaymath}
that, collectively, can be used to describe the thermo\-elastic response of a volume of material.  

How these are to be parameterized will be addressed in next year's work.

\section{Finite Element Implementation of Constitutive Equations}
\label{secFE_CE}

The above constitutive models are implemented into our finite element model as hypo-elastic material models \cite{Truesdell55} described by\footnote{
    Constitutive equation $\mathrm{d} \boldsymbol{\sigma} = \mathbf{M} ( \boldsymbol{\sigma}, \boldsymbol{\epsilon} ) \, \mathrm{d} \boldsymbol{\epsilon}$ is that of a hypo-elastic solid \cite{Truesdell55}.  A reasonable visco\-elastic constitutive equation that one could consider would be a Zener \cite{Zener48} solid, which would look something like
    $$ \boldsymbol{\sigma} + \boldsymbol{\tau} \, \mathrm{d} \boldsymbol{\sigma} = \boldsymbol{E}_{\infty} \boldsymbol{\epsilon} + \boldsymbol{\tau} \mathbfsf{E}_0 \, \mathrm{d} \boldsymbol{\epsilon} 
    \quad \text{or} \quad 
    \mathrm{d} \boldsymbol{\sigma} + \boldsymbol{\tau} \, \mathrm{d}^2 \boldsymbol{\sigma} = \boldsymbol{E}_{\infty} \mathrm{d} \boldsymbol{\epsilon} + \boldsymbol{\tau} \mathbfsf{E}_0 \, \mathrm{d}^2 \boldsymbol{\epsilon}$$ 
    where $\mathbfsf{E}_{\infty}$ is a matrix of rubbery moduli, $\mathbfsf{E}_0$ is a matrix of glassy moduli, and $\boldsymbol{\tau}$ is a matrix of characteristic relaxation times.  This is a topic for future work.
}
\begin{equation}
\mathrm{d} \boldsymbol{\sigma} = \mathbf{M} ( \boldsymbol{\sigma}, \boldsymbol{\epsilon} ) \, \mathrm{d} \boldsymbol{\epsilon} 
\quad \text{wherein} \quad
\mathrm{d} \boldsymbol{\epsilon} = 
\frac{\mathrm{d} \boldsymbol{\varepsilon} ( \boldsymbol{\lambda} )}
{\mathrm{d} \boldsymbol{\lambda}} \, \mathrm{d} \boldsymbol{\lambda}
\label{hypoelastic}
\end{equation}
where $\boldsymbol{\sigma}$, $\boldsymbol{\epsilon}$ and $\boldsymbol{\lambda}$ are arrays of stress, strain and stretch attributes, respectively, with matrix $\mathbf{M} ( \boldsymbol{\sigma}, \boldsymbol{\epsilon} )$ containing the constitutive tangent moduli, i.e., $\mathrm{d} \boldsymbol{\sigma} / \mathrm{d} \boldsymbol{\epsilon}$, that, in general, may depend upon both stress $\boldsymbol{\sigma}$ and strain $\boldsymbol{\epsilon}$, while strain depends solely upon stretch $\boldsymbol{\lambda}$.

The two-step PECE algorithm presented in \S\ref{sec:1stOrderPECE} presumes the following information.  There is an initial condition for the thermo\-dynamic stresses $\boldsymbol{\sigma}_0$.  An initial far-field deformation gradient $\mathbfsf{F}_0$ is also known, from which initial conditions for the thermo\-dynamic stretches $\boldsymbol{\lambda}_0$ and strains $\boldsymbol{\epsilon}_0$ can be readily obtained.  All nodes of integration are to be spaced uniformly in time with $\mathrm{d}t > 0$ designating their separation in time.  

A far-field deformation gradient is considered to be known at the end of the first integration step, i.e., $\mathbfsf{F}_1$, from which the thermo\-dynamic stretches $\boldsymbol{\lambda}_1$ and strains $\boldsymbol{\epsilon}_1$ can be calculated, with differential rates $\mathrm{d} \boldsymbol{\lambda}_0$ and $\mathrm{d} \boldsymbol{\lambda}_1$ coming from finite difference formul\ae, so that $\mathrm{d} \boldsymbol{\epsilon}_0 = [ \mathrm{d} \boldsymbol{\varepsilon} ( \boldsymbol{\lambda}_0 ) / \mathrm{d} \boldsymbol{\lambda}_0 ] \, \mathrm{d} \boldsymbol{\lambda}_0$ and $\mathrm{d} \boldsymbol{\epsilon}_1 = [ \mathrm{d} \boldsymbol{\varepsilon} ( \boldsymbol{\lambda}_1 ) / \mathrm{d} \boldsymbol{\lambda}_1 ] \, \mathrm{d} \boldsymbol{\lambda}_1$ follow.  Consequently, an initial stress rate can be established, i.e., $\mathrm{d} \boldsymbol{\sigma}_0 = \mathbf{M} ( \boldsymbol{\sigma}_0 , \boldsymbol{\epsilon}_0 ) \, \mathrm{d} \boldsymbol{\epsilon}_0$.  With this information, Heun's method (Eqn.~\ref{startUp1stOrderODEs}) can be called upon to integrate Eqn.~(\ref{hypoelastic}) to determine the thermo\-dynamic stresses and their differential rates out to the end of the first integration step, viz., $\boldsymbol{\sigma}_1$ after which one can determine $\mathrm{d} \boldsymbol{\sigma}_1 = \mathbf{M} ( \boldsymbol{\sigma}_1 , \boldsymbol{\epsilon}_1 ) \, \mathrm{d} \boldsymbol{\epsilon}_1$.    

For the next step, and those that follow, a more stable two-step method in Eqn.~(\ref{1stOrderODEs}) can be called upon to advance a solution for the thermo\-dynamic stresses and their differential rates.  What needs to be stored are variables from the previous step, viz., $\boldsymbol{\sigma}_{n-1}$, $\boldsymbol{\lambda}_{n-1}$ and $\mathrm{d} \boldsymbol{\sigma}_{n-1}$, along with like variables from the current step, viz.,  $\boldsymbol{\sigma}_n$, $\boldsymbol{\lambda}_n$ and $\mathrm{d} \boldsymbol{\sigma}_n$.  Then, given a far-field deformation gradient for the next step, i.e., $\mathbfsf{F}_{n+1}$, one can determine $\boldsymbol{\lambda}_{n+1}$ and $\boldsymbol{\epsilon}_{n+1}$ along with the differential rate $\mathrm{d} \boldsymbol{\lambda}_{n+1}$ obtained from finite difference formul\ae, after which $\mathrm{d} \boldsymbol{\epsilon}_{n+1} = [ \mathrm{d} \boldsymbol{\varepsilon} ( \boldsymbol{\lambda}_{n+1} ) / \mathrm{d} \boldsymbol{\lambda}_{n+1} ] \, \mathrm{d} \boldsymbol{\lambda}_{n+1}$ can be established.  With this information, the PECE method (Eqn.~\ref{1stOrderODEs}) can be called upon to integrate Eqn.~(\ref{hypoelastic}) to determine the thermo\-dynamic stresses and their differential rates out to the end of the next integration step, viz., $\boldsymbol{\sigma}_{n+1}$ along with $\mathrm{d} \boldsymbol{\sigma}_{n+1} = \mathbf{M} ( \boldsymbol{\sigma}_{n+1} , \boldsymbol{\epsilon}_{n+1} ) \, \mathrm{d} \boldsymbol{\epsilon}_{n+1}$.

\subsection{1D Formulation}

For any given alveolar chord, we know its reference length $L_0$ and its current length $L$ from which chordal stretch is described by $\lambda \defeq L / L_0$ and its strain is defined as $e \defeq \ln \lambda$ whose differential rate of change is $\mathrm{d} e = L^{-1} \, \mathrm{d} L$; consequently,
\begin{equation}
    \left\{ \begin{matrix}
    \mathrm{d} \theta \\ \mathrm{d} e
    \end{matrix} \right\} = \begin{bmatrix}
    1 & 0 \\ 0 & 1/L
    \end{bmatrix} \left\{ \begin{matrix}
    \mathrm{d} \theta \\ \mathrm{d} L
    \end{matrix} \right\}
\end{equation}
where the above matrix establishes $\mathrm{d} \boldsymbol{\varepsilon} ( \boldsymbol{\lambda} ) / \mathrm{d} \boldsymbol{\lambda}$ in Eqn.~(\ref{hypoelastic}) for a chord, with $\mathrm{d} \boldsymbol{\lambda}$ being the vector to the right of this matrix, and $\mathrm{d} \boldsymbol{\varepsilon}$ being the vector on the left-hand side.  Finite difference formul\ae\ are used to quantify $\mathrm{d} \theta$ and $\mathrm{d} L$.

The matrix of tangent moduli $\mathbf{M} ( \boldsymbol{\sigma} , \boldsymbol{\varepsilon} )$ given in Eqn.~(\ref{hypoelastic}) is Eqn.~(\ref{Helmholtz1D}) for the case of a chord, whose elastic modulus is $E(\theta , s , e)$ with stress defined as $s \defeq F/A = \lambda F / A_0$, where $F$ is the force carried by the chord, and where $A_0$ and $A$ are the chordal cross-sectional areas in its reference and current states, respectively.  Here it is assumed that chordal volume is preserved, which is a reasonable assumption for soft biological structures.  The vector being operated on by this matrix of tangent moduli is $\mathrm{d} \boldsymbol{\varepsilon}$ above, with the constitutive equation returning a vector $\mathrm{d} \boldsymbol{\sigma} = \{ \mathrm{d} \eta , \mathrm{d} s \}^{\mathsf{T}}$.  The numerical method presented in \S\ref{sec:1stOrderPECE} can then be called upon to integrate this hypo-elastic constitutive equation for its thermo\-dynamic stresses $\boldsymbol{\sigma} = \{ \eta , s \}^{\mathsf{T}}$.  


\subsection{2D Formulation}

Consider an incoming deformation gradient $\mathbfsf{F} = \mathcal{F}_{ij} \, \vec{\mathbfsf{e}}_i \otimes \vec{\mathbfsf{e}}_j$, $i, j = 1, 2$, whose components $\mathcal{F}_{ij}$ are evaluated in a co-ordinate system associated with a membrane whose base vectors $( \vec{\mathbfsf{e}}_1 , \vec{\mathbfsf{e}}_2 )$, cf.\ Fig.~\ref{figPentagonCoord}, have been re-indexed via an orthogonal matrix $\mathbfsf{P}$ according to \S\ref{secRemedy}.  It is in this co-ordinate system that the components $\mathcal{U}_{ij}$ of Laplace stretch $\boldsymbol{\mathcal{U}} = \mathcal{U}_{ij} \, \vec{\mathbfsf{e}}_i \otimes \vec{\mathbfsf{e}}_j$ and its inverse $\boldsymbol{\mathcal{U}}^{-1}$ are quantified, their physical attributes $a$, $b$, $g$ are determined, and their associated thermo\-dynamic strains $\xi$, $\varepsilon$, $\gamma$ and differential strain rates $\mathrm{d} \xi$, $\mathrm{d} \varepsilon$, $\mathrm{d} \gamma$ are established, cf.\ \S\ref{strainsAndStrainRates2D}.

The differential rates of these physical attributes relate to the differential rates of their thermo\-dynamic variables according to 
\begin{equation}
    \left\{ \begin{matrix}
    \mathrm{d} \theta \\ \mathrm{d} \xi \\
    \mathrm{d} \varepsilon \\ \mathrm{d} \gamma
    \end{matrix} \right\} = \begin{bmatrix}
    1 & 0 & 0 & 0 \\
    0 & 1/2a & 1/2b & 0 \\
    0 & 1/2a & -1/2b & 0 \\
    0 & 0 & 0 & 1
    \end{bmatrix} \left\{ \begin{matrix} 
    \mathrm{d} \theta \\ \mathrm{d} a \\
    \mathrm{d} b \\ \mathrm{d} g
    \end{matrix} \right\}
\end{equation}
where the above matrix establishes $\mathrm{d} \boldsymbol{\varepsilon} ( \boldsymbol{\lambda} ) / \mathrm{d} \boldsymbol{\lambda}$ in Eqn.~(\ref{hypoelastic}) for a membrane, with $\mathrm{d} \boldsymbol{\lambda}$ being the vector to the right of this matrix, and $\mathrm{d} \boldsymbol{\varepsilon}$ being the vector on the left-hand side.

The matrix of tangent moduli $\mathbf{M} ( \boldsymbol{\sigma} , \boldsymbol{\varepsilon} )$ given in Eqn.~(\ref{hypoelastic}) is Eqn.~(\ref{HelmholtzMembraneODEs}) for the case of a membrane, whose moduli are: an areal modulus $M(\theta , \pi , \xi)$, a squeeze modulus $N( \sigma , \varepsilon )$, and a shear modulus $G( \tau , \gamma )$.  The vector being operated on by this matrix of tangent moduli is $\mathrm{d} \boldsymbol{\varepsilon}$ above, with the constitutive equation returning a vector $\mathrm{d} \boldsymbol{\sigma} = \{ \mathrm{d} \eta , \mathrm{d} \pi , \mathrm{d} \sigma , \mathrm{d} \tau \}^{\mathsf{T}}$.  The numerical method presented in \S\ref{sec:1stOrderPECE} can then be called upon to integrate this hypo-elastic constitutive equation for its thermo\-dynamic stresses $\boldsymbol{\sigma}$.  

After this constitutive equation has been integrated, the ensuing thermo\-dynamic stress attributes map into components $\mathcal{S}_{ij}$ of a stress tensor $\boldsymbol{\mathcal{S}} = \mathcal{S}_{ij} \, \vec{\mathbfsf{e}}_i \otimes \vec{\mathbfsf{e}}_j$ evaluated in the basis $( \vec{\mathbfsf{e}}_1 , \vec{\mathbfsf{e}}_2 )$ of a membrane; specifically,
\begin{equation}
   \left\{ \begin{matrix}
   \eta \\ \mathcal{S}_{11} \\ \mathcal{S}_{22} \\ \mathcal{S}_{12} = \mathcal{S}_{21}
   \end{matrix} \right\} = \begin{bmatrix}
   1 & 0 & 0 & 0 \\
   0 & 1/2 & 1/2 & 0 \\
   0 & 1/2 & -1/2 & 0 \\
   0 & 0 & 0 & b / a
   \end{bmatrix}
   \left\{ \begin{matrix}
   \eta \\ \pi \\ \sigma \\ \tau
   \end{matrix} \right\} .
\end{equation}
These physical stress components $\mathcal{S}_{ij}$ can be pulled back into components $S_{ij}$ belonging to the Lagrangian, second, Piola-Kirchhoff stress $\mathbfsf{S} = S_{ij} \, \vec{\mathbfsf{e}}_i \otimes \vec{\mathbfsf{e}}_j$ or rotated into components $s_{ij}$ belonging to the Eulerian Kirchhoff stress $\mathbfsf{s} = s_{ij} \, \vec{\mathbfsf{e}}_i \otimes \vec{\mathbfsf{e}}_j$ via 
\begin{equation}
    S_{ij} = \mathcal{U}^{-1}_{ik} \mathcal{S}_{k\ell\,} \mathcal{U}^{-1}_{j\ell}
    \quad \text{or} \quad
    s_{ij} = \mathcal{F}_{ik} S_{k\ell} \mathcal{F}_{j\ell} = 
    \mathcal{R}_{ik} \mathcal{S}_{k\ell} \mathcal{R}_{j\ell}
    \label{membraneStresses}
\end{equation}
where $\boldsymbol{\mathcal{R}} = \mathbfsf{F} \hspace{0.5pt} \boldsymbol{\mathcal{U}}^{-1}$ is the Gram rotation that associates with Laplace stretch, viz., $\mathbfsf{F} = \boldsymbol{\mathcal{RU}}$. 

Components $S_{ij}$ belonging to the Lagrangian, second, Piola-Kirchhoff stress $\mathbfsf{S}$ and components $s_{ij}$ belonging to the Eulerian Kirchhoff stress $\boldsymbol{s}$, as established in Eqn.~(\ref{membraneStresses}), are evaluated in a re-indexed co-ordinate system with base vectors $( \vec{\mathbfsf{e}}_1 , \vec{\mathbfsf{e}}_2 )$.  To map these components back into the co-ordinate system of the `user', one must apply the linear transformations
\begin{displaymath}
    P_{ik} S_{k\ell} P_{j\ell}
    \quad \text{and} \quad
    P_{ik} s_{k\ell} P_{j\ell} 
\end{displaymath}
where $\mathbfsf{P}$ is the orthogonal matrix defined in Eqn.~(\ref{membraneRelabling}).

\subsection{3D Formulation}

Consider an incoming deformation gradient $\mathbfsf{F} = \mathcal{F}_{ij} \, \vec{\mathbfsf{E}}_i \otimes \vec{\mathbfsf{E}}_j$, $i, j = 1, 2, 3$, whose components $\mathcal{F}_{ij}$ are evaluated in a re-indexed co-ordinate system with base vectors $( \vec{\mathbfsf{E}}_1 , \vec{\mathbfsf{E}}_2 , \vec{\mathbfsf{E}}_3 )$, as established in \S\ref{reindexing3D}.  It is in this co-ordinate system that the components $\mathcal{U}_{ij}$ of Laplace stretch $\boldsymbol{\mathcal{U}} = \mathcal{U}_{ij} \, \vec{\mathbfsf{E}}_i \otimes \vec{\mathbfsf{E}}_j$ and its inverse $\boldsymbol{\mathcal{U}}^{-1}$ are evaluated (described in Eqn.~\ref{LagrangianLaplaceStretch3D}), their associated stretch attributes $a$, $b$, $c$, $\alpha$, $\beta$, $\gamma$ are determined, and their associated thermo\-dynamic strains $\Xi$, $\varepsilon_1$, $\varepsilon_2$, $\gamma_1$, $\gamma_2$, $\gamma_3$ and differential strain rates $\mathrm{d} \Xi$, $\mathbf{d} \varepsilon_1$, $\mathrm{d} \varepsilon_2$, $\mathrm{d} \gamma_1$, $\mathrm{d} \gamma_2$, $\mathrm{d} \gamma_3$ are established, cf.\ \S\ref{strainsAndStrainRates3D}.

The differential rates of these physical attributes relate to the differential rates of their thermo\-dynamic variables according to
\begin{equation}
    \left\{ \begin{matrix}
    \mathrm{d} \theta \\ \mathrm{d} \Xi \\ \mathrm{d} \varepsilon_1 \\
    \mathrm{d} \varepsilon_2 \\ \mathrm{d} \gamma_1 \\ \mathrm{d} \gamma_2 \\
    \mathrm{d} \gamma_3
    \end{matrix} \right\} = \begin{bmatrix}
    1 & 0 & 0 & 0 & 0 & 0 & 0 \\
    0 & 1/3a & 1/3b & 1/3c & 0 & 0 & 0 \\
    0 & 1/3a & -1/3b & 0 & 0 & 0 & 0 \\
    0 & 0 & 1/3b & -1/3c & 0 & 0 & 0 \\
    0 & 0 & 0 & 0 & 1 & 0 & 0 \\
    0 & 0 & 0 & 0 & 0 & 1 & 0 \\
    0 & 0 & 0 & 0 & 0 & 0 & 1
    \end{bmatrix} \left\{ \begin{matrix}
    \mathrm{d} \theta \\ \mathrm{d} a \\ \mathrm{d} b \\ \mathrm{d} c \\
    \mathrm{d} \alpha \\ \mathrm{d} \beta \\ \mathrm{d} \gamma
    \end{matrix} \right\}
\end{equation}
where the above matrix establishes $\mathrm{d} \boldsymbol{\varepsilon} ( \boldsymbol{\lambda} ) / \mathrm{d} \boldsymbol{\lambda}$ in Eqn.~(\ref{hypoelastic}), with $\mathrm{d} \boldsymbol{\lambda}$ being the vector at the right of this matrix, and $\mathrm{d} \boldsymbol{\varepsilon}$ being the vector on the left-hand side. 

The matrix of tangent moduli $\mathbf{M} ( \boldsymbol{\sigma} , \boldsymbol{\varepsilon} )$ given in Eqn.~(\ref{hypoelastic}) is Eqn.~(\ref{HelmholtzODEs}) for the general case, whose moduli are: a bulk modulus $K(\theta , \Pi , \Xi)$, two squeeze moduli $N_1 = N ( \sigma_1 , \varepsilon_1 )$ and $N_2 = N ( \sigma_2 , \varepsilon_2 )$, and three shear moduli $G_1 = G ( \tau_1 , \gamma_1 )$, $G_2 = G ( \tau_2 , \gamma_2 )$ and $G_3 = G ( \tau_3 , \gamma_3 )$.  The vector being operated on by this matrix of tangent moduli is $\mathrm{d} \boldsymbol{\varepsilon}$ above, with the constitutive equation returning a vector $\mathrm{d} \boldsymbol{\sigma} = \{ \mathrm{d} \eta , \mathrm{d} \Pi , \mathrm{d} \sigma_1 , \mathrm{d} \sigma_2 , \mathrm{d} \tau_1 , \mathrm{d} \tau_2 , \mathrm{d} \tau_3 \}^{\mathsf{T}}$.  The numerical method presented in \S\ref{sec:1stOrderPECE} can be called upon to integrate this hypo-elastic constitutive equation.  

After this constitutive equation has been integrated, the ensuing thermo\-dynamic stress attributes can be mapped into components $\mathcal{S}_{ij}$ belonging to a stress tensor $\boldsymbol{\mathcal{S}} = \mathcal{S}_{ij} \, \vec{\mathbfsf{E}}_i \otimes \vec{\mathbfsf{E}}_j$, $i, j = 1, 2, 3$, evaluated in the re-indexed basis $( \vec{\mathbfsf{E}}_1 , \vec{\mathbfsf{E}}_2 , \vec{\mathbfsf{E}}_3 )$ of our dodecahedron; specifically,
\begin{equation}
\left\{ \begin{matrix}
\eta \\ \mathcal{S}_{11} \\ \mathcal{S}_{22} \\ \mathcal{S}_{33} \\
\mathcal{S}_{23} = \mathcal{S}_{32} \\
\mathcal{S}_{13} = \mathcal{S}_{31} \\
\mathcal{S}_{12} = \mathcal{S}_{21}
\end{matrix} \right\} = \begin{bmatrix}
1 & 0 & 0 & 0 & 0 & 0 & 0 \\
0 & 1/3 & 2/3 & 1/3 & 0 & 0 & 0 \\
0 & 1/3 & -1/3 & 1/3 & 0 & 0 & 0 \\
0 & 1/3 & -1/3 & -2/3 & 0 & 0 & 0 \\
0 & 0 & 0 & 0 & c/b & 0 & 0 \\
0 & 0 & 0 & 0 & 0 & c/a & 0 \\
0 & 0 & 0 & 0 & 0 & \alpha b/a & b/a
\end{bmatrix}
\left\{ \begin{matrix}
\eta \\ \Pi \\ \sigma_1 \\ \sigma_2 \\ \tau_1 \\ \tau_2 \\ \tau_3
\end{matrix} \right\} .
\end{equation}
These physical components $\mathcal{S}_{ij}$ can be pulled back into components $S_{ij}$ belonging to the Lagrangian, second, Piola-Kirchhoff stress $\mathbfsf{S} = S_{ij} \, \vec{\mathbfsf{E}}_i \otimes \vec{\mathbfsf{E}}_j$ or rotated into components $s_{ij}$ belonging to the Eulerian Kirchhoff stress $\mathbfsf{s} = s_{ij} \, \vec{\mathbfsf{E}}_i \otimes \vec{\mathbfsf{E}}_j$ via 
\begin{equation}
S_{ij} = \mathcal{U}^{-1}_{ik} \mathcal{S}_{k\ell\,} \mathcal{U}^{-1}_{j\ell}
\quad \text{or} \quad
s_{ij} = \mathcal{F}_{ik} S_{k\ell} \mathcal{F}_{j\ell} = 
\mathcal{R}_{ik} \mathcal{S}_{k\ell} \mathcal{R}_{j\ell}
\end{equation}
where $\boldsymbol{\mathcal{R}} = \mathbfsf{F} \hspace{0.5pt} \boldsymbol{\mathcal{U}}^{-1}$ is the Gram rotation that associates with Laplace stretch, viz., $\mathbfsf{F} = \boldsymbol{\mathcal{RU}}$.

Components $S_{ij}$ belonging to the Lagrangian, second, Piola-Kircchoff stress $\mathbfsf{S}$ and components $s_{ij}$ belonging to the Eulerian Kirchhoff stress $\boldsymbol{s}$, as established in Eqn.~(\ref{membraneStresses}), are evaluated in a re-indexed co-ordinate system with base vectors $( \vec{\mathbfsf{E}}_1 , \vec{\mathbfsf{E}}_2 , \vec{\mathbfsf{E}}_3 )$.  To map these components back into the co-ordinate system of the `user', i.e., $( \vec{\boldsymbol{\imath}} , \vec{\boldsymbol{\jmath}} , \vec{\boldsymbol{k}} )$, one must apply the linear transformations
\begin{displaymath}
P_{ik} S_{k\ell} P_{j\ell}
\quad \text{and} \quad
P_{ik} s_{k\ell} P_{j\ell} 
\end{displaymath}
where $\mathbfsf{P}$ is an orthogonal matrix defined in \S\ref{reindexing3D}.

\section{Code Verification and Constitutive Parameterization}
\label{secCE_verifyCode}

All material parameters are assigned from their respective statistical distributions.  For example, there are thirty chords in a dodecahedron comprised of collagen and elastin fibers that are loaded in parallel.  Figure~\ref{figStressStrainFibers} presents a typical set of stress\slash strain response curves for the chords of a dodecahedron whose parameters are displayed in Table~\ref{tabStressStrainFibers}.

\begin{figure}
    \centering
    \includegraphics[width=0.45\textwidth]{figures/collagenStressStrain.jpg}
    \includegraphics[width=0.45\textwidth]{figures/elastinStressStrain.jpg}
    \caption{Typical stress\slash strain curves for collagen (left) and elastin (right) fibers that make up a septal chord.  Their material parameters have been assigned via statistical distributions.}
    \label{figStressStrainFibers}
\end{figure}

\begin{table}
    \centering
    \begin{tabular}{|l|l|l|l|}
        \hline
        \multicolumn{2}{|c|}{Collagen$\vphantom{|^{|^|}}$} & 
        \multicolumn{2}{|c|}{Elastin} \\ \hline
        $E_1^c$ \hfill [barye] & $5.0 \times 10^{5^{\vphantom{|}}} \pm 2.0 \times 10^5$ &  
        $E_1^e$ \hfill [barye] & $2.3 \times 10^6 \pm 1.0 \times 10^6$ \\
        $E_2^c$ \hfill [barye] & $5.0 \times 10^7 \pm 1.0 \times 10^7$ &  
        $E_2^e$ \hfill [barye] & $1.0 \times 10^7 \pm 2.0 \times 10^6$ \\
        $e^c_t$ & $0.09 \pm 0.015$ &
        $e^e_t$ & $0.4 \pm 0.1$ \\ 
        \hline
    \end{tabular}
    \caption{Physical properties for hydrated collagen and elastin fibers when described with the fiber model of Freed \& Rajagopal \cite{FreedRajagopal16}.}
    \label{tabStressStrainFibers}
\end{table}

Recently, Birzle \textit{et~al}.\ \cite{Birzleetal19} performed experiments on thin slices of rat parenchyma loaded in tension where they removed the collagen and\slash or elastin fiber through collagenase and elastase baths to study their individual behaviors and their interactions under load.


\newpage
\setcounter{equation}{0}
\setcounter{figure}{0}
\setcounter{section}{0}
\setcounter{table}{0}
\part{Numerical Integrators}
\label{partNumericalMethods}

This analysis tool, which models alveolar geometry as a dodecahedron, requires numerical methods for the temporal integration of its constitutive equations (systems of first-order ODEs) and their governing equations of motion (systems of second-order ODEs), and for the spatial integrations of: length of line, area of surface, and volume of space that pertain to the various finite-element geometries required.  Results obtained at the Guass points need to be mapped out to their nodal locations, so extrapolation procedures that are consistent with the shape (interpolation) functions used are derived for the various elements and quadratures selected.

\section{ODE Solvers}

The constitutive equations used to describe our alveolar model present themselves as ordinary differential equations that need to be integrated, cf.\ \S\ref{secFE_CE}.  To this end, we employ the PECE (Predict, Evaluate, Correct, re-Evaluate) algorithms of Freed \cite{Freed17a} suitable for solving stiff systems of first- and second-order differential equations.  These methods are based upon Gear's well-known, second-order, backward, difference formula (BDF2) that appears in Eqns.~(\ref{1stOrderCorrector} \& \ref{velocityCorrector}) below.

Time $t$ is considered to be the independent variable, discretized over an interval in time $[t_0, t_N]$ for which $N$ solutions are to be extracted at nodes $n=1, 2, \ldots, N$ spaced at uniform intervals in time with a common step size of $\mathrm{d}t = (t_N - t_0)/N$ separating them, where time $t_0$ associates with the initial condition.

\subsection{PECE Solver for First-Order ODEs}
\label{sec:1stOrderPECE}

Let $\mathbf{x}(t)$ be a vector of independent control variables described in terms of time $t$, and let $\mathbf{y} (\mathbf{x})$ be a vector of dependent response variables obeying a differential equation of evolution $\dot{\mathbf{y}} = \mathbf{f} (\mathbf{x}, \mathbf{y}) \, \dot{\mathbf{x}}$, or equivalently $\mathrm{d} \mathbf{y} = \mathbf{f} (\mathbf{x}, \mathbf{y}) \, \dot{\mathbf{x}} \, \mathrm{d} t = \mathbf{f} (\mathbf{x}, \mathbf{y}) \, \mathrm{d}\mathbf{x}$, subject to an initial condition $\mathbf{y}_0 = \mathbf{y}(\mathbf{x}_0)$ where $\mathbf{x}_0 = \mathbf{x} (t_0)$ with matrix $\mathbf{f} (\mathbf{x}, \mathbf{y})$ establishing the constitutive response for the system.

The two-step method put forward here incrementally solves such an ODE, returning solutions associated with the next moment in time $t_{n+1}$, i.e., it acquires $\mathbf{y}_{n+1}$, given knowledge of the  previous $\mathbf{y}_{n-1}$ and current $\mathbf{y}_n$ solutions plus their rates $\dot{\mathbf{y}}_{n-1}$ and $\dot{\mathbf{y}}_n$, with the corrector also depending upon $\dot{\mathbf{y}}_{n+1}$; consequently, the corrector is an implicit method, which is the source of the method's stability properties.

\subsubsection{Start-Up Algorithm}

Multi-step methods are not self starting.  As such, Heun's method (a forward-Euler predictor with a trapezoidal corrector) is used to start this integrator; specifically,
\begin{subequations}
    \label{startUp1stOrderODEs}
    \begin{align}
    \mbox{} & \text{Predict} & 
    \mathbf{y}_1^p & = \mathbf{y}_0 + \dot{\mathbf{y}}_0 \, \mathrm{d}t + 
    \mathcal{O} \bigl( (\mathrm{d}t)^2 \bigr)
    \label{startUp1stOrderPredictor} \\
    \mbox{} & \text{Evaluate} & 
    \dot{\mathbf{y}}^p_1 & = \mathbf{f} (\mathbf{x}_1 , \mathbf{y}_1^p) \, 
    \dot{\mathbf{x}}_1
    \label{startUp1stEvaluate} \\
    \mbox{} & \text{Correct} &
    \mathbf{y}_1 & = \mathbf{y}_0 + \tfrac{1}{2} 
    \bigl( \dot{\mathbf{y}}_1^p + \dot{\mathbf{y}}_0 \bigr) \mathrm{d}t + 
    \mathcal{O} \bigl( (\mathrm{d}t)^3 \bigr)
    \label{startUp1stOrderCorrector} \\
    \mbox{} & \text{re-Evaluate} & 
    \dot{\mathbf{y}}_1 & = \mathbf{f} (\mathbf{x}_1 , \mathbf{y}_1) \,
    \dot{\mathbf{x}}_1 
    \label{startUp1stReEvaluate}
    \end{align}
\end{subequations}
wherein $\dot{\mathbf{y}}_0 = \mathbf{f}(\mathbf{x}_0, \mathbf{y}_0) \, \dot{\mathbf{x}}_0$.  The predictor is the forward Euler method, while the corrector is the trapezoidal rule.  The order of accuracy for a method (the exponent on $\mathrm{d}t$ in $\mathcal{O}$), as they appear in the above big $\mathcal{O}$ operators, pertains to a single step of integration.  The overall order of the method, when integrated over a sequence of steps, is one less than the exponent inside the $\mathcal{O}$ operator.  Therefore, Euler's method is first-order accurate, and Heun's method is second-order accurate.

\subsubsection{Two-Step ODE Solver}

The two-step method of Freed \cite{Freed17a} for solving first-order ODEs is
\begin{subequations}
    \label{1stOrderODEs}
    \begin{align}
    \mbox{} & \text{Predict} & 
    \mathbf{y}_{n+1}^p & = \tfrac{1}{3} 
    \bigl( 4 \mathbf{y}_n - \mathbf{y}_{n-1} \bigr) + 
    \tfrac{2}{3} \bigl( 2 \dot{\mathbf{y}}_n - \dot{\mathbf{y}}_{n-1} 
    \bigr) \mathrm{d}t + \mathcal{O} \bigl( (\mathrm{d}t)^3 \bigr)
    \label{1stOrderPredictor} \\
    \mbox{} & \text{Evaluate} & 
    \dot{\mathbf{y}}^p_{n+1} & = \mathbf{f} (\mathbf{x}_{n+1} , \mathbf{y}_{n+1}^p) \, \dot{\mathbf{x}}_{n+1}
    \label{1stOrderEvaluate} \\
    \mbox{} & \text{Correct} &
    \mathbf{y}_{n+1} & = \tfrac{1}{3} 
    \bigl( 4 \mathbf{y}_n - \mathbf{y}_{n-1} \bigr) + 
    \tfrac{2}{3} \dot{\mathbf{y}}^{p}_{n+1} \mathrm{d}t + 
    \mathcal{O} \bigl( (\mathrm{d}t)^3 \bigr)
    \label{1stOrderCorrector} \\
    \mbox{} & \text{re-Evaluate} & 
    \dot{\mathbf{y}}_{n+1} & = \mathbf{f} (\mathbf{x}_{n+1} , \mathbf{y}_{n+1}) \, 
    \dot{\mathbf{x}}_{n+1}
    \label{1stOrderReEvaluate}
    \end{align}
\end{subequations} 
whose corrector is the well-known BDF2 formula made popular by Gear, for which Freed has provided a predictor.  This method is second-order accurate in both its predictor and corrector.

Both the predictor and corrector of this PECE scheme have a solution $\mathbf{y}$ with a weight of 1, and a rate $\dot{\mathbf{y}}$ with a weight of $\tfrac{2}{3} \mathrm{d}t$; hence, this predictor\slash corrector pair is internally consistent, i.e., the predictor and corrector will produce the same result whenever they integrate over a constant $\dot{\mathbf{y}}$ field. 

The correct\slash re-evaluate (CE) steps of a PECE method are often iterated over until a convergence criterion is satisfied.  Such methods are typically denoted as $\text{PE}(\text{CE})^m$, where $m$ specifies the number of iterations imposed.


\subsection{A Relevant Example}

In our finite element implementation, a hypo-elastic material model \cite{Truesdell55} is introduced to describe the constitutive response of an alveolus whereby
\begin{displaymath}
    \dot{\boldsymbol{\sigma}} = \mathbf{M} ( \boldsymbol{\epsilon} , \boldsymbol{\sigma} ) \, \dot{\boldsymbol{\epsilon}} 
    \quad \text{or equivalently \cite{Noll55}} \quad
    \mathrm{d} \boldsymbol{\sigma} = \mathbf{M} ( \boldsymbol{\epsilon} , \boldsymbol{\sigma} ) \, \mathrm{d} \boldsymbol{\epsilon}
\end{displaymath} 
where $\boldsymbol{\epsilon}$ is a vector of thermo\-dynamic strains, $\boldsymbol{\sigma}$ is a vector of thermo\-dynamic stresses, and $\mathbf{M}$ is a square matrix comprised of their tangent moduli, which can depend upon both stress and strain in our application; specifically,
\begin{displaymath}
   \boldsymbol{\sigma}_{1D} = \{ \eta , s \}^{\mathsf{T}} , \quad
   \boldsymbol{\sigma}_{2D} = \{ \eta , s^{\pi} , s^{\sigma} , s^{\tau} \}^{\mathsf{T}} , \quad
   \boldsymbol{\sigma}_{3D} = \{ \eta , \Pi , \sigma_1 , \sigma_2 , \tau_1 , \tau_2 , \tau_3 \}^{\mathsf{T}}
\end{displaymath}
where $\eta$ is entropy and the rest of its constituents are stress attributes.  Their thermo\-dynamic conjugates are the control variables
\begin{displaymath}
\boldsymbol{\epsilon}_{1D} = \{ \theta , e \}^{\mathsf{T}} , \quad
\boldsymbol{\epsilon}_{2D} = \{ \theta , \xi , \varepsilon , \gamma \}^{\mathsf{T}} , \quad
\boldsymbol{\epsilon}_{3D} = \{ \theta , \Xi , \varepsilon_1 , \varepsilon_2 , \gamma_1 , \gamma_2 , \gamma_3 \}^{\mathsf{T}}
\end{displaymath}
where $\theta$ is temperature and the rest of its constituents are strain attributes.  In the 2- and 3-D cases, these stress\slash strain attributes arise from Gram-Schmidt decompositions of their respective deformation gradients (cf.\ Sections~\ref{secQR2D}, \ref{secConjugatePairs} and \ref{secFE_CE}).  Constructing tangent moduli $\mathbf{M} ( \boldsymbol{\epsilon} , \boldsymbol{\sigma} )$ is the topic of Part~\ref{partConstitutive}.  Both $\boldsymbol{\sigma}$ and $\mathrm{d} \boldsymbol{\sigma} = \mathbf{M} \, \mathrm{d} \boldsymbol{\epsilon}$ arise in the construction of our stiffness matrices, cf.\ \S\ref{secStiffnessMatrices}.

Equation (\ref{startUp1stOrderODEs}) is used to take the first step of integration; specifcally, 
\begin{subequations}
    \notag
    \begin{align}
    \mbox{} & \text{Predict} & 
    \boldsymbol{\sigma}_1^p & = \boldsymbol{\sigma}_0 + \dot{\boldsymbol{\sigma}}_0 \, \mathrm{d}t \\
    \mbox{} & \text{Evaluate} & 
    \dot{\boldsymbol{\sigma}}^p_1 & = \mathbf{M} ( \boldsymbol{\epsilon} , \boldsymbol{\sigma}_1^p ) \, \dot{\boldsymbol{\epsilon}}_1 \\
    \mbox{} & \text{Correct} &
    \boldsymbol{\sigma}_1 & = \boldsymbol{\sigma}_0 + \tfrac{1}{2} 
    \bigl( \dot{\boldsymbol{\sigma}}_1^p + 
    \dot{\boldsymbol{\sigma}}_0 \bigr) \mathrm{d} t \\
    \mbox{} & \text{re-Evaluate} & 
    \dot{\boldsymbol{\sigma}}_1 & = \mathbf{M} ( \boldsymbol{\epsilon}_1 , \boldsymbol{\sigma}_1 ) \, \dot{\boldsymbol{\epsilon}}_1
    \end{align}
\end{subequations}
where $\dot{\boldsymbol{\sigma}}_0 = \mathbf{M} ( \boldsymbol{\epsilon}_0 , \boldsymbol{\sigma}_0 ) \, \dot{\boldsymbol{\epsilon}}_0$.  The remaining steps of integration follow according to Eqn.~(\ref{1stOrderODEs}); specifically,
\begin{subequations}
    \notag
    \begin{align}
    \mbox{} & \text{Predict} & 
    \boldsymbol{\sigma}_{n+1}^p & = \tfrac{1}{3} 
    \bigl( 4 \boldsymbol{\sigma}_n - \boldsymbol{\sigma}_{n-1} \bigr) + 
    \tfrac{2}{3} \bigl( 2 \, \dot{\boldsymbol{\sigma}}_n - 
    \dot{\boldsymbol{\sigma}}_{n-1} \bigr) \mathrm{d} t \\
    \mbox{} & \text{Evaluate} & 
    \dot{\boldsymbol{\sigma}}^p_{n+1} & = \mathbf{M} ( \boldsymbol{\epsilon}_{n+1} , \boldsymbol{\sigma}_{n+1}^p ) \, \dot{\boldsymbol{\epsilon}}_{n+1} \\
    \mbox{} & \text{Correct} &
    \boldsymbol{\sigma}_{n+1} & = \tfrac{1}{3} 
    \bigl( 4 \boldsymbol{\sigma}_n - \boldsymbol{\sigma}_{n-1} \bigr) + 
    \tfrac{2}{3} \, \dot{\boldsymbol{\sigma}}^{p}_{n+1} \, \mathrm{d}t \\
    \mbox{} & \text{re-Evaluate} & 
    \dot{\boldsymbol{\sigma}}_{n+1} & = \mathbf{M} ( \boldsymbol{\epsilon}_{n+1} , 
    \boldsymbol{\sigma}_{n+1} ) \, \dot{\boldsymbol{\epsilon}}_{n+1}
    \end{align}
\end{subequations} 
whose strain rates $\dot{\boldsymbol{\epsilon}}$ are computed according to \S\ref{secFE_CE}.


\subsection{PECE Solver for Second-Order ODEs}
\label{sec:2ndOrderPECE}

Now let $\mathbf{u}$ denote a vector of dependent variables obeying a differential equation of evolution $\mathrm{d}^2 \mathbf{u}(t) / \mathrm{d} t^2 = \ddot{\mathbf{u}} = \mathbf{f} (t, \mathbf{u}, \dot{\mathbf{u}})$ subjected to the pair of initial conditions $\mathbf{u}_0 = \mathbf{u}(t_0)$ and $\dot{\mathbf{u}}_0 = \dot{\mathbf{u}}(t_0)$.  One may think of $\mathbf{u}$ as being displacements whose rates $\dot{\mathbf{u}}$ are velocities $\mathbf{v}$, with $\ddot{\mathbf{u}} = \dot{\mathbf{v}}$ representing their accelerations $\mathbf{a}$. 

The two-step method put forward here incrementally solves such an ODE, returning solutions associated with the next moment in time $t_{n+1}$ for both displacement $\mathbf{u}_{n+1}$ and velocity $\dot{\mathbf{u}}_{n+1}$.  To update the displacement to $\mathbf{u}_{n+1}$, the predictor requires knowledge of the previous fields $\mathbf{u}_{n-1}$, $\dot{\mathbf{u}}_{n-1}$ and $\ddot{\mathbf{u}}_{n-1}$ plus the current fields $\mathbf{u}_n$, $\dot{\mathbf{u}}_n$ and $\ddot{\mathbf{u}}_n$, with the corrector also requiring knowledge of $\dot{\mathbf{u}}_{n+1}$ and $\ddot{\mathbf{u}}_{n+1}$.  Likewise, to update the velocity to $\dot{\mathbf{u}}_{n+1}$, the predictor requires knowledge of the previous fields $\dot{\mathbf{u}}_{n-1}$ and $\ddot{\mathbf{u}}_{n-1}$ plus the current fields $\dot{\mathbf{u}}_n$ and $\ddot{\mathbf{u}}_n$, with the corrector also requiring knowledge of $\ddot{\mathbf{u}}_{n+1}$.  Both predictors are explicit, and both correctors are implicit.  It is this implicit quality that provides numeric stability for the integrator.

\subsubsection{Start-Up Algorithm}

Again, multi-step methods are not self starting, so a one-step method is needed to take the first step of integration; specifically, we employ
\begin{subequations}
    \label{pairedStartUp}
    \begin{align}
    \mbox{} & \text{Predict} & 
    \mathbf{u}_1^p & = \mathbf{u}_0 + \dot{\mathbf{u}}_0 \, \mathrm{d}t +
    \tfrac{1}{2} \ddot{\mathbf{u}}_0 (\mathrm{d}t)^2 + \mathcal{O} \bigl( ( \mathrm{d}t )^3 \bigr)
    \label{startupDisplacementPredictor} \\
    \mbox{} & &
    \dot{\mathbf{u}}^p_1 & = 
    \dot{\mathbf{u}}_0 + \ddot{\mathbf{u}}_0 \, \mathrm{d}t + 
    \mathcal{O} \bigl( ( \mathrm{d}t )^2 \bigr) 
    \label{startUpVelocityPredictor} \\
    \mbox{} & \text{Evaluate} &
    \ddot{\mathbf{u}}^p_1 & = \mathbf{f} (t_1, \mathbf{u}^p_1, \dot{\mathbf{u}}^p_1)
    \label{startUpEvaluate} \\
    \mbox{} & \text{Correct} &
    \mathbf{u}_1 & = \mathbf{u}_0 + \tfrac{1}{2} 
    ( \dot{\mathbf{u}}^p_1 + \dot{\mathbf{u}}_0 ) \mathrm{d}t -
    \tfrac{1}{12} ( \ddot{\mathbf{u}}^p_1 - 
    \ddot{\mathbf{u}}_0 ) (\mathrm{d}t)^2 + \mathcal{O} \bigl( (\mathrm{d}t)^4 \bigr) 
    \label{startupDisplacementCorrector} \\
    \mbox{} & &
    \dot{\mathbf{u}}_1 & = \dot{\mathbf{u}}_0 + \tfrac{1}{2} 
    ( \ddot{\mathbf{u}}_1^p + \ddot{\mathbf{u}}_0 ) \mathrm{d}t + 
    \mathcal{O} \bigl( (\mathrm{d}t)^3 \bigr)
    \label{startUpVelocityCorrector} \\
    \mbox{} & \text{re-Evaluate} &
    \ddot{\mathbf{u}}_1 & = \mathbf{f} (t_1, \mathbf{u}_1, \dot{\mathbf{u}}_1) 
    \label{startUpReEvaluate}
    \end{align}
\end{subequations}
wherein $\ddot{\mathbf{u}}_0 = \mathbf{f}(t_0, \mathbf{u}_0, \dot{\mathbf{u}}_0)$ and $t_1 = t_0 + \mathrm{d}t$. 


\subsubsection{Two-Step ODE Solver}

The two-step method of Freed \cite{Freed17a} for solving second-order ODEs is
\begin{subequations}
    \label{pairedMethods}
    \begin{align}
    \mbox{} & \text{Predict} &
    \mathbf{u}_{n+1}^p & = \tfrac{1}{3} (
    4 \mathbf{u}_n - \mathbf{u}_{n-1} ) + 
    \tfrac{1}{6} ( 3 \dot{\mathbf{u}}_n + 
    \dot{\mathbf{u}}_{n-1} ) \mathrm{d}t \notag \\ 
    \mbox{} & & & \qquad + 
    \tfrac{1}{36} ( 31 \ddot{\mathbf{u}}_n - 
    \ddot{\mathbf{u}}_{n-1} ) (\mathrm{d}t)^2 + 
    \mathcal{O} \bigl( (\mathrm{d}t)^4 \bigr) 
    \label{displacementPredictor} \\
    \mbox{} & &
    \dot{\mathbf{u}}_{n+1}^p & = \tfrac{1}{3} 
    ( 4 \dot{\mathbf{u}}_n - \dot{\mathbf{u}}_{n-1} ) + 
    \tfrac{2}{3} ( 2\ddot{\mathbf{u}}_n - \ddot{\mathbf{u}}_{n-1} )
    \mathrm{d}t + \mathcal{O} \bigl( (\mathrm{d}t)^3 \bigr)
    \label{velocityPredictor} \\
    \mbox{} & \text{Evaluate} &
    \ddot{\mathbf{u}}^p_{n+1} & = \mathbf{f} (t_{n+1}, \mathbf{u}^p_{n+1}, \dot{\mathbf{u}}^p_{n+1}) 
    \label{2ndEvaluate} \\
    \mbox{} & \text{Correct} & 
    \mathbf{u}_{n+1} & = \tfrac{1}{3} (
    4  \mathbf{u}_n - \mathbf{u}_{n-1} ) +
    \tfrac{1}{24} ( \dot{\mathbf{u}}^p_{n+1} +
    14 \dot{\mathbf{u}}_n + \dot{\mathbf{u}}_{n-1} ) \mathrm{d}t 
    \notag \\
    \mbox{} & & & \qquad +
    \tfrac{1}{72} ( 10 \ddot{\mathbf{u}}^p_{n+1} + 
    51 \ddot{\mathbf{u}}_n - \ddot{\mathbf{u}}_{n-1} ) (\mathrm{d}t)^2 + 
    \mathcal{O} \bigl( (\mathrm{d}t)^4 \bigr)
    \label{displacementCorrector} \\ 
    \mbox{} & &
    \dot{\mathbf{u}}_{n+1} & = \tfrac{1}{3} 
    ( 4 \dot{\mathbf{u}}_n - \dot{\mathbf{u}}_{n-1} ) + 
    \tfrac{2}{3} \ddot{\mathbf{u}}^p_{n+1} \, \mathrm{d}t + 
    \mathcal{O} \bigl( (\mathrm{d}t)^3 \bigr)
    \label{velocityCorrector} \\
    \mbox{} & \text{re-Evaluate} & 
    \ddot{\mathbf{u}}_{n+1} & = \mathbf{f} (t_{n+1}, \mathbf{u}_{n+1}, \dot{\mathbf{u}}_{n+1})
    \label{2ndReEvaluate}
    \end{align}
\end{subequations}
which is a second-order method for integrating velocities $\dot{\mathbf{u}}$, and a third-order method for integrating displacements $\mathbf{u}$.  

This PECE solver for velocity $\dot{\mathbf{u}}$ has a predictor and a corrector, i.e., Eqns.~(\ref{velocityPredictor} \& \ref{velocityCorrector}), that are the same as those of method~(\ref{1stOrderPredictor} \& \ref{1stOrderCorrector}), and as such, this predictor\slash corrector pair for integrating velocity is consistent.  Likewise, in both the predictor and corrector for displacement $\mathbf{u}$, contributions from the solution $\mathbf{u}$ have a weight of 1, contributions from the velocities $\dot{\mathbf{u}}$ have a weight of $\tfrac{2}{3} \mathrm{d}t$, and contributions from the accelerations $\ddot{\mathbf{u}}$ have a weight of $\tfrac{5}{6} (\mathrm{d}t)^2$; hence, this predictor\slash corrector pair is internally consistent, too.

\subsection{A Relevant Example}
\label{sec:solve2ndOrderODE}

The finite element problem that we consider here requires solutions for the second-order ODE\footnote{
    This solver can also be used to get solutions for the system of equations $\mathbf{M} \ddot{\mathbf{u}} + \mathbf{C} \dot{\mathbf{u}} + \mathbf{K} \mathbf{u} = \mathbf{f}(t)$ wherein $\mathbf{M}$ is a mass matrix, $\mathbf{C}$ is a damping matrix, $\mathbf{K}$ is a stiffness matrix, and $\mathbf{f}$ is a forcing function.
}
\begin{displaymath}
    \mathbf{M} \ddot{\mathbf{u}} + \mathbf{K}\mathbf{u} = \mathbf{f}(t)
\end{displaymath}
where $\mathbf{u}$ is a generalized displacement vector, $\ddot{\mathbf{u}}$ is its acceleration, $\mathbf{M}$ and $\mathbf{K}$ are mass and stiffness matrices, and $\mathbf{f}(t)$ is a forcing function evaluated at current time $t$.  For this system of ODEs, the first step to be taken follows algorithm (\ref{pairedStartUp}) and is implemented as
\begin{subequations}
    \notag
    \begin{align}
    \mbox{} & \text{Predict} & 
    \mathbf{u}_1^p & = \mathbf{u}_0 + \dot{\mathbf{u}}_0 \, \mathrm{d}t +
    \tfrac{1}{2} \ddot{\mathbf{u}}_0 ( \mathrm{d}t )^2 \\
    \mbox{} & &
    \dot{\mathbf{u}}^p_1 & = \dot{\mathbf{u}}_0 + \ddot{\mathbf{u}}_0 \, \mathrm{d}t \\
    \mbox{} & \text{Evaluate} &
    \ddot{\mathbf{u}}^p_1 & = \mathbf{M}^{-1} \bigl( \mathbf{f}(t_{1} ) - 
    \mathbf{K} \mathbf{u}_1^p \bigr) \\
    \mbox{} & \text{Correct} &
    \mathbf{u}_1 & = \mathbf{u}_0 + \tfrac{1}{2} 
    ( \dot{\mathbf{u}}^p_1 + \dot{\mathbf{u}}_0 ) \mathrm{d}t -
    \tfrac{1}{12} ( \ddot{\mathbf{u}}^p_1 - \ddot{\mathbf{u}}_0 ) 
    ( \mathrm{d}t )^2 \\
    \mbox{} & &
    \dot{\mathbf{u}}_1 & = \dot{\mathbf{u}}_0 + \tfrac{1}{2}  
    ( \ddot{\mathbf{u}}_1^p + \ddot{\mathbf{u}}_0 ) \mathrm{d}t \\
    \mbox{} & \text{re-Evaluate} &
    \ddot{\mathbf{u}}_1 & = \mathbf{M}^{-1} \bigl( \mathbf{f}(t_1 ) - 
    \mathbf{K} \mathbf{u}_1 \bigr)
    \end{align}
\end{subequations}
with continued steps being governed by algorithm (\ref{pairedMethods}), which takes on the form of
\begin{subequations}
    \notag
    \begin{align}
    \mbox{} & \text{Predict} &
    \mathbf{u}_{n+1}^p & = \tfrac{1}{3} (
    4 \mathbf{u}_n - \mathbf{u}_{n-1} ) + 
    \tfrac{1}{6} ( 3 \dot{\mathbf{u}}_n + 
    \dot{\mathbf{u}}_{n-1} ) \mathrm{d}t \\ & & & \qquad + 
    \tfrac{1}{36} ( 31 \ddot{\mathbf{u}}_n - 
    \ddot{\mathbf{u}}_{n-1} ) ( \mathrm{d}t )^2 \\
    \mbox{} & &
    \dot{\mathbf{u}}_{n+1}^p & = \tfrac{1}{3} 
    ( 4 \dot{\mathbf{u}}_n - \dot{\mathbf{u}}_{n-1} ) + 
    \tfrac{2}{3} ( 2 \ddot{\mathbf{u}}_n - \ddot{\mathbf{u}}_{n-1} ) \mathrm{d}t \\
    \mbox{} & \text{Evaluate} &
    \ddot{\mathbf{u}}^p_{n+1} & = \mathbf{M}^{-1} \bigl( \mathbf{f}(t_{n+1} ) - 
    \mathbf{K} \mathbf{u}_{n+1}^p \bigr) \\
    \mbox{} & \text{Correct} & 
    \mathbf{u}_{n+1} & = \tfrac{1}{3} (
    4  \mathbf{u}_n - \mathbf{u}_{n-1} ) +
    \tfrac{1}{24} ( \dot{\mathbf{u}}^p_{n+1} +
    14 \dot{\mathbf{u}}_n + \dot{\mathbf{u}}_{n-1} ) \mathrm{d}t  \\
    \mbox{} & & & \qquad +
    \tfrac{1}{72} ( 10 \ddot{\mathbf{u}}^p_{n+1} + 
    51 \ddot{\mathbf{u}}_n - \ddot{\mathbf{u}}_{n-1} ) ( \mathrm{d}t )^2 \\ 
    \mbox{} & &
    \dot{\mathbf{u}}_{n+1} & = \tfrac{1}{3} 
    ( 4 \dot{\mathbf{u}}_n - \dot{\mathbf{u}}_{n-1} ) + 
    \tfrac{2}{3} \ddot{\mathbf{u}}^p_{n+1} \, \mathrm{d}t \\
    \mbox{} & \text{re-Evaluate} & 
    \ddot{\mathbf{u}}_{n+1} & = \mathbf{M}^{-1} \bigl( \mathbf{f}(t_{n+1} ) - 
    \mathbf{K} \mathbf{u}_{n+1} \bigr) 
    \end{align}
\end{subequations}
where, in this example, velocity $\dot{\mathbf{u}}$ is not needed for the evaluation steps, but it is used by both the prediction and correction steps of integration.

We observe that the mass matrix must not be ill conditioned in order for this algorithm to work as intended.  In those cases where the mass matrix does not change with time, it will only need to be evaluated and inverted once.  This is an advantage over using the popular Newmark \cite{Newmark59} integrator, where matrix evaluation and inversion is required at every step along its solution paths.

A small amount of damping is often introduced into finite element problems of this type, i.e., a damping matrix $\mathbf{C}$ is introduced so that $\mathbf{M}\ddot{\mathbf{u}} + \mathbf{Ku} = \mathbf{f}(t)$ becomes $\mathbf{M}\ddot{\mathbf{u}} + \mathbf{C}\dot{\mathbf{u}} + \mathbf{Ku} = \mathbf{f}(t)$, where elements of the damping matrix $\mathbf{C}$ are small compared to those of the stiffness matrix $\mathbf{K}$.  This is done to enhance solution stability.  Presently, it is not known if this gimmick will be required or not for our application.

\section{Quadrature Rules for Spatial Integration}
\label{secGauss}

The quadrature rules of Gauss are usually selected to integrate over individual elements within a finite element model, because this class of methods have integrators with the smallest errors of approximation.   All integrations occur in their natural co-ordinate systems.   Four sets of Gauss quadrature rules are cataloged here: for 1D rods, for 2D triangles, for 2D pentagons, and for 3D tetrahedra.  Formul\ae\ presented in the following tables integrate $1^{\text{st}}$, $3^{\text{rd}}$ and $5^{\text{th}}$ order polynomials exactly in their respective geometries.  Formul\ae\ for the pentagon cannot found elsewhere.  Later, in \S\ref{sec:extrapolation}, quadrature rules are presented where the number of Gauss points equals the number of nodal points, which are the integrators that we implement.

\subsection{Gauss Integration Along a Rod}

Quadrature rules that integrate a 1D chord in its natural co-ordinate system, which spans the interval $-1 \leq \xi \leq 1$, are presented in Table~\ref{tabQuadrature1D}.  These formul\ae\ are well known and can be found in any standard textbook on finite elements.  Here the integral of some function $f( \xi )$ is approximated via the quadrature rule
\begin{equation}
    \mathcal{I} = \int_{-1}^1 f ( \xi ) \, \mathrm{d} \xi 
    = \sum_{i=1}^n w_i f( \xi_i ) + E
    \quad \text{with} \quad
    \int_{-1}^1 \mathrm{d} \xi = \sum_{i=1}^n w_i = 2
    \label{Gauss1D}
\end{equation}
where $\xi_i$ and $w_i$ are the nodes and weights of integration, respectively, for which there are $n$ pairs, with $E$ denoting the error of approximation.

\begin{table}
    \centering
    \begin{tabular}{|c|rr|}
        \hline
        nodes & \centering $\xi$ co-ordinate \phantom{123}  & 
        weight \phantom{123456} \\ \hline
        & \multicolumn{2}{|c|}{Exact for Polynomials of Degree $1^{\phantom{|^|}}$} \\ 
        \hline
        1 & 0.000000000000000 & 2.000000000000000 \\ 
        \hline
        & \multicolumn{2}{|c|}{Exact for Polynomials of Degree $3^{\phantom{|^|}}$} \\ \hline
        1 & -0.577350269189626 & 1.000000000000000 \\
        2 & 0.577350269189626 & 1.000000000000000 \\ 
        \hline
        & \multicolumn{2}{|c|}{Exact for Polynomials of Degree $5^{\phantom{|^|}}$} \\ \hline
        1 & -0.774596669241483 & 0.555555555555556 \\
        2 & 0.000000000000000 & 0.888888888888889 \\
        3 & 0.774596669241483 & 0.555555555555556 \\ 
        \hline
    \end{tabular}
    \caption{Gauss quadrature weights and nodes for integrating over a line in its natural co-ordinate system $\xi$.  These weights sum to 2, which is the span of its natural co-ordinate.  We note that $\sqrt{3} / 3 \approx 0.577350269189626$ and that $\sqrt{15} / 25 \approx 0.774596669241483$.}
    \label{tabQuadrature1D}
\end{table}


\subsection{Multi-Dimensional Integration}

When integrating some function $f$ over, say, a square, one often introduces tensor products of the 1D formul\ae\ (\ref{Gauss1D}) whose quadratures are listed in Table~\ref{tabQuadrature1D}; specifically, one might construct a quadrature rule that looks like
\begin{multline*}
    \mathcal{I} = \int_{-1}^1 \int_{-1}^1 f ( \xi , \eta ) \, \mathrm{d} \xi \, \mathrm{d} \eta =
    \int_{-1}^1 \sum_{i=1}^n w_i f ( \xi_i , \eta ) \, \mathrm{d} \eta + E \\ =
    \sum_{i=1}^n \sum_{j=1}^n w_i w_j f ( \xi_i , \eta_ j ) + E  
    \quad \text{with} \quad
    \int_{-1}^1 \int_{-1}^1 \mathrm{d} \xi \, \mathrm{d} \eta = 
    \sum_{i=1}^n \sum_{j=1}^n w_i w_j = 4
\end{multline*}
which is the integration scheme presented in most textbooks on finite elements.  In this approach there is an $n \times n$ grid of nodes that associate with $n^2$ weights.  

The above multi-dimensional formul\ae\ are not optimal, as they require more function evaluations than are usually needed to secure a quadrature rule at some specified order of accuracy.  It is more efficient to adopt non-tensor product quadrature rules where, e.g., when integrating over a square, one would have
\begin{multline*}
    \mathcal{I} = \int_{-1}^1 \int_{-1}^1 f ( \xi , \eta ) \, \mathrm{d} \xi \, \mathrm{d} \eta =
    \sum_{i=1}^n w_i f ( \xi_i , \eta_i ) + E  \\ 
    \quad \text{with} \quad
    \int_{-1}^1 \int_{-1}^1 \mathrm{d} \xi \, \mathrm{d} \eta = 
    \sum_{i=1}^n w_i = 4
\end{multline*}
where co-ordinates $\xi_i$ and $\eta_i$ place the $i^{\text{th}}$ node of integration inside a square of area 4 whose weights of integration are $w_i$, $i=1,2,\ldots,n$.  We employ such methods.

\subsection{Gauss Integration of a Triangle}

A triangle has natural co-ordinates $( \xi , \eta )$ that span regions of $0 \leq \xi \leq 1$ and $0 \leq \eta \leq 1 - \xi$ so that an integral of $f(\xi, \eta)$ becomes
\begin{multline}
\mathcal{I} = \int_0^1 \int_{\eta =0}^{1-\xi} f ( \xi , \eta ) \, \mathrm{d} \eta \, \mathrm{d} \xi =
\sum_{i=1}^n w_i f ( \xi_i , \eta_i ) + E \\
\quad \text{with} \quad
\int_0^1 \int_{\eta =0}^{1-\xi} \mathrm{d} \eta \, \mathrm{d} \xi = 
\sum_{i=1}^n w_i = \frac{1}{2}
\label{GaussTriangle}
\end{multline}
where co-ordinates $\xi_i$ and $\eta_i$ place the $i^{\text{th}}$ node of integration inside a triangle, and whose corresponding weight of integration is $w_i$, with their being $n$ pairs of nodes and weights.  The sum of its weights must equal the area of this triangle, which is \textfrac{1}{2}. Table~\ref{tabQuadrature2D} provides a selection of quadrature rules for such triangles.  This table can be found in some finite element textbooks.

\begin{table}
    \centering
    \begin{tabular}{|c|rrr|}
        \hline
        nodes & \centering $\xi$ co-ordinate \phantom{123} & 
        \centering $\eta$ co-ordinate \phantom{123} &
        weight \phantom{123456} \\ \hline
        & \multicolumn{3}{|c|}{Exact for Polynomials of Degree $1^{\phantom{|^|}}$} \\ 
        \hline
        1 & 0.333333333333333 & 0.333333333333333 & 0.500000000000000 \\ 
        \hline
        & \multicolumn{3}{|c|}{Exact for Polynomials of Degree $3^{\phantom{|^|}}$} \\ \hline
        1 & 0.333333333333333 & 0.333333333333333 & -0.281250000000000 \\
        2 & 0.200000000000000 & 0.600000000000000 & 0.260416666666667 \\ 
        3 & 0.200000000000000 & 0.200000000000000 & 0.260416666666667 \\
        4 & 0.600000000000000 & 0.200000000000000 & 0.260416666666667 \\ 
        \hline
        & \multicolumn{3}{|c|}{Exact for Polynomials of Degree $5^{\phantom{|^|}}$} \\ \hline
        1 & 0.333333333333333 & 0.333333333333333 & 0.112500000000000 \\
        2 & 0.101286507323456 & 0.797426985353087 & 0.062969590272413 \\ 
        3 & 0.101286507323456 & 0.101286507323456 & 0.062969590272413 \\
        4 & 0.797426985353087 & 0.101286507323456 & 0.062969590272413 \\
        5 & 0.470142064105115 & 0.059715871789770 & 0.066197076394253 \\ 
        6 & 0.470142064105115 & 0.470142064105115 & 0.066197076394253 \\
        7 & 0.059715871789770 & 0.470142064105115 & 0.066197076394253 \\
        \hline
    \end{tabular}
    \caption{Symmetric weights and nodes for Gauss quadratures that integrate over a triangle in its natural co-ordinate system $( \xi , \eta )$, where $0 \leq \xi \leq 1$ and $0 \leq \eta \leq 1 - \xi$.  These weights sum to \textfrac{1}{2}, which is the area of a triangle when evaluated in its natural co-ordinate system.}
    \label{tabQuadrature2D}
\end{table}


\subsection{Gauss Integration of a Pentagon}
\label{sec:pentagonQuadrature}

Gauss quadrature rules for a regular pentagon described in its natural co-ordinate system, i.e., oriented according to Fig.~\ref{figRegPentagon}, are presented in Table~\ref{tabQuadrature}, which describe integrations of the type
\begin{multline}
    \mathcal{I} = \iint_{\pentagon} f ( \xi , \eta) \, \mathrm{d} \eta \, \mathrm{d} \xi = 
    \sum_{i=1}^n w_i f ( \xi_i , \eta_i ) + E \\
    \text{with} \quad
    \iint_{\pentagon} \mathrm{d} \eta \, \mathrm{d} \xi = \sum_{i=1}^n w_i =
    A^p = 2.3776412907378837 
\end{multline}
where co-ordinates $\xi_i$ and $\eta_i$ place the $i^{\text{th}}$ node inside a pentagon, and whose corresponding weight of integration is $w_i$, with their being $n$ pairs of nodes and weights.  The sum of these weights must equal the area of this pentagon $A^p = 5 \sin \omega \cos \omega$ where $2 \omega = 108^{\circ}$ is an inside angle of a regular pentagon, here inscribing the unit circle. 

\begin{table}
    \centering
    \begin{tabular}{|c|rrr|}
        \hline
        node & \centering $\xi$ co-ordinate \phantom{123}  & 
        $\eta$ co-ordinate \phantom{123} & weight \phantom{12345} \\ \hline
        & \multicolumn{3}{|c|}{Exact for Polynomials of Degree $1^{\phantom{|^|}}$} \\ \hline
        1 & 0.0000000000000000 & 0.0000000000000000 &
        2.3776412907378837\vphantom{$|^{|^|}$} \\ 
        \hline
        & \multicolumn{3}{|c|}{Exact for Polynomials of Degree $3^{\phantom{|^|}}$} \\ \hline
        1 & -0.0349156305831802 &  0.6469731019095136 &
        0.5449124407446143\vphantom{$|^{|^|}$} \\
        2 & -0.5951653065516678 & -0.0321196846022659 & 0.6439082046243272 \\
        3 &  0.0349156305831798 & -0.6469731019095134 & 0.5449124407446146 \\
        4 &  0.5951653065516677 &  0.0321196846022661 & 0.6439082046243275 \\ 
        \hline
        & \multicolumn{3}{|c|}{Exact for Polynomials of Degree $5^{\phantom{|^|}}$} \\ \hline
        1 & -0.0000000000000000 & -0.0000000000000002 &
        0.6257871064166934\vphantom{$|^{|^|}$} \\
        2 & -0.1351253857178451 &  0.7099621260052327 & 0.3016384608809768 \\
        3 & -0.6970858746672087 &  0.1907259121533272 & 0.3169910433902452 \\ 
        4 & -0.4651171392611024 & -0.5531465782166917 & 0.3155445150066620 \\
        5 &  0.2842948078559476 & -0.6644407817506509 & 0.2958801959111726 \\
        6 &  0.7117958231685716 & -0.1251071394727008 & 0.2575426306970870 \\
        7 &  0.5337947578638855 &  0.4872045224587945 & 0.2642573384350463 \\
        \hline
    \end{tabular}
    \caption{Gauss quadrature weights and nodes (a.k.a., cubature rules) for integrating over a regular pentagon in its natural co-ordinate system $( \xi , \eta )$.  These weights sum to the area of a pentagon inscribing an unit circle, the formula for which is given in Eqn.~(\ref{regPentagonArea}).}
    \label{tabQuadrature}
\end{table}

The quadrature rules presented in Table~\ref{tabQuadrature} were supplied to the authors by Prof.\ N.\ Sukumar from the University of California at Davis, which he derived for us at our request using a methodology that he had published. \cite{Mousavietal10}  In tabulated form, they cannot be found elsewhere in the literature.  In their document, the authors derived non-symmetric cubature formul\ae\ for determining the nodes and weights of quadrature for a class of methods.  They applied their technique to quadrilaterals, pentagons, hexagons, heptagons and octagons, of which they only published their nodes and weights of quadrature for the hexagon, as hexagons tile two space.  The node for the $1^{\mathrm{st}}$ order method for integrating over the area of a pentagon is located at its centroid.  Nodes for the $3^{\mathrm{rd}}$ and $5^{\mathrm{th}}$ order methods of Table~\ref{tabQuadrature} are displayed in Fig.~\ref{figQuadrature}.

\begin{figure}
    \centering
    \includegraphics[width=4cm]{figures/pentagon_degree3.pdf}
    \hspace{1cm}
    \includegraphics[width=4cm]{figures/pentagon_degree5.pdf}
    \caption{Locations of generalized, Gaussian, quadrature nodes for the $3^{\mathrm{rd}}$ (left) and $5^{\mathrm{th}}$ (right) degree integration methods presented in Table~\ref{tabQuadrature}.  Vertex 1 is located at the top of the pentagon, cf.\ Fig.~\ref{figRegPentagon}, while the coordinate origin is located at its centroid (node 1 in the right figure).  It is readily apparent that these quadrature rules are not symmetric.}
    \label{figQuadrature}
\end{figure}

The Gaussian quadrature rules of Mousavi, Xiao \& Sukumar \cite{Mousavietal10} presented in Table~\ref{tabQuadrature} are compatible with the shape functions of Wachspress \cite{Wachspress75,Wachspress16} and Dasgupta \cite{Dasgupta03} presented in \S\ref{secShapeFns}.  More recently, Chakrabort \textit{et~al}.\ \cite{Chakrabortyetal18} have provided alternative quadrature schemes for pentagons that are also compatible with the Wachspress shape functions.  Again, Table~\ref{tabQuadrature} cannot be found in the literature.

\subsection{Gauss Integration of a Tetrahedron}

Integrating over the volume of a tetrahedron, when expressed in its natural co-ordinates $( \xi , \eta , \zeta )$, which span the ranges of $0 \leq \xi \leq 1$, $0 \leq \eta \leq 1 - \xi$, and $0 \leq \zeta \leq 1 - \xi - \eta$, approximates an integral of some function $f ( \xi , \eta , \zeta )$ as
\begin{multline}
     \mathcal{I} = \int_0^1 \int_{\eta=0}^{1-\xi} \int_{\zeta = 0}^{1 - \xi - \eta} 
     f ( \xi , \eta , \zeta ) \, \mathrm{d} \zeta \, \mathrm{d} \eta \, 
     \mathrm{d} \xi = \sum_{i=1}^n w_i f( \xi_i , \eta_i , \zeta_i ) + E \\
     \text{with} \quad 
     \int_0^1 \int_{\eta=0}^{1-\xi} \int_{\zeta = 0}^{1 - \xi - \eta} 
     \mathrm{d} \zeta \, \mathrm{d} \eta \, \mathrm{d} \xi = \sum_{i=1}^n w_i = 
     \frac{1}{6}
\end{multline}
where co-ordinates $\xi_i$, $\eta_i$ and $\zeta_i$ place the $i^{\text{th}}$ node of integration inside a tetrahedron, and whose corresponding weight of integration is $w_i$, with there being $n$ pairs of nodes and weights.  The volume of this tetradedron is \textfrac{1}{6}.  Table~\ref{tabQuadraturetetra} provides a selection of quadrature rules for tetrahedra when expressed in their natural co-ordinate system.  This table can be found in some finite element textbooks.

\footnotesize
\begin{table}
    \hspace{-1.75cm}
    \begin{tabular}{|c|rrrr|}
        \hline
        node & \centering $\xi$ co-ordinate \phantom{12} & $\eta$ co-ordinate \phantom{12} & 
        $\zeta$ co-ordinate \phantom{12} & weight \phantom{12345} \\ \hline        
        & \multicolumn{4}{|c|}{Exact for Polynomials of Degree $1^{\phantom{|^|}}$} \\ 
        \hline
        1 & 0.250000000000000 & 0.250000000000000 & 0.250000000000000 & 
            0.166666666666667 \\ 
        \hline
        & \multicolumn{4}{|c|}{Exact for Polynomials of Degree $3^{\phantom{|^|}}$} \\ \hline
        1 & 0.250000000000000 & 0.250000000000000 & 0.250000000000000 & 
           -0.133333333333333 \\
        2 & 0.500000000000000 & 0.166666666666667 & 0.166666666666667 &  
            0.075000000000000 \\
        3 & 0.166666666666667 & 0.500000000000000 & 0.166666666666667 &  
            0.075000000000000 \\ 
        4 & 0.166666666666667 & 0.166666666666667 & 0.500000000000000 & 
            0.075000000000000 \\
        5 & 0.166666666666667 & 0.166666666666667 & 0.166666666666667 & 
            0.075000000000000 \\
        \hline
        & \multicolumn{4}{|c|}{Exact for Polynomials of Degree $5^{\phantom{|^|}}$} \\ \hline
        1 & 0.250000000000000 & 0.250000000000000 & 0.250000000000000 &    
            0.030283678097089 \\
        2 & 0.000000000000000 & 0.333333333333333 & 0.333333333333333 & 
            0.006026785714286 \\
        3 & 0.333333333333333 & 0.000000000000000 & 0.333333333333333 & 
            0.006026785714286 \\ 
        4 & 0.333333333333333 & 0.333333333333333 & 0.000000000000000 & 
            0.006026785714286 \\
        5 & 0.333333333333333 & 0.333333333333333 & 0.333333333333333 & 
            0.006026785714286 \\
        6 & 0.727272727272727 & 0.090909090909091 & 0.090909090909091 & 
            0.011645249086029 \\
        7 & 0.090909090909091 & 0.727272727272727 & 0.090909090909091 & 
            0.011645249086029 \\
        8 & 0.090909090909091 & 0.090909090909091 & 0.727272727272727 & 
            0.011645249086029 \\ 
        9 & 0.090909090909091 & 0.090909090909091 & 0.090909090909091 & 
            0.011645249086029 \\
        10 & 0.066550153573664 & 0.433449846426336 & 0.433449846426336 & 
             0.010949141561386 \\
        11 & 0.433449846426336 & 0.066550153573664 & 0.433449846426336 & 
             0.010949141561386 \\
        12 & 0.433449846426336 & 0.433449846426336 & 0.066550153573664 & 
             0.010949141561386 \\
        13 & 0.433449846426336 & 0.066550153573664 & 0.066550153573664 & 
             0.010949141561386 \\ 
        14 & 0.066550153573664 & 0.433449846426336 & 0.066550153573664 & 
             0.010949141561386 \\
        15 & 0.066550153573664 & 0.066550153573664 & 0.433449846426336 & 
             0.010949141561386 \\ 
        \hline
    \end{tabular}
    \caption{Symmetric weights and nodes for Gauss quadratures that integrate over a tetrahedron in its natural co-ordinate system $(\xi , \eta , \zeta)$, where $0 \leq \xi \leq 1 - \eta - \zeta$, $0 \leq \eta \leq 1 - \zeta$ and $0 \leq \zeta \leq 1$.  These weights sum to \textfrac{1}{6}, which is the volume of a tetrahedron measured in its natural co-ordinate system.}
    \label{tabQuadraturetetra}
\end{table}
\normalsize

\section{Interpolation: Nodal Points $\mapsto$ Gauss Points \\ 
         \qquad Extrapolation: Gauss Points $\mapsto$ Nodal Points}
\label{sec:extrapolation}

In a general finite element setting, information comes into the nodes of an element that then gets interpolated down to its Gauss points for their use there.  In many applications, and in particular, in ours, one needs to also be able to take fields, in our case the stress and entropy that have been determined at the Gauss points of an element, and extrapolate this information out to the exterior nodes of the element.

Particular to our application, a suite of nodes is common betwixt three, separate, finite element models comprised of twenty common vertices that belong to a dodecahedron used as a geometric model for an alveolus.  The resultant force at each vertex arises from: a finite element model of thirty 1D rods representing the alveolar chords, a finite element model of twelve 2D pentagons representing the alveolar membranes, and a finite element model of sixty 3D tetrahedra representing the alveolar sac.  The micro\-scopic forces coming from these three models are summed at these twenty common vertices. These resultant forces are then collectively homogenized to yield an averaged macroscopic state of stress for the parenchyma.  Feasibility of this solution strategy hinges upon one's ability to \textit{i\/}) extrapolate stresses evaluated at the Gauss points out to their nodal positions, and \textit{ii\/}) the conversion of these nodal stresses into nodal forces.  We address the first of these two issues in this section, and the second of these two issues in the next section.

Shape functions are introduced for interpolating within an element; specifically, consider an arbitrary field, say $f$, whose values are known at the nodes, then
\begin{subequations}
    \label{extrapolationProcedure}
    \begin{align}
    f ( \boldsymbol{\xi}_k ) & = \sum_{i=1}^n 
    N_i ( \boldsymbol{\xi}_k ) f ( \boldsymbol{x}_i ) &
    k & = 1, 2, \ldots, m 
    \label{interpolation} \\
    \intertext{where the $\boldsymbol{x}_i$ are co-ordinates that locate one of the $n$ vertices in an element of interest, and where the $\boldsymbol{\xi}_i$ are co-ordinates that locate one of its $m$ Gauss points, both being evaluated in the natural co-ordinate system of the element.  Functions $N_i$ are the so-called shape (interpolation) functions.  They obey $\sum_{i=1}^n N_i (\boldsymbol{\xi}) = 1$.
    \bigskip\newline
    A corresponding extrapolation scheme can therefore be written down as}
    f ( \boldsymbol{x}_k ) & = \sum_{i=1}^m 
    M_i ( \boldsymbol{x}_k ) f ( \boldsymbol{\xi}_i ) &
    k & = 1, 2, \ldots, n 
    \label{extrapolation} \\
    \intertext{where the $M_i$ denote extrapolation functions, i.e., they take values of function $f$, now assumed to be known at all Gauss points $\boldsymbol{\xi}_i$, $i=1,2,\ldots,m$, and extrapolate them out to their individual nodal points $\boldsymbol{x}_k$, $k \in \{ 1, 2, \ldots , n \}$. They obey $\sum_{i=1}^m M_i (\boldsymbol{x}) = 1$.
    \bigskip\newline
    These interpolation\slash extrapolation functions must also obey the following constraints: either}
    1 & = \sum_{i=1}^n N_i ( \boldsymbol{\xi}_j )  
    M_j ( \boldsymbol{x}_i ) & j & = 1, 2, \ldots, m 
    \label{extrapolationConstraint1} \\
    0 & = \sum_{i=1}^n N_i ( \boldsymbol{\xi}_j )  
    M_k ( \boldsymbol{x}_i ) & j, k & = 1, 2, \ldots, m , 
    \quad j \neq k
    \label{extrapolationConstraint3}  \\
    \intertext{or}
    1 & = \sum_{i=1}^m  M_i ( \boldsymbol{x}_j )
    N_j ( \boldsymbol{\xi}_i ) & j & = 1, 2, \ldots, n 
    \label{extrapolationConstraint2} \\
    0 & = \sum_{i=1}^m  M_i ( \boldsymbol{x}_j )
    N_k ( \boldsymbol{\xi}_i ) & j, k & = 1, 2, \ldots, n ,
    \quad\; j \neq k
    \label{extrapolationConstraint4}
    \end{align}
\end{subequations}
which follow upon substituting Eqn.~(\ref{extrapolation}) into Eqn.~(\ref{interpolation}), or vice versa.  In this regard, such a pair of interpolation\slash extrapolation functions are self consistent.  

Specifically, whenever $m=n$, the matrices that come about from the interpolation and extrapolation coefficients are reciprocal to one another, with the 0's and 1's of Eqns.~(\ref{extrapolationConstraint1} \& \ref{extrapolationConstraint3} or \ref{extrapolationConstraint2} \& \ref{extrapolationConstraint4}) associating with the individual components of an identity matrix.  Consequently, our need to extrapolate information as-well-as interpolate it strongly suggests that the number of Gauss points selected ought to equal the number of nodal points for any given element geometry.

For example, in the case of a tetrahedron one interpolates via
\begin{subequations}
    \begin{align}
    \left\{ \begin{matrix}
    f ( \boldsymbol{\xi}_1 ) \\ 
    f ( \boldsymbol{\xi}_2 ) \\ 
    f ( \boldsymbol{\xi}_3 ) \\ 
    f ( \boldsymbol{\xi}_4 )
    \end{matrix} \right\} & = \begin{bmatrix}
    N_1 (\boldsymbol{\xi}_1) & N_2 (\boldsymbol{\xi}_1) & 
    N_3 (\boldsymbol{\xi}_1) & N_4 (\boldsymbol{\xi}_1) \\
    N_1 (\boldsymbol{\xi}_2) & N_2 (\boldsymbol{\xi}_2) &
    N_3 (\boldsymbol{\xi}_2) & N_4 (\boldsymbol{\xi}_2) \\
    N_1 (\boldsymbol{\xi}_3) & N_2 (\boldsymbol{\xi}_3) & 
    N_3 (\boldsymbol{\xi}_3) & N_4 (\boldsymbol{\xi}_3) \\
    N_1 (\boldsymbol{\xi}_4) & N_2 (\boldsymbol{\xi}_4) & 
    N_3 (\boldsymbol{\xi}_4) & N_4 (\boldsymbol{\xi}_4)
    \end{bmatrix} \left\{ \begin{matrix} 
    f ( \boldsymbol{x}_1 ) \\ 
    f ( \boldsymbol{x}_2 ) \\ 
    f ( \boldsymbol{x}_3 ) \\
    f ( \boldsymbol{x}_4 ) 
    \end{matrix} \right\}
    \notag \\ 
    \intertext{and extrapolates via}
    \left\{ \begin{matrix} 
    f ( \boldsymbol{x}_1 ) \\ 
    f ( \boldsymbol{x}_2 ) \\ 
    f ( \boldsymbol{x}_3 ) \\
    f ( \boldsymbol{x}_4 )
    \end{matrix} \right\} & = \begin{bmatrix}
    M_1 (\boldsymbol{x}_1) & M_2 (\boldsymbol{x}_1) & 
    M_3 (\boldsymbol{x}_1) & M_4 (\boldsymbol{x}_1) \\
    M_1 (\boldsymbol{x}_2) & M_2 (\boldsymbol{x}_2) &
    M_3 (\boldsymbol{x}_2) & M_4 (\boldsymbol{x}_2) \\
    M_1 (\boldsymbol{x}_3) & M_2 (\boldsymbol{x}_3) & 
    M_3 (\boldsymbol{x}_3) & M_4 (\boldsymbol{x}_3) \\
    M_1 (\boldsymbol{x}_4) & M_2 (\boldsymbol{x}_4) & 
    M_3 (\boldsymbol{x}_4) & M_4 (\boldsymbol{x}_4)
    \end{bmatrix} \left\{ \begin{matrix}
    f ( \boldsymbol{\xi}_1 ) \\ 
    f ( \boldsymbol{\xi}_2 ) \\ 
    f ( \boldsymbol{\xi}_3 ) \\ 
    f ( \boldsymbol{\xi}_4 )
    \end{matrix} \right\}
    \notag
    \end{align}
\end{subequations}
where vectors $\boldsymbol{x}_1$, $\boldsymbol{x}_2$, $\boldsymbol{x}_3$ and $\boldsymbol{x}_4$ hold the co-ordinates of the nodal points, and where vectors $\boldsymbol{\xi}_1$, $\boldsymbol{\xi}_2$, $\boldsymbol{\xi}_3$ and $\boldsymbol{\xi}_4$ hold the co-ordinates of their Gauss points, all evaluated in the natural co-ordinate system of the element.  The matrices in the above mappings are inverses of one another in this construction.
\addtocounter{equation}{-1}

Our three-model finite element analysis of an alveolus requires the use of rods with two nodes, triangles with three nodes, tetrahedra with four nodes, and pentagons with five nodes.  We now provide consistent interpolation\slash extrapolation procedures for these geometries.  This requires the selection of a two-point quadrature rule for rods, a three-point quadrature rule for triangles, a four point quadrature rule for tetrahedra, and a five-point quadrature rule for pentagons.  Our selections, and their associated interpolation\slash extrapolation pairs, are presented below.

\subsection{Self-Consistent Interpolation\slash Extrapolation Procedures for Rods}

Considering a rod with two Gauss points, the interpolation of an arbitrary field, say $f$, whose values are known at nodal points $x_i$, $i=1,2$, into approximated values located at Gauss points $\xi_i$, assigned according to Table~\ref{tab:2nodeRod}, while selecting shape (interpolation) functions $N_1 = \tfrac{1}{2} ( 1 - \xi )$ and $N_2 = \tfrac{1}{2} ( 1 + \xi )$, where $-1 \leq \xi \leq 1$, results in an interpolation that maps values for a field from nodes to Gauss point via
\begin{subequations}
    \begin{align}
     \left\{ \begin{matrix}
    f ( \textfrac{-\sqrt{3}\,}{\,3} ) \\ f ( \textfrac{\sqrt{3}\,}{\,3} )
    \end{matrix} \right\} & = \frac{1}{6} \begin{bmatrix}
        3 + \sqrt{3} & 3 - \sqrt{3} \\
        3 - \sqrt{3} & 3 + \sqrt{3}
    \end{bmatrix} \left\{ \begin{matrix} 
    f ( -1 ) \\ f ( 1 )
    \end{matrix} \right\} 
    \label{interpolateRod} \\
    \intertext{that, upon applying the methodology put forward in Eqn.~(\ref{extrapolationProcedure}), leads to a straight\-forward extrapolation formula that maps values for the field from Gauss points to nodes via}
    \left\{ \begin{matrix} 
    f ( -1 ) \\ f ( 1 )
    \end{matrix} \right\} & 
    = \frac{1}{2 \sqrt{3}} \begin{bmatrix}
    \sqrt{3} + 3 & \sqrt{3} - 3 \\
    \sqrt{3} - 3 & \sqrt{3} + 3
    \end{bmatrix} \left\{ \begin{matrix}
    f ( \textfrac{-\sqrt{3}\,}{\,3} ) \\ f ( \textfrac{\sqrt{3}\,}{\,3} )
    \end{matrix} \right\} .
    \label{extrapolateRod}
    \end{align}
\end{subequations}
This extrapolation matrix can be found in O{\~n}ate \cite[pg.~332]{Onate09}.  As a check, each row in this matrix sums to 1.  Furthermore, the matrices in Eqns.~(\ref{interpolateRod} \& \ref{extrapolateRod}) are reciprocals to one another, as they must be.

\begin{table}
    \begin{center}
        \begin{tabular}{|c|cc|}
            \hline
            node & $\xi$ co-ordinate & weight \\ \hline        
            1 & $-\sqrt{3} / 3^{\vphantom{|^|}}$ & 1 \\ 
            2 & $\phantom{-}\sqrt{3} / 3$ & 1 \\ 
            \hline
        \end{tabular}
    \end{center}
    \caption{A Gauss quadrature rule for integrating functions over the lengths of rods.  It approximates $\int_{-1}^1 f(\xi) \, \mathrm{d}\xi$ using two quadrature points.  The weights of quadrature sum to its length because $L = \int_{-1}^1 \mathrm{d} \xi = 2$.  This quadrature rule is from Christoffel.  It integrates polynomials along a line exactly up through second order.}
    \label{tab:2nodeRod}
\end{table}

\subsection{Self-Consistent Interpolation\slash Extrapolation Procedures for Triangles}

Now, considering a triangle with three Gauss points, the interpolation of an arbitrary field $f$ whose values are known at nodal points $\boldsymbol{x}_i$, $i=1,2,3$, into approximated values located at Gauss points $\boldsymbol{\xi}$, assigned according to Table~\ref{tab:3nodeTriangle}, while selecting shape (interpolation) functions $N_1 = 1 - \xi - \eta$, $N_2 = \xi$, and $N_3 = \eta$, where $0 \leq \xi \leq 1$ and $0 \leq \eta \leq 1 - \xi$, results in an interpolation that maps according to
\begin{subequations}
    \begin{align}
    \left\{ \begin{matrix}
    f ( \textfrac{1}{6} , \textfrac{1}{6} ) \\ 
    f ( \textfrac{2}{3} , \textfrac{1}{6} ) \\ 
    f ( \textfrac{1}{6} , \textfrac{2}{3} )
    \end{matrix} \right\} & = \frac{1}{6} \begin{bmatrix}
    4 & 1 & 1 \\
    1 & 4 & 1 \\
    1 & 1 & 4
    \end{bmatrix} \left\{ \begin{matrix} 
    f ( 0, 0 ) \\ f ( 1, 0 ) \\ f ( 0, 1 )
    \end{matrix} \right\}
    \label{interpolateTriangle} \\
    \intertext{that, upon applying the methodology put forward in Eqn.~(\ref{extrapolationProcedure}), which requires some algebra, leads to a simple extrapolation formula applicable for triangles when evaluated in their natural co-ordinate system, viz.,}
    \left\{ \begin{matrix} 
    f ( 0, 0 ) \\ f ( 1, 0 ) \\ f ( 0, 1 )
    \end{matrix} \right\} & 
    = \frac{1}{3} \begin{bmatrix}
        5 & -1 & -1 \\
        -1 & 5 & -1 \\
        -1 & -1 & 5
    \end{bmatrix} \left\{ \begin{matrix}
        f ( \textfrac{1}{6} , \textfrac{1}{6} ) \\ 
        f ( \textfrac{2}{3} , \textfrac{1}{6} ) \\ 
        f ( \textfrac{1}{6} , \textfrac{2}{3} )
    \end{matrix} \right\} .
    \label{extrapolateTriangle}
    \end{align}
\end{subequations}
As a check, each row in both matrices sums to 1 and, as expected, plus these matrices are reciprocals to one another.

\begin{table}
    \begin{center}
    \begin{tabular}{|c|ccc|}
        \hline
        node & $\xi$ co-ordinate & $\eta$ co-ordinate  & weight \\ \hline        
        1 & 1/6 & 1/6 & 1/6 \\ 
        2 & 2/3 & 1/6 & 1/6 \\ 
        3 & 1/6 & 2/3 & 1/6 \\ 
        \hline
    \end{tabular}
    \end{center}
    \caption{A Gauss quadrature rule for integrating functions over the areas of triangles.  It approximates $\int_0^1 \int_0^{1-\xi} f(\xi , \eta) \, \mathrm{d} \eta \, \mathrm{d} \xi$  using three quadrature points.  The weights of quadrature sum to its area because $A = \int_0^1 \int_0^{1-\xi} \mathrm{d}\eta \, \mathrm{d} \xi = \textfrac{1}{2}$.  This quadrature rule is from Strang.  It integrates polynomials over a triangular region exactly up through second order.}
    \label{tab:3nodeTriangle}
\end{table}
    
\subsection{Self-Consistent Interpolation\slash Extrapolation Procedures for Tetrahedra}
    
We now consider a tetrahedron with four Gauss points.  Here the interpolation of an arbitrary field $f$ whose values are known at nodal points $\boldsymbol{x}_i$, $i=1,2,3,4$, into approximated values located at Gauss points $\boldsymbol{\xi}_i$, assigned according to Table~\ref{tab:4nodedTet}, while selecting shape functions $N_1 = 1 - \xi - \eta - \zeta$, $N_2 = \xi$, $N_3 = \eta$, and $N_4 = \zeta$, bounded by $0 \leq \xi \leq 1$, $0 \leq \eta \leq 1 - \xi$ and $0 \leq \zeta \leq 1 - \xi - \eta$, leads to the following interpolation formula
\begin{subequations}
    \label{extrapolationTetrahedron}
    \begin{align}
         \left\{ \begin{matrix}
        f ( a, a, a ) \\ 
        f ( b, a, a ) \\ 
        f ( a, b, a ) \\
        f ( a, a, b )
        \end{matrix} \right\} & = \begin{bmatrix}
        1 - 3a & a & a & a \\
        1-2a-b & b & a & a \\
        1-2a-b & a & b & a \\
        1-2a-b & a & a & b
        \end{bmatrix} \left\{ \begin{matrix} 
        f ( 0, 0, 0) \\ f ( 1, 0, 0 ) \\ f ( 0, 1, 0 ) \\ f ( 0, 0, 1 )
        \end{matrix} \right\} 
        \label{interpolateTet} \\
    \intertext{that, upon applying the methodology put forward in Eqn.~(\ref{extrapolationProcedure}), which now requires a good deal of algebra, results in the following extrapolation formula for tetrahedra}
    \left\{ \begin{matrix} 
        f ( 0, 0, 0) \\ f ( 1, 0, 0 ) \\ f ( 0, 1, 0 ) \\ f ( 0, 0, 1 )
        \end{matrix} \right\} & 
    = \frac{1}{b-a}\begin{bmatrix}
        2a+b & -a & -a & -a \\
        2a+b-1 & 1-a & -a & -a \\
        2a+b-1 & -a & 1-a & -a \\
        2a+b-1 & -a & -a & 1-a
    \end{bmatrix} \left\{ \begin{matrix}
        f ( a, a, a ) \\ 
        f ( b, a, a ) \\ 
        f ( a, b, a ) \\
        f ( a, a, b )
    \end{matrix} \right\} 
    \label{extrapolateTet}
    \end{align}
\end{subequations}
where $a = 0.1381966011250105$ and $b = 0.5854101966249685$ from Table~\ref{tab:4nodedTet}.  As a check, each row in the above matrices sums to 1.  Unlike the interpolation\slash extrapolation formul\ae\ for rods and triangles, whose matrices are symmetric, here these matrices of interpolation\slash extrapolation coefficients are not symmetric. 
    
\begin{table}
    \hspace{-7mm}
    \begin{tabular}{|c|cccc|}
        \hline
        node & $\xi$ co-ordinate & $\eta$ co-ordinate & 
        $\zeta$ co-ordinate & weight \\ \hline        
        1 & 0.1381966011250105 & 0.1381966011250105 & 0.1381966011250105 & 1/24 \\
        2 & 0.5854101966249685 & 0.1381966011250105 & 0.1381966011250105 & 1/24 \\
        3 & 0.1381966011250105 & 0.5854101966249685 & 0.1381966011250105 & 1/24 \\
        4 & 0.1381966011250105 & 0.1381966011250105 & 0.5854101966249685 & 1/24 \\
        \hline
    \end{tabular}
    \caption{A Gauss quadrature rule for integrating functions over the volumes of tetrahedra.  It approximates $\int_0^1 \int_0^{1-\xi} \int_0^{1-\xi-\eta} f(\xi, \eta, \zeta) \, \mathrm{d}\zeta \, \mathrm{d}\eta \, \mathrm{d}\xi$ using four quadrature points.  The weights of quadrature sum to its volume because $V = \int_0^1 \int_0^{1-\xi} \int_0^{1-\xi-\eta} \mathrm{d}\zeta \, \mathrm{d}\eta \, \mathrm{d}\xi = \textfrac{1}{6}$.  This quadrature rule is from Keast.  It integrates polynomials over a tetrahedral region exactly up through second order.  As a point of reference, the centroid has co-ordinates (\textfrac{1}{4}, \textfrac{1}{4}, \textfrac{1}{4}).}
    \label{tab:4nodedTet}
\end{table}

\subsection{Self-Consistent Interpolation\slash Extrapolation Procedures for Pentagons}

We only know of two papers where quadrature formul\ae\ have been created for integrating over the area of pentagons. \cite{Mousavietal10,Chakrabortyetal18}  Neither presents tables for their nodes and weights of quadrature, cf.\ \S\ref{sec:pentagonQuadrature} for such a table.  Only mathematical methodologies from which one can numerically construct such tables are published.  Also, neither of their strategies exploits the symmetry properties of a pentagon.  

Because we seek a quadrature rule for regular pentagons that employs five Gauss points, and pentagons posses five radial lines of symmetry, it is reasonable to consider that the five nodes of quadrature that we seek lie along these five radial lines.  Specifically, we seek a quadrature rule for a pentagon whose vertices are located at $\boldsymbol{x}_i$, $i=1,2,\ldots,5$, and whose nodes of quadrature are located at $\boldsymbol{\xi}_i$ such that
\begin{subequations}
    \label{pentagonCoordinates}
    \begin{align}
    \boldsymbol{x}_1 & = \bigl( \cos ( \textfrac{\pi}{2} ) ,
        \sin ( \textfrac{\pi}{2} ) \bigr) & \boldsymbol{\xi}_1 & = \ell \boldsymbol{x}_1 \\
    \boldsymbol{x}_2 & = \bigl( \cos ( \textfrac{9\pi}{10}) ,
    \sin ( \textfrac{9\pi}{10} ) \bigr) & \boldsymbol{\xi}_2 & = \ell \boldsymbol{x}_2 \\
    \boldsymbol{x}_3 & = \bigl( \cos ( \textfrac{13\pi}{10}) ,
    \sin ( \textfrac{13\pi}{10} ) \bigr) & \boldsymbol{\xi}_3 & = \ell \boldsymbol{x}_3 \\
    \boldsymbol{x}_4 & = \bigl( \cos ( \textfrac{17\pi}{10}) ,
    \sin ( \textfrac{17\pi}{10} ) \bigr) & \boldsymbol{\xi}_4 & = \ell \boldsymbol{x}_4 \\
    \boldsymbol{x}_5 & = \bigl( \cos ( \textfrac{\pi}{10}) ,
    \sin ( \textfrac{\pi}{10} ) \bigr) & \boldsymbol{\xi}_5 & = \ell \boldsymbol{x}_5
    \end{align}
\end{subequations}
where lines radiating from the origin out to vertices $\boldsymbol{x}_i$ each have unit length, while those that radiate out to the Gauss points $\boldsymbol{\xi}$ have a shorter length of $\ell$.

Implementing the strategies that underlie Gauss quadrature, length $\ell$ represents a distance out to the centroid of an area.  In our case, this area (one of five equivalent areas) is a four-sided polygon whose apex has an inside angle of $108^{\circ}$, whose shoulders have right angles, and whose inside angle at the origin is $72^{\circ}$.  A little bit of geometry and algebra leads to the result that
\begin{subequations}
    \label{pentagonQuadrature}
    \begin{align}
    \ell & = \frac{1 + \sin ( \textfrac{3\pi}{10} )}
    {3 \sin ( \textfrac{3\pi}{10} ) } \approx 0.7454 \\
    \intertext{whose associated area becomes the weight of quadrature}
    w & = \sin ( \textfrac{3\pi}{10} ) \cos ( \textfrac{3\pi}{10} ) 
    \approx 0.4755 
    \end{align}
\end{subequations}
which is one-fifth the area of a pentagon, cf.\ Eqn.~(\ref{regPentagonArea}).  To the best of our knowledge, the quadrature rule put forward in Eqns.~(\ref{pentagonCoordinates} \& \ref{pentagonQuadrature}) for regular pentagons is new to the literature. 

Interpolation is described through shape functions.  Adopting the shape functions of Wachspress, which were constructed in \S\ref{secShapeFns} for a pentagon, while using the quadrature rule of Eqns.~(\ref{pentagonCoordinates} \& \ref{pentagonQuadrature}), results in a symmetric interpolation map of
\begin{subequations}
    \label{extrapolationPentagon}
    \begin{align} 
    \left\{ \begin{matrix}
    f ( \boldsymbol{\xi}_1 ) \\ 
    f ( \boldsymbol{\xi}_2 ) \\ 
    f ( \boldsymbol{\xi}_3 ) \\ 
    f ( \boldsymbol{\xi}_4 ) \\ 
    f ( \boldsymbol{\xi}_5 )
    \end{matrix} \right\} & = \begin{bmatrix}
    a & b & c & c & b \\
    b & a & b & c & c \\
    c & b & a & b & c \\
    c & c & b & a & b \\
    b & c & c & b & a
    \end{bmatrix} 
    \left\{ \begin{matrix} 
    f ( \boldsymbol{x}_1 ) \\ 
    f ( \boldsymbol{x}_2 ) \\ 
    f ( \boldsymbol{x}_3 ) \\
    f ( \boldsymbol{x}_4 ) \\
    f ( \boldsymbol{x}_5 )
    \end{matrix} \right\} \\
    \intertext{with elements sized as $a = 0.6901471673508344$, $b = 0.1367959452017669$ and $c = 0.0181304711228159$, and an ensueing extrapolation map that is described by}
    \left\{ \begin{matrix} 
    f ( \boldsymbol{x}_1 ) \\ 
    f ( \boldsymbol{x}_2 ) \\ 
    f ( \boldsymbol{x}_3 ) \\
    f ( \boldsymbol{x}_4 ) \\
    f ( \boldsymbol{x}_5 )
    \end{matrix} \right\} & = \frac{1}{\Delta} \begin{bmatrix}
    x & y & z & z & y \\
    y & x & y & z & z \\
    z & y & x & y & z \\
    z & z & y & x & y \\
    y & z & z & y & x
    \end{bmatrix} \left\{ \begin{matrix}
    f ( \boldsymbol{\xi}_1 ) \\ 
    f ( \boldsymbol{\xi}_2 ) \\ 
    f ( \boldsymbol{\xi}_3 ) \\ 
    f ( \boldsymbol{\xi}_4 ) \\ 
    f ( \boldsymbol{\xi}_5 )
    \end{matrix} \right\}  \\
    \intertext{wherein}
    x & = a^2 + (a-b)(b+c) - c^2 \\
    y & = (b+c)c - (a+b)b \\
    z & = b^2 - (a-b)c - c^2 \\
    \Delta & = a^2 - (a+b)b - (a-3b)c - c^2
    \end{align}
\end{subequations}
where, as a check, $\sum_{i=1}^5 N_i (\boldsymbol{\xi}_j) = 1$ and $\sum_{i=1}^5 M_i (\boldsymbol{x}_j) = 1$ for $j=1,2,\ldots,5$, and therefore, $a + 2b + 2c = 1$ and $x + 2y + 2z = \Delta$.  Furthermore, these coefficient matrices for interpolation and extrapolation are inverses to one another.  


\section{Converting Nodal Stresses into Nodal Forces}




\newpage
\setcounter{equation}{0}
\setcounter{figure}{0}
\setcounter{section}{0}
\setcounter{table}{0}
\setcounter{section}{0}
\part{Variational Formulation}
\label{partVariational}

The problem we have set up to solve takes on the general form of 
\begin{equation}
	\mathbf{M} \ddot{\mathbf{x}} + \mathbf{K} \mathbf{x} = \mathbf{f}(t)
\end{equation}
where $\mathbf{M}$ is a mass matrix, $\mathbf{K}$ is a stiffness matrix, $\mathbf{f}$ is a forcing function, and $\mathbf{x}$ is a displacement vector.  

For our problem of interest, 
\begin{subequations}
\begin{align}
	\mathbf{x} & = \sum_{v=1}^{20} \{ x_v , y_v , z_v \}^{\mathsf{T}} \\
	\intertext{are the co-ordinates of vertex $v$ located in the co-ordinate frame of the dodecahedron $(\boldsymbol{\imath} , \boldsymbol{\jmath} , \vec{\mathbfit{k}} )$ so that vectors $\mathbf{f}$ and $\mathbf{x}$ have length 60 while matrices $\mathbf{M}$ and $\mathbf{K}$ have dimension $60 \times 60$ with}
	\mathbf{M} & = \mathbf{M}_{1D} + \mathbf{M}_{2D} + \mathbf{M}_{3D} \\
	\mathbf{K} & = \mathbf{K}_{1D} + \mathbf{K}_{2D} + \mathbf{K}_{3D} \\
	\mathbf{f} & = \mathbf{f}_{1D} + \mathbf{f}_{2D} + \mathbf{f}_{3D}
\end{align}
\end{subequations}
where subscript `$\mbox{}_{1D}$' applies to the alveolar chords, subscript `$\mbox{}_{2D}$' applies to the alveolar septa, and subscript `$\mbox{}_{3D}$' applies to the alveolar volume.

\section{Mass Matrix}
A consistent mass matrix \cite{Archer65} established in its natural co-ordinate system is defined as
\begin{equation}
	\mathbf{M}_{C} = \sum_m \int_{V_m} \rho_m \, \mathbf{N}_m^{\mathsf{T}} \mathbf{N}_m \,
	\mathrm{d} V_m
	\label{consistentMassMatrix}
\end{equation}
wherein $\mathbf{N}_m$ is the shape function matrix used to also construct the stiffness matrix for element $m$.

The row-sum techniques is considered to construct the lumped mass matrix, that is the sum of the elements of each row of the consistent mass matrix is used as the diagonal element \cite{Reddy93}:
\begin{equation}
{M}_{L_{ii}} = \sum_{j=1}^n \int_{V_m} \rho \, N_i \, N_j \, \mathrm{d}  V_m 
\end{equation}
wherein $\sum_{j=1}^n N_j = 1$, and $n$ is the Gauss integration points.

The lumped-consistent weighted mass matrix $\mathbf{M}_{LC} $ is defined as follow
\begin{equation}
\mathbf{M}_{LC}  = (1 - \mu) \, \mathbf{M}_{C} + \mu \, \mathbf{M}_{L}
\end{equation}
wherein $\mu$ is a free scalar parameter that is considered to be $\mu = 1/2$ to minimize low frequency dispersion
\begin{equation}
\mathbf{M}_{LC}  = \frac{1}{2} \, (\mathbf{M}_{C} + \mathbf{M}_{L})
\label{LumconsMass}
\end{equation}

\subsection{Mass Matrix of Chord}
The determinant of the Jacobian matrix is used for the transformation of integral from the global coordinate system to the natural coordinate system by
\begin{equation}
     |\mathbf{J}| = \mathrm{\det} \mathbf{J} = \mathrm{\det} \begin{bmatrix}\frac{\partial x }{\partial\xi} \end{bmatrix} = \sum\nolimits_{i=1}^n N_{i,\xi} (\xi) \, x_i
     \label{detJac1D}
\end{equation}
wherein $N_{i}$ are the shape functions for a two-node alveolar chords in its natural coordinate system which are defined in a matrix form as
\begin{equation}
	\mathbf{N} = \begin{bmatrix}
    \frac{1}{2} \, (1 - \xi) &  \frac{1}{2} \, (1 + \xi)
\end{bmatrix} 
\end{equation}
wherein $\xi$ is abscissae of the Gauss integration rule. 

The consistent mass matrix of the 1-D alveolar chord that is evaluated numerically in its natural coordinate system can be described as
\begin{equation}
    \mathbf{M}_{1C} = \int_{\Gamma} \rho \, \mathbf{N}^{\mathsf{T}} \mathbf{N} \, A \, \mathrm{d} x  = \int_{-1}^{1} \rho \, \mathbf{N}^{\mathsf{T}} \mathbf{N}\, A \, |\mathbf{J}|\,  \mathrm{d} \xi =  \sum_{i=1}^{n}  \rho  \, \mathbf{N}^{\mathsf{T}} \mathbf{N} \, A\, |\mathbf{J}| \, \mathrm{w}_i
\end{equation}
 with $\mathrm{w}_i$ being the  weighting coefficients of the Gauss integration rule, and $A$ being the cross section area of alveolar chord. Table~\ref{tabQuadrature1D} demonstrates the values of $\xi$ and $\mathrm{w}_i$ for $n = 1, 2$, and $3$ Gauss integration points.
\begin{table}
    \centering
    \begin{tabular}{|c|rr|}
        \hline
        node & \centering $\xi$ coordinate \phantom{12}  & 
        weight \phantom{12} \\ \hline
        & \multicolumn{2}{|c|}{Exact for Polynomials of Degree $1^{\phantom{|^|}}$} \\ 
        \hline
        1 & 0.0000000000000 & 2.0000000000000 \\ 
        \hline
        & \multicolumn{2}{|c|}{Exact for Polynomials of Degree $3^{\phantom{|^|}}$} \\ \hline
        1 & -0.577350269189 & 1.000000000000\\
        2 & 0.577350269189 & 1.000000000000\\ 
        \hline
        & \multicolumn{2}{|c|}{Exact for Polynomials of Degree $5^{\phantom{|^|}}$} \\ \hline
        1 & -0.774596669241 & 0.555555555556 \\
        2 & 0.000000000000 & 0.888888888889\\
        3 & 0.774596669241 & 0.555555555556\\ 
        \hline
    \end{tabular}
    \caption{Generalized, Gaussian, quadrature, weights and nodes for integrating over a alveolar chord in its natural coordinate system.}
    \label{tabQuadrature1D}
\end{table}

The lumped mass matrix for 1-D alveolar chord in its natural coordinate system becomes 
\begin{equation}
{M}_{1L_{ii}} = \sum_{j=1}^n \int_{\Gamma} \rho \, N_i \, N_j \, A \, \mathrm{d} x  = \int_{-1}^{1} \rho \, N_i\, A \, |\mathbf{J}|\,  \mathrm{d} \xi =  \sum_{i=1}^n  \rho  \, N_i\, A\, |\mathbf{J}| \, \mathrm{w}_i
\end{equation}
wherein $\sum_{j=1}^n N_j = 1$. 

For instance, the consistent mass matrix for alveolar chords with $1$ Gauss integration point that is approximated by the weighted sum of function at the center of chord becomes
\begin{equation}
\mathbf{M}_{1C}  = \frac{\rho \, A \, L}{4}\begin{bmatrix}
1 & 1 \\
1 & 1
\end{bmatrix} 
	\label{ConsMassMatrix1D}
\end{equation}
where $L$ is the length of alveolar chord. The row-sum techniques, gives the lumped mass matrix
\begin{equation}
\mathbf{M}_{1L}  = \frac{\rho \, A \, L}{2}\begin{bmatrix}
1 & 0 \\
0 & 1
\end{bmatrix} 
\label{LumMassMatrix1D}
\end{equation}
and the lumped-consistent weighted mass matrix is constructed as follow 
\begin{equation}
\mathbf{M}_{1LC}  = \frac{\rho \, A \, L}{8}\begin{bmatrix}
3 & 1 \\
1 & 3
\end{bmatrix} 
\label{LumconsMassMatrix1D}
\end{equation}


\subsection{Mass Matrix of Pentagon}
For the alveolar septa, the matrix of shape functions $\mathbf{N}$ is arranged as
 \begin{equation}
	\mathbf{N} = 
	\begin{bmatrix}
	N_1 & 0 & N_2 & 0 & N_3 & 0 & N_4 & 0 & N_5 & 0 \\ 0 & N_1 & 0 & N_2 & 0 & N_3 & 0 & N_4 & 0 & N_5 
\end{bmatrix} 
	\label{shape2D}
\end{equation}
in which $\mathrm{N}_i (i = 1, 2, 3, 4, 5)$ are five shape functions corresponding to the five vertices of the pentagon that are defined in Eq.~(\ref{shapeFunctions}).
The consistent mass matrix $\mathbf{M}_{2C}$ can also be obtained by substituting the above shape function matrix into 
\begin{equation}
    \mathbf{M}_{2C} = \int_{V} \rho \, \mathbf{N}^{\mathsf{T}} \mathbf{N} \, \mathrm{d} V = \int_{\pentagon} \int_{\pentagon} \rho \, \mathbf{N}^{\mathsf{T}} \mathbf{N} \,|\mathbf{J}| \, h \, \mathrm{d} \xi \, \mathrm{d} \eta
    \label{massintegral2d}
\end{equation}
wherein $h$ being membrane thickness, and $|\mathbf{J}|$ being the determinant of the Jacobian matrix for pentagon. 
In quadrilateral derivations,  the Jacobian of two-dimensional transformations that connect the ${x, y}$ to ${\xi, \eta}$ coordinate systems is needed. The components of Jacobian matrix are calculated using derivatives of shape functions with respect to the local coordinates at the $i^{\mathrm{th}}$ vertex via
\begin{equation}
\mathbf{J} = 
\begin{bmatrix}
\partial x / \partial\xi & \partial y / \partial\xi \\
\partial x / \partial\eta & \partial y / \partial\eta 
\end{bmatrix}  
= \begin{bmatrix}
\sum\nolimits_{i=1}^5 N_{i,\xi} (\xi,\eta) \, x_i & \sum\nolimits_{i=1}^5 N_{i,\xi} (\xi,\eta) \, y_i \\
\sum\nolimits_{i=1}^5 N_{i,\eta} (\xi,\eta) \, x_i & \sum\nolimits_{i=1}^5 N_{i,\eta} (\xi,\eta) \, y_i
\end{bmatrix}
\label{jacobianpent}
\end{equation}
The numerical integration of Eq.~(\ref{massintegral2d}) result in 
\begin{equation}
    \mathbf{M}_{2C} = \sum_{i=1}^{n} \rho \, \mathbf{N}^{\mathsf{T}} \mathbf{N} \,|\mathbf{J}| \, h \, w_i
\end{equation}
where $n$ stands for number of Gauss points, and $\mathrm{w}_i$ denotes the natural weight of the element.
There are three mass matrix for each pentagon based upon three Gauss, quadrature rules in Table~\ref{tabQuadrature} which are integrating polynomials of order 3 and 5, respectively. 

The Lumped mass matrix for pentagon makes the diagonal parameters as follow
\begin{equation}
{M}_{2L_{ii}} = \sum_{j=1}^n \int_{V}  \rho \, N_i \, N_j \, \mathrm{d} V  = \int_{\pentagon} \int_{\pentagon} \rho \, N_i \, |\mathbf{J}| \, h \,  \mathrm{d} \xi \, \mathrm{d} \eta =  \sum_{i=1}^n  \rho  \, N_i \, |\mathbf{J}| \, h \, \mathrm{w}_i
\label{LumMass2D}
\end{equation}
wherein $\sum_{j=1}^n N_j = 1$. 

For instance,the lumped-consistent mass matrix of a pentagon  with $1$ Gauss integration point at the center of pentagon that is constructed by averaging consistent mass matrix and the lumped mass matrix becomes 
\begin{equation}
\mathbf{M}_{2LC}  = \rho \, h
\begin{bmatrix}
0.28532 & 0 & 0.04755 & 0 & 0.04755 & 0 & 0.04755 & 0 \\ 0.04755 & 0 \\
0 & 0.28532 & 0 & 0.04755 & 0 & 0.04755 & 0 & 0.04755\\ 0 & 0.04755
\\
0.04755 & 0 & 0.28532 & 0 & 0.04755 & 0 & 0.04755 & 0 \\ 0.04755 & 0\\
0 & 0.04755 & 0 & 0.28532 & 0 & 0.04755 & 0 & 0.04755\\ 0 & 0.04755 \\
0.04755 & 0 & 0.04755 & 0 & 0.28532 & 0 & 0.04755 & 0 \\ 0.04755 & 0 \\
0 & 0.04755 & 0 & 0.04755 & 0 & 0.28532 & 0 & 0.04755\\ 0 & 0.04755 \\
0.04755 & 0 & 0.04755 & 0 & 0.04755 & 0 & 0.28532 & 0\\ 0.04755 & 0\\
0 & 0.04755 & 0 & 0.04755 & 0 & 0.04755 & 0 & 0.28532\\ 0 & 0.04755\\
0.04755 & 0 & 0.04755 & 0 & 0.04755 & 0 & 0.04755 & 0 \\ 0.28532 & 0 & 0.04755\\
0 & 0.04755 & 0 & 0.04755 & 0 & 0.04755 & 0 & 0.04755\\ 0 & 0.28532\\
\end{bmatrix} 
\label{LumconsMassMatrix2D}
\end{equation}


\subsection{Mass Matrix of Tetrahedron}

The dodecahedron has 60 individual tetrahedral whereas its origin being the common vertex of all tetrahedrons. Hence, the analysis to find the mass matrix of a tetrahedron is used to reach the mass matrix of whole alveolar volume.

The matrix of shape functions $\mathbf{N}$ for a tetrahedon has the form of
\begin{equation}
	\mathbf{N} =  
\begin{bmatrix*}[r]
	N_1 & 0 & 0 & N_2 & 0 & 0 & N_3 & 0 & 0 & N_4 & 0 & 0 \\
	0 & N_1 & 0 & 0 & N_2 & 0 & 0 & N_3 & 0 & 0 & N_4 & 0 \\
	0 & 0 & N_1 & 0 & 0 & N_2 & 0 & 0 & N_3 & 0 & 0 & N_4
\end{bmatrix*} 
	\label{shape3D}
\end{equation}
in which $\mathrm{N}_i (i = 1, 2, 3, 4)$ are four shape functions corresponding to the four vertices of the tetrahedron that are defined as follow
\begin{subequations}
\begin{align}
	N_1 & = 1 - \xi - \eta - \zeta \\
	N_2 & = \xi \\
	N_3 & = \eta \\
	N_4 & = \zeta
\end{align}
\end{subequations}
The numerical integration is used to obtain the mass matrix of tetrahedron via
\begin{equation}
    \mathbf{M}_{3C} = \sum_{i=1}^{n_i} \rho  \mathbf{N}^{\mathsf{T}} \mathbf{N} \,  \,|\mathbf{J}| \, w_i
\end{equation}
wherein $|\mathbf{J}|$ being the determinant of the Jacobian matrix in a tetrahedron that are calculated using derivatives of shape functions with respect to the local coordinates $(\xi, \eta, \zeta)$, and the current global coordinates $(x_i, y_i, z_i)$ at the $i^{\mathrm{th}}$ vertex via
\begin{equation}
\begin{aligned}
 \mathbf{J}= &
\begin{bmatrix}
\partial x / \partial\xi & \partial y / \partial\xi & \partial z / \partial\xi\\
\partial x / \partial\eta & \partial y / \partial\eta & \partial z / \partial\eta \\
\partial x / \partial\zeta & \partial y / \partial\zeta & \partial z / \partial\zeta 
\end{bmatrix}\\
  = & \begin{bmatrix}
\sum\nolimits_{i=1}^4 N_{i,\xi} (\xi,\eta,\zeta) \, x_i & \sum\nolimits_{i=1}^4 N_{i,\xi} (\xi,\eta,\zeta) \, y_i &
\sum\nolimits_{i=1}^4 N_{i,\xi} (\xi,\eta,\zeta) \, z_i\\
\sum\nolimits_{i=1}^4 N_{i,\eta} (\xi,\eta,\zeta) \, x_i & \sum\nolimits_{i=1}^4 N_{i,\eta} (\xi,\eta,\zeta) \, y_i &
\sum\nolimits_{i=1}^4 N_{i,\eta} (\xi,\eta,\zeta) \, z_i\\
\sum\nolimits_{i=1}^4 N_{i,\zeta} (\xi,\eta,\zeta) \, x_i & \sum\nolimits_{i=1}^4 N_{i,\zeta} (\xi,\eta,\zeta) \, y_i &
\sum\nolimits_{i=1}^4 N_{i,\zeta} (\xi,\eta,\zeta) \, z_i
\end{bmatrix}
\end{aligned}
\label{jacobiantet}
\end{equation}

There are three mass matrix for each tetrahedron based upon three Gauss, quadrature rules in Table~\ref{tabQuadraturetetra} which are integrating polynomials of order 1, 2 and 3, respectively. 

\begin{table}
    \centering
    \begin{tabular}{|c|rrrr|}
        \hline
        node & \centering $\xi$ coordinate \phantom{1234}  & 
        $\eta$ coordinate \phantom{1234} & 
        $\zeta$ coordinate \phantom{1234} & weight \phantom{12345} \\ \hline        
        & \multicolumn{4}{|c|}{Exact for Polynomials of Degree $1^{\phantom{|^|}}$} \\ 
        \hline
        1 & 1/4 & 1/4 & 1/4 & 1/6 \\ 
        \hline
        & \multicolumn{4}{|c|}{Exact for Polynomials of Degree $2^{\phantom{|^|}}$} \\ \hline
        1 & (5 - $\sqrt{5}$)/20 & (5 - $\sqrt{5}$)/20 & (5 - $\sqrt{5}$)/20 & 1/24\\
        2 & (5 - $\sqrt{5}$)/20 & (5 - $\sqrt{5}$)/20 & (5 + 3 $\sqrt{5}$)/20 & 1/24\\
        3 & (5 - $\sqrt{5}$)/20 & (5 + 3 $\sqrt{5}$)/20 & (5 - $\sqrt{5}$)/20 & 1/24\\ 
        4 & (5 + 3 $\sqrt{5}$)/20 & (5 - $\sqrt{5}$)/20 & (5 - $\sqrt{5}$)/20 & 1/24\\ 
        \hline
        & \multicolumn{4}{|c|}{Exact for Polynomials of Degree $3^{\phantom{|^|}}$} \\ \hline
        1 & 1/4 & 1/4 & 1/4 & -2/15 \\
        2 & 1/6 & 1/6 & 1/6 & 3/40\\
        3 & 1/6 & 1/6 & 1/2 & 3/40\\ 
        4 & 1/6 & 1/2 & 1/6 & 3/40 \\
        5 & 1/2 & 1/6 & 1/6 & 3/40 \\
        \hline
    \end{tabular}
    \caption{Generalized, Gaussian, quadrature, weights and nodes  for integrating over a tetrahedron in its natural coordinate system.}
    \label{tabQuadraturetetra}
\end{table}


The lumped mass matrix for tetrahedron makes the diagonal parameters as follow
\begin{equation}
{M}_{3L_{ii}} = \sum_{j=1}^n \int_{V}  \rho \, N_i \, N_j \, \mathrm{d} V = \sum_{i=1}^n  \rho  \, N_i\, |\mathbf{J}| \, \mathrm{w}_i
\label{LumMass3D}
\end{equation}
wherein $\sum_{j=1}^n N_j = 1$. 

\section{Stiffness Matrix}
For the nonlinear elastic material like soft tissues, as they generally become stiffer with increased deformation, the slope of stress-strain curve will change, and therefore the instantaneous stiffness will change. So, it is required to derive the tangent stiffness for our purpose. 

The total potential energy of a deformed body can be expressed as the differences between the potential energy of deformation $\delta{U}$ and the potential energy of the external loading $\delta{W}$ \cite{Yangetal10}. 
\begin{subequations}
\begin{align}
\delta{W} & = \mathbf{F} \, \mathrm{d} \boldsymbol{\Delta}\\
\delta{U} & = \int_{V} \, \bar{\mathbf{B}}^{\mathsf{T}} \, \boldsymbol{\sigma} \, \mathrm{d} V \, \mathrm{d} \boldsymbol{\Delta}
\end{align}
\end{subequations}
where $\mathbf{F}$ is the vector of external forces, and $\bar{\mathbf{B}}$ is the matrix of strain displacement for large deflection which make the relation between strain increments and displacement increments 
\begin{equation}
\mathrm{d} \boldsymbol{\epsilon} = \bar{\mathbf{B}} \, \mathrm{d} \boldsymbol{\Delta}
\end{equation}
and it can be defined as a sum of linear and nonlinear strain displacements as follow
\begin{equation}
\bar{\mathbf{B}} = \mathbf{B}_L + \mathbf{B}_N
\label{straindis}
\end{equation} 
where $\mathbf{B}_L$ can be obtained from the linear analysis $\boldsymbol{\epsilon} = \mathbf{B}_L \, \boldsymbol{\Delta}$, and $\mathbf{B}_N$ is a function of displacement $ \mathbf{u}$ that can be defined as 
\begin{equation}
\mathbf{B}_N = \mathbf{A} \, \mathbf{G}
\end{equation} 
wherein $\mathbf{A}= \partial \mathbf{u} / \partial \mathbf{x}$ is the derivative of displacements, and $\mathbf{G}= \partial \mathbf{N} / \partial \mathbf{x}$ is the derivative of shape functions respect to the global coordinate system. The displacement fields are interpolated as
\begin{equation}
\bar{\mathbfsf{ u}} = 
\begin{Bmatrix}
u \\
v     
\end{Bmatrix}
=
\begin{Bmatrix}
\sum_{i=1}^{n} u_j \, N_j \\
\sum_{i=1}^{n} v_j \, N_j     
\end{Bmatrix}
= \mathbf{N} \, \mathbf{\Delta}
\end{equation}
with
\begin{equation}
\mathbf{\Delta}^T  = 
\begin{Bmatrix}
u_1 , v_1, u_2 , v_2, ..., u_n , v_n
\end{Bmatrix}
\end{equation}

In order to satisfy the equilibrium, the first variation of the total potential energy which is known as the residual force should be zero. 
\begin{equation}
\mathbf{R} = \int_{V} \, \bar{\mathbf{B}}^{\mathsf{T}} \, \boldsymbol{\sigma} \, \mathrm{d} V - \mathbf{F} = 0
\label{residual}
\end{equation}
differentiation of the residual force yield 
\begin{equation}
\mathrm{d} \mathbf{R} = \int_{V} \, \mathrm{d} \bar{\mathbf{B}}^{\mathsf{T}} \, \boldsymbol{\sigma} \, \mathrm{d} V + \int_{V} \, \bar{\mathbf{B}}^{\mathsf{T}} \, \mathrm{d} \boldsymbol{\sigma} \, \mathrm{d} V 
\label{diffresidual}
\end{equation}
where $\mathrm{d} \boldsymbol{\sigma} = \mathbf{M} \, \mathrm{d} \boldsymbol{\epsilon} = \mathbf{M} \, \bar{\mathbf{B}} \, \mathrm{d} \boldsymbol{\Delta}$, with $\mathbf{M}$ being the matrix of elastic modulus. 
The incremental form of  Eq.~(\ref{straindis}) yield
\begin{equation}
\mathrm{d} \bar{\mathbf{B}} = \mathrm{d} \left(\mathbf{B}_L + \mathbf{B}_N \right) = \mathrm{d} \mathbf{B}_N = \mathrm{d} (\mathbf{A} \, \mathbf{G}) = \mathrm{d} \mathbf{A} \, \mathbf{G}
\end{equation} 

The tangent stiffness matrix can be obtained by substituting the definition of $\mathrm{d} \boldsymbol{\sigma}$ into Eq.~(\ref{diffresidual}) as follow
\begin{equation}
\begin{aligned}
\mathrm{d} \mathbf{R} & = \int_{V} \, \mathrm{d} \bar{\mathbf{B}}^{\mathsf{T}} \, \boldsymbol{\sigma} \, \mathrm{d} V + \int_{V} \, \bar{\mathbf{B}}^{\mathsf{T}} \, \mathbf{M} \, \bar{\mathbf{B}} \, \mathrm{d} V \, \mathrm{d} \boldsymbol{\Delta} \\ 
& = \int_{V} \, \mathbf{G}^{\mathsf{T}} \, \mathrm{d} \mathbf{A}^{\mathsf{T}} \boldsymbol{\sigma} \, \mathrm{d} V + \int_{V} \, \left(\mathbf{B}_L + \mathbf{B}_N \right)^{\mathsf{T}} \, \mathbf{M} \,  \left(\mathbf{B}_L + \mathbf{B}_N \right) \, \mathrm{d} V \, \mathrm{d} \boldsymbol{\Delta} \\ 
& = \mathbf{K}_{\boldsymbol{\sigma}} \, \mathrm{d} \boldsymbol{\Delta} + \int_{V} \, \mathbf{B}_L^{\mathsf{T}} \, \mathbf{M} \, \mathbf{B}_L \, \mathrm{d} V \, \mathrm{d} \boldsymbol{\Delta}\\ 
& + \int_{V} \,  \left(\mathbf{B}_L^{\mathsf{T}} \, \mathbf{M} \, \mathbf{B}_N \, + \mathbf{B}_N^{\mathsf{T}} \, \mathbf{M} \, \mathbf{B}_L \, + \mathbf{B}_N^{\mathsf{T}} \, \mathbf{M} \, \mathbf{B}_N \,  \right) \, \mathrm{d} V \, \mathrm{d} \boldsymbol{\Delta}\\
& = \left(\mathbf{K}_{\boldsymbol{\sigma}} + \mathbf{K}_L + \mathbf{K}_N\right)\mathrm{d} \boldsymbol{\Delta} = \mathbf{K}_T \, \mathrm{d} \boldsymbol{\Delta}
\end{aligned} 
\end{equation} 
wherein $\mathbf{K}_{\boldsymbol{\sigma}}$ being the stiffness matrix associated with the initial stress, $\mathbf{K}_L$ being the conventional small displacement stiffness matrix, $\mathbf{K}_N$ being the the large displacement stiffness matrix, and $\mathbf{K}_T$ being the tangent-stiffness matrix 

\subsection{Stiffness Matrix for Chord}
The stress stiffness matrix $\mathbf{K}_{\boldsymbol{\sigma}}$ expresses the influence of membrane stresses on the lateral deflection that is independent of material properties.
\begin{equation}
\mathbf{K}_{\boldsymbol{\sigma}} = \int_{V} \, \mathbf{G}^{\mathsf{T}} \, \mathrm{d} \mathbf{A}^{\mathsf{T}} \boldsymbol{\sigma} \, \mathrm{d} V = \int_{\Gamma} \, \mathbf{G}^{\mathsf{T}} \,  \boldsymbol{\sigma} \, \mathbf{G} \, A \, \mathrm{d} x 
= \int_{-1}^{1} \mathbf{G}^{\mathsf{T}} \,  \boldsymbol{\sigma} \, \mathbf{G} \, |\mathbf{J}|  \, G \,  \mathrm{d} \xi =  \sum_{i=1}^{n}  \mathbf{G}^{\mathsf{T}} \,  \boldsymbol{\sigma} \, \mathbf{G}  \, |\mathbf{J}| \, A \, \mathrm{w}_i
\end{equation}
where 
\begin{equation}
\begin{aligned}
\mathbf{G} = \begin{bmatrix}
\partial N_1 / \partial x & \partial N_2 / \partial x
\end{bmatrix} = \begin{bmatrix}
\partial N_1 / \partial \xi \cdot \partial \xi / \partial x &  \partial N_2 / \partial \xi \cdot \partial \xi / \partial x
\end{bmatrix} = \begin{bmatrix}
\partial N_1 / \partial \xi &  \partial N_2 / \partial \xi \, 
\end{bmatrix} \, \mathbf{J}^{-1}
\end{aligned}
\end{equation}


The small displacement stiffness matrix for a 1-D alveolar chord that is transformed from global coordinate system to the natural coordiante system by the determinant of the Jacobian matrix, i.e., Eq.~(\ref{detJac1D}) is evaluated numerically as 
\begin{equation}
\mathbf{K}_{L} = \int_{\Gamma} \, \mathbf{B}_L^{\mathsf{T}} \, \mathbf{M} \, \mathbf{B}_L \, A \, \mathrm{d} x  = \int_{-1}^{1} \mathbf{B}_L^{\mathsf{T}} \, \mathbf{M} \, \mathbf{B}_L \, |\mathbf{J}|  \, A \,  \mathrm{d} \xi =  \sum_{i=1}^{n}  \mathbf{B}_L^{\mathsf{T}} \, \mathbf{M} \, \mathbf{B}_L \, |\mathbf{J}| \, A \, \mathrm{w}_i
\end{equation}
with $\mathrm{w}_i$ being the  weighting coefficients of the Gauss integration rule, and $A$ being the cross section area of alveolar chord. The values of $\xi$ and $\mathrm{w}_i$ for $n = 1, 2$, and $3$ Gauss integration points are demonstrated in Table~\ref{tabQuadrature1D}.
In calculation of the stiffness matrix for an element, the linear strain-displacement matrix $\mathbf{B}_L$ is required that is the derivatives of the shape functions with respect to the global coordinate system  
\begin{equation}
\mathbf{B}_L = \begin{bmatrix}
\partial N_1 / \partial x &  \partial N_2 / \partial x
\end{bmatrix} = \begin{bmatrix}
\partial N_1 / \partial \xi \cdot \partial \xi / \partial x &  \partial N_2 / \partial \xi \cdot \partial \xi / \partial x
\end{bmatrix} = \begin{bmatrix}
\partial N_1 / \partial \xi &  \partial N_2 / \partial \xi \, 
\end{bmatrix} \, \mathbf{J}^{-1}
\end{equation}
wherein $\xi$ is abscissae of the Gauss integration rule. 

The large displacement stiffness matrix for chord can be presented as follow
\begin{equation}
\begin{aligned}
\mathbf{K}_{N} & =  \int_{V} \,  \left(\mathbf{B}_L^{\mathsf{T}} \, \mathbf{M} \, \mathbf{B}_N \, + \mathbf{B}_N^{\mathsf{T}} \, \mathbf{M} \, \mathbf{B}_L \, + \mathbf{B}_N^{\mathsf{T}} \, \mathbf{M} \, \mathbf{B}_N \,  \right) \mathrm{d} V \\
& = \int_{-1}^{1} \left(\mathbf{B}_L^{\mathsf{T}} \, \mathbf{M} \, \mathbf{B}_N \, + \mathbf{B}_N^{\mathsf{T}} \, \mathbf{M} \, \mathbf{B}_L \, + \mathbf{B}_N^{\mathsf{T}} \, \mathbf{M} \, \mathbf{B}_N \,  \right) \, |\mathbf{J}|  \, A \,  \mathrm{d} \xi \\
& = \sum_{i=1}^{n}  \left(\mathbf{B}_L^{\mathsf{T}} \, \mathbf{M} \, \mathbf{B}_N + \mathbf{B}_N^{\mathsf{T}} \, \mathbf{M} \, \mathbf{B}_L + \mathbf{B}_N^{\mathsf{T}} \, \mathbf{M} \, \mathbf{B}_N \right) \, |\mathbf{J}| \, A \, \mathrm{w}_i
\end{aligned}
\end{equation}
where $\mathbf{B}_N$ have the expression 
\begin{equation}
\begin{aligned}
\mathbf{B}_{N} & =  \mathbf{A} \, \mathbf{G} = \begin{bmatrix}
\partial \mathbf{u} / \partial \mathbf{x} 
\end{bmatrix} \, \begin{bmatrix}
\partial N_1 / \partial x & \partial N_2 / \partial x
\end{bmatrix} \\
& = \begin{bmatrix}
\sum\nolimits_{i=1}^n N_{i,x} \, u_i
\end{bmatrix} \, \begin{bmatrix}
\partial N_1 / \partial \xi &  \partial N_2 / \partial \xi \, 
\end{bmatrix} \, \mathbf{J}^{-1}
\end{aligned}
\end{equation}
Thereby the total stiffness matrix $\mathbf{K}_T$ can be obtained by summation of stress stiffness matrix, small and large displacement stiffness matrices.

\subsection{Stiffness Matrix for Pentagon}
The components of Laplace stretch $\boldsymbol{\mathcal{U}}$ associated with a planar membrane has a  Cholesky factorization expressed in terms of the right Cauchy-Green deformation tensor $\mathbf{C} \defeq \mathbf{F}^{\mathsf{T}} \mathbf{F} = \boldsymbol{\mathcal{U}}^{\mathsf{T}} \boldsymbol{\mathcal{U}}$, which is a symmetric second-order tensor. 
\begin{equation}
\begin{aligned}
{\mathcal{U}}_{11} & = \sqrt{C_{11}} \;\; & 
{\mathcal{U}}_{12} & = C_{12} / {\mathcal{U}_{11}} \\
{\mathcal{U}}_{21} & = 0 &
{\mathcal{U}}_{22} & = \sqrt{C_{22} - ({\mathcal{U}}_{12})^2} 
\end{aligned}
\label{Laplace stretchComponents}
\end{equation} 
where ${C_{11}}$, ${C_{12}}$, ${C_{21}}$ and ${C_{22}}$ are components of the Green deformation matrix $\mathbf{C}$ wherein  $\mathbf{F}$ is the deformation gradient.
\begin{subequations}
	\begin{align}
	\mathbf{C} &= \mathbf{F}^{\mathsf{T}} \mathbf{F} =
	\begin{bmatrix}
	C_{11} & C_{12}  \\
	C_{12} & C_{22} 
	\end{bmatrix} \quad \textrm{,} \quad
	\mathbf{F} =  
	\begin{bmatrix}
	1+\mathrm{\partial u / \partial x} & \mathrm{\partial u / \partial y}  \\
	\mathrm{\partial v / \partial x} & 1+\mathrm{\partial v / \partial y}
	\end{bmatrix}\\
	\intertext{with}
	\mathrm{C_{11}}&= \mathrm{\partial u / \partial x}^2+ {\mathrm{\partial v / \partial x}}^2 + 2\, \mathrm{\partial u / \partial x}  + 1\\
	\mathrm{C_{12}}&= \mathrm{\partial u / \partial y} + \mathrm{\partial v / \partial x} + \mathrm{\partial u / \partial x} \cdot \mathrm{\partial u / \partial y} + \mathrm{\partial v / \partial x} \cdot \mathrm{\partial v / \partial y}\\
	\mathrm{C_{22}}&= \mathrm{\partial u / \partial y}^2 + {\mathrm{\partial v / \partial y}}^2 + 2\, \mathrm{\partial v / \partial y} + 1
	\end{align}
\end{subequations}

It is useful to define the total virtual strains Eq.~(\ref{conjugateStrains}) in terms of linear and nonlinear incremental strains as follow
\begin{subequations}
	\begin{align}
	\mathrm  \xi & =\mathrm \xi_{L}+\mathrm  \xi_{N}\\
	\mathrm  \varepsilon & =\mathrm  \varepsilon_{L}+\mathrm  \varepsilon_{N}\\
	\mathrm \gamma & =\mathrm  \gamma_{L}+\mathrm  \gamma_{N}
	\end{align}
	\label{totalvirtualstrain}
\end{subequations}
Taylor series are used here to obtain the linear and nonlinear part of displacements 
\begin{subequations}
	\begin{align}
	\mathrm \xi_{L} & =\frac{1}{2}\, \left(\mathrm{\frac{\partial u}{\partial x}} + \mathrm{\frac{\partial v}{\partial y}}\right)\\
	\mathrm \xi_{N} & = \frac{1}{4}\, \left(- \mathrm{\frac{ \partial v}{\partial y}}\, \mathrm{\frac{ \partial v}{\partial y}} -\mathrm{\frac{\partial u}{\partial x}}\, \mathrm{\frac{\partial u}{\partial x}} - 2 \, \mathrm{\frac{\partial u}{\partial y}}\, \mathrm{\frac{\partial v}{\partial x}}\right)\\
	\mathrm \varepsilon_{L} & =\frac{1}{2}\, \left(\mathrm{\frac{\partial u}{\partial x}} - \mathrm{\frac{\partial v}{\partial y}}\right)\\
	\mathrm \varepsilon_{N} & = \frac{1}{4}\, \left(2 \, \mathrm{\frac{\partial v}{\partial x}}\, \mathrm{\frac{\partial v}{\partial x}} + \mathrm{\frac{ \partial v}{\partial y}}\, \mathrm{\frac{ \partial v}{\partial y}} -\mathrm{\frac{\partial u}{\partial x}}\, \mathrm{\frac{\partial u}{\partial x}} + 2 \, \mathrm{\frac{\partial u}{\partial y}}\, \mathrm{\frac{\partial v}{\partial x}}\right)\\
	\mathrm \gamma_{L} & = \mathrm{\frac{\partial u}{\partial y}} + \mathrm{\frac{\partial v}{\partial x}}\\
	\mathrm \gamma_{N} & = \mathrm{\frac{\partial v}{\partial x}}\, \mathrm{\frac{\partial v}{\partial y}} - 2\, \mathrm{\frac{ \partial u}{\partial x}}\, \mathrm{\frac{ \partial v}{\partial x}} 
	-\mathrm{\frac{\partial u}{\partial x}}\, \mathrm{\frac{\partial u}{\partial y}}
	\end{align}
\end{subequations}
The linear strain displacement matrix $\mathbf{B}_L$ can be obtained by differentiation of displacements expressed through the nodal displacements and shape functions from infinitesimal linear strain vector that take the form of
\begin{equation}
\mathbf{B}_L = \frac{1}{2} \, \begin{bmatrix}
\partial N_i / \partial x &  \partial N_i / \partial y \\
\partial N_i / \partial x & -\partial N_i / \partial y \\
2 \, \partial N_i / \partial y & 2 \, \partial N_i / \partial x \end{bmatrix} 
\end{equation}
The nonlinear strain terms can be written as 
\begin{equation}
\boldsymbol{\epsilon}_ N = \frac{1}{2} \, \begin{bmatrix}
-\frac{1}{2} \, \partial u_i / \partial x &  -4 \, \partial u_i / \partial y & 0 & -\frac{1}{2} \, \partial v_i / \partial y \\
-\frac{1}{2} \, \partial u_i / \partial x &  4 \, \partial v_i / \partial x & 4 \, \partial v_i / \partial x & \frac{1}{2} \, \partial v_i / \partial y \\
-2 \, \partial u_i / \partial y &  -4 \, \partial u_i / \partial x & 0 & 2 \, \partial v_i / \partial x \end{bmatrix} \, \begin{Bmatrix}
\partial u_i / \partial x\\
\partial v_i / \partial x\\
\partial u_i / \partial y\\
\partial v_i / \partial y
\end{Bmatrix}
= \frac{1}{2} \, \mathbf{A} \, \boldsymbol{\theta}
\end{equation}
the derivative of displacement can be related to the nodal parameters via
\begin{equation}
\boldsymbol{\theta} =  \begin{Bmatrix}
\partial u_i / \partial x\\
\partial v_i / \partial x\\
\partial u_i / \partial y\\
\partial v_i / \partial y
\end{Bmatrix}
= \begin{Bmatrix}
\sum\nolimits_{i=1}^5 N_{i,x} \, u_i\\
\sum\nolimits_{i=1}^5 N_{i,x} \, v_i\\
\sum\nolimits_{i=1}^5 N_{i,y} \, u_i\\
\sum\nolimits_{i=1}^5 N_{i,y} \, v_i
\end{Bmatrix} 
= \begin{bmatrix}
[\mathbf{g}_1], & [\mathbf{g}_2], & [\mathbf{g}_3], & [\mathbf{g}_4], & [\mathbf{g}_5] 
\end{bmatrix}  \begin{Bmatrix} \boldsymbol{\Delta} \end{Bmatrix}  
= [\mathbf{G}] \begin{Bmatrix} \boldsymbol{\Delta} \end{Bmatrix} 
\end{equation}
where 
\begin{equation}
\mathbf{g}_i = \begin{bmatrix}
N_{i,x} &  0 \\
0 & N_{i,x} \\
N_{i,y} & 0 \\
0 & N_{i,y} \end{bmatrix} 
\end{equation}
hence $\mathbf{B}_N$ become
\begin{equation}
\begin{aligned}
\mathbf{B}_{N}  =  \mathbf{A} \, \mathbf{G} = \begin{bmatrix}
-\frac{1}{2} \, \partial u_i / \partial x &  -4 \, \partial u_i / \partial y & 0 & -\frac{1}{2} \, \partial v_i / \partial y \\
-\frac{1}{2} \, \partial u_i / \partial x &  4 \, \partial v_i / \partial x & 4 \, \partial v_i / \partial x & \frac{1}{2} \, \partial v_i / \partial y \\
-2 \, \partial u_i / \partial y &  -4 \, \partial u_i / \partial x & 0 & 2 \, \partial v_i / \partial x \end{bmatrix}  \, \begin{bmatrix}
N_{i,x} &  0 \\
0 & N_{i,x} \\
N_{i,y} & 0 \\
0 & N_{i,y} \end{bmatrix} 
\end{aligned}
\end{equation}

The stress stiffness matrix for a 2-D alveolar septa take the form of
\begin{equation}
\mathbf{K}_{\boldsymbol{\sigma}} = \int_{V} \, \mathbf{G}^{\mathsf{T}} \, \mathrm{d} \mathbf{A}^{\mathsf{T}} \boldsymbol{\sigma} \, \mathrm{d} V
= \int_{\pentagon} \int_{\pentagon} \mathbf{G}^{\mathsf{T}} \,  \boldsymbol{\sigma} \, \mathbf{G} \, |\mathbf{J}|  \, h \,  \mathrm{d} \xi \,  \mathrm{d} \eta =  \sum_{i=1}^{n}  \mathbf{G}^{\mathsf{T}} \,  \boldsymbol{\sigma} \, \mathbf{G} \, |\mathbf{J}| \, h \, \mathrm{w}_i
\end{equation}
where $n$ stands for number of Gauss points, and $\mathrm{w}_i$ denotes the natural weight of the element demonstrated in Table~\ref{tabQuadrature}.

The small displacement stiffness matrix for pentagon is evaluated numerically as 
\begin{equation}
\mathbf{K}_{L} = \int_{V} \, \mathbf{B}_L^{\mathsf{T}} \, \mathbf{M} \, \mathbf{B}_L \, \mathrm{d} V  = \int_{\pentagon} \int_{\pentagon} \mathbf{B}_L^{\mathsf{T}} \, \mathbf{M} \, \mathbf{B}_L \, |\mathbf{J}| \, h \,  \mathrm{d} \xi \,  \mathrm{d} \eta =  \sum_{i=1}^{n}  \mathbf{B}_L^{\mathsf{T}} \, \mathbf{M} \, \mathbf{B}_L \, |\mathbf{J}|  \, h \, \mathrm{w}_i
\end{equation}

The large displacement stiffness matrix for chord can be presented as follow
\begin{equation}
\begin{aligned}
\mathbf{K}_{N} & =  \int_{V} \,  \left(\mathbf{B}_L^{\mathsf{T}} \, \mathbf{M} \, \mathbf{B}_N \, + \mathbf{B}_N^{\mathsf{T}} \, \mathbf{M} \, \mathbf{B}_L \, + \mathbf{B}_N^{\mathsf{T}} \, \mathbf{M} \, \mathbf{B}_N \,  \right) \mathrm{d} V \\
& = \int_{\pentagon} \int_{\pentagon} \left(\mathbf{B}_L^{\mathsf{T}} \, \mathbf{M} \, \mathbf{B}_N \, + \mathbf{B}_N^{\mathsf{T}} \, \mathbf{M} \, \mathbf{B}_L \, + \mathbf{B}_N^{\mathsf{T}} \, \mathbf{M} \, \mathbf{B}_N \,  \right)  \, |\mathbf{J}|\, h \, \mathrm{d} \xi \, \mathrm{d} \eta \\
& = \sum_{i=1}^{n}  \left(\mathbf{B}_L^{\mathsf{T}} \, \mathbf{M} \, \mathbf{B}_N + \mathbf{B}_N^{\mathsf{T}} \, \mathbf{M} \, \mathbf{B}_L + \mathbf{B}_N^{\mathsf{T}} \, \mathbf{M} \, \mathbf{B}_N \right) \, |\mathbf{J}| \, h \, \mathrm{w}_i
\end{aligned}
\end{equation}

\subsection{Stiffness Matrix for Tetrahedron}
The stress stiffness matrix for a tetrahedron can be expressed as follow
\begin{equation}
\mathbf{K}_{\boldsymbol{\sigma}} = \int_{V} \, \mathbf{G}^{\mathsf{T}} \, \mathrm{d} \mathbf{A}^{\mathsf{T}} \boldsymbol{\sigma} \, \mathrm{d} V
=  \sum_{i=1}^{n}  \mathbf{G}^{\mathsf{T}} \,  \boldsymbol{\sigma} \, \mathbf{G} \, |\mathbf{J}| \, \mathrm{w}_i
\end{equation}   

The small displacement stiffness matrix for tetrahedron is evaluated numerically as 
\begin{equation}
\mathbf{K}_{L} = \int_{V} \, \mathbf{B}_L^{\mathsf{T}} \, \mathbf{M} \, \mathbf{B}_L \, \mathrm{d} V  =  \sum_{i=1}^{n}  \mathbf{B}_L^{\mathsf{T}} \, \mathbf{M} \, \mathbf{B}_L \, |\mathbf{J}| \, \mathrm{w}_i
\end{equation}

The large displacement stiffness matrix for chord can be presented as follow
\begin{equation}
\begin{aligned}
\mathbf{K}_{N} & =  \int_{V} \,  \left(\mathbf{B}_L^{\mathsf{T}} \, \mathbf{M} \, \mathbf{B}_N \, + \mathbf{B}_N^{\mathsf{T}} \, \mathbf{M} \, \mathbf{B}_L \, + \mathbf{B}_N^{\mathsf{T}} \, \mathbf{M} \, \mathbf{B}_N \,  \right) \mathrm{d} V \\
& = \sum_{i=1}^{n}  \left(\mathbf{B}_L^{\mathsf{T}} \, \mathbf{M} \, \mathbf{B}_N + \mathbf{B}_N^{\mathsf{T}} \, \mathbf{M} \, \mathbf{B}_L + \mathbf{B}_N^{\mathsf{T}} \, \mathbf{M} \, \mathbf{B}_N \right) \, |\mathbf{J}| \, \mathrm{w}_i
\end{aligned}
\end{equation}




\newpage
\small
\bibliographystyle{elsarticle-num}
\bibliography{dodec}
\normalsize

\setcounter{equation}{0}
\setcounter{figure}{0}
\setcounter{table}{0}

\appendix

\section{Implicit Elasticity}
\label{appImplicitElasticity}

Both explicit (Green \cite{Green41} elastic) and implicit (Rajagopal \cite{Rajagopal03} elastic) material models are put forward in this appendix for one's consideration when choosing a material model to represent biologic fibers and membranes.  We discuss thermo\-elastic fibers first, and then thermo\-elastic membranes.  We have no need to address thermo\-elastic volumes for our application beyond what has been presented in \S\ref{sec:IdealGasLaw}.  We employ Gibbs' free-energy potentials instead of internal-energy potentials to the extent that is possible.  They relate to one another via a well-known Legendre transformation.  A Gibbs energy approach implies that a change in the intensive variables (thermo\-dynamic forces) will cause a response in the extensive variables (thermo\-dynamic displacements), which is the exact opposite of an internal energy approach.  Cause and effect are correct in a Gibbs formulation.

\subsection{Alveolar Chords as Green (Explicit) Thermoelastic Fibers}

For a 1D fiber with a mass density of $\rho$ per unit length, the conjugate fields are: temperature $\theta$ and entropy $\eta$, plus force $F$ and strain $e \defeq \ln (L/L_0)$ with $L_0$ and $L$ denoting the initial and current fiber lengths.  A Green elastic fiber adopts a Gibbs free-energy function with an explicit dependence upon state, viz., $\mathcal{G} (\theta , F)$ such that $\mathrm{d} \mathcal{G} = -\eta \, \mathrm{d} \theta - \tfrac{1}{\rho} e \, \mathrm{d}F$ (cf.~Eqn.~\ref{thermoelastic1Dlaw}), from which one derives its governing constitutive equations as being
\begin{subequations}
    \label{fiberConstitutiveTheory}
    \begin{align}
\eta & = -\partial_{\theta\,} \mathcal{G} (\theta , F)
\quad \text{and} \quad
e = -\rho \, \partial_{F\,} \mathcal{G} (\theta, F)
\label{entropyStrain1D} \\
\intertext{that, when differentiated, can be rearranged into the following hypo-elastic equation}
\left\{ \begin{matrix}
\mathrm{d} \eta \\ \mathrm{d} e 
\end{matrix} \right\} & = -\begin{bmatrix}
\partial_{\theta\theta\,} \mathcal{G} & \partial_{\theta F\,} \mathcal{G} \\
\rho \, \partial_{F\theta\,} \mathcal{G} & \rho \, \partial_{FF\,} \mathcal{G}
\end{bmatrix} 
\left\{ \begin{matrix}
\mathrm{d} \theta \\ \mathrm{d} F
\end{matrix} \right\}
= \begin{bmatrix}
C / \theta & \alpha / \rho \\
\alpha & 1 / E
\end{bmatrix}
\left\{ \begin{matrix}
\mathrm{d} \theta \\ \mathrm{d} F
\end{matrix} \right\}
\label{thermoelasticCE1D} \\
    \intertext{whose thermo-physical material properties are, in general, functions of state defined by}
    C & \defeq \theta \, \partial_{\theta\,} \eta |_F = 
    -\theta \, \partial_{\theta\theta\,} \mathcal{G} (\theta , F)
    \label{specificHeat} \\
    \alpha & \defeq L^{-1} \, \partial_{\theta\,} L |_F = \partial_{\theta\,} e |_F =
    -\rho \, \partial_{F\theta\,} \mathcal{G} (\theta , F) \equiv
    -\rho \, \partial_{\theta F\,} \mathcal{G} (\theta , F)
    \label{thermalExpansion} \\
    1 / E & \defeq L^{-1} \, \partial_{F\,} L |_{\theta} = \partial_{F\,} e |_{\theta} =
    -\rho \, \partial_{FF\,} \mathcal{G} (\theta , F)
    \label{compliance}
    \end{align}
\end{subequations}
where the elastic compliance $1/E = L^{-1} \, \partial_{F\,} L |_{\theta } = \partial_{F} \ln (L / L_0) |_{\theta } = \partial_{F\,} e |_{\theta }$ has units of reciprocal force evaluated at constant temperature, while the thermal expansion coefficient $\alpha = L^{-1} \, \partial_{\theta\,} L |_F = \partial_{\theta} \ln (L/L_0) |_F = \partial_{\theta\,} e |_F$ has units of reciprocal temperature evaluated at constant force.  The mass density $\rho$ is one-dimensional in this presentation, i.e., it has units of mass per unit length of fiber; likewise, modulus $E$ has units of force, not stress.

\subsubsection{A Hookean Fiber}

A thermoelastic Hookean fiber is Green elastic with a Gibbs free energy described by
\begin{equation}
    \mathcal{G} (\theta , F) = -C \left( \theta \ln \left( \frac{\theta}{\theta_0} \right) - 
    (\theta - \theta_0) \right) - 
    \frac{F}{\rho} \left( \alpha ( \theta - \theta_0 ) + \frac{F}{2E} \right)
    \label{GreenEnergy}
\end{equation}
which is a function of temperature $\theta$ and force $F$, normalized so that $\mathcal{G} (\theta_0 , 0) = 0$, and is compatible with the thermo-physical material properties put forward in Eqn.~(\ref{specificHeat}--\ref{compliance}).  In this model the material constants $\rho$, $C$, $\alpha$ and $E$ are all considered to be of constant value across state space.

\subsection{Alveolar Chords as Rajagopal (Implicit) Thermoelastic Fibers}

In 2003, Rajagopal \cite{Rajagopal03} introduced the idea of an implicit elastic solid.  In 2016, Freed \&\ Rajagopal \cite{FreedRajagopal16} constructed an elastic fiber model that convolves an explicit energy with an implicit energy.  In their approach they decomposed strain $e \defeq \ln (L / L_0)$ into a sum of two strains, viz., $e = e_1 + e_2$ where $e_1 \defeq \ln (L_1 / L_0)$ and $e_2 \defeq \ln (L / L_1)$.  Length $L_0$ is an initial fiber length, viz., its length whenever $F = 0$.  Length $L_1$ can be thought of as a fiber length caused solely by an unraveling of molecular configuration (e.g., an unraveling of collagen crimp) under an applied load of $F$.  The state associated with length $L_1$ is non-physical in that one cannot unravel molecules without also stretching them to some extent.  Final length $L$ is the actual fiber length under an applied load $F$ caused by both a reconfiguration and a stretching of its molecular network.  Here we present their ideas in terms of a Gibbs free-energy function. (Freed \& Rajagopal originally used a Helmholtz free-energy function.)

Let the Gibbs free energy be described by a function of the form\footnote{
    One might be tempted to consider an implicit energy function of the form $\mathcal{G} = \mathcal{G}_1 (\theta ,  e_1 , F ) + \mathcal{G}_2 (\theta , F)$, but this would lead to a non-symmetric susceptibility matrix.  Consequently, it would not satisfy Maxwell's thermo\-dynamic constraint for integrability; hence, it is inadmissible as a Gibbs potential.
}
\begin{equation}
\mathcal{G} (\theta , e , F) \defeq \mathcal{G}_1 ( e_1 , F ) + \mathcal{G}_2 ( \theta , F )
\quad \text{where} \quad
\mathrm{d} \mathcal{G} = -\eta \, \mathrm{d} \theta - 
\tfrac{1}{\rho} e \, \mathrm{d} F
\label{GibbsFreeEnergy}
\end{equation}
with $\mathcal{G}_1$ being an implicit potential (a configuration energy) and $\mathcal{G}_2$ being an explicit potential (a strain energy).  This energy function leads to the same constitutive equation displayed in Eqn.~(\ref{thermoelasticCE1D}), but whose material properties (\ref{specificHeat}--\ref{compliance}) are now interpreted according to the expressions
\begin{subequations}
    \label{physicalFields1Dfiber}
    \begin{align}
    C & \defeq \theta \, \partial_{\theta\,} \eta |_F = 
    -\theta \, \partial_{\theta\theta\,} \mathcal{G}_2(\theta , F)
    \label{specificHeat1D} \\
    \alpha & \defeq \partial_{\theta\,} e |_F = 
    -\rho \, \partial_{F\theta\,} \mathcal{G}_2(\theta , F) \equiv
    -\rho \, \partial_{\theta F\,} \mathcal{G}_2(\theta , F)
    \label{thermalExpansion1D} \\
    1/E & \defeq \partial_{F\,} e |_{\theta} = -
    \bigl( \rho \, \partial_{e_1} \mathcal{G}_1 ( e_1, F ) \bigr)^{-1} 
    \bigl( e + \rho \, \partial_{F\,} \mathcal{G} (\theta , e , F) \bigr) -
    \rho \, \partial_{FF\,} \mathcal{G}_2(\theta , F)
    \label{compliance1D}
    \end{align}
\end{subequations}
whose elastic compliance $1/E$ is found to be the sum of two compliances: one explicit in origin and the other implicit in origin, and where the mass density $\rho$ is per unit length. 

\medskip\noindent
\textbf{Derivation}: 
Because Gibbs energy is a state function, its differential is exact allowing one to write the left-hand side of the thermo\-dynamic expression $\mathrm{d} \mathcal{G} = -\eta \, \mathrm{d} \theta - \tfrac{1}{\rho} e \, \mathrm{d}F$ as $\mathrm{d} \mathcal{G} = \partial_{e_1} \mathcal{G}_1 \, \mathrm{d} e_1 + \partial_{F\,} \mathcal{G}_1 \, \mathrm{d}F + \partial_{\theta\,} \mathcal{G}_2 \, \mathrm{d} \theta + \partial_{F\,} \mathcal{G}_2 \, \mathrm{d}F$. Recalling that $e = e_1 + e_2$, the explicit (hyperelastic) terms combine to produce constitutive expressions of
\begin{displaymath}
\eta = -\partial_{\theta\,} \mathcal{G}_2 (\theta , F) 
\quad \text{and} \quad
e_2 = -\rho \, \partial_{F\,} \mathcal{G}_2 (\theta , F)
\end{displaymath} 
with the remaining implicit terms collecting to produce the following differential equation
\begin{displaymath}
\rho \, \partial_{e_1} \mathcal{G}_1 ( e_1 , F ) \, \mathrm{d} e_1 = 
-\bigl( e_1 + \rho \, \partial_{F\,} \mathcal{G}_1 ( e_1 , F )
\bigr) \mathrm{d}F .
\end{displaymath}
Differentiating the constitutive equation for entropy with respect to state leads directly to the expressions for specific heat $C$ and thermal expansion $\alpha$ stated in Eqns.~(\ref{specificHeat1D} \& \ref{thermalExpansion1D}).  Recalling that the strains add, i.e., $e = e_1 + e_2$, and therefore so do their rates, viz., $\mathrm{d} e = \mathrm{d} e_1 + \mathrm{d} e_2$, a consequence of them being logarithmic in construction, then upon rearranging the implicit constitutive equation to solve for $\mathrm{d} e_1$, while differentiating the explicit constitutive equation for $e_2$, and finally adding these strain increments to get $\mathrm{d} e$, one obtains the elastic compliance function stated in Eqn.~(\ref{compliance1D}).  \hfill $\qed$

\subsubsection{A Biologic Fiber}
\label{secBioFiber}

The fiber model of Freed \&\ Rajagopal \cite{FreedRajagopal16} imposes a strain-limiting constraint onto internal strain $e_1$ whenever one considers a Gibbs free-energy function of the form
\begin{subequations}
    \label{RajagopaleanFiber}
    \begin{align}
    \mathcal{G}_1 ( e_1 , F ) & = - \frac{1}{\rho} \Bigl(
    e_t ( E_1 e_1 - F ) + 2 e_1 F \Bigr)
    \label{FreedEnergy} \\
    \mathcal{G}_2(\theta , F) & = -C \left( \theta \ln \left( \frac{\theta}{\theta_0} \right) - 
    (\theta - \theta_0) \right) - 
    \frac{F}{\rho} \left( \alpha ( \theta - \theta_0 ) + \frac{F}{2E_2} \right)
    \label{HookeanEnergy} \\
    \intertext{which depend upon tempreature $\theta$, force $F$, and internal strain $e_1$, normalized so that $\mathcal{G}_1(0, 0) = 0$ and $\mathcal{G}_2(\theta_0,0)=0$.  In fact, the explicit energy adopted here is Hookean, cf.\ Eqn.~(\ref{GreenEnergy}).  The resulting constitutive responses for entropy $\eta$ and strain $e=\ln (L/L_0)$ are therefore described by the following matrix differential equation}
    \left\{ \begin{matrix}
    \mathrm{d} \eta \\ \mathrm{d} e 
    \end{matrix} \right\} & = \begin{bmatrix}
    C / \theta & \alpha / \rho  \\
    \alpha & 1 / E
    \end{bmatrix}
    \left\{ \begin{matrix}
    \mathrm{d} \theta \\ \mathrm{d} F
    \end{matrix} \right\}
    \label{FungCE} \\
    \intertext{with an elastic compliance whose tangent response is described by}
    \frac{1}{E(\theta , e , F)} & = 
    \frac{e_t - e_1}{E_1 e_t + 2F} + \frac{1}{E_2} 
    \quad \text{wherein} \quad
    e_1 = e - \alpha (\theta - \theta_0) - \frac{F}{E_2}
    \label{FRcompliance}
    \end{align}
\end{subequations}
with an initial tangent modulus $E(\theta_0, 0, 0)$ of $E_1 E_2 / (E_1 + E_2) \approx E_1$ whenever $E_2 \gg E_1 > 0$, while the terminal tangent modulus $E(e_1 \! = \! e_t)$ is $E_2$, with a transition strain occurring at $e_t > 0$ that establishes a limiting state for internal strain $e_1$, i.e., $0 \leq e_1 < e_t$, which is that strain whereat a fiber's molecular configuration has been completely unraveled.  Here $\rho$ is a mass per unit length of fiber, $C$ is a specific heat per unit mass at constant force, and $\alpha$ is a coefficient of linear thermal expansion at constant force, all of which have the same physical interpretation as their counterparts for the Hookean fiber.  Only their elastic compliances are interpreted differently.  This model has been found to be superior to other models commonly employed in the literature for modeling biologic fibers \cite{AkintundeMiller18}.

Biologic fibers, per our application, are long and slender.  Consequently, they will buckle under compression.  Buckling is not accounted for in our modeling of alveolar chords.  Rather, it is assumed that the compliant response at the origin, with modulus $E_1 E_2 / ( E_1 + E_2 )$, continues into compression, thereby ensuring a measure of numeric stability in our software.

The above methodology would allow us to construct a suite of thermo\-dynamically admissible elastic compliance functions, but we only have need for the one considered above.

\subsection{Alveolar Septa as Green (Explicit) Thermoelastic Membranes}

We observed in \S\ref{secNonuniform2D} that an alveolar membrane has a response comprised of an uniform contribution and a non-uniform contribution, and that these two contributions are uncoupled; consequently, their internal energies add in that $\mathcal{U} (\eta , \xi , \varepsilon , \gamma) = \mathcal{U}_u (\eta , \xi) + \mathcal{U}_n (\varepsilon , \gamma)$ with $\mathcal{U}_u$ being the uniform contribution of $\mathcal{U}$, and $\mathcal{U}_n$ being the non-uniform contribution of $\mathcal{U}$.  It is advantageous to relate the material constants to a Gibbs free-energy approach for the uniform contribution, while retaining the internal energy approach for its non-uniform contribution.

\subsubsection{Uniform Response}

From thermo\-dynamics, Eqn.~(\ref{thermoelastic2Dlaw}), comes $\mathrm{d}\hspace{1pt}\mathcal{U}_u = \theta \, \mathrm{d} \eta + \tfrac{1}{\rho} T \, \mathrm{d}A / \! A = \theta \, \mathrm{d} \eta + \tfrac{1}{\rho} \pi \, \mathrm{d}\xi$ where $\pi = 2T$ and $\mathrm{d}\xi = \tfrac{1}{2} A^{-1} \, \mathrm{d}A$ whose mass density $\rho$ now has units of mass per unit area.  Upon writing this expression in its Gibbs form $\mathrm{d} \mathcal{G}_u = -\eta \, \mathrm{d} \theta - \tfrac{1}{\rho} \xi \, \mathrm{d} \pi$ via a Legendre transformation comes a constitutive equation appropriate for describing the uniform response of a thermo\-elastic planar membrane, namely
\begin{subequations}
    \label{uniformMembraneModel}
    \begin{align}
\left\{ \begin{matrix}
\mathrm{d} \eta \\ \mathrm{d} \xi
\end{matrix} \right\} & = -\begin{bmatrix}
\partial_{\theta\theta\,} \mathcal{G}_u & \partial_{\theta\pi\,} \mathcal{G}_u \\
\rho \, \partial_{\pi\theta\,} \mathcal{G}_u & \rho \, \partial_{\pi\pi\,} \mathcal{G}_u
\end{bmatrix} 
\left\{ \begin{matrix}
\mathrm{d} \theta \\ \mathrm{d} \pi
\end{matrix} \right\} = \begin{bmatrix}
C / \theta & \alpha / 2 \rho \\ \alpha / 2 & 1 / 4 M
\end{bmatrix} \left\{ \begin{matrix}
\mathrm{d} \theta \\ \mathrm{d} \pi
\end{matrix} \right\}
\label{thermoelasticCE2D} \\
\intertext{with material constants defined accordingly}
    C & \defeq \theta \, \partial_{\theta\,} \eta |_{\pi} = -\theta \, \partial_{\theta\theta\,} \mathcal{G}_u \\
    \alpha & \defeq A^{-1} \, \partial_{\theta\,} A |_T = 2 \, \partial_{\theta\,} \xi |_{\pi} = -2\rho \, \partial_{\pi\theta\,} \mathcal{G}_u = -2\rho \, \partial_{\theta\pi\,} \mathcal{G}_u \\
    1/M & \defeq A^{-1} \, \partial_{T\,} A |_{\theta} = 2 \, \partial_{T\,} \xi |_{\theta} = 4 \, \partial_{\pi\,} \xi |_{\theta} = -4\rho \, \partial_{\pi\pi\,} \mathcal{G}_u
    \end{align}
\end{subequations}
where $C$ is the specific heat at constant pressure, $\alpha$ is the areal coefficient of thermal expansion, and $M$ is the modulus of dilation (uniform expansion).  Here $\eta$ is entropy, $\theta$ is temperature, $T \defeq \tfrac{1}{2} ( \sigma_{11} + \sigma_{22})$ is surface tension (an invariant in 2D whose intensive variable is $\pi = 2T = \sigma_{11} + \sigma_{22}$ wherein $\sigma_{ij}$ are components of Cauchy stress quantified in 2D), and $A$ is area (whose extensive variable is $\xi = \ln \sqrt{A / \! A_0}$, which denotes dilation, i.e., that strain describing an uniform areal expansion).  

\subsubsection{Non-Uniform Response}

Because $\mathrm{d} \hspace{1pt}\mathcal{U} = \mathrm{d}\hspace{1pt} \mathcal{U}_u + \mathrm{d}\hspace{1pt} \mathcal{U}_n$ with $\mathrm{d}\hspace{1pt} \mathcal{U}_u = \theta \, \mathrm{d} \eta + \tfrac{1}{\rho} \pi \, \mathrm{d} \xi$ it follows that $\mathrm{d}\hspace{1pt} \mathcal{U}_n = \tfrac{1}{\rho} ( \sigma \, \mathrm{d} \varepsilon + \tau \, \mathrm{d} \gamma )$ out of which come constitutive equations that govern the non-uniform response of a Green elastic membrane, viz., $\sigma = \rho \, \partial_{\varepsilon\,} \mathcal{U}_n$ and $\tau = \rho \, \partial_{\gamma\,} \mathcal{U}_n$ which, assuming these are continuous and differentiable functions of state, become the following system of differential equations
\begin{subequations}
    \begin{align}
    \left\{ \begin{matrix}  
    \mathrm{d} \sigma \\ \mathrm{d} \tau
    \end{matrix} \right\} & = \rho \begin{bmatrix}
    \partial_{\varepsilon\varepsilon\,} \mathcal{U}_n & 
    \partial_{\varepsilon\gamma\,} \mathcal{U}_n \\
    \partial_{\gamma\varepsilon\,} \mathcal{U}_n &
    \partial_{\gamma\gamma\,} \mathcal{U}_n
    \end{bmatrix} \left\{ \begin{matrix}
    \mathrm{d} \varepsilon \\ \mathrm{d} \gamma
    \end{matrix} \right\} = \begin{bmatrix}
    2 N & 2 \tau \\ 
    2 \tau & G
    \end{bmatrix} \left\{ \begin{matrix}
    \mathrm{d} \varepsilon \\ \mathrm{d} \gamma
    \end{matrix} \right\} \\
    \intertext{with material constants defined accordingly}
    N & \defeq \Gamma \, \partial_{\Gamma} ( \mathcal{S}_{11} - \mathcal{S}_{22} ) |_g = \tfrac{1}{2} \, \partial_{\varepsilon\,} \pi |_{\gamma} = \tfrac{1}{2} \rho \, \partial_{\varepsilon\varepsilon\,} \mathcal{U}_n \\
    G & \defeq \Gamma \, \partial_{g\,} \mathcal{S}_{21} |_{\Gamma} = \partial_{\gamma\,} \tau |_{\varepsilon} = \rho \, \partial_{\gamma\gamma\,} \mathcal{U}_n \\
    \intertext{subject to the constraint}
    2\tau & = \rho \, \partial_{\varepsilon\gamma\,} \mathcal{U}_n = \rho \, \partial_{\gamma\varepsilon\,} \mathcal{U}_n
    \label{thermoConstraint}
    \end{align}
\end{subequations}
which follows from Eqn.~(\ref{HelmholtzMembraneODEs}).  This constraint suggests that the non-uniform response is actually Rajagopal elastic, because $\tau$ is a response variable, with $\mathcal{U}_n$ therefore being an implicit function of state.


\subsubsection{A Hookean Membrane}

A Green elastic membrane whose uniform response is governed by a Gibbs free-energy function of the form
\begin{subequations}
\label{membraneHookeanEnergies}
\begin{align}
\mathcal{G}_u (\theta , \pi) & = -C \left( \theta \ln \left( \frac{\theta}{\theta_0} \right) -
(\theta - \theta_0) \right) - 
\frac{\pi}{2\rho} \left( \alpha ( \theta - \theta_0 ) + \frac{\pi}{4M} \right)
\label{Gibbs2DHookean} \\
   \intertext{and whose non-uniform response is governed by an internal energy function of the form}
\mathcal{U}_n (\varepsilon , \gamma) & = \mathcal{U}_n (-\varepsilon , \gamma) = \mathcal{U}_n (\varepsilon , -\gamma) = \mathcal{U}_n (-\varepsilon , -\gamma) = \tfrac{1}{\rho} \bigl( 2 \tau \varepsilon \gamma + N \varepsilon^2 + \tfrac{1}{2} G \gamma^2 \bigr) \\
\intertext{with negative strain arguments accounting for symmetries in strain, which are pertinent for all terms, except for the coupling term $2 \tau \varepsilon \gamma$ that arises because of the constraint in Eqn.~(\ref{thermoConstraint}).  Form these formul\ae\ come the following system of differential equations}
\left\{ \begin{matrix}
\mathrm{d} \eta \\ \mathrm{d} \pi \\
\mathrm{d} \sigma \\ \mathrm{d} \tau
\end{matrix} \right\} & = \begin{bmatrix}
C / \theta - M \alpha^2 / \rho & 2 M \alpha / \rho & 0 & 0 \\
-2 M \alpha & 4 M & 0 & 0 \\
0 & 0 & 2 N & 2 \tau \\
0 & 0 & 2 \tau & G 
\end{bmatrix} \left\{ \begin{matrix}
\mathrm{d} \theta \\ \mathrm{d} \xi \\
\mathrm{d} \varepsilon \\ \mathrm{d} \gamma
\end{matrix} \right\}
\end{align}
\end{subequations}
which describes a thermo\-elastic Hookean membrane whose material parameters $\rho$, $C$, $\alpha$, $M$, $N$ and $G$ are each constant valued across state space.

\subsection{Alveolar Septa as Rajagopal (Implicit) Thermoelastic Membranes}

We employ implicit elasticity here to derive a constitutive theory suitable for describing biologic membranes.

\subsubsection{Biologic Membrane Under Uniform Motions}

Like the implicit elastic fiber introduced in Eqn.~(\ref{RajagopaleanFiber}), the uniform response of an implicit elastic membrane with a strain-limiting dilation can be modeled using a Gibbs free energy $\mathcal{G}_u (\theta , \xi , \pi ) \defeq \mathcal{G}_1 (\xi_1 ,\pi) + \mathcal{G}_2 (\theta , \pi)$ where the dilation $\xi \defeq \ln \sqrt{A / \! A_0}$ is considered to decompose into a sum of two dilations: $\xi_1 \defeq \ln \sqrt{A_1 / \! A_0}$ and $\xi_2 \defeq \ln \sqrt{A / \! A_1}$ so that $\xi = \xi_1 + \xi_2$, with like interpretations as those from their linear counterparts, viz., $e$, $e_1$ and $e_2$; specifically, 
\begin{subequations}
    \label{physicalFields2Dmembrane}
    \begin{align}
    C & \defeq \theta \, \partial_{\theta\,} \eta |_{\pi} = 
    -\theta \, \partial_{\theta\theta\,} \mathcal{G}_2(\theta , \pi)
    \label{specificHeat2Dmembrane} \\
    \alpha & \defeq \partial_{\theta\,} \xi |_{\pi} = 
    -\rho \, \partial_{\pi\theta\,} \mathcal{G}_2(\theta , \pi) \equiv
    -\rho \, \partial_{\theta\pi\,} \mathcal{G}_2(\theta , \pi)
    \label{thermalExpansion2Dmembrane} \\
    1/M & \defeq 4 \, \partial_{\pi\,} \xi |_{\theta} = -4 \Bigl(
    \bigl( \rho \, \partial_{\xi_1} \mathcal{G}_1 ( \xi_1, \pi ) \bigr)^{-1} 
    \bigl( \xi + \rho \, \partial_{\pi\,} \mathcal{G}_u (\theta , \xi , \pi ) \bigr) +
    \rho \, \partial_{\pi\pi\,} \mathcal{G}_2(\theta , \pi) \Bigr)
    \label{compliance2Dmembrane}
    \end{align}
\end{subequations}
whose derivation is analogous to that of the implicit fiber above.  In this case, we consider a Gibbs free-energy function of the form
\begin{subequations} 
    \label{uniformMembrane}
    \begin{align}
    \mathcal{G}_1 (\xi_1 , \pi) & = - \frac{1}{\rho} 
    \Bigl( \xi_t (4M_1 \xi_1 - \pi ) + 2 \xi_1 \pi \Bigr) \\
    \mathcal{G}_2 (\theta , \pi) & = -C \left( \theta \ln \left( \frac{\theta}{\theta_0} \right) -
    (\theta - \theta_0) \right) - \frac{\pi}{2\rho} \left( 
    \alpha ( \theta - \theta_0 ) + \frac{\pi}{4M_2} \right) \\
    \intertext{whose resulting elastic compliance is}
    \frac{1}{M(\theta, \xi, \pi)} & = 
    \frac{\xi_t - \xi_1}{M_1 \xi_t + \pi / 2} + \frac{1}{M_2} 
    \quad \text{wherein} \quad 
    \xi_1 = \xi - \tfrac{1}{2} \alpha (\theta - \theta_0) - \frac{\pi}{4M_2}
    \label{membraneCompliance}
    \end{align}
\end{subequations}
with $\xi_t > 0$ being an upper bound on strain $\xi_1$ in that $0 \leq \xi_1 < \xi_t$.  Such a membrane has an initial tangent stiffness $M(\theta_0, 0, 0)$ of $M_1 M_2 / ( M_1 + M_2 ) \approx M_1$ whenever $M_2 \gg M_1 > 0$, and a terminal tangent stiffness $M(\xi_1 \! = \! \xi_t)$ of $M_2$.  

Membranes will wrinkle under states of negative surface tension (or dilation).  In alveolar mechanics, surfactant helps to prevent this, and a possible ensuing alveolar collapse.  Wrinkling is not accounted for in our modeling of alveolar septa.  Rather, like fibers, membranes are assumed to support compression with a modulus of $M_1 M_2 / ( M_1 + M_2 )$, which associates with the compliant response found at the origin (zero tension, zero dilation).  This is done to help ensure numeric stability in our software.

The difference between a Green and Rajagopal thermo\-elastic membrane undergoing a dilation is in their definitions for elastic compliance.  There is no difference in their properties for specific heat or thermal expansion.  The above model has been successfully applied to a visceral pleura membrane \cite{Freedetal17}.

\subsection{Biologic Membrane Under Non-Uniform Motions}

We seek an energetic construction that is consistent with that of the Freed \& Rajagopal fiber model \cite{FreedRajagopal16}, but which is applicable to the non-uniform responses of planar membranes.  A Rajagopal elastic solid is implicit; therefore, we consider an internal energy with the following special structure
\begin{equation}
\mathcal{U}_n ( \varepsilon , \gamma , \sigma , \tau ) = \tfrac{1}{\rho} 2 \varepsilon \gamma \tau + \mathcal{U}_1 ( \varepsilon_1 , \sigma ) + \mathcal{U}_2 ( \varepsilon_2 ) + \mathcal{U}_3 ( \gamma_1 , \tau ) + \mathcal{U}_4 ( \gamma_2 )
\label{nonuniformEnergy}
\end{equation}
that depends upon squeeze strains $\varepsilon \defeq \ln \sqrt{\Gamma \! / \Gamma_0}$, $\varepsilon_1 \defeq \ln \sqrt{ \Gamma_1 / \Gamma_0}$ and $\varepsilon_2 \defeq \ln \sqrt{ \Gamma \! / \Gamma_1}$, and shear strains $\gamma \defeq g - g_0$, $\gamma_1 \defeq g_1 - g_0$ and $\gamma_2 \defeq g - g_1$, both of which are additive in that $\varepsilon = \varepsilon_1 + \varepsilon_2$ and $\gamma = \gamma_1 + \gamma_2$, and as such, so are their differential rates of change $\mathrm{d} \varepsilon = \mathrm{d} \varepsilon_1 + \mathrm{d} \varepsilon_2$ and $\mathrm{d} \gamma = \mathrm{d} \gamma_1 + \mathrm{d} \gamma_2$.  Strains $\varepsilon_1$ and $\gamma_1$ may be thought of as describing unravelings of molecular configuration, analogous to $e_1$ in the fiber model of Eqn.~(\ref{RajagopaleanFiber}) and $\xi_1$ in the uniform membrane model of Eqn.~(\ref{membraneCompliance}).  The first term on the right-hand side of Eqn.~(\ref{nonuniformEnergy}) ensures that the constraint in Eqn.~(\ref{thermoConstraint}) is satisfied.  Other than this term, no coupling between squeeze and shear is assumed in this energy function.  Energies $\mathcal{U}_1$ and $\mathcal{U}_3$ are Rajagopal elastic (implicit), while energies $\mathcal{U}_2$ and $\mathcal{U}_4$ are Green elastic (explicit).

Given a non-uniform internal energy in the form of Eqn.~(\ref{nonuniformEnergy}), then the squeeze compliance is found to be
\begin{subequations}
    \label{nonuniformCompliances}
    \begin{align}
    \frac{1}{N} & = 2 \left( \frac{\rho \, \partial_{\sigma\,} \hspace{0.5pt} \mathcal{U}_1 ( \varepsilon_1 , \sigma )}{\rho \, \partial_{\varepsilon_2}  \mathcal{U}_2 ( \varepsilon_2 ) - \rho \, \partial_{\varepsilon_1} \mathcal{U}_1 ( \varepsilon_1 , \sigma )} + \frac{1}{\rho \, \partial_{\varepsilon_2 \varepsilon_2} \mathcal{U}_2 ( \varepsilon_2 )} \right) 
    \label{squeezeCompliance} \\
    \intertext{while the shear compliance is found to be}
    \frac{1}{G} & = \Gamma \left( \frac{\rho \, \partial_{\tau\,} \hspace{0.5pt} \mathcal{U}_3 ( \gamma_1 , \tau ) + 2 \varepsilon \gamma}{\rho \, \partial_{\gamma_2} \mathcal{U}_4 ( \gamma_2 ) - \rho \, \partial_{\gamma_1}  \mathcal{U}_3 ( \gamma_1 , \tau )} + \frac{1 - 2\varepsilon}{\rho \, \partial_{\gamma_2 \gamma_2} \mathcal{U}_4 ( \gamma_2 )} \right)
    \label{shearCompliance}
    \end{align}
\end{subequations}
whose mathematical structure is similar to that of the fiber model presented in Eqn.~(\ref{RajagopaleanFiber}).  The first terms in the parentheses are Rajagopal elastic. The second terms are Green elastic.  The shear compliance $1/G$ has extra terms of $2 \varepsilon \gamma$ and $(1 - 2\varepsilon)$ that arise because of the constraint energy $\tfrac{1}{\rho} 2 \varepsilon \gamma \tau$, which is a direct consequence of defining shear as $\tau \defeq \Gamma \mathcal{S}_{21}$.  These extra terms are missing in the original, implicit, membrane model derived by Freed \textit{et~al}.~\cite{Freedetal17}.

\medskip\noindent
\textbf{Derivation}: The First and Second Laws of Thermo\-dynamics, as they pertain to non-uniform contributions of stress power, obey $\mathrm{d} \hspace{1pt} \mathcal{U}_n ( \varepsilon , \gamma , \sigma , \tau ) = \tfrac{1}{\rho} \mathrm{d} W_n$ where, from Eqn.~(\ref{nonuniformEnergy}), one gets $\rho \, \mathrm{d} \hspace{1pt} \mathcal{U}_n = \bigl( 2 \gamma \tau + \rho \, \partial_{\varepsilon_1} \mathcal{U}_1 ( \varepsilon_1 , \sigma ) \bigr) \mathrm{d} \varepsilon_1 + \bigl( 2 \gamma \tau + \rho \, \partial_{\varepsilon_2} \mathcal{U}_2 ( \varepsilon_2 ) \bigr) \mathrm{d} \varepsilon_2 + \bigl( \rho \, \partial_{\sigma\,} \mathcal{U}_1 ( \varepsilon_1 , \sigma ) \bigr) \mathrm{d} \sigma + \bigl( 2 \varepsilon \tau + \rho \, \partial_{\gamma_1} \mathcal{U}_3 ( \gamma_1 , \tau ) \bigr) \mathrm{d} \gamma_1 + \bigl( 2 \varepsilon \tau + \rho \, \partial_{\gamma_2} \mathcal{U}_4 ( \gamma_2 ) \bigr) \mathrm{d} \gamma_2 + \bigl( 2 \varepsilon \gamma + \rho \, \partial_{\tau\,} \mathcal{U}_3 ( \gamma_1 , \tau ) \bigr) \mathrm{d} \tau$ while the right side becomes $\mathrm{d}W_n = \sigma \, \mathrm{d} \varepsilon_1 + \sigma \, \mathrm{d} \varepsilon_2 + \tau \, \mathrm{d} \gamma_1 + \tau \, \mathrm{d} \gamma_2$ because of the additivity in strain rates.  Gathering like terms result in two Green elastic formul\ae\ that produce the stresses
\begin{displaymath}
\sigma = 2 \gamma \tau + \rho \, \partial_{\varepsilon_2} \hspace{0.5pt} \mathcal{U}_2 ( \varepsilon_2 ) 
\quad \text{and} \quad
\tau = 2 \varepsilon \tau + \rho \, \partial_{\gamma_2} \hspace{0.5pt} \mathcal{U}_4 ( \gamma_2 )
\end{displaymath}
and two Rajagopal elastic formul\ae\ that govern the internal strains
\begin{align*}
\rho \, \partial_{\sigma\,} \mathcal{U}_1 ( \varepsilon_1 , \sigma ) \, \mathrm{d} \sigma & = \bigl( \pi - 2 \gamma \tau - \rho \, \partial_{\varepsilon_1} \mathcal{U}_1 ( \varepsilon_1 , \sigma ) \bigr) \mathrm{d} \varepsilon_1 \\
\bigl( 2 \varepsilon \gamma + \rho \, \partial_{\tau\,} \mathcal{U}_3 ( \gamma_1 , \tau ) \bigr) \mathrm{d} \tau & = \bigl( \tau ( 1 - 2 \varepsilon ) - \rho \, \partial_{\gamma_1} \mathcal{U}_3 ( \gamma_1 , \tau ) \bigr) \mathrm{d} \gamma_1 
\end{align*}
that when combined produce the constitutive responses
\begin{align*}
\rho \, \partial_{\sigma\,} \mathcal{U}_1 ( \varepsilon_1 , \sigma ) \, \mathrm{d} \sigma & = \bigl( \rho \, \partial_{\varepsilon_2} \mathcal{U}_2 ( \varepsilon_2 ) - \rho \, \partial_{\varepsilon_1} \mathcal{U}_1 ( \varepsilon_1 , \sigma ) \bigr) \mathrm{d} \varepsilon_1 \\
\bigl( 2 \varepsilon \gamma + \rho \, \partial_{\tau\,} \mathcal{U}_3 ( \gamma_1 , \tau ) \bigr) \mathrm{d} \tau & = \bigl( \rho \, \partial_{\gamma_2} \mathcal{U}_4 ( \gamma_2 ) - \rho \, \partial_{\gamma_1} \mathcal{U}_3 ( \gamma_1 , \tau ) \bigr) \mathrm{d} \gamma_1 .
\end{align*}
We now differentiate the Green elastic constitutive equations, thereby putting them into differential form.  When doing this, we impose a constraint that the squeeze compliance $1/N$ is to be evaluated at constant shear $\gamma$, while the shear compliance $1/G$ is to be evaluated at constant squeeze $\varepsilon$, consequently
\begin{displaymath}
\mathrm{d} \sigma |_{\gamma} = \rho \, \partial_{\varepsilon_2 \varepsilon_2} \mathcal{U}_2 ( \varepsilon_2 ) \, \mathrm{d} \varepsilon_2
\quad \text{and} \quad
\mathrm{d} \tau |_{\varepsilon} = 2 \varepsilon \, \mathrm{d} \tau + \rho \, \partial_{\gamma_2 \gamma_2} \mathcal{U}_4 ( \gamma_2 ) \, \mathrm{d} \gamma_2 .
\end{displaymath}
Because $\mathrm{d} \varepsilon = \mathrm{d} \varepsilon_1 + \mathrm{d} \varepsilon_2$ and $\mathrm{d} \gamma = \mathrm{d} \gamma_1 + \mathrm{d} \gamma_2$, one solves the above equations for strain rate, adds them appropriately, and from these one can thereby construct the two compliances found in Eqn.~(\ref{nonuniformCompliances}) via the expressions
\begin{displaymath}
\frac{1}{N} = 2 \left. \frac{\mathrm{d} \varepsilon}{\mathrm{d}\sigma} \right|_{\gamma} = 2 \left( \left. \frac{\mathrm{d} \varepsilon_1}{\mathrm{d}\sigma} \right|_{\gamma} + \left. \frac{\mathrm{d} \varepsilon_2}{\mathrm{d}\sigma} \right|_{\gamma} \right) 
\quad \text{and} \quad
\frac{1}{G} = \Gamma \left. \frac{\mathrm{d} \gamma}{\mathrm{d} \tau} \right|_{\varepsilon} = \Gamma \left(  \left. \frac{\mathrm{d} \gamma_1}{\mathrm{d} \tau} \right|_{\varepsilon} + \left. \frac{\mathrm{d} \gamma_2}{\mathrm{d} \tau} \right|_{\varepsilon} \right)
\end{displaymath}
that when all terms are collected together become Eqn.~(\ref{nonuniformCompliances}). \hfill $\qed$

We now seek an energy function (\ref{nonuniformEnergy}) that produces compliances $1/N$ and $1/G$ with a like mathematical structure to that of Eqn.~(\ref{membraneCompliance}) for dilation, viz., $1/M$; specifically, we shall consider
\begin{subequations}
    \label{nonuniformComplianceEnergies}
    \begin{align}
    \rho \, \mathcal{U}_1 ( \varepsilon_1 , \sigma ) & = \varepsilon_t \bigl( N_1 \varepsilon_1 - \sigma \bigr) + \varepsilon_1 \sigma &
    \rho \, \mathcal{U}_2 ( \varepsilon_2 ) & = \tfrac{1}{2} N_2 \varepsilon_2^{\phantom{2}2} 
    \label{squeezeEnergy} \\
    \rho \, \mathcal{U}_3 ( \gamma_1 , \tau ) & = \gamma_t \bigl( G_1 \gamma_1 - \tau \bigr) + \gamma_1 \tau &
    \rho \, \mathcal{U}_4 ( \gamma_2 ) & = \tfrac{1}{2} G_2 \gamma_2^{\phantom{2}2}
    \label{shearEnergy}
    \end{align}
\end{subequations}
which have the same mathematical structure as the energies for the biologic fiber (\ref{RajagopaleanFiber}) and uniform membrane (\ref{uniformMembrane}).  When substituted into Eqn.~(\ref{nonuniformCompliances}), they produce the following thermo\-elastic compliances
\begin{subequations}
    \label{nonuniformComplianceFns}
    \begin{align}
    \frac{1}{N} & = 2 \left( \frac{ \varepsilon_t - | \varepsilon_1 |}{N_1 \varepsilon_t + 2 \gamma \tau} + \frac{1}{N_2} \right) &
    \varepsilon_1 & = \varepsilon - \frac{\sigma - 2 \gamma \tau}{N_2}
    \label{squeezeCompliance2D} \\
    \frac{1}{G} & = \Gamma \left( \frac{\gamma_t - 2\varepsilon \gamma - | \gamma_1 |}{G_1 \gamma_t + 2 \varepsilon \tau} + \frac{1-2\varepsilon}{G_2} \right) & 
    \gamma_1 & = \gamma - \frac{(1 - 2\varepsilon) \tau}{G_2}
    \label{shearCompliance2D}
    \end{align}
\end{subequations}
where absolute values are introduced because the squeeze $\varepsilon$ and shear $\gamma$ strains can take on both positive and negative values, viz., they are odd functions, cf.\ \cite{Freedetal17}.  (Recall that in our construction the material parameters are tangents to response curves---the models are differential.)  Like our other biologic models, the squeeze compliance $1/N$ is described by three material parameters: an asymptotic modulus at the reference state of $N_1 N_2 / (N_1 + N_2) \approx N_1$ whenever $N_2 \gg N_1 > 0$, where $N_1$ may be thought of as the stiffness of an unstretched molecular network; a terminal modulus $N_2$ designating a stiffness after this molecular network has been stretched out; and a limiting state of configurational squeeze $\varepsilon_t$.  The shear compliance $1/G$ is also described by three material parameters: an asymptotic modulus at the reference state of $G_1 G_2 / ( G_1 + G_2 ) \approx G_1$ whenever $G_2 \gg G_1 > 0$, a terminal modulus $G_2$, and a limiting state of configurational shear $\gamma_t$ that shifts by $2\varepsilon\gamma$.  This shift is a consequence of the stress-strain coupling introduced in shear $\tau \defeq \Gamma \mathcal{S}_{21}$.

\newpage
\section{Overview}

These appendices describe interfaces for a software package written in Python whose intent is to model the micro-mechanical response of alveolar sacs that comprise the bulk of the parenchyma in lung tissue.  A flow chart for this software is presented in Fig.~\ref{figFlow}.  The software assumes the following design strategy: \textit{i\/}) a reference configuration exists (see \P\ below) and three sequential configurations exist that are separated in time by an uniform time step, \textit{ii\/}) the three sequential configurations associate with steps $n \! - \! 1$, $n$ and $n \! + \! 1$, \textit{iii\/}) co-ordinates for the next configuration are assigned through a vertex's \texttt{update} method (they can be reassigned multiple times at any step along a solution path), \textit{iv\/}) the \texttt{advance} method relabels the current data to their associated previous data, and then relabels the next data to their associated current data, thereby preparing the data structure of each object within a dodecahedral object for its next step along a solution path, and \textit{v\/}) the mechanical response is isotropic and can be described by three modes of deformation: dilatation\slash dilation, squeeze and shear \cite{Freedetal17,FreedZamani19}.

\begin{sidewaysfigure}
	\centering
	\includegraphics[width=\columnwidth]{figures/flow.pdf}
	\caption{Flow chart for a dodecahedral model of an alveolus.}
	\label{figFlow}
\end{sidewaysfigure}

The initial co-ordinates that locate each vertex in a dodecahedron used to model the alveoli of lung are assigned according to a reference configuration where the pleural pressure (the negative pressure surrounding lung in the pleural cavity) and the transpulmonary pressure (the difference between aleolar and pleural pressures) are both at zero gauge pressure, i.e., \textit{all pressures are atmospheric pressure in the reference state.}  This is not a typical physiological state.  The pleural pressure is normally negative, sucking the pleural membrane against the wall of the chest.  During expiration, the diaphragm is pushed up, reducing the pleural pressure.  The pleural pressure remains negative during breating at rest, but it can become positive during active expiration.  The surface tension created by surfactant helps keep most alveoli open during excursions into positive pleural pressures, but not all will remain open. Alveolar recruitment is not addressed here.  Under normal conditions, alveoli are their smallest at max expiration.  Alveolar size is predominately determined by the transpulmonary pressure.  The greater the transpulmonary pressure the greater the alveolar size.

Numerous methods belonging to the classes of these appendices have a string argument that is denoted as \texttt{state}  which can take on any of the following values:
\begin{labeling}{`p', `prev', `previous'}
\item [`c', `curr', `current'] gets the value for a current configuration 
\item [`n', `next'] gets the value for a next configuration 
\item [`p', `prev', `previous'] gets the value for a previous configuration 
\item [`r', `ref', `reference'] gets the value for the reference configuration 
\end{labeling}
Several strings can be used to denote each \texttt{state}.


\newpage
\appendix{Dodecahedra}
\label{appDodecahedra}

Module \texttt{dodecahedra.py} is Python code that exports class \texttt{dodecahedron}.  An object of type \texttt{dodecahedron} is comprised of twenty vertices labeled according to Table~\ref{TableDodecahedron}, as visualized in Fig.~\ref{figDodecahedron}, thirty chords assigned according to Table~\ref{Tablechordae}, and twelve pentagons assigned according to Table~\ref{TablePentagons}.  This class has the following interface:

\noindent
\textbf{class} \texttt{dodecahedron}

\medskip\noindent
\textit{constructor}

\medskip\noindent
\texttt{d = dodecahedron(F0, h, gaussPts=1, alveolarDiameter=1.952400802898434)} \\
\indent \texttt{F0} \qquad\qquad a deformation gradient: distortion from regular shape in a reference state \\
\indent \texttt{h} \qquad\qquad\; time step separating two neighboring configurations \\
\indent \texttt{gaussPts} \quad number of Gauss points in each pentagonal surface: $\in \{ 1 , 4, 7 \}$ \\
\indent \texttt{alveolarDiameter} \; mean diameter of an alveolar sac

\medskip\noindent
The default alveolar diameter results in vertices of the dodecahedron taking on co-ordinate values that associate with its natural configuration, i.e., all vertices touch the unit sphere from within.  Adopting the labeling scheme presented in Fig.~\ref{figDodecahedron}, the vertices are indexed according to Table~\ref{TableDodecahedron}, the chords are indexed according to Table~\ref{Tablechordae}, and the pentagons are indexed according to Table~\ref{TablePentagons}.  If \texttt{F0} is the identity matrix, then the shape will be that of a regular dodecahedron in its reference state; otherwise, the shape will be that of an irregular dodecahedron in its reference state.  The number of \texttt{gaussPts}, viz., 1, 4 or 7, establishes the quadrature scheme to be used for numeric integration, in accordance with Fig.~\ref{figQuadrature}.

\medskip\noindent
\textit{methods}

\medskip\noindent
\texttt{s = d.verticesToString(state)}

\medskip\noindent
Returns a formatted string description for this dodecahedron's vertices in configuration \texttt{state}.

\medskip\noindent
\texttt{s = d.chordsToString(state)}

\medskip\noindent
Returns a formatted string description for this dodecahedron's chords in configuration \texttt{state}.

\medskip\noindent
\texttt{s = d.pentagonsToString(state)}

\medskip\noindent
Returns a formatted string description for this dodecahedron's pentagons in configuration \texttt{state}.

\medskip\noindent
\texttt{v = d.getVertex(number)}

\medskip\noindent
Returns that vertex indexed with \texttt{number}, which must be in interval [1, 20].

\medskip\noindent
\texttt{c = d.getChord(number)}

\medskip\noindent
Returns that chord indexed with \texttt{number}, which must be in interval [1, 30].

\medskip\noindent
\texttt{p = d.getPentagon(number)}

\medskip\noindent
Returns that irregular pentagon indexed with \texttt{number}, which must be in interval [1, 12].

\newpage
\medskip\noindent
\texttt{d.update(nextF)}

\medskip\noindent
Assuming that the deformation imposed on an alveolus is homogeneous, described by a deformation gradient '\texttt{nextF}', this procedure assigns new co-ordinate values to all vertices of the dodecahedron for its next configuration such that whenever \texttt{nextF} is the identity matrix the dodecahedron is in its reference state.  This method calls the \texttt{update} methods for all of its vertices, chords and pentagons, after which it updates the local fields of the dodecahedron object itself.  This method may be called multiple times before freezing its values with a call to \texttt{advance}.

\medskip\noindent
\texttt{d.advance()}

\medskip\noindent
Calls method \texttt{advance} for all of the vertices, chords and pentagons comprising this dodecahedron, where current fields are assigned to previous fields, and then next fields are assigned to current fields for these objects.  Afterwords, it assigns the current fields to the previous fields and then assigns the next fields to the current fields of the dodecahedron itself, thereby freezing the present next-fields in preparation for advancing the solution along its path.

\medskip\noindent
\textit{The geometric fields associated with a dodecahedron.}

\medskip\noindent
\texttt{v = d.volume(state)}

\medskip\noindent
Returns the volume of this dodecahedron in configuration \texttt{state}.

\medskip\noindent
\texttt{vLambda = d.volumetricStretch(state)}

\medskip\noindent
Returns the cube root of the volume at \texttt{state} divided by reference volume, i.e., $\sqrt[3]{V / V_0}$.

\medskip\noindent
\texttt{vStrain = d.volumetricStrain(state)}

\medskip\noindent
Returns the logarithm of volumetric stretch evaluated at \texttt{state}, i.e., $\Xi = \ln \sqrt[3]{V / V_0}$.

\medskip\noindent
\texttt{dvStrain = d.dVolumetricStrain(state)}

\medskip\noindent
Returns the rate of volumetric strain at \texttt{state}, viz., $\mathrm{d} \Xi = \tfrac{1}{3} V^{-1} \, \mathrm{d} V$.


\newpage
\appendix{Vertices}
\label{appVertices}

Module \texttt{vertices.py} is Python code that exports class \texttt{vertex}.  There are twenty vertices in a dodecahedron.  Their normalized reference co-ordinates are presented in Table~\ref{TableDodecahedron}, which are indexed according to Fig.~\ref{figDodecahedron}.  These normalized co-ordinates are uniformly scaled by the factor \texttt{alveolarDiameter}/1.952400802898434 supplied to the \texttt{dodecahedron} constructor, and then transformed by the linear operator \texttt{F0} also supplied to the \texttt{dodecahedron} constructor; the vertices are created within the \texttt{dodecahedron} constructor.  (The user does not call the \texttt{vertex} constructor.)  This module has the following interface:

\medskip\noindent
\textit{function}

\medskip\noindent
\texttt{s = coordinatesToString(x, y, z)} \\
\indent \texttt{x} \; the 1 co-ordinate \\
\indent \texttt{y} \; the 2 co-ordiante \\
\indent \texttt{z} \; the 3 co-ordiante 

\medskip\noindent
Returns a formatted string representation for the assigned set of co-ordinates.

\bigskip\noindent
\textbf{class} \texttt{vertex}

\medskip\noindent
\textit{constructor}

\medskip\noindent
\texttt{v = vertex(number, x0, y0, z0, h)} \\
\indent \texttt{number} \; an immutable value unique to this vertex \\
\indent \texttt{x0} \qquad\;\, the initial $x$ co-ordinate at zero pleural pressure \\
\indent \texttt{y0} \qquad\;\, the initial $y$ co-ordinate at zero pleural pressure \\
\indent \texttt{z0} \qquad\;\, the initial $z$ co-ordinate at zero pleural pressure \\
\indent \texttt{h\phantom{0}} \qquad\;\, the time-step size between two neighboring configurations

\medskip\noindent
co-ordinates \texttt{x0, y0, z0} have values assigned in the reference co-ordinate frame of a dodecahedron.  The natural co-ordinates for the vertices of a regular dodecahedron are listed in Table~\ref{TableDodecahedron}.

\medskip\noindent
\textit{methods}

\medskip\noindent
\texttt{s = v.toString(state)}

\medskip\noindent 
Returns a formatted string representation for this vertex in configuration \texttt{state} of its dodecahedron.

\medskip\noindent
\texttt{n = v.number()} 

\medskip\noindent 
Returns the unique number affiliated with this vertex.

\medskip\noindent
\texttt{x, y, z = v.coordinates(state)} 

\medskip\noindent 
Returns the co-ordinates for this vertex in configuration \texttt{state}, which are evaluated in the co-ordinate system of its dodecahedron.

\newpage
\medskip\noindent
\texttt{v.update(x, y, z)} 

\medskip\noindent 
Assigns a new set of co-ordinate values to the vertex affiliated with the next configuration of its dodecahedron, as quantified in the co-ordinate system of its dodecahedron.  This method may be called multiple times before freezing its value with a call to \texttt{advance}.  (This method is called internally by \texttt{dodecahedron} objects.)

\medskip\noindent
\texttt{v.advance()} 

\medskip\noindent 
Assigns all of the object's data associated with the current configuration into their affiliated data associated with the previous configuration, and then assigns all of the object's data associated with the next configuration into their affiliated data associated with the current configuration, thereby freezing these data from external change. (This method is called internally by \texttt{dodecahedron} objects.)

\medskip\noindent
\textit{Kinematic fields associated with a point (vertex) in 3 space.}

\medskip\noindent
\texttt{[ux, uy, uz] = v.displacement(state)} 

\medskip\noindent 
Returns the displacement vector of this vertex for configuration \texttt{state} whose components are evaluated in the co-ordinate system of its dodecahedron.  Displacements interpolate quadraticly between consecutive states, because only three locations are maintained at any step $n$ along a solution path.

\medskip\noindent
\texttt{[vx, vy, vz] = v.velocity(state)} 

\medskip\noindent 
Returns the velocity vector of this vertex for configuration \texttt{state} whose components are evaluated in the co-ordinate system of its dodecahedron.  Velocities are calculated using second-order difference formul\ae. Velocities interpolate linearly between consecutive states, because only three locations are maintained at any step $n$ along a solution path.

\medskip\noindent
\texttt{[ax, ay, az] = v.acceleration(state)} 

\medskip\noindent 
Returns the acceleration vector of this vertex for configuration \texttt{state} whose components are evaluated in the co-ordinate system of its dodecahedron.  Accelerations are equivalent for the previous, current and next states, i.e., accelerations are constant over an interval $(n \! - \! 1, n \! + \! 1)$; consequently, accelerations are discontinuous along a solution path.  This is because only three locations are maintained at any step $n$ along a solution path.


\newpage
\appendix{Chords}
\label{appchords}

Module \texttt{chords.py} is Python code that exports class \texttt{chord}.  There are thirty chords in a dodecahedron.  They are assigned vertices according to Table~\ref{Tablechordae} that index according to Fig.~\ref{figDodecahedron}.  They are created within the \texttt{dodecahedron} constructor.  (The user does not call the \texttt{chord} constructor.)  This class has the following interface:

\bigskip\noindent
\textbf{class} \texttt{chord}

\medskip\noindent
\textit{constructor}

\medskip\noindent
\texttt{c = chord(number, vertex1, vertex2, h)} \\
\indent \texttt{number\phantom{1}} \; an immutable value unique to this chord \\
\indent \texttt{vertex1} \; an end point of the chord, an object of class \texttt{vertex} \\
\indent \texttt{vertex2} \; an end point of the chord, an object of class \texttt{vertex} \\
\indent \texttt{h} \qquad\quad\;\; timestep size between two neighboring configurations

\medskip\noindent
Vertices \texttt{vertex1} and \texttt{vertex2} must be different.  The chordal numbering scheme is specified in Table~\ref{Tablechordae}, given the vertex numbering scheme for the dodecahedron listed in Table~\ref{TableDodecahedron} that is visible in Fig.~\ref{figDodecahedron}.

\medskip\noindent
\textit{methods}

\medskip\noindent
\texttt{s = c.toString(state)}

\medskip\noindent
Returns a formatted string representation for this chord in configuration \texttt{state} of its dodecahedron.

\medskip\noindent
\texttt{n = c.number()}

\medskip\noindent
Returns the unique number affiliated with this chord.

\medskip\noindent
\texttt{v1, v2 = c.vertexNumbers()}

\medskip\noindent
Returns the vertex numbers assigned to the two vertices of this chord.

\medskip\noindent
\texttt{truth = c.hasVertex(number)}

\medskip\noindent
Returns \texttt{True} if one of the two vertices has this vertex number; otherwise, it returns \texttt{False}.

\medskip\noindent
\texttt{v = c.getVertex(number)}

\medskip\noindent
Returns the vertex with identifier \texttt{number}.  Typically, it is to be called inside a \texttt{c.hasVertex if} clause.

\medskip\noindent
\texttt{c.update()}

\medskip\noindent
Establishes the fields that pertain to this instance of \texttt{chord} which affiliate with the next configuration.  It is to be called after all vertices
have had their co-ordinates updated.  This method does \textbf{not} call the \texttt{update} method for the two vertices at its end points.  This method may be called multiple times before freezing its values with a call to \texttt{advance}.  (This method is called internally by \texttt{dodecahedron} objects.)

\newpage
\medskip\noindent
\texttt{c.advance()}

\medskip\noindent
Assigns all of the object's data associated with the current configuration into their affiliated data associated with the previous configuration, and then assigns all of the object's data associated with the next configuration into their affiliated data associated with the current configuration, thereby freezing these data from external change.  This method does \textbf{not} call the \texttt{advance} method for the two vertices at its end points. (This method is called internally by \texttt{dodecahedron} objects.)

\medskip\noindent
\textit{The geometric fields associated with a chord in 3 space.}

\medskip\noindent
\texttt{ell = c.length(state)}

\medskip\noindent
Returns the chordal length in configuration \texttt{state} of its dodecahedron.

\medskip\noindent
\texttt{lambda = c.stretch(state)}

\medskip\noindent
Returns the stretch of this chord in configuration \texttt{state} of its dodecahedron.

\medskip\noindent
\textit{The kinematic fields associated with the centroid of a chord in 3 space.}

\medskip\noindent
\texttt{[x, y, z] = c.centroid(state)}

\medskip\noindent
Returns the position vector for this chord locating its mid-point in configuration \texttt{state} of its dodecahedron, i.e., it is the $\boldsymbol{\chi}$ vector of Fig.~\ref{figchord}.

\medskip\noindent
\texttt{[ux, uy, uz] = c.displacement(state)}

\medskip\noindent
Returns the displacement vector of the centroid for this chord in configuration \texttt{state} of its dodecahedron.

\medskip\noindent
\texttt{[vx, vy, vz] = c.velocity(state)}

\medskip\noindent
Returns the velocity vector of the centroid for this chord in configuration \texttt{state} of its dodecahedron.

\medskip\noindent
\texttt{[ax, ay, az] = c.acceleration(state)}

\medskip\noindent
Returns the acceleration vector of the centroid for this chord in configuration \texttt{state} of its dodecahedron.

\medskip\noindent
\textit{The rotation and spin matrices for this chord, as measured relative to its dodecahedron's co-ordinate system.}

\medskip\noindent
\texttt{pMtx = c.rotation(state)}

\medskip\noindent
Returns a 3$\times$3 orthogonal matrix $\mathbfsf{P}$ that rotates the base vectors from its dodecahedral frame of reference into a set of local base vectors where the 1~direction is tangent to the chordal axis, the 2~direction is the normal for this curve in 3~space, and the 3~direction is its binormal.  The returned matrix associates with configuration \texttt{state} of its dodecahedron.

\medskip\noindent
\texttt{omegaMtx = c.spin(state)}

\medskip\noindent
Returns a 3$\times$3 skew symmetric matrix $\boldsymbol{\Omega} \defeq \dot{\mathbfsf{P}} \mathbfsf{P}^{\mathsf{T}}$ that describes the time rate of
rotation, i.e., the spin of the local chordal co-ordinate system about the
fixed   system of its dodecahedron.  The returned matrix
associates with configuration \texttt{state} of its dodecahedron.

\medskip\noindent
\textit{The thermodynamic strain and strain-rate fields associated with a chord.}

\medskip\noindent
\texttt{strain = c.strain(state)}

\medskip\noindent
Returns the logarithmic strain for this chord in configuration \texttt{state} of its dodecahedron, i.e., $e = \ln (L / L_0)$.

\medskip\noindent
\texttt{dStrain = c.dStrain(state)}

\medskip\noindent
Returns the logarithmic strain-rate for this chord in configuration \texttt{state} of its dodecahedron, viz., $\mathrm{d} e = L^{-1} \, \mathrm{d}L$.


\newpage
\section{Modules for Planar Pentagons}
\label{appPentagons}

There are three separate modules that collectively allow a description to be cast for a response of the twelve pentagons that comprise the surface of a dodecahedron.  The first module provides their shape functions.  The second module provides their kinematics, modeled as membranes.  While the third module incorporates these features into a viable description for pentagons.

\subsection{Shape Functions}
\label{appShapeFunctions}

Module \texttt{shapeFunctions} is Python code that exports class \texttt{shapeFunction}.  These shape functions are for the geometry of an irregular planar pentagon.  Shape functions for other geometries are not needed.  These objects are created within the \texttt{pentagon} constructor, and are utilized by the objects of that class.  Class \texttt{shapeFunction} has the following interface:

\medskip\noindent
\textbf{class} \texttt{shapeFunction}

\medskip\noindent
\textit{constructor}

\medskip\noindent
\texttt{sf = shapeFunction(xi, eta)} \\
\indent \texttt{xi} \;\;\;\: the $x$ co-ordinate in the natural co-ordinate system \\
\indent \texttt{eta} \;\;  the $y$ co-ordinate in the natural co-ordinate system

\medskip\noindent
where admissible co-ordinates \texttt{xi} and \texttt{eta} are any values that lie within the area of that pentagon which inscribes a unit circle or that reside along its boundary, as drawn in Fig.~\ref{figRegPentagon}.

\medskip\noindent
\textit{methods}

\medskip\noindent
\texttt{y = sf.interpolate(y1, y2, y3, y4, y5)}

\medskip\noindent
Returns an interpolation for field \texttt{y} at location (\texttt{xi, eta}) given values \texttt{y1, y2, y3, y4, y5} for some field of interest, which are evaluated at the five vertices of the pentagon, indexed according to Fig.~\ref{figRegPentagon}.  Arguments may be of any numeric type or that of a NumPy array.

\medskip\noindent
\texttt{Gmtx = sf.G(x1, x2, x3, x4, x5, x01, x02, x03, x04, x05)}

\medskip\noindent
Returns the displacement gradient 
\begin{displaymath}
   \mathbf{G} = \begin{bmatrix}
        \partial u / \partial x & \partial u / \partial y \\
        \partial v / \partial x & \partial v / \partial y 
   \end{bmatrix}
   \qquad \text{where} \qquad
   \begin{aligned}
        u & = x - X \\
        v & = y - Y
   \end{aligned}
\end{displaymath}
at that location with natural co-ordinates (\texttt{xi, eta}) residing within a pentagonal plane.  The arguments are tuples providing co-ordinate positions for the vertices of a pentagon in their pentagonal frame of reference, e.g., \texttt{x1} = ($x_1, y_1$), $\ldots$, \texttt{x01} = ($X_1, Y_1$), $\ldots$ with subscripts indexing according to Fig.~\ref{figRegPentagon}.

\medskip\noindent
\texttt{Fmtx = sf.F(x1, x2, x3, x4, x5, x01, x02, x03, x04, x05)}

\medskip\noindent
Returns the deformation gradient 
\begin{displaymath}
\mathbf{F} = \begin{bmatrix}
\partial x / \partial X & \partial x / \partial Y \\
\partial y / \partial X & \partial y / \partial Y 
\end{bmatrix}
\end{displaymath}
at that location with natural co-ordinates (\texttt{xi, eta}) residing within a pentagonal plane.  The arguments are tuples providing co-ordinate positions for the vertices of a pentagon in their pentagonal frame of reference, e.g., \texttt{x1} = ($x_1, y_1$), $\ldots$, \texttt{x01} = ($X_1, Y_1$), $\ldots$ with subscripts indexing according to Fig.~\ref{figRegPentagon}.

\medskip\noindent
\texttt{dFdXi = sf.dFdXi(x1, x2, x3, x4, x5, x01, x02, x03, x04, x05)}

\medskip\noindent
Returns the gradient of a deformation gradient taken with respect to the $\xi$ direction 
\begin{displaymath}
\frac{\partial\mathbf{F}}{\partial \xi} = \frac{\partial}{\partial \xi} 
\begin{bmatrix}
\partial x / \partial X & \partial x / \partial Y \\
\partial y / \partial X & \partial y / \partial Y 
\end{bmatrix}
\end{displaymath}
at that location with natural co-ordinates (\texttt{xi, eta}) residing within a pentagonal plane.  The arguments are tuples providing co-ordinate positions for the vertices of a pentagon in their pentagonal frame of reference, e.g., \texttt{x1} = ($x_1, y_1$), $\ldots$, \texttt{x01} = ($X_1, Y_1$), $\ldots$ with subscripts indexing according to Fig.~\ref{figRegPentagon}.

\medskip\noindent
\texttt{dFdEta = sf.dFdEta(x1, x2, x3, x4, x5, x01, x02, x03, x04, x05)}

\medskip\noindent
Returns the gradient of a deformation gradient taken with respect to the $\eta$ direction
\begin{displaymath}
\frac{\partial\mathbf{F}}{\partial \eta} = \frac{\partial}{\partial \eta} 
\begin{bmatrix}
\partial x / \partial X & \partial x / \partial Y \\
\partial y / \partial X & \partial y / \partial Y
\end{bmatrix}
\end{displaymath}
at that location with natural co-ordinates (\texttt{xi, eta}) residing within a pentagonal plane.  The arguments are tuples providing co-ordinate positions for the vertices of a pentagon in their pentagonal frame of reference, e.g., \texttt{x1} = ($x_1, y_1$), $\ldots$, \texttt{x01} = ($X_1, Y_1$), $\ldots$ with subscripts indexing according to Fig.~\ref{figRegPentagon}.


\subsection{Membranes}
\label{appMembranes}

Module \texttt{membranes.py} is Python code that exports class \texttt{membrane}.  Objects of this class are used to describe kinematic fields at the Gauss points of a pentagon.  These objects are created within the \texttt{pentagon} constructor, and are utilized by the objects of that class.  Class \texttt{membrane} has the following interface:

\medskip\noindent
\textbf{class} \texttt{membrane}

\medskip\noindent
\textit{constructor}

\medskip\noindent
\texttt{m = membrane(h)} \\
\indent \texttt{h} \;\; uniform time step separating any two neighboring configurations

\medskip\noindent
\textit{methods}

\medskip\noindent
\texttt{m.update(nextF)}
\medskip\noindent

Establishes the fields that pertain to this instance of \texttt{membrane} which affiliate with the deformation gradient \texttt{nextF} of the next configuration.  This is a $2 \times 2$ matrix describing the deformation gradient in the plane of the pentagon.  (It is not the deformation gradient sent to \texttt{d.update(nextF)} of \texttt{dodecahedron} objects.)  This method may be called multiple times before freezing its values with a call to \texttt{advance}.  (This method is called internally by \texttt{pentagon} objects.)

\medskip\noindent
\texttt{m.advance()}

\medskip\noindent
Assigns all of the object's data associated with the current configuration into their affiliated data associated with the previous configuration, and then assigns all of the object's data associated with the next configuration into their affiliated data associated with the current configuration, thereby freezing these data from external change.  (This method is called internally by \texttt{pentagon} objects.)

\medskip\noindent
\textit{Tensor fields that associate with a Gram-Schmidt factorization of the deformation gradient.}

\medskip\noindent
\texttt{qMtx = m.Q(state)}

\medskip\noindent
Returns the $2 \times 2$ re-indexing matrix that is applied to the deformation gradient prior to its Gram-Schmidt decomposition in configuration \texttt{state}.

\medskip\noindent
\texttt{rMtx = m.R(state)}

\medskip\noindent
Returns the $2 \times 2$ rotation matrix \textbf{Q} derived from a \textbf{QR} decomposition of the re-indexed deformation gradient in configuration \texttt{state}.

\medskip\noindent
\texttt{omega = m.spin(state)}

\medskip\noindent
Returns the $2 \times 2$ spin matrix caused by planar deformation, i.e., $\mathrm{d} \mathbf{R} \, \mathbf{R}^{\mathsf{T}}$, in configuration \texttt{state}.

\medskip\noindent
\texttt{uMtx = m.U(state)}

\medskip\noindent
Returns the $2 \times 2$ Laplace stretch \textbf{R} (denoted herein as $\boldsymbol{\mathcal{U}}$) derived from a \textbf{QR} decomposition of the re-indexed deformation gradient in configuration \texttt{state}.

\medskip\noindent
\texttt{uInvMtx = m.UInv(state)}

\medskip\noindent
Returns the $2 \times 2$ inverse Laplace stretch \textbf{R} (denoted herein as $\boldsymbol{\mathcal{U}}^{-1}$) derived from a \textbf{QR} decomposition of the re-indexed deformation gradient in configuration \texttt{state}.

\medskip\noindent
\texttt{duMtx = m.dU(state)}

\medskip\noindent
Returns the differential change of the Laplace stretch \textbf{R} derived from a \textbf{QR} decomposition of the re-indexed deformation gradient in configuration \texttt{state}.

\medskip\noindent
\texttt{duInvMtx = m.dUInv(state)}

\medskip\noindent
Returns the differential change of the inverse Laplace stretch \textbf{R} derived from a \textbf{QR} decomposition of the re-indexed deformation gradient in configuration \texttt{state}.

\medskip\noindent
\textit{Scalar attributes that arise as extensive thermodynamic variables.}

\medskip\noindent
\texttt{xi = m.dilation(state)}

\medskip\noindent
Returns the planar dilation derived from a \textbf{QR} decomposition of the re-indexed deformation gradient in configuration \texttt{state}, i.e., $\xi = \ln \sqrt{A / \! A_0}$.

\newpage
\medskip\noindent
\texttt{epsilon = m.squeeze(state)}

\medskip\noindent
Returns the in-plane squeeze derived from a \textbf{QR} decomposition of the re-indexed deformation gradient in configuration \texttt{state}, i.e., $\varepsilon = \ln \sqrt{ \Gamma \! / \Gamma_0}$.

\medskip\noindent
\texttt{gamma = m.shear(state)}

\medskip\noindent
Returns the in-plane shear derived from a \textbf{QR} decomposition of the re-indexed deformation gradient in configuration \texttt{state}.

\medskip\noindent
\texttt{dXi = m.dDilation(state)}

\medskip\noindent
Returns a differential change in the planar dilation derived from a \textbf{QR} decomposition of the re-indexed deformation gradient in configuration \texttt{state}, viz., $\mathrm{d}\xi = \tfrac{1}{2} A^{-1} \, \mathrm{d}A$.

\medskip\noindent
\texttt{dEpsilon = m.dSqueeze(state)}

\medskip\noindent
Returns a differential change in the planar squeeze derived from a \textbf{QR} decomposition of the re-indexed deformation gradient in configuration \texttt{state}, viz., $\mathrm{d} \varepsilon = \tfrac{1}{2} \Gamma^{-1} \, \mathrm{d} \Gamma$.

\medskip\noindent
\texttt{dGamma = m.dShear(state)}

\medskip\noindent
Returns a differential change in the in-plane shear derived from a \textbf{QR} decomposition of the re-indexed deformation gradient in configuration \texttt{state}.

\subsection{Pentagons}

Module \texttt{pentagons.py} is Python code that exports class \texttt{pentagon}.  There are twelve pentagons in a dodecahedron.  They are assigned chords according to Table~\ref{TablePentagons} with vertices assigned according to Table~\ref{TableDodecahedron} that index according to Fig.~\ref{figDodecahedron}.  Pentagon objects are created within the \texttt{dodecahedron} constructor.  Membrane and shape function objects are created within the \texttt{pentagon} constructor.  Class \texttt{pentagon} has the following interface:

\bigskip\noindent
\textbf{class} \texttt{pentagon}

\medskip\noindent
\textit{constructor}

\medskip\noindent\small
\texttt{p = pentagon(number, chord1, chord2, chord3, chord4, chord5, h, gaussPts)} \\
\normalsize
\indent \texttt{number} \;\;\;\;\, an immutable value unique to this pentagon \\
\indent \texttt{chord1} \;\;\;\;\, an edge of the pentagon, an object of class \texttt{chord} \\
\indent \texttt{chord2} \;\;\;\;\, an edge of the pentagon, an object of class \texttt{chord} \\
\indent \texttt{chord3} \;\;\;\;\, an edge of the pentagon, an object of class \texttt{chord} \\
\indent \texttt{chord4} \;\;\;\;\, an edge of the pentagon, an object of class \texttt{chord} \\
\indent \texttt{chord5} \;\;\;\;\, an edge of the pentagon, an object of class \texttt{chord} \\
\indent \texttt{h} \qquad\qquad\hspace{0pt} timestep size between two neighboring configurations \\
\indent \texttt{gaussPts} \; number of Gauss points in a pentagonal surface: $\in \{ 1 , 4, 7 \}$

\medskip\noindent
Chords \texttt{chord1}, \texttt{chord2}, \texttt{chord3}, \texttt{chord4}, and \texttt{chord5} must have five vertices that are common, assigned according to the scheme presented in Fig.~\ref{figPentagon}.  The pentagon numbering scheme is specified in Table~\ref{TablePentagons}, given the vertex numbering scheme for the dodecahedron listed in Table~\ref{TableDodecahedron}.  When assigning the five chords of a pentagon, do so according to Fig.~\ref{figPentagon} when looking inward from the outside of a dodecahedron.  By numbering the chords in a counter\-clockwise direction, the algorithm used to compute its area will be positive; otherwise, if the chords were numbered clockwise then the area derived by Eq.~(\ref{irregularPentagonArea}) would become negative.

\begin{figure}
	\centering
	\includegraphics[width=4cm]{figures/pentagon.pdf}
	\caption{Vertex and chord labeling scheme for a pentagon, which coincides with the labeling scheme used by the pentagonal shape functions.}
	\label{figPentagon}
\end{figure}

\medskip\noindent
\textit{methods}

\medskip\noindent
\texttt{s = p.toString(state)}

\medskip\noindent
Returns a formatted string representation for this pentagon in configuration \texttt{state}.

\medskip\noindent
\texttt{n = p.number()}

\medskip\noindent
Returns the unique number affiliated with this pentagon.

\medskip\noindent
\texttt{n1, n2, n3, n4, n5 = p.chordNumbers()}

\medskip\noindent
Returns the chordal numbers associated with the chords of this pentagon, sorted from smallest to largest, i.e., not in accordance to Fig.~\ref{figPentagon}.

\medskip\noindent
\texttt{n1, n2, n3, n4, n5 = p.vertexNumbers()}

\medskip\noindent
Returns the vertex numbers associated with the vertices of this pentagon, sorted from smallest to largest, i.e., not in accordance to Fig.~\ref{figPentagon}.

\medskip\noindent
\texttt{truth = p.hasChord(number)}

\medskip\noindent
Returns \texttt{True} if one of the five chords has this chordal number; otherwise, it returns \texttt{False}.

\medskip\noindent
\texttt{truth = p.hasVertex(number)}

\medskip\noindent
Returns \texttt{True} if one of the five vertices has this vertex number; otherwise, it returns \texttt{False}.

\medskip\noindent
\texttt{c = p.getChord(number)}

\medskip\noindent
Returns the chord with specified \texttt{number}.  Typically called inside a \texttt{p.haschord if} clause.

\medskip\noindent
\texttt{v = p.getVertex(number)}

\medskip\noindent
Returns the vertex with specified \texttt{number}.  Typically called inside a \texttt{p.hasVertex if} clause.

\newpage
\medskip\noindent
\texttt{v = p.gaussPoints()}

\medskip\noindent
Returns the number of Gauss points associated with this pentagon.

\medskip\noindent
\texttt{p.update()}

\medskip\noindent
Establishes the fields that pertain to this object of class \texttt{pentagon} that affiliate with the next configuration.  It is to be called after all vertices and all chords have been updated.  This method does \textbf{not} call the \texttt{update} methods for the five vertices nor for the five chords that comprise this pentagon, but it does call the \texttt{update} methods for the \texttt{shapeFunction} and \texttt{membrane} objects contained within.  This method may be called multiple times before freezing its values with a call to \texttt{advance}.  (This method is called internally by \texttt{dodecahedron} objects.)

\medskip\noindent
\texttt{p.advance()}

\medskip\noindent
Assigns all of the object's data associated with the current configuration into their affiliated data associated with the previous configuration, and then assigns all of the object's data associated with the next configuration into their affiliated data associated with the current configuration, thereby freezing these data from external change.  This method does \textbf{not} call the \texttt{advance} methods for the five vertices nor for the five chords that make up this pentagon, but it does call the \texttt{advance} methods for the \texttt{shapeFunction} and \texttt{membrane} objects contained within.  (This method is called internally by \texttt{dodecahedron} objects.)

\medskip\noindent
\textit{The geometric fields associated with a pentagonal surface embedded in 3 space.}

\medskip\noindent
\texttt{a = p.area(state)}

\medskip\noindent
Returns the area of this irregular pentagon in configuration \texttt{state}.

\medskip\noindent
\texttt{aLambda = p.arealStretch(state)}

\medskip\noindent
Returns the square root of the area at \texttt{state} divided by reference area, i.e., $\sqrt{A / A_0}$.

\medskip\noindent
\texttt{aStrain = p.arealStrain(state)}

\medskip\noindent
Returns the logarithm of areal stretch evaluated at 'state', i.e., $\ln \sqrt{A / A_0}$.

\medskip\noindent
\texttt{daStrain = p.dArealStrain(state)}

\medskip\noindent
Returns the rate of areal strain at 'state', viz., $\tfrac{1}{2} A^{-1} \, \mathrm{d} A$.

\medskip\noindent
\texttt{[nx, ny, nz] = p.normal(state)}

\medskip\noindent
Returns the outward unit normal vector to this pentagon in configuration \texttt{state}.

\medskip\noindent
\texttt{[dnx, dny, dnz] = p.dNormal(state)}

\medskip\noindent
Returns the rate-of-change of the outward unit normal vector to this pentagon in configuration \texttt{state}.

\newpage
\medskip\noindent
\textit{Kinematic vector fields associated with the centroid of a pentagon in 3 space.}

\medskip\noindent
\texttt{[cx, cy, cz] = p.centroid(state)}

\medskip\noindent
Returns the centroid of this irregular pentagon in configuration \texttt{state}, i.e., this is vector $\boldsymbol{\chi}$ in Fig.~\ref{figPentagonCoord}.

\medskip\noindent
\texttt{[ux, uy, uz] = p.displacement(state)}

\medskip\noindent
Returns the displacement vector of its centroid in configuration \texttt{state}.

\medskip\noindent
\texttt{[vx, vy, vz] = p.velocity(state)}

\medskip\noindent
Returns the velocity vector of its centroid in configuration \texttt{state}.

\medskip\noindent
\texttt{[ax, ay, az] = p.acceleration(state)}

\medskip\noindent
Returns the acceleration vector of its centroid in configuration \texttt{state}.

\medskip\noindent
\textit{The rotation and spin of a pentagonal surface as it moves through 3 space.}

\medskip\noindent
\texttt{pMtx = p.rotation(state)}

\medskip\noindent
Returns a 3$\times$3 orthogonal matrix $\mathbfsf{P}$ that rotates the base vectors from its dodecahedral frame of reference into a set of local base vectors pertaining to this irregular pentagon whose outward normal aligns with the 3~direction, i.e., the irregular pentagon resides in the local pentagonal 1-2 plane.  The local 1~direction connects the shoulders of the pentagon, while the 2~direction is rooted at the head of the pentagon.  The returned matrix associates with configuration \texttt{state}.

\medskip\noindent
\texttt{omegaMtx = p.spin(state)}

\medskip\noindent
Returns a 3$\times$3 skew symmetric matrix $\boldsymbol{\Omega} \defeq \mathrm{d} \mathbfsf{P} \, \mathbfsf{P}^{\mathsf{T}}$ that describes the time rate of
rotation, i.e., a spin of the local pentagonal co-ordinate system about the
fixed co-ordinate system of its dodecahedron.  The returned matrix associates with configuration \texttt{state}.

\medskip\noindent
\textit{The kinematic tensor fields of a planar membrane evaluated in the reference co-ordinate system of the pentagon.  To rotate a tensor field from the pentagonal frame, say $\bar{\mathbfsf{A}}$, into its dodecahedral frame producing $\mathbfsf{A}$, apply the following map: $\mathbfsf{A} = \mathbfsf{P} \bar{\mathbfsf{A}} \mathbfsf{P}^{\mathsf{T}}$ where $\mathbfsf{P}$ is the orthogonal matrix returned by\/} \texttt{rotation}.

\medskip\noindent
\texttt{fMtx = p.F(gaussPt, state)}

\medskip\noindent
Returns the 2$\times$2 planar deformation gradient $\mathbfsf{F}$ located at \texttt{gaussPt} in configuration \texttt{state}.

\medskip\noindent
\texttt{dFdX = p.dFdX(gaussPt, state)}

\medskip\noindent
Returns the partial derivative taken with respect to the X direction of a 2$\times$2 planar deformation gradient, viz., $\mathbfsf{F}_{,1}$, located at \texttt{gaussPt} in configuration \texttt{state}.

\medskip\noindent
\texttt{dFdY = p.dFdY(gaussPt, state)}

\medskip\noindent
Returns the partial derivative taken with respect to the Y direction of a 2$\times$2 planar deformation gradient, viz., $\mathbfsf{F}_{,2}$, located at \texttt{gaussPt} in configuration \texttt{state}.

\newpage
\medskip\noindent
\texttt{gMtx = p.G(gaussPt, state)}

\medskip\noindent
Returns the 2$\times$2 planar displacement gradient $\mathbfsf{G}$ located at \texttt{gaussPt} in configuration \texttt{state}.

\medskip\noindent
\texttt{qMtx = p.Q(gaussPt, state)}

\medskip\noindent
Returns the 2$\times$2 re-indexing matrix $\mathbfsf{Q}$ located at \texttt{gaussPt} in configuration \texttt{state}.

\medskip\noindent
\texttt{rMtx = p.R(gaussPt, state)}

\medskip\noindent
Returns the 2$\times$2 rotation matrix $\mathbfsf{R}$ derived from a \textbf{QR} decomposition of the re-indexed deformation gradient, denoted as $\mathbfsf{R}\hspace{1pt}\boldsymbol{\mathcal{U}}$, located at \texttt{gaussPt} in configuration \texttt{state}.

\medskip\noindent
\texttt{uMtx = p.U(gaussPt, state)}

\medskip\noindent
Returns the 2$\times$2 Laplace stretch $\boldsymbol{\mathcal{U}}$ derived from a \textbf{QR} decomposition of the re-indexed deformation gradient located at \texttt{gaussPt} in configuration \texttt{state}.

\medskip\noindent
\texttt{uInvMtx = p.UInv(gaussPt, state)}

\medskip\noindent
Returns the inverse of a 2$\times$2 Laplace stretch $\boldsymbol{\mathcal{U}}^{-1}$ derived from a \textbf{QR} decomposition of the re-indexed deformation gradient located at \texttt{gaussPt} in configuration \texttt{state}.

\medskip\noindent
\texttt{duMtx = p.dU(gaussPt, state)}

\medskip\noindent
Returns a differential change in the 2$\times$2 Laplace stretch $\boldsymbol{\mathcal{U}}$ derived from a \textbf{QR} decomposition of the re-indexed deformation gradient located at \texttt{gaussPt} in configuration \texttt{state}.

\medskip\noindent
\texttt{duInvMtx = p.dUInv(gaussPt, state)}

\medskip\noindent
Returns a differential change in the inverse of a 2$\times$2 Laplace stretch $\boldsymbol{\mathcal{U}}^{-1}$ derived from a \textbf{QR} decomposition of the re-indexed deformation gradient located at \texttt{gaussPt} in configuration \texttt{state}.

\medskip\noindent
\textit{Scalar attributes that are extensive thermodynamic variables, and their rates.}

\medskip\noindent
\texttt{xi = p.dilation(gaussPt, state)}

\medskip\noindent
Returns the planar dilation derived from a \textbf{QR} decomposition of the re-indexed deformation gradient located at \texttt{gaussPt} in configuration \texttt{state}, i.e., $\xi = \ln \sqrt{ab/a_0b_0}$.

\medskip\noindent
\texttt{epsilon = p.squeeze(gaussPt, state)}

\medskip\noindent
Returns the planar squeeze derived from a \textbf{QR} decomposition of the re-indexed deformation gradient located at \texttt{gaussPt} in configuration \texttt{state}, i.e., $\varepsilon = \ln \sqrt{ab_0/a_0b}$.

\medskip\noindent
\texttt{gamma = p.shear(gaussPt, state)}

\medskip\noindent
Returns the planar shear derived from a \textbf{QR} decomposition of the re-indexed deformation gradient located at \texttt{gaussPt} in configuration \texttt{state}.

\medskip\noindent
\texttt{dDelta = p.dDilation(gaussPt, state)}

\medskip\noindent
Returns differential change in the dilation of an irregular pentagon located at \texttt{gaussPt} in configuration \texttt{state}.

\newpage
\medskip\noindent
\texttt{dEpsilon = p.dSqueeze(gaussPt, state)}

\medskip\noindent
Returns differential change in the squeeze of an irregular pentagon located at \texttt{gaussPt} in configuration \texttt{state}.

\medskip\noindent
\texttt{dGamma = p.dGamma(gaussPt, state)}

\medskip\noindent
Returns differential change in the shear of an irregular pentagon located at \texttt{gaussPt} in configuration \texttt{state}.


\newpage
\section{Constitutive Models}
\label{appModels}

There are elastic constitutive models for 1D fibers, chordal fibers, 2D membranes, and 3D volumes.  These are provided for in module \texttt{constitutiveEqns.py}.

\subsection{Elastic Fibers}
\label{appElasticFibers}

Five constitutive models are considered for 1D elastic fibers.  Their mathematical representations are summarized in Eqn.~(\ref{elasticModuli}).  All five inherit the base class \texttt{elasticFibers} whose interface is:

\medskip\noindent
\textbf{class} \texttt{elasticFiber}

\bigskip\noindent
\textit{implemented mmethods}

\bigskip\noindent
These methods are intended to be called via a super call from all classes that extend class \texttt{elasticFiber}.  

\bigskip\noindent
\texttt{<object>.\_\_init\_\_()}
    
\medskip\noindent
This is the constructor, but it is not to be called externally, only internally from those classes that extend this base class.

\bigskip\noindent
\texttt{name = <object>.fiberType()}

\medskip\noindent
Returns a string that contains the name of the fiber model.

\medskip\noindent
\texttt{E = <object>.modulus(stress, strain, temperature)}

\medskip\noindent
Returns the elastic tangent modulus at the specified \texttt{stress, strain} and \texttt{temperature}.  It is the inverse of its elastic compliance.

\bigskip\noindent
\textit{virtual method}

\bigskip\noindent
This method must be overridden by every fiber model that extends this base type.  Virtual methods only provide an interface; their implementation is empty. 

\medskip\noindent
\texttt{C = <object>.compliance(stress, strain, temperature)}

\medskip\noindent
Returns the elastic tangent compliance at the specified \texttt{stress, strain} and \texttt{temperature}.


\subsubsection{Hookean Fibers}

This class provides the elastic compliance and modulus for a Hookean fiber per Eqn.~(\ref{HookeanModulus}).

\bigskip\noindent
\textbf{class} \texttt{hooke(elasticFiber)}

\medskip\noindent
\textit{constructor}

\medskip\noindent
\texttt{elasFiber = hooke(E)} \\
\indent \texttt{E} \;\; the elastic modulus of the fiber, i.e., Young's modulus

\bigskip\noindent
\texttt{name = elasFiber.fiberType()}

\medskip\noindent
Returns a string that contains the name of the fiber model, viz., \texttt{`Hooke'}.

\medskip\noindent
\texttt{C = elasFiber.compliance(stress, strain, temperature)}

\medskip\noindent
Returns the elastic compliance at a specified \texttt{stress, strain} and \texttt{temperature}.  None of these arguments are required.

\medskip\noindent
\texttt{E = elasFiber.modulus(stress, strain, temperature)}

\medskip\noindent
Returns the elastic modulus at a specified \texttt{stress, strain} and \texttt{temperature}, which is the inverse of its elastic compliance.  None of these arguments are required.


\subsubsection{Fungean Fibers}

This class provides the elastic compliance and modulus for a Fungean fiber per Eqn.~(\ref{FungeanModulus}).

\bigskip\noindent
\textbf{class} \texttt{fung1(elasticFiber)}

\medskip\noindent
\textit{constructor}

\medskip\noindent
\texttt{elasFiber = fung1(E, beta)} \\
\indent \texttt{E} \qquad the elastic tangent modulus at zero stress and zero strain \\
\indent \texttt{beta} \: strength of the exponential response

\bigskip\noindent
\texttt{name = elasFiber.fiberType()}

\medskip\noindent
Returns a string that contains the name of the fiber model, viz., \texttt{`Fung'}.

\medskip\noindent
\texttt{C = elasFiber.compliance(stress, strain, temperature)}

\medskip\noindent
Returns the elastic tangent compliance at a specified \texttt{stress, strain} and \texttt{temperature}.  Only argument \texttt{stress} is required.

\medskip\noindent
\texttt{E = elasFiber.modulus(stress, strain, temperature)}

\medskip\noindent
Returns the elastic tangent modulus at a specified \texttt{stress, strain} and \texttt{temperature}, which is the inverse of its elastic compliance.  Only argument \texttt{stress} is required.


\subsubsection{Fungean\slash Hookean Fibers}

This class provides the elastic response functions for Fungean and Hookean fibers whose compliances sum, per Eqn.~(\ref{FungLikeModulus}).

\bigskip\noindent
\textbf{class} \texttt{fung2(elasticFiber)}

\medskip\noindent
\textit{constructor}

\medskip\noindent
\texttt{elasFiber = fung2(E1, E2, beta)} \\
\indent \texttt{E1} \quad\: the elastic tangent modulus at zero stress and zero strain \\
\indent \texttt{E2} \quad\: the elastic tangent modulus at terminal stress \\
\indent \texttt{beta} \: strength of the exponential response

\newpage
\bigskip\noindent
\texttt{name = elasFiber.fiberType()}

\medskip\noindent
Returns a string that contains the name of the fiber model, viz., \texttt{`Fung/Hooke'}.

\medskip\noindent
\texttt{C = elasFiber.compliance(stress, strain, temperature)}

\medskip\noindent
Returns the elastic tangent compliance at a specified \texttt{stress, strain} and \texttt{temperature}.  Only argument \texttt{stress} is required.

\medskip\noindent
\texttt{E = elasFiber.modulus(stress, strain, temperature)}

\medskip\noindent
Returns the elastic tangent modulus at a specified \texttt{stress, strain} and \texttt{temperature}, which is the inverse of its elastic compliance.  Only argument \texttt{stress} is required.


\subsubsection{Freed-Rajagopal Fibers}

This class provides the elastic compliance and modulus for a Freed-Rajagopalean fiber per Eqn.~(\ref{FRmodulus}).

\bigskip\noindent
\textbf{class} \texttt{freed1(elasticFiber)}

\medskip\noindent
\textit{constructor}

\medskip\noindent
\texttt{elasFiber = freed1(E, e\_t)} \\
\indent \texttt{E} \quad\, the elastic tangent modulus at zero stress and zero strain \\
\indent \texttt{e\_t} \: the limit strain, i.e., the maximum strain allowed

\bigskip\noindent
\texttt{name = elasFiber.fiberType()}

\medskip\noindent
Returns a string that contains the name of the fiber model, viz., \texttt{`Freed-Rajagopal'}.

\medskip\noindent
\texttt{C = elasFiber.compliance(stress, strain, temperature)}

\medskip\noindent
Returns the elastic tangent compliance at a specified \texttt{stress, strain} and \texttt{temperature}.  Argument \texttt{temperature} is not required, but \texttt{stress} and \texttt{strain} are.

\medskip\noindent
\texttt{E = elasFiber.modulus(stress, strain, temperature)}

\medskip\noindent
Returns the elastic tangent modulus at a specified \texttt{stress, strain} and \texttt{temperature}, which is the inverse of its elastic compliance.  Argument \texttt{temperature} is not required, but \texttt{stress} and \texttt{strain} are.


\subsubsection{Freed-Rajagopal\slash Hookean Fibers}

This class provides the elastic response functions for Freed-Rajagopalean and Hookean fibers whose compliances sum, per Eqn.~(\ref{FRbiologicModulus}).  This version does not account for thermal straining.

\bigskip\noindent
\textbf{class} \texttt{freed2(elasticFiber)}

\medskip\noindent
\textit{constructor}

\medskip\noindent
\texttt{elasFiber = freed2(E1, E2, e\_t)} \\
\indent \texttt{E1} \quad\;\;\: the elastic tangent modulus at zero stress and strain \\
\indent \texttt{E2} \quad\;\;\: the elastic tangent modulus at terminal stress \\
\indent \texttt{e\_t} \quad\;\, the limit strain, i.e., the maximum strain allowed 

\bigskip\noindent
\texttt{name = elasFiber.fiberType()}

\medskip\noindent
Returns a string that contains the name of the fiber model, viz., \texttt{`Freed-Rajagopal/Hooke'}.

\medskip\noindent
\texttt{C = elasFiber.compliance(stress, strain, temperature)}

\medskip\noindent
Returns the elastic tangent compliance at a specified \texttt{stress} and \texttt{strain}.    Argument \texttt{temperature} is not required, but \texttt{stress} and \texttt{strain} are.

\medskip\noindent
\texttt{E = elasFiber.modulus(stress, strain, temperature)}

\medskip\noindent
Returns the elastic tangent modulus at a specified \texttt{stress} and \texttt{strain}; it is the inverse of its elastic compliance.  Argument \texttt{temperature} is not required, but \texttt{stress} and \texttt{strain} are.


\subsubsection{Freed-Rajagopal\slash Kelvin\slash Hookean Fibers}

This class provides the elastic response functions for Freed-Rajagopalean and Kelvin\slash Hookean fibers whose compliances sum, per Eqn.~(\ref{FRbiologicModulus}).  This version does account for thermal straining.

\bigskip\noindent
\textbf{class} \texttt{freed3(elasticFiber)}

\medskip\noindent
\textit{constructor}

\medskip\noindent
\texttt{elasFiber = freed3(E1, E2, e\_t, alpha, T0)} \\
\indent \texttt{E1} \quad\;\;\: the elastic tangent modulus at zero stress and strain \\
\indent \texttt{E2} \quad\;\;\: the elastic tangent modulus at terminal stress \\
\indent \texttt{e\_t} \quad\;\, the limit strain, i.e., the maximum strain allowed \\
\indent \texttt{alpha} \; thermal strain coefficient \\
\indent \texttt{T0} \quad\;\;\; the reference temperature for thermal strain, typically body temperature

\bigskip\noindent
\texttt{name = elasFiber.fiberType()}

\medskip\noindent
Returns a string that contains the name of the fiber model, viz., \texttt{`Freed-Rajagopal/Kelvin/ Hooke'}.

\medskip\noindent
\texttt{C = elasFiber.compliance(stress, strain, temperature)}

\medskip\noindent
Returns the elastic tangent compliance at a specified \texttt{stress, strain} and \texttt{temperature}.  All arguments are required.

\medskip\noindent
\texttt{E = elasFiber.modulus(stress, strain, temperature)}

\medskip\noindent
Returns the elastic tangent modulus at a specified \texttt{stress, strain} and \texttt{temperature}, which is the inverse of its elastic compliance.  All arguments are required.


\subsection{Chordal Fibers}



\newpage
\section{Solvers}
\label{appSolvers}

Two ODE solvers are included in this software.  The first, \texttt{peceVtoX.py}, uses a two-step PECE (Predict, Evaluate, Correct, re-Evaluate) method to solve a first-order, ordinary differential equation $\dot{\mathbf{x}}(t) = \mathbf{v}(t) = \mathbf{f} (t, \mathbf{x})$ given an initial condition $\mathbf{x}_0 = \mathbf{x}(t_0)$ where, as an analogy, $\mathbf{x}$ denotes displacement and $\mathbf{v}$ denotes velocity.  The second, \texttt{peceAtoVandX.py}, uses another two-step PECE method to solve a second-order, ordinary, differential equation $\ddot{\mathbf{x}}(t) = \mathbf{a}(t) = \mathbf{f}(t, \mathbf{x}, \mathbf{v})$ given initial conditions $\mathbf{x}_0 = \mathbf{x}(t_0)$ and $\mathbf{v}_0 = \mathbf{v}(t_0) = \dot{\mathbf{x}}(t_0)$ where, as an analogy, $\mathbf{x}$ denotes displacement, $\mathbf{v}$ denotes velocity, and $\mathbf{a}$ denotes acceleration.

\subsection{$\mathit{1}^{\text{st}}$ Order ODE Solver}
\label{app1stOrderODEs}

Module \texttt{peceVtoX.py} is a Python code that exports class \texttt{pece} which solves first-order, ordinary, differential equations using a two-step method; in particular, it solves
\begin{displaymath}
    \mathbf{v} = \mathbf{f}(t,\mathbf{x}) 
    \quad \text{where} \quad
    \mathbf{v} = \dot{\mathbf{x}}
    \quad \text{satisfying IC} \quad
    \mathbf{x}_0 = \mathbf{x}(t_0)  
\end{displaymath}
where the dependent variables of integration $\mathbf{x}$ are analogous to displacements, while the ODEs $\dot{\mathbf{x}} = \mathbf{f}(t, \mathbf{x})$ are analogous to velocities $\mathbf{v} = \dot{\mathbf{x}}$.

\bigskip\noindent
\textbf{class} \texttt{pece}

\medskip\noindent
\textit{constructor}

\medskip\noindent
\texttt{solver = pece(ode, t0, x0, h, tol=0.0001)} \\
\indent \texttt{ode} \; the differential equation to be solved, i.e., $\dot{\mathbf{x}} = \mathbf{v} = \mathbf{f} (t, \mathbf{x})$ where \texttt{ode} = $\mathbf{f}(t,\mathbf{x})$ \\
\indent \texttt{t0} \;\;\; the initial time $t$, viz., time at the start of integration \\
\indent \texttt{x0} \;\;\; the initial condition, viz., displacements at the start of integration $\mathbf{x}_0 = \mathbf{x}(t_0)$  \\
\indent \texttt{h} \;\;\;\;\; the global time-step size separating two neighboring states \\
\indent \texttt{tol} \;\hspace{1pt} the maximum allowed local truncation error, with a default set at $10^{-4}$

\medskip\noindent
\textit{methods}

\medskip\noindent
\texttt{solver.integrate()}

\medskip\noindent
A command that integrates the ODE from current time $t_n$ to the next time $t_{n+1} = t_n + h$.  This command may be called multiple times before committing a solution.  A local time stepper is used to integrate over the global time step.  The local time-step size is controlled by a PI controller that runs in the background.  This controller bounds the local truncation error from above.  If the error is too small then the controller increases the local step size.  If the error is too large then the controller decreases the local step size.

\medskip\noindent
\texttt{solver.advance()}

\medskip\noindent
A command that updates the internal data structure of the integrator by relabeling variables assigned to current time $t_n$ to their counterparts associated with previous time $t_{n-1}$, and then assigning the variables just solved for at time $t_{n+1}$ to their counterparts at time $t_n$.  This performs an incremental advancement of the solution along its trajectory, with $t_n + h$ now becoming the current time.

\newpage
\medskip\noindent
\textit{The following methods are to be called after a solution has been advanced\slash committed, but before the next integration step is taken.}

\medskip\noindent
\texttt{n, nd, nh, nr = solver.getStatistics()} \\
\indent \texttt{n} \;\;\;\: total number of local steps taken \\
\indent \texttt{nd} \;\; total number of local steps taken where the step-size was doubled \\
\indent \texttt{nh} \;\; total number of local steps taken where the step-size was halved \\
\indent \texttt{nr} \;\; total number of local steps taken where the integrator was restarted 

\medskip\noindent
\texttt{t = solver.getT()}

\medskip\noindent
Returns the current time \texttt{t}, i.e., the independent variable of integration.

\medskip\noindent
\texttt{x = solver.getX()}

\medskip\noindent
Returns the solution vector \texttt{x} at current time, i.e., the dependent variables of integration.

\medskip\noindent
\texttt{v = solver.getV()} 

\medskip\noindent
Returns the time rate-of-change in the dependent variables at current time, i.e., the ODEs being solved, their analog being velocities.

\medskip\noindent
\texttt{err = solver.getError()} 

\medskip\noindent
Returns an estimate for the local truncation error \texttt{err} at current time.

\medskip\noindent
\texttt{x = solver.interpolate(atT)}

\medskip\noindent
Returns the solution \texttt{x(atT)} at time \texttt{atT} using cubic Hermite interpolation, where \texttt{atT} is located somewhere between the previous $t_{n-1}$ and current $t_n$ times of the integrator.

\subsection{$\mathit{2}^{\text{nd}}$ Order ODE Solver}
\label{app2ndOrderODEs}

Module \texttt{peceAtoVandX.py} is a Python code that exports class \texttt{pece} which solves second-order, ordinary, differential equations using a two-step method; in particular, it solves
\begin{displaymath}
\mathbf{a} = \mathbf{f}(t,\mathbf{x},\mathbf{v}) 
\quad \text{where} \quad
\mathbf{a} = \ddot{\mathbf{x}}
\quad \text{and} \quad
\mathbf{v} = \dot{\mathbf{x}}
\quad \text{with ICs} \quad
\mathbf{x}_0 = \mathbf{x}(t_0)
\quad \text{and} \quad 
\mathbf{v}_0 = \mathbf{v}(t_0)
\end{displaymath}
where the dependent variables of integration $\mathbf{x}$ are analogous to displacements whose rates $\mathbf{v} = \dot{\mathbf{x}}$ are analogous to velocities, while the ODEs $\ddot{\mathbf{x}} = \mathbf{f}(t, \mathbf{x}, \mathbf{v})$ are analogous to accelerations $\mathbf{a} = \dot{\mathbf{v}} = \ddot{\mathbf{x}}$.  

This solver is useful when solving dynamics problems, e.g.,
\begin{displaymath}
    \mathbf{M} \mathbf{a} + \mathbf{C} \mathbf{v} + \mathbf{K} \mathbf{x} =
    \boldsymbol{\phi} (t, \mathbf{x}, \mathbf{v} )
    \quad \text{or} \quad
    \mathbf{a} = \mathbf{f} (t, \mathbf{x}, \mathbf{v}) 
    \quad \text{with} \quad
    \mathbf{f} = \mathbf{M}^{-1} \bigl( \boldsymbol{\phi}(t) - \mathbf{C} \mathbf{v} - \mathbf{K} \mathbf{x} \bigr)
\end{displaymath}
where $\mathbf{M}$ is a mass matrix, $\mathbf{C}$ is a damping matrix, $\mathbf{K}$ is a stiffness matrix, and $\boldsymbol{\phi}$ is a forcing function.  Typically $\mathbf{M}$ is diagonal so its inverse is trivial.

\newpage
\bigskip\noindent
\textbf{class} \texttt{pece}

\medskip\noindent
\textit{constructor}

\medskip\noindent
\texttt{solver = pece(aFn, t0, x0, v0, h, tol=0.0001)} \\
\indent \texttt{aFn} \; the differential equation to be solved, i.e., $\ddot{\mathbf{x}} = \mathbf{a} = \mathbf{f} (t, \mathbf{x}, \dot{\mathbf{x}})$ where \texttt{aFn} = $\mathbf{f}(t,\mathbf{x},\mathbf{v})$ \\
\indent \texttt{t0} \;\;\; the initial time $t$, viz., time at the start of integration \\
\indent \texttt{x0} \;\;\; an initial condition, viz., displacements at the start of integration $\mathbf{x}_0 = \mathbf{x}(t_0)$  \\
\indent \texttt{v0} \;\;\; an initial condition, viz., velocities at the start of integration $\mathbf{v}_0 = \mathbf{v}(t_0) = \dot{\mathbf{x}}(t_0)$  \\
\indent \texttt{h} \;\;\;\;\; the global time-step size separating two neighboring states \\
\indent \texttt{tol} \;\hspace{1pt} the maximum allowed local truncation error, with a default set at $10^{-4}$

\medskip\noindent
\textit{methods}

\medskip\noindent
\texttt{solver.integrate()}

\medskip\noindent
A command that integrates the ODE from current time $t_n$ to the next time $t_{n+1} = t_n + h$.  This command may be called multiple times before committing a solution.  A local time stepper is used to integrate over the global time step.  The local time-step size is controlled by a PI controller that runs in the background.  This controller bounds the local truncation error from above.  If the error is too small then the controller increases the local step size.  If the error is too large then the controller decreases the local step size.

\medskip\noindent
\texttt{solver.advance()}

\medskip\noindent
A command that updates the internal data structure of the integrator by relabeling variables assigned to current time $t_n$ to their counterparts associated with previous time $t_{n-1}$, and then assigning the variables just solved for at time $t_{n+1}$ to their counterparts at time $t_n$.  This performs an incremental advancement of the solution along its trajectory, with $t_n + h$ now becoming the current time.

\medskip\noindent
\textit{The following methods are to be called after a solution has been advanced\slash committed, but before the next integration step is taken.}

\medskip\noindent
\texttt{n, nd, nh, nr = solver.getStatistics()} \\
\indent \texttt{n} \;\;\;\: total number of local steps taken \\
\indent \texttt{nd} \;\; total number of local steps taken where the step-size was doubled \\
\indent \texttt{nh} \;\; total number of local steps taken where the step-size was halved \\
\indent \texttt{nr} \;\; total number of local steps taken where the integrator was restarted 

\medskip\noindent
\texttt{t = solver.getT()}

\medskip\noindent
Returns the current time \texttt{t}, i.e., the independent variable of integration.

\medskip\noindent
\texttt{x = solver.getX()}

\medskip\noindent
Returns the solution vector \texttt{x} at current time, i.e., first set of the dependent variables of integration.

\newpage
\medskip\noindent
\texttt{v = solver.getV()}

\medskip\noindent
Returns the solution vector \texttt{v} at current time, i.e., second set of the dependent variables of integration.

\medskip\noindent
\texttt{a = solver.getA()} 

\medskip\noindent
Returns the time rate-of-change in the velocity variables at current time, i.e., the ODEs being solved, their analog being accelerations.

\medskip\noindent
\texttt{err = solver.getError()} 

\medskip\noindent
Returns an estimate for the local truncation error \texttt{err} at current time.

\medskip\noindent
\texttt{x = solver.interpolateX(atT)}

\medskip\noindent
Returns the solution \texttt{x(atT)} at time \texttt{atT} using cubic Hermite interpolation, where \texttt{atT} is located somewhere between the previous $t_{n-1}$ and current $t_n$ times of the integrator.

\medskip\noindent
\texttt{x = solver.interpolateV(atT)}

\medskip\noindent
Returns the solution \texttt{v(atT)} at time \texttt{atT} using cubic Hermite interpolation, where \texttt{atT} is located somewhere between the previous $t_{n-1}$ and current $t_n$ times of the integrator.


\end{document}
