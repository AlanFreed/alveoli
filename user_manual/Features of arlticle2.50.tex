\documentclass{arlticle}
\usepackage{lipsum}
% IF REPORT CITATIONS TO REAPPEAR IN APPENDICES, YOU NEED 
% THE NEXT TWO LINES
\usepackage{bibentry}
\nobibliography*
% THEN, INSTEAD OF \cite{<bibtag>},
% USE \footnotecite[<tag>]{\bibentry{<bibtag>}} FOR THE APPENDIX
% CITATION.  IF A RECITE IS NEEDED THEREAFTER, \recite{<tag>} WILL DO.
% NOTE THAT tag AND bibtag CAN BE THE SAME STRING.
\def\MandatoryDL{ARL-02-01-19(Unlimited).dls}
\def\UserDL{blank.dls}
\sectspace
\distcodes{A}{}{}
\arltitle{Testing New Features of arlticle V2.50\\Including New Logos}
\pubdate{January 2019}
\allauthors{Steven B Segletes}
\authorsA  {Steven B Segletes}
\organizationA	{Weapons and Materials Research Directorate, ARL}
\usepackage{filecontents}
\begin{filecontents*}{mybib.bib}
@TECHREPORT{segl83,
	AUTHOR	= "Segletes, Steven B.",
	TITLE   = "Drift Velocity Computations for Shaped-Charge Jets",
        NUMBER  = "ARBRL-MR-03306 (ADA 133 756)",
        INSTITUTION = "Army Ballistic Research Laboratory (US)",
        ADDRESS = "Aberdeen Proving Ground (" # md.us # ")",
	YEAR	= "1983",
	MONTH	= sep				}
@TECHREPORT{segl98a,
	AUTHOR	= "Segletes, S. B.",
	TITLE	= "The Vibrational Stiffness of an Atomic Lattice",
   NUMBER  = "ARL-TR-1757",
   INSTITUTION = "U.S. Army Research Laboratory",
   ADDRESS = "Aberdeen Proving Ground, " # md.us,
	YEAR	= "1998",
	MONTH	= sep				}
\end{filecontents*}
\begin{document}
\ARLcover{
}% Place a zero in the braces to get an ink-saving "LoInk" cover
\arltitlepage

\tableofcontents
\clearpage

\section{Report Section}

Seven major changes are incorporated into the \textsf{arlticle} document
  class, version 2.50.
They include
\begin{enumerate}
\item New logos on the cover and title page, to reflect our realignment
  with CCDCOM.  The logos may yet be upgraded to vector graphics, if
  available, to avoid zoom pixelation.

\item Footnotes should appear at the bottom margin, even if the page text 
  ends partially down the page.  That is, the gap (if any) is before the
  footnote and not after it.

\item The macros \verb|\footnotecite| and \verb|\recite| have been
  provided to assist the user in citing works from the appendices.

\item The definitions associated with appendices have been changed
  so that subsections and lower sectioning of appendices does not
  show up in the Table of Contents.

\item Newer preferred macros have been introduced to  support
  portion marking (FOUO in unclassified reports and classification
  marking in classified reports) for tables and figures.
  These macros are \verb|\setportionmark| and \verb|\portionmark|.

\item The \textsf{stackengine} package is now included by default,
  to support the portion-marking features of the document class.

\item When creating a distribution list, the package has, for some time,
  created an ancillary \verb|tex| document that is used to create the
  email data for the distribution list.
  A bug has been fixed so that this feature does not break when the 
  original document has spaces in the filename.
\end{enumerate}

\subsection{Report Sectioning}

Within the main document, report sectioning shows up in the Table of Contents.
Report subsectioning (all the way down to subsubsections) will also show
  up in the Table of Contents.
Please use this new way to portion mark.

\subsection{Portion Marking for FOUO (and/or classification)}

New macros for portion marking have been introduced.

\subsubsection{Former Technique (deprecated...please use no longer)}
\verb|\FOUOunderset[<gap>]{<content>}| and \\
\verb|\FOUOincludegraphics[<options>]{<image-file>}|
\begin{ARLtable}[ht]
\caption{(FOUO) Table title includes FOUO portion marking}
\FOUOunderset{%
  \begin{tabular}[b]{ccccp{1in}}
   \hline
  A & B & C & D & Text\\
   \hline
  1.2 & Red & Blue & 100 & Here is my wrapping text that is in the table\\
   \hline
  \end{tabular}%
  }%
\qquad\FOUOincludegraphics[width=1in]{example-image}
\end{ARLtable}

\subsubsection{Preferred Technique}
\verb|\setportionmark{<default marking>}| and \\
\verb|\portionmark[<marking>][<gap>]{<content>}|

\setportionmark{FOUO}
\begin{ARLtable}[ht]
\caption{(FOUO) Table title includes FOUO portion marking}
\portionmark[UNCLASSIFIED][7pt]{%
  \begin{tabular}[b]{ccccp{1in}}
   \hline
  A & B & C & D & Text\\
   \hline
  1.2 & Red & Blue & 100 & Here is my wrapping text that is in the table\\
   \hline
  \end{tabular}%
  }%
\qquad\portionmark{\includegraphics[width=1in]{example-image}}
\end{ARLtable}

\paragraph{Footnote change}
My footnote\footnote{Right here, at bottom margin.} should be at the 
  bottom margin and not immediately below the last line of text.
It is not numbered, but denoted with a symbol.
Here are also several report reference,\cite{segl98a,segl83} which 
  are numbered.

\arlbibliography{mybib}

\appendix*{Sole Appendix}

The appendix will show in the Table of Contents.

\subsection{Appendix Test}

However, subsections and lower sectioning of appendices will \textit{not} show
in the Table of Contents.
And while footnotes\footnote{Here is regular footnote.} in appendices
  should operate as in the main report, appendix citations do not.
Here is an appendix citation\footnotecite[Rsegl83]{\bibentry{segl83}.}
  of a reference that already appeared in the main report.
It gets [possibly] new numbering here.  The actual reference for the
  citation shows up in a numbered footnote, using the \verb|bibitem|
  package.
Use \verb|\footnotecite[<tag>]{\bibentry{bibtex-tag}.}|

The \verb|tag| is optionally specified should you later need to re-cite
  the same report later in the same appendix.
And here it is: a recitation\recite{Rsegl83} of a reference already 
  cited elsewhere in the appendix.
The syntax is \verb|\recite{<tag>}|.  
The \verb|tag| can be specified the same as the \verb|bibtex-tag|, if desired.

\distlistsetup
\begin{distributionlist}
\input{\MandatoryDL}
\pdlitem{1}
A file\\
DL\_ARL-TR-XXXX.tex\\
is created for generating\\
the distribution e-mailing\\
list.\\
This feature no longer\\
fails when this document has\\
spaces in the filename
\end{distributionlist}
\distlistcleanup

\end{document}