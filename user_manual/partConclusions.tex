\section{Conclusions}
\label{partConclusions}

This report develops a micro\-scopic alveolar model whose homogenized response describes the macro\-scopic behavior of parenchyma in lung.  Such a model can be used in lieu of physical experiments to help develop and parameterize a better continuum lung model for use in finite element analyses.   The need for such a model is to aid Army engineers in their development of improved PPE to better protect Soldiers from BABT and BLI when impacted by ballistic projectiles or blast waves.  

The geometry of an individual alveolus is modeled as an irregular dodecahedron comprising 20 alveolar vertices, 30 1D alveolar chords, and 12 2D pentagonal alveolar septa, all enveloping a 3D alveolar sac.  Implicit elastic constitutive equations are used to model these alveolar chords and septa.  Alveolar chords are modeled as collagen and elastin fibers loaded in parallel.  Damage is accounted for through the rupture of individual alveolar fibers and septa, and the tearing of capillaries that lead to blood and interstitial fluids leaking into its alveolar sac.  Material properties for the individual fibers and septa are assigned through probability distribution functions to account for their biologic variability.

It is shown that geometric strains for the three physical dimensions that arise in this analysis are equivalent during uniform deformations when they are defined as $\ln (L / L_0)$ for 1D rods, $\ln \sqrt{A / \! A_0}$ for 2D membranes, and $\ln \sqrt[3]{V \! / V_0}$ for 3D volumes.  Adopting Laplace stretch as our fundamental kinematic variable, thermo\-dynamic conjugate pairs are established for these three geometric dimensions.  These thermo\-dynamic strains equate with the above geometric strains under conditions of uniform deformation, plus they allow for the handling of nonuniform deformations, in particular, pure and simple shears.

New to this report are the following: \textit{i\/}) Sets of consistent interpolation\slash extrapolation procedures for 1D rods, 2D triangles and pentagons, and 3D tetrahedra, which allow physical fields to be mapped between the nodes and Gauss points of an element in a reproducible manner; \textit{ii\/}) Shape functions and a Gauss integration formula suitable for constructing a pentagonal finite element, which is used to model alveolar septa; \textit{iii\/}) Nonlinear strain-displacement matrices for 2D pentagons and 3D tetrahedra that employ Laplace stretch as their kinematic variable; and \textit{iv\/})  A numerical algorithm that employs both secant and tangent stiffness matrices in its finite element solver.
