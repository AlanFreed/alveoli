\section{Constitutive Models}
\label{appModels}

There are elastic constitutive models for 1D fibers, chordal fibers, 2D membranes, and 3D volumes.  These are provided for in module \texttt{constitutiveEqns.py}.

\subsection{Elastic Fibers}
\label{appElasticFibers}

Five constitutive models are considered for 1D elastic fibers.  Their mathematical representations are summarized in Eqn.~(\ref{elasticModuli}).  All five inherit the base class \texttt{elasticFibers} whose interface is:

\medskip\noindent
\textbf{class} \texttt{elasticFiber}

\bigskip\noindent
\textit{implemented mmethods}

\bigskip\noindent
These methods are intended to be called via a super call from all classes that extend class \texttt{elasticFiber}.  

\bigskip\noindent
\texttt{<object>.\_\_init\_\_()}
    
\medskip\noindent
This is the constructor, but it is not to be called externally, only internally from those classes that extend this base class.

\bigskip\noindent
\texttt{name = <object>.fiberType()}

\medskip\noindent
Returns a string that contains the name of the fiber model.

\medskip\noindent
\texttt{E = <object>.modulus(stress, strain, temperature)}

\medskip\noindent
Returns the elastic tangent modulus at the specified \texttt{stress, strain} and \texttt{temperature}.  It is the inverse of its elastic compliance.

\bigskip\noindent
\textit{virtual method}

\bigskip\noindent
This method must be overridden by every fiber model that extends this base type.  Virtual methods only provide an interface; their implementation is empty. 

\medskip\noindent
\texttt{C = <object>.compliance(stress, strain, temperature)}

\medskip\noindent
Returns the elastic tangent compliance at the specified \texttt{stress, strain} and \texttt{temperature}.


\subsubsection{Hookean Fibers}

This class provides the elastic compliance and modulus for a Hookean fiber per Eqn.~(\ref{HookeanModulus}).

\bigskip\noindent
\textbf{class} \texttt{hooke(elasticFiber)}

\medskip\noindent
\textit{constructor}

\medskip\noindent
\texttt{elasFiber = hooke(E)} \\
\indent \texttt{E} \;\; the elastic modulus of the fiber, i.e., Young's modulus

\bigskip\noindent
\texttt{name = elasFiber.fiberType()}

\medskip\noindent
Returns a string that contains the name of the fiber model, viz., \texttt{`Hooke'}.

\medskip\noindent
\texttt{C = elasFiber.compliance(stress, strain, temperature)}

\medskip\noindent
Returns the elastic compliance at a specified \texttt{stress, strain} and \texttt{temperature}.  None of these arguments are required.

\medskip\noindent
\texttt{E = elasFiber.modulus(stress, strain, temperature)}

\medskip\noindent
Returns the elastic modulus at a specified \texttt{stress, strain} and \texttt{temperature}, which is the inverse of its elastic compliance.  None of these arguments are required.


\subsubsection{Fungean Fibers}

This class provides the elastic compliance and modulus for a Fungean fiber per Eqn.~(\ref{FungeanModulus}).

\bigskip\noindent
\textbf{class} \texttt{fung1(elasticFiber)}

\medskip\noindent
\textit{constructor}

\medskip\noindent
\texttt{elasFiber = fung1(E, beta)} \\
\indent \texttt{E} \qquad the elastic tangent modulus at zero stress and zero strain \\
\indent \texttt{beta} \: strength of the exponential response

\bigskip\noindent
\texttt{name = elasFiber.fiberType()}

\medskip\noindent
Returns a string that contains the name of the fiber model, viz., \texttt{`Fung'}.

\medskip\noindent
\texttt{C = elasFiber.compliance(stress, strain, temperature)}

\medskip\noindent
Returns the elastic tangent compliance at a specified \texttt{stress, strain} and \texttt{temperature}.  Only argument \texttt{stress} is required.

\medskip\noindent
\texttt{E = elasFiber.modulus(stress, strain, temperature)}

\medskip\noindent
Returns the elastic tangent modulus at a specified \texttt{stress, strain} and \texttt{temperature}, which is the inverse of its elastic compliance.  Only argument \texttt{stress} is required.


\subsubsection{Fungean\slash Hookean Fibers}

This class provides the elastic response functions for Fungean and Hookean fibers whose compliances sum, per Eqn.~(\ref{FungLikeModulus}).

\bigskip\noindent
\textbf{class} \texttt{fung2(elasticFiber)}

\medskip\noindent
\textit{constructor}

\medskip\noindent
\texttt{elasFiber = fung2(E1, E2, beta)} \\
\indent \texttt{E1} \quad\: the elastic tangent modulus at zero stress and zero strain \\
\indent \texttt{E2} \quad\: the elastic tangent modulus at terminal stress \\
\indent \texttt{beta} \: strength of the exponential response

\newpage
\bigskip\noindent
\texttt{name = elasFiber.fiberType()}

\medskip\noindent
Returns a string that contains the name of the fiber model, viz., \texttt{`Fung/Hooke'}.

\medskip\noindent
\texttt{C = elasFiber.compliance(stress, strain, temperature)}

\medskip\noindent
Returns the elastic tangent compliance at a specified \texttt{stress, strain} and \texttt{temperature}.  Only argument \texttt{stress} is required.

\medskip\noindent
\texttt{E = elasFiber.modulus(stress, strain, temperature)}

\medskip\noindent
Returns the elastic tangent modulus at a specified \texttt{stress, strain} and \texttt{temperature}, which is the inverse of its elastic compliance.  Only argument \texttt{stress} is required.


\subsubsection{Freed-Rajagopal Fibers}

This class provides the elastic compliance and modulus for a Freed-Rajagopalean fiber per Eqn.~(\ref{FRmodulus}).

\bigskip\noindent
\textbf{class} \texttt{freed1(elasticFiber)}

\medskip\noindent
\textit{constructor}

\medskip\noindent
\texttt{elasFiber = freed1(E, e\_t)} \\
\indent \texttt{E} \quad\, the elastic tangent modulus at zero stress and zero strain \\
\indent \texttt{e\_t} \: the limit strain, i.e., the maximum strain allowed

\bigskip\noindent
\texttt{name = elasFiber.fiberType()}

\medskip\noindent
Returns a string that contains the name of the fiber model, viz., \texttt{`Freed-Rajagopal'}.

\medskip\noindent
\texttt{C = elasFiber.compliance(stress, strain, temperature)}

\medskip\noindent
Returns the elastic tangent compliance at a specified \texttt{stress, strain} and \texttt{temperature}.  Argument \texttt{temperature} is not required, but \texttt{stress} and \texttt{strain} are.

\medskip\noindent
\texttt{E = elasFiber.modulus(stress, strain, temperature)}

\medskip\noindent
Returns the elastic tangent modulus at a specified \texttt{stress, strain} and \texttt{temperature}, which is the inverse of its elastic compliance.  Argument \texttt{temperature} is not required, but \texttt{stress} and \texttt{strain} are.


\subsubsection{Freed-Rajagopal\slash Hookean Fibers}

This class provides the elastic response functions for Freed-Rajagopalean and Hookean fibers whose compliances sum, per Eqn.~(\ref{FRbiologicModulus}).  This version does not account for thermal straining.

\bigskip\noindent
\textbf{class} \texttt{freed2(elasticFiber)}

\medskip\noindent
\textit{constructor}

\medskip\noindent
\texttt{elasFiber = freed2(E1, E2, e\_t)} \\
\indent \texttt{E1} \quad\;\;\: the elastic tangent modulus at zero stress and strain \\
\indent \texttt{E2} \quad\;\;\: the elastic tangent modulus at terminal stress \\
\indent \texttt{e\_t} \quad\;\, the limit strain, i.e., the maximum strain allowed 

\bigskip\noindent
\texttt{name = elasFiber.fiberType()}

\medskip\noindent
Returns a string that contains the name of the fiber model, viz., \texttt{`Freed-Rajagopal/Hooke'}.

\medskip\noindent
\texttt{C = elasFiber.compliance(stress, strain, temperature)}

\medskip\noindent
Returns the elastic tangent compliance at a specified \texttt{stress} and \texttt{strain}.    Argument \texttt{temperature} is not required, but \texttt{stress} and \texttt{strain} are.

\medskip\noindent
\texttt{E = elasFiber.modulus(stress, strain, temperature)}

\medskip\noindent
Returns the elastic tangent modulus at a specified \texttt{stress} and \texttt{strain}; it is the inverse of its elastic compliance.  Argument \texttt{temperature} is not required, but \texttt{stress} and \texttt{strain} are.


\subsubsection{Freed-Rajagopal\slash Kelvin\slash Hookean Fibers}

This class provides the elastic response functions for Freed-Rajagopalean and Kelvin\slash Hookean fibers whose compliances sum, per Eqn.~(\ref{FRbiologicModulus}).  This version does account for thermal straining.

\bigskip\noindent
\textbf{class} \texttt{freed3(elasticFiber)}

\medskip\noindent
\textit{constructor}

\medskip\noindent
\texttt{elasFiber = freed3(E1, E2, e\_t, alpha, T0)} \\
\indent \texttt{E1} \quad\;\;\: the elastic tangent modulus at zero stress and strain \\
\indent \texttt{E2} \quad\;\;\: the elastic tangent modulus at terminal stress \\
\indent \texttt{e\_t} \quad\;\, the limit strain, i.e., the maximum strain allowed \\
\indent \texttt{alpha} \; thermal strain coefficient \\
\indent \texttt{T0} \quad\;\;\; the reference temperature for thermal strain, typically body temperature

\bigskip\noindent
\texttt{name = elasFiber.fiberType()}

\medskip\noindent
Returns a string that contains the name of the fiber model, viz., \texttt{`Freed-Rajagopal/Kelvin/ Hooke'}.

\medskip\noindent
\texttt{C = elasFiber.compliance(stress, strain, temperature)}

\medskip\noindent
Returns the elastic tangent compliance at a specified \texttt{stress, strain} and \texttt{temperature}.  All arguments are required.

\medskip\noindent
\texttt{E = elasFiber.modulus(stress, strain, temperature)}

\medskip\noindent
Returns the elastic tangent modulus at a specified \texttt{stress, strain} and \texttt{temperature}, which is the inverse of its elastic compliance.  All arguments are required.


\subsection{Chordal Fibers}

