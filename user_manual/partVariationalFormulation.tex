\setcounter{section}{0}
\part{Variational Formulation}
\label{partVariational}

The problem we have set up to solve takes on the general form of 
\begin{equation}
	\mathbf{M} \ddot{\mathbf{x}} + \mathbf{K} \mathbf{x} = \mathbf{f}(t)
\end{equation}
where $\mathbf{M}$ is a mass matrix, $\mathbf{K}$ is a stiffness matrix, $\mathbf{f}$ is a forcing function, and $\mathbf{x}$ is a displacement vector.  

For our problem of interest, 
\begin{subequations}
\begin{align}
	\mathbf{x} & = \sum_{v=1}^{20} \{ x_v , y_v , z_v \}^{\mathsf{T}} \\
	\intertext{are the co-ordinates of vertex $v$ located in the co-ordinate frame of the dodecahedron $(\boldsymbol{\imath} , \boldsymbol{\jmath} , \vec{\mathbfit{k}} )$ so that vectors $\mathbf{f}$ and $\mathbf{x}$ have length 60 while matrices $\mathbf{M}$ and $\mathbf{K}$ have dimension $60 \times 60$ with}
	\mathbf{M} & = \mathbf{M}_{1D} + \mathbf{M}_{2D} + \mathbf{M}_{3D} \\
	\mathbf{K} & = \mathbf{K}_{1D} + \mathbf{K}_{2D} + \mathbf{K}_{3D} \\
	\mathbf{f} & = \mathbf{f}_{1D} + \mathbf{f}_{2D} + \mathbf{f}_{3D}
\end{align}
\end{subequations}
where subscript `$\mbox{}_{1D}$' applies to the alveolar chords, subscript `$\mbox{}_{2D}$' applies to the alveolar septa, and subscript `$\mbox{}_{3D}$' applies to the alveolar volume.

\section{Mass Matrix}
A consistent mass matrix \cite{Archer65} established in its natural co-ordinate system is defined as
\begin{equation}
	\mathbf{M}_{C} = \sum_m \int_{V_m} \rho_m \, \mathbf{N}_m^{\mathsf{T}} \mathbf{N}_m \,
	\mathrm{d} V_m
	\label{consistentMassMatrix}
\end{equation}
wherein $\mathbf{N}_m$ is the shape function matrix used to also construct the stiffness matrix for element $m$.

The row-sum techniques is considered to construct the lumped mass matrix, that is the sum of the elements of each row of the consistent mass matrix is used as the diagonal element \cite{Reddy93}:
\begin{equation}
{M}_{L_{ii}} = \sum_{j=1}^n \int_{V_m} \rho \, N_i \, N_j \, \mathrm{d}  V_m 
\end{equation}
wherein $\sum_{j=1}^n N_j = 1$, and $n$ is the Gauss integration points.

The lumped-consistent weighted mass matrix $\mathbf{M}_{LC} $ is defined as follow
\begin{equation}
\mathbf{M}_{LC}  = (1 - \mu) \, \mathbf{M}_{C} + \mu \, \mathbf{M}_{L}
\end{equation}
wherein $\mu$ is a free scalar parameter that is considered to be $\mu = 1/2$ to minimize low frequency dispersion
\begin{equation}
\mathbf{M}_{LC}  = \frac{1}{2} \, (\mathbf{M}_{C} + \mathbf{M}_{L})
\label{LumconsMass}
\end{equation}

\subsection{Mass Matrix of Chord}
The determinant of the Jacobian matrix is used for the transformation of integral from the global coordinate system to the natural coordinate system by
\begin{equation}
     |\mathbf{J}| = \mathrm{\det} \mathbf{J} = \mathrm{\det} \begin{bmatrix} \frac{\partial x }{\partial\xi} \end{bmatrix} = \sum\nolimits_{i=1}^n N_{i,\xi} (\xi) \, x_i
     \label{detJac1D}
\end{equation}
wherein $N_{i}$ are the shape functions for a two-node alveolar chords in its natural coordinate system which are defined in a matrix form as
\begin{equation}
	\mathbf{N} = \begin{bmatrix}
    \frac{1}{2} \, (1 - \xi) &  \frac{1}{2} \, (1 + \xi)
\end{bmatrix} 
\end{equation}
wherein $\xi$ is abscissae of the Gauss integration rule. 

The consistent mass matrix of the 1-D alveolar chord that is evaluated numerically in its natural coordinate system can be described as
\begin{equation}
    \mathbf{M}_{1C} = \int_{\Gamma} \rho \, \mathbf{N}^{\mathsf{T}} \mathbf{N} \, A \, \mathrm{d} x  = \int_{-1}^{1} \rho \, \mathbf{N}^{\mathsf{T}} \mathbf{N}\, A \, |\mathbf{J}|\,  \mathrm{d} \xi =  \sum_{i=1}^{n}  \rho  \, \mathbf{N}^{\mathsf{T}} \mathbf{N} \, A\, |\mathbf{J}| \, \mathrm{w}_i
\end{equation}
 with $\mathrm{w}_i$ being the  weighting coefficients of the Gauss integration rule, and $A$ being the cross section area of alveolar chord. Table~\ref{tabQuadrature1D} demonstrates the values of $\xi$ and $\mathrm{w}_i$ for $n = 1, 2$, and $3$ Gauss integration points.
\begin{table}
    \centering
    \begin{tabular}{|c|rr|}
        \hline
        node & \centering $\xi$ coordinate \phantom{12}  & 
        weight \phantom{12} \\ \hline
        & \multicolumn{2}{|c|}{Exact for Polynomials of Degree $1^{\phantom{|^|}}$} \\ 
        \hline
        1 & 0.0000000000000 & 2.0000000000000 \\ 
        \hline
        & \multicolumn{2}{|c|}{Exact for Polynomials of Degree $3^{\phantom{|^|}}$} \\ \hline
        1 & -0.577350269189 & 1.000000000000\\
        2 & 0.577350269189 & 1.000000000000\\ 
        \hline
        & \multicolumn{2}{|c|}{Exact for Polynomials of Degree $5^{\phantom{|^|}}$} \\ \hline
        1 & -0.774596669241 & 0.555555555556 \\
        2 & 0.000000000000 & 0.888888888889\\
        3 & 0.774596669241 & 0.555555555556\\ 
        \hline
    \end{tabular}
    \caption{Generalized, Gaussian, quadrature, weights and nodes for integrating over a alveolar chord in its natural coordinate system.}
    \label{tabQuadrature1D}
\end{table}

The lumped mass matrix for 1-D alveolar chord in its natural coordinate system becomes 
\begin{equation}
{M}_{1L_{ii}} = \sum_{j=1}^n \int_{\Gamma} \rho \, N_i \, N_j \, A \, \mathrm{d} x  = \int_{-1}^{1} \rho \, N_i\, A \, |\mathbf{J}|\,  \mathrm{d} \xi =  \sum_{i=1}^n  \rho  \, N_i\, A\, |\mathbf{J}| \, \mathrm{w}_i
\end{equation}
wherein $\sum_{j=1}^n N_j = 1$. 

For instance, the consistent mass matrix for alveolar chords with $1$ Gauss integration point that is approximated by the weighted sum of function at the center of chord becomes
\begin{equation}
\mathbf{M}_{1C}  = \frac{\rho \, A \, L}{4}\begin{bmatrix}
1 & 1 \\
1 & 1
\end{bmatrix} 
	\label{ConsMassMatrix1D}
\end{equation}
where $L$ is the length of alveolar chord. The row-sum techniques, gives the lumped mass matrix
\begin{equation}
\mathbf{M}_{1L}  = \frac{\rho \, A \, L}{2}\begin{bmatrix}
1 & 0 \\
0 & 1
\end{bmatrix} 
\label{LumMassMatrix1D}
\end{equation}
and the lumped-consistent weighted mass matrix is constructed as follow 
\begin{equation}
\mathbf{M}_{1LC}  = \frac{\rho \, A \, L}{8}\begin{bmatrix}
3 & 1 \\
1 & 3
\end{bmatrix} 
\label{LumconsMassMatrix1D}
\end{equation}


\subsection{Mass Matrix of Pentagon}
For the alveolar septa, the matrix of shape functions $\mathbf{N}$ is arranged as
 \begin{equation}
	\mathbf{N} = 
	\begin{bmatrix}
	N_1 & 0 & N_2 & 0 & N_3 & 0 & N_4 & 0 & N_5 & 0 \\ 0 & N_1 & 0 & N_2 & 0 & N_3 & 0 & N_4 & 0 & N_5 
\end{bmatrix} 
	\label{shape2D}
\end{equation}
in which $\mathrm{N}_i (i = 1, 2, 3, 4, 5)$ are five shape functions corresponding to the five vertices of the pentagon that are defined in Eq.~(\ref{shapeFunctions}).
The consistent mass matrix $\mathbf{M}_{2C}$ can also be obtained by substituting the above shape function matrix into 
\begin{equation}
    \mathbf{M}_{2C} = \int_{V} \rho \, \mathbf{N}^{\mathsf{T}} \mathbf{N} \, \mathrm{d} V = \int_{\pentagon} \int_{\pentagon} \rho \, \mathbf{N}^{\mathsf{T}} \mathbf{N} \,|\mathbf{J}| \, h \, \mathrm{d} \xi \, \mathrm{d} \eta
    \label{massintegral2d}
\end{equation}
wherein $h$ being membrane thickness, and $|\mathbf{J}|$ being the determinant of the Jacobian matrix for pentagon. 
In quadrilateral derivations,  the Jacobian of two-dimensional transformations that connect the ${x, y}$ to ${\xi, \eta}$ coordinate systems is needed. The components of Jacobian matrix are calculated using derivatives of shape functions with respect to the local coordinates at the $i^{\mathrm{th}}$ vertex via
\begin{equation}
\mathbf{J} = 
\begin{bmatrix}
\partial x / \partial\xi & \partial y / \partial\xi \\
\partial x / \partial\eta & \partial y / \partial\eta 
\end{bmatrix}  
= \begin{bmatrix}
\sum\nolimits_{i=1}^5 N_{i,\xi} (\xi,\eta) \, x_i & \sum\nolimits_{i=1}^5 N_{i,\xi} (\xi,\eta) \, y_i \\
\sum\nolimits_{i=1}^5 N_{i,\eta} (\xi,\eta) \, x_i & \sum\nolimits_{i=1}^5 N_{i,\eta} (\xi,\eta) \, y_i
\end{bmatrix}
\label{jacobianpent}
\end{equation}
The numerical integration of Eq.~(\ref{massintegral2d}) result in 
\begin{equation}
    \mathbf{M}_{2C} = \sum_{i=1}^{n} \rho \, \mathbf{N}^{\mathsf{T}} \mathbf{N} \,|\mathbf{J}| \, h \, w_i
\end{equation}
where $n$ stands for number of Gauss points, and $\mathrm{w}_i$ denotes the natural weight of the element.
There are three mass matrix for each pentagon based upon three Gauss, quadrature rules in Table~\ref{tabQuadrature} which are integrating polynomials of order 3 and 5, respectively. 

The Lumped mass matrix for pentagon makes the diagonal parameters as follow
\begin{equation}
{M}_{2L_{ii}} = \sum_{j=1}^n \int_{V}  \rho \, N_i \, N_j \, \mathrm{d} V  = \int_{\pentagon} \int_{\pentagon} \rho \, N_i \, |\mathbf{J}| \, h \,  \mathrm{d} \xi \, \mathrm{d} \eta =  \sum_{i=1}^n  \rho  \, N_i \, |\mathbf{J}| \, h \, \mathrm{w}_i
\label{LumMass2D}
\end{equation}
wherein $\sum_{j=1}^n N_j = 1$. 

For instance,the lumped-consistent mass matrix of a pentagon  with $1$ Gauss integration point at the center of pentagon that is constructed by averaging consistent mass matrix and the lumped mass matrix becomes 
\begin{equation}
\mathbf{M}_{2LC}  = \rho \, h
\scalefont{0.75}{ \begin{bmatrix}
0.28532 & 0 & 0.04755 & 0 & 0.04755 & 0 & 0.04755 & 0 & 0.04755 & 0 \\
0 & 0.28532 & 0 & 0.04755 & 0 & 0.04755 & 0 & 0.04755 & 0 & 0.04755
\\
0.04755 & 0 & 0.28532 & 0 & 0.04755 & 0 & 0.04755 & 0 & 0.04755 & 0\\
0 & 0.04755 & 0 & 0.28532 & 0 & 0.04755 & 0 & 0.04755 & 0 & 0.04755 \\
0.04755 & 0 & 0.04755 & 0 & 0.28532 & 0 & 0.04755 & 0 & 0.04755 & 0 \\
0 & 0.04755 & 0 & 0.04755 & 0 & 0.28532 & 0 & 0.04755 & 0 & 0.04755 \\
0.04755 & 0 & 0.04755 & 0 & 0.04755 & 0 & 0.28532 & 0 & 0.04755 & 0\\
0 & 0.04755 & 0 & 0.04755 & 0 & 0.04755 & 0 & 0.28532 & 0 & 0.04755\\
0.04755 & 0 & 0.04755 & 0 & 0.04755 & 0 & 0.04755 & 0 & 0.28532 & 0 \\
0 & 0.04755 & 0 & 0.04755 & 0 & 0.04755 & 0 & 0.04755 & 0 & 0.28532\\
\end{bmatrix} }
\label{LumconsMassMatrix2D}
\end{equation}


\subsection{Mass Matrix of Tetrahedron}

The dodecahedron has 60 individual tetrahedral whereas its origin being the common vertex of all tetrahedrons. Hence, the analysis to find the mass matrix of a tetrahedron is used to reach the mass matrix of whole alveolar volume.

The matrix of shape functions $\mathbf{N}$ for a tetrahedon has the form of
\begin{equation}
	\mathbf{N} =  
\begin{bmatrix*}[r]
	N_1 & 0 & 0 & N_2 & 0 & 0 & N_3 & 0 & 0 & N_4 & 0 & 0 \\
	0 & N_1 & 0 & 0 & N_2 & 0 & 0 & N_3 & 0 & 0 & N_4 & 0 \\
	0 & 0 & N_1 & 0 & 0 & N_2 & 0 & 0 & N_3 & 0 & 0 & N_4
\end{bmatrix*} 
	\label{shape3D}
\end{equation}
in which $\mathrm{N}_i (i = 1, 2, 3, 4)$ are four shape functions corresponding to the four vertices of the tetrahedron that are defined as follow
\begin{subequations}
\begin{align}
	N_1 & = 1 - \xi - \eta - \zeta \\
	N_2 & = \xi \\
	N_3 & = \eta \\
	N_4 & = \zeta
\end{align}
\end{subequations}
The numerical integration is used to obtain the mass matrix of tetrahedron via
\begin{equation}
    \mathbf{M}_{3C} = \sum_{i=1}^{n_i} \rho  \mathbf{N}^{\mathsf{T}} \mathbf{N} \,  \,|\mathbf{J}| \, w_i
\end{equation}
wherein $|\mathbf{J}|$ being the determinant of the Jacobian matrix in a tetrahedron that are calculated using derivatives of shape functions with respect to the local coordinates $(\xi, \eta, \zeta)$, and the current global coordinates $(x_i, y_i, z_i)$ at the $i^{\mathrm{th}}$ vertex via
\begin{equation}
\begin{aligned}
 \mathbf{J}= &
\begin{bmatrix}
\partial x / \partial\xi & \partial y / \partial\xi & \partial z / \partial\xi\\
\partial x / \partial\eta & \partial y / \partial\eta & \partial z / \partial\eta \\
\partial x / \partial\zeta & \partial y / \partial\zeta & \partial z / \partial\zeta 
\end{bmatrix}\\
  = & \begin{bmatrix}
\sum\nolimits_{i=1}^4 N_{i,\xi} (\xi,\eta,\zeta) \, x_i & \sum\nolimits_{i=1}^4 N_{i,\xi} (\xi,\eta,\zeta) \, y_i &
\sum\nolimits_{i=1}^4 N_{i,\xi} (\xi,\eta,\zeta) \, z_i\\
\sum\nolimits_{i=1}^4 N_{i,\eta} (\xi,\eta,\zeta) \, x_i & \sum\nolimits_{i=1}^4 N_{i,\eta} (\xi,\eta,\zeta) \, y_i &
\sum\nolimits_{i=1}^4 N_{i,\eta} (\xi,\eta,\zeta) \, z_i\\
\sum\nolimits_{i=1}^4 N_{i,\zeta} (\xi,\eta,\zeta) \, x_i & \sum\nolimits_{i=1}^4 N_{i,\zeta} (\xi,\eta,\zeta) \, y_i &
\sum\nolimits_{i=1}^4 N_{i,\zeta} (\xi,\eta,\zeta) \, z_i
\end{bmatrix}
\end{aligned}
\label{jacobiantet}
\end{equation}

There are three mass matrix for each tetrahedron based upon three Gauss, quadrature rules in Table~\ref{tabQuadraturetetra} which are integrating polynomials of order 1, 2 and 3, respectively. 

\begin{table}
    \centering
    \begin{tabular}{|c|rrrr|}
        \hline
        node & \centering $\xi$ coordinate \phantom{1234}  & 
        $\eta$ coordinate \phantom{1234} & 
        $\zeta$ coordinate \phantom{1234} & weight \phantom{12345} \\ \hline        
        & \multicolumn{4}{|c|}{Exact for Polynomials of Degree $1^{\phantom{|^|}}$} \\ 
        \hline
        1 & 1/4 & 1/4 & 1/4 & 1/6 \\ 
        \hline
        & \multicolumn{4}{|c|}{Exact for Polynomials of Degree $2^{\phantom{|^|}}$} \\ \hline
        1 & (5 - $\sqrt{5}$)/20 & (5 - $\sqrt{5}$)/20 & (5 - $\sqrt{5}$)/20 & 1/24\\
        2 & (5 - $\sqrt{5}$)/20 & (5 - $\sqrt{5}$)/20 & (5 + 3 $\sqrt{5}$)/20 & 1/24\\
        3 & (5 - $\sqrt{5}$)/20 & (5 + 3 $\sqrt{5}$)/20 & (5 - $\sqrt{5}$)/20 & 1/24\\ 
        4 & (5 + 3 $\sqrt{5}$)/20 & (5 - $\sqrt{5}$)/20 & (5 - $\sqrt{5}$)/20 & 1/24\\ 
        \hline
        & \multicolumn{4}{|c|}{Exact for Polynomials of Degree $3^{\phantom{|^|}}$} \\ \hline
        1 & 1/4 & 1/4 & 1/4 & -2/15 \\
        2 & 1/6 & 1/6 & 1/6 & 3/40\\
        3 & 1/6 & 1/6 & 1/2 & 3/40\\ 
        4 & 1/6 & 1/2 & 1/6 & 3/40 \\
        5 & 1/2 & 1/6 & 1/6 & 3/40 \\
        \hline
    \end{tabular}
    \caption{Generalized, Gaussian, quadrature, weights and nodes  for integrating over a tetrahedron in its natural coordinate system.}
    \label{tabQuadraturetetra}
\end{table}


The lumped mass matrix for tetrahedron makes the diagonal parameters as follow
\begin{equation}
{M}_{3L_{ii}} = \sum_{j=1}^n \int_{V}  \rho \, N_i \, N_j \, \mathrm{d} V = \sum_{i=1}^n  \rho  \, N_i\, |\mathbf{J}| \, \mathrm{w}_i
\label{LumMass3D}
\end{equation}
wherein $\sum_{j=1}^n N_j = 1$. 

\section{Stiffness Matrix}
For the nonlinear elastic material like soft tissues, as they generally become stiffer with increased deformation, the slope of stress-strain curve will change, and therefore the instantaneous stiffness will change. So, it is required to derive the tangent stiffness for our purpose. 

The total potential energy of a deformed body can be expressed as the differences between the variation in the potential energy of deformation $\delta{U}$ and the variation in the potential energy of the external loading $\delta{W}$ \cite{Yangetal10}, where 
\begin{subequations}
\begin{align}
\delta{W} & = \mathbf{F} \, \mathrm{d} \boldsymbol{\Delta}\\
\delta{U} & = \int_{V} \, \bar{\mathbf{B}}^{\mathsf{T}} \, \boldsymbol{\sigma} \, \mathrm{d} V \, \mathrm{d} \boldsymbol{\Delta}
\end{align}
\end{subequations}
where $\mathbf{F}$ is the vector of external forces, and $\bar{\mathbf{B}}$ is the matrix of strain displacement for large deflection which make the relation between strain and displacement 
\begin{equation}
\boldsymbol{\epsilon} = \bar{\mathbf{B}} \,  \boldsymbol{\Delta}
\label{strain}
\end{equation}
The displacement fields are interpolated as
\begin{equation}
\bar{\mathbf{ u}} = 
\begin{Bmatrix}
u \\
v     
\end{Bmatrix}
=
\begin{Bmatrix}
\sum_{i=1}^{n} u_j \, N_j \\
\sum_{i=1}^{n} v_j \, N_j     
\end{Bmatrix}
= \mathbf{N} \, \boldsymbol{\Delta}
\end{equation}
with $\boldsymbol{\Delta}$ being the vector of displacement. 
The strain displacement can be defined as sum of linear and nonlinear strain displacements as follow
\begin{equation}
\bar{\mathbf{B}} = \mathbf{B}_L + \mathbf{B}_N
\label{straindis}
\end{equation} 
$\mathbf{B}_L$ can be obtained from the linear analysis $\boldsymbol{\epsilon}_L = \mathbf{B}_L \, \boldsymbol{\Delta}$, and $\mathbf{B}_N$ that is a function of displacement $ \mathbf{u}$ can be obtained from the nonlinear part of strain $\boldsymbol{\epsilon}_N$ which can be formed as follow 
\begin{equation}
\boldsymbol{\epsilon}_N = \tfrac{1}{2} \, {\mathbf{A}} \,  \boldsymbol{\theta}
\end{equation}
wherein ${\mathbf{A}}$ being the matrix of  derivative of displacements, and $\boldsymbol{\theta}$ being the vector of derivative of displacements that can be obtained via $\boldsymbol{\theta} = \mathbf{H} \, \boldsymbol{\Delta} $ where $\mathrm{d} \boldsymbol{\theta} = \mathbf{H} \, \mathrm{d} \boldsymbol{\Delta}$ with $\mathbf{H}$ being the derivative of shape functions respect to the global coordinate system. Thus, the derivative of strain can be obtained via
\begin{equation}
\mathrm{d} \boldsymbol{\epsilon} = \mathrm{d} (\boldsymbol{\epsilon}_L + \boldsymbol{\epsilon}_N) = \mathrm{d} \boldsymbol{\epsilon}_N = \tfrac{1}{2} \, (\mathrm{d} \mathbf{A} \, \boldsymbol{\theta} + \mathbf{A} \, \mathrm{d} \boldsymbol{\theta}) = \tfrac{1}{2} \, (\mathrm{d} \mathbf{A} \, \mathbf{H} \, \boldsymbol{\Delta} + \mathbf{A} \, \mathbf{H} \, \mathrm{d} \boldsymbol{\Delta})
\label{dstrain1}
\end{equation}
where $\mathrm{d} \mathbf{A} = \mathbf{A}^{'} \, \mathrm{d} \boldsymbol{\Delta} $. Besides, the derivative of strain in Eq.~(\ref{strain}) become 
\begin{equation}
\mathrm{d} \boldsymbol{\epsilon} = \mathrm{d} \bar{\mathbf{B}} \, \boldsymbol{\Delta} + \bar{\mathbf{B}} \, \mathrm{d} \boldsymbol{\Delta}
\label{dstrain2}
\end{equation}
wherein $\mathrm{d} \bar{\mathbf{B}} = \mathrm{d} (\mathbf{B}_L + \mathbf{B}_N) = \mathrm{d} \mathbf{B}_N$. Therefor, the nonlinear strain displacement and its derivative can be obtained by equating Eq.~(\ref{dstrain1}) and Eq.~(\ref{dstrain2}) which yield
\begin{equation}
\mathbf{B}_N = \tfrac{1}{2} \, \mathbf{A} \, \mathbf{H} - \mathbf{B}_L \quad \text{with} \quad \mathrm{d} \mathbf{B}_N = \tfrac{1}{2} \, \mathrm{d} \mathbf{A} \, \mathbf{H}
\end{equation} 
 

In order to satisfy the equilibrium, the first variation of the total potential energy which is known as the residual force should be zero \cite{Elseifi98}. 
\begin{equation}
\mathbf{R} = \int_{V} \, \bar{\mathbf{B}}^{\mathsf{T}} \, \boldsymbol{\sigma} \, \mathrm{d} V - \mathbf{F} = 0
\label{residual}
\end{equation}
derivative of the residual force yield 
\begin{equation}
\mathrm{d} \mathbf{R} = \int_{V} \, \mathrm{d} \bar{\mathbf{B}}^{\mathsf{T}} \, \boldsymbol{\sigma} \, \mathrm{d} V + \int_{V} \, \bar{\mathbf{B}}^{\mathsf{T}} \, \mathrm{d} \boldsymbol{\sigma} \, \mathrm{d} V 
\label{diffresidual}
\end{equation}
where $\mathrm{d} \boldsymbol{\sigma} = \mathbf{M} \, \mathrm{d} \boldsymbol{\epsilon} = \mathbf{M} \, \mathrm{d} \mathbf{B}_N \, \boldsymbol{\Delta} + \mathbf{M} \,  \bar{\mathbf{B}} \, \mathrm{d} \boldsymbol{\Delta}$, with $\mathbf{M}$ being the matrix of elastic modulus. 

The tangent stiffness matrix can be obtained by substituting the definition of $\mathrm{d} \boldsymbol{\sigma}$ into Eq.~(\ref{diffresidual}) as follow 
\begin{equation}
\begin{aligned}
\mathrm{d} \mathbf{R} & = \int_{V} \, \mathrm{d} \mathbf{B}_N^{\mathsf{T}} \, \boldsymbol{\sigma} \, \mathrm{d} V + \int_{V} \, \bar{\mathbf{B}}^{\mathsf{T}} \, \mathbf{M} \, \mathrm{d} \mathbf{B}_N \, \boldsymbol{\Delta} \, \mathrm{d} V  + \int_{V} \, \bar{\mathbf{B}}^{\mathsf{T}} \, \mathbf{M} \, \bar{\mathbf{B}} \, \mathrm{d} V \, \mathrm{d} \boldsymbol{\Delta} \\ 
& = \int_{V} \, \frac{1}{2} \, (\mathbf{H}^{\mathsf{T}} \, \mathrm{d} \mathbf{A}^{\mathsf{T}} \boldsymbol{\sigma}) \, \mathrm{d} V + \int_{V} \, \frac{1}{2} \, \left(\mathbf{B}_L + \mathbf{B}_N \right)^{\mathsf{T}} \, \mathbf{M} \, \mathrm{d} \mathbf{A} \, \mathbf{H} \, \boldsymbol{\Delta} \, \mathrm{d} V  \\
& + \int_{V} \, \left(\mathbf{B}_L + \mathbf{B}_N \right)^{\mathsf{T}} \, \mathbf{M} \,  \left(\mathbf{B}_L + \mathbf{B}_N \right) \, \mathrm{d} V \, \mathrm{d} \boldsymbol{\Delta} \\ 
& = \frac{1}{2} \, \int_{V} \, \mathbf{H}^{\mathsf{T}} \, \boldsymbol{\sigma} \, \mathbf{A}^{'} \mathrm{d} V \, \mathrm{d} \boldsymbol{\Delta} + \frac{1}{2} \, \int_{V} \mathbf{B}_L^{\mathsf{T}} \, \mathbf{M} \, \mathbf{A}^{'} \, \mathbf{H} \, \boldsymbol{\Delta} \, \mathrm{d} V \, \mathrm{d} \boldsymbol{\Delta}\\
& + \frac{1}{2} \, \int_{V} \mathbf{B}_N^{\mathsf{T}} \, \mathbf{M} \, \mathbf{A}^{'} \, \mathbf{H} \, \boldsymbol{\Delta} \, \mathrm{d} V \, \mathrm{d} \boldsymbol{\Delta} + \int_{V} \, \mathbf{B}_L^{\mathsf{T}} \, \mathbf{M} \, \mathbf{B}_L \, \mathrm{d} V \, \mathrm{d} \boldsymbol{\Delta}\\ 
& + \int_{V} \,  \left(\mathbf{B}_L^{\mathsf{T}} \, \mathbf{M} \, \mathbf{B}_N + \mathbf{B}_N^{\mathsf{T}} \, \mathbf{M} \, \mathbf{B}_L + \mathbf{B}_N^{\mathsf{T}} \, \mathbf{M} \, \mathbf{B}_N \, \right) \, \mathrm{d} V \, \mathrm{d} \boldsymbol{\Delta}\\
& = \left(\mathbf{K}_{\boldsymbol{\sigma}} + \mathbf{K}_L + \mathbf{K}_N\right)\mathrm{d} \boldsymbol{\Delta} = \mathbf{K}_T \, \mathrm{d} \boldsymbol{\Delta}
\end{aligned} 
\end{equation} 
wherein $\mathbf{K}_{\boldsymbol{\sigma}}$ being the stiffness matrix associated with the initial stress, $\mathbf{K}_L$ being the conventional small displacement stiffness matrix, $\mathbf{K}_N$ being the the large displacement stiffness matrix, and $\mathbf{K}_T$ being the tangent-stiffness matrix. 

\subsection{Stiffness Matrix for Chord}
The components of Laplace stretch $\boldsymbol{\mathcal{U}}$ has a Cholesky factorization expressed in terms of the right Cauchy-Green deformation tensor $\mathbf{C} \defeq \mathbf{F}^{\mathsf{T}} \mathbf{F} = \boldsymbol{\mathcal{U}}^{\mathsf{T}} \boldsymbol{\mathcal{U}}$, which is a symmetric second-order tensor. 
A 1-D chord has an axial strain of $e = \ln ( a / a_0 ) = \ln ( L / L_0 )$ 
where $a$ that is the first component of Laplace stretch can be written in terms of derivative of displacements as follow
\begin{equation}
a = {\mathcal{U}}_{11} = \sqrt{C_{11}}  \quad \text{with} \quad \mathrm{C_{11}} = \left(\frac{\mathrm{\partial u}}{\partial x}\right)^2 + 2\, \frac{\mathrm{\partial u}}{\partial x}  + 1
\end{equation} 
The total axial strain of chord can be written in terms of linear and nonlinear incremental strains as follow
\begin{equation}
 e = e_{L} + e_{N}
\end{equation}
Taylor series are used here to obtain the linear and nonlinear part of displacements 
\begin{subequations}
	\begin{align}
	e_{L} & = \mathrm{\frac{\partial u}{\partial x}} \\
	e_{N} & = -\frac{1}{2} \, \mathrm{\frac{\partial u}{\partial x}}\, \mathrm{\frac{\partial u}{\partial x}} 
	\end{align}
\end{subequations}

The linear strain displacement matrix $\mathbf{B}_L$ can be obtained by derivative of displacements as follow
\begin{subequations}
	\begin{align}
e_L  = 
\partial u / \partial x   = &
\sum\nolimits_{i=1}^2 N_{i,x} \, u_i = \begin{bmatrix}
[\mathbf{b}_1], & [\mathbf{b}_2]
\end{bmatrix}  \begin{Bmatrix} \boldsymbol{\Delta} \end{Bmatrix}  
= [\mathbf{B}_L] \begin{Bmatrix} \boldsymbol{\Delta} \end{Bmatrix}\\
 \intertext{with} \mathbf{b}_i  &= 
N_{i,x} \quad {and} \quad
\mathbf{\Delta}^T  = 
\begin{Bmatrix}
u_1 ,u_2
\end{Bmatrix} 
\end{align}
\end{subequations}


The nonlinear strain terms can be written as 
\begin{equation}
e_ N =  \tfrac{1}{2} \,
 [-\partial u / \partial x ]
\{\partial u / \partial x\}
= \tfrac{1}{2} \, \mathbf{A} \, \boldsymbol{\theta}
\end{equation}
the derivative of displacement can be related to the nodal parameters via
\begin{equation}
\boldsymbol{\theta} =  \begin{Bmatrix}
\partial u / \partial x
\end{Bmatrix}
= \begin{Bmatrix}
\sum\nolimits_{i=1}^2 N_{i,x} \, u_i
\end{Bmatrix} 
= \begin{bmatrix}
[\mathbf{h}_1], & [\mathbf{h}_2]
\end{bmatrix}  \begin{Bmatrix} \boldsymbol{\Delta} \end{Bmatrix}  
= [\mathbf{H}] \begin{Bmatrix} \boldsymbol{\Delta} \end{Bmatrix} \quad \text{with} \quad \mathbf{h}_i = N_{i,x}  
\end{equation}
hence $\mathbf{B}_N$ become
\begin{equation}
\mathbf{B}_N = \tfrac{1}{2} \, \mathbf{A} \, \mathbf{H} - \mathbf{B}_L =  \begin{bmatrix}
[\mathbf{b}_1], & [\mathbf{b}_2] 
\end{bmatrix} 
\end{equation}
where
\begin{equation}
\mathbf{b}_i  =   
\tfrac{1}{2} \, [-\partial u / \partial x] \cdot
N_{i,x} - N_{i,x} 
\end{equation}


The stress stiffness matrix $\mathbf{K}_{\boldsymbol{\sigma}}$ expresses the influence of membrane stresses on the lateral deflection that is independent of material properties.
\begin{equation}
\begin{aligned}
\mathbf{K}_{\boldsymbol{\sigma}} & = \frac{1}{2} \, \int_{V} \, \mathbf{H}^{\mathsf{T}} \, \boldsymbol{\sigma} \, \mathbf{A}^{'} \, \mathrm{d} V = \frac{1}{2} \, \int_{\Gamma} \, \mathbf{H}^{\mathsf{T}} \, \boldsymbol{\sigma} \, \mathbf{A}^{'} \, A \, \mathrm{d} x 
= \frac{1}{2} \, \int_{-1}^{1} \mathbf{H}^{\mathsf{T}} \,  \, \boldsymbol{\sigma} \, \mathbf{A}^{'} \, |\mathbf{J}|  \, A \,  \mathrm{d} \xi\\
& = \frac{1}{2} \, \sum_{i=1}^{n}  \mathbf{H}^{\mathsf{T}} \, \boldsymbol{\sigma} \, \mathbf{A}^{'}  \, |\mathbf{J}| \, A \, \mathrm{w}_i
\end{aligned}
\end{equation}
where 
\begin{equation}
\begin{aligned}
\mathbf{H} = \begin{bmatrix}
 N_{1,x} &  N_{2,x}
\end{bmatrix} = \begin{bmatrix}
 N_{1,\xi} &  \partial N_{2,\xi} \, 
\end{bmatrix} \, \mathbf{J}^{-1}
\end{aligned}
\end{equation}
and $\mathbf{A}^{'}$ can be defined as follow
\begin{equation}
\mathrm{d} \mathbf{A} = \mathbf{A}^{'} \, \mathrm{d} \boldsymbol{\Delta} = \begin{bmatrix}
[\mathbf{a}^{'}_1], & [\mathbf{a}^{'}_2]
\end{bmatrix}  \begin{Bmatrix} \boldsymbol{\Delta} \end{Bmatrix}  
= [\mathbf{A}^{'}] \begin{Bmatrix} \mathrm{d} \boldsymbol{\Delta} \end{Bmatrix} \quad \text{with} \quad \mathbf{a}^{'}_i = - N_{i,x}  
\end{equation}


The small displacement stiffness matrix for a 1-D alveolar chord that is transformed from global coordinate system to the natural coordiante system by the determinant of the Jacobian matrix, i.e., Eq.~(\ref{detJac1D}) is evaluated numerically as 
\begin{equation}
\mathbf{K}_{L} = \int_{\Gamma} \, \mathbf{B}_L^{\mathsf{T}} \, \mathbf{M} \, \mathbf{B}_L \, A \, \mathrm{d} x  = \int_{-1}^{1} \mathbf{B}_L^{\mathsf{T}} \, \mathbf{M} \, \mathbf{B}_L \, |\mathbf{J}|  \, A \,  \mathrm{d} \xi =  \sum_{i=1}^{n}  \mathbf{B}_L^{\mathsf{T}} \, \mathbf{M} \, \mathbf{B}_L \, |\mathbf{J}| \, A \, \mathrm{w}_i
\end{equation}
with $\mathrm{w}_i$ being the  weighting coefficients of the Gauss integration rule, and $A$ being the cross section area of alveolar chord. The values of $\xi$ and $\mathrm{w}_i$ for $n = 1, 2$, and $3$ Gauss integration points are demonstrated in Table~\ref{tabQuadrature1D}.
In calculation of the stiffness matrix for an element, the linear strain-displacement matrix $\mathbf{B}_L$ is required that is the derivatives of the shape functions with respect to the global coordinate system  
\begin{equation}
\mathbf{B}_L = \begin{bmatrix}
 N_{1,x} &   N_{2,x}
\end{bmatrix} = \begin{bmatrix}
 N_{1,\xi} &   N_{2,\xi} \, 
\end{bmatrix} \, \mathbf{J}^{-1}
\end{equation}
wherein $\xi$ is abscissae of the Gauss integration rule. 

The large displacement stiffness matrix for chord can be presented as follow
\begin{equation}
\mathbf{K}_{N}  = \int_{V} \mathbf{D} \, \mathrm{d} V \\
 = \int_{-1}^{1} \mathbf{D} \, |\mathbf{J}|  \, A \,  \mathrm{d} \xi 
= \sum_{i=1}^{n} \mathbf{D} \, |\mathbf{J}| \, A \, \mathrm{w}_i
\end{equation}
where $\mathbf{D}$ take the form of
\begin{equation}
\mathbf{D} = \tfrac{1}{2} \, \mathbf{B}_L^{\mathsf{T}} \, \mathbf{M} \, \mathbf{A}^{'} \, \mathbf{H} \, \boldsymbol{\Delta} + \tfrac{1}{2} \, \mathbf{B}_N^{\mathsf{T}} \, \mathbf{M} \, \mathbf{A}^{'} \, \mathbf{H} \, \boldsymbol{\Delta} + \mathbf{B}_L^{\mathsf{T}} \, \mathbf{M} \, \mathbf{B}_N \, + \mathbf{B}_N^{\mathsf{T}} \, \mathbf{M} \, \mathbf{B}_L \, + \mathbf{B}_N^{\mathsf{T}} \, \mathbf{M} \, \mathbf{B}_N 
\end{equation}
wherein $\mathbf{B}_N$ have the expression 
\begin{equation}
\begin{aligned}
\mathbf{B}_{N} & =  \tfrac{1}{2} \, \begin{bmatrix}
\partial \mathbf{u} / \partial \mathbf{x} 
\end{bmatrix} \, \begin{bmatrix}
 N_{1,x} &  N_{2,x}
\end{bmatrix} - \begin{bmatrix}
N_{1,x} &   N_{2,x}
\end{bmatrix}  \\
& = \tfrac{1}{2} \, \begin{bmatrix}
\sum\nolimits_{=1}^n N_{i,x} \, u_i
\end{bmatrix} \, \begin{bmatrix}
 N_{1,\xi} &   N_{2,\xi} \, 
\end{bmatrix} \, \mathbf{J}^{-1} - \begin{bmatrix}
N_{1,\xi} &   N_{2,\xi} \, 
\end{bmatrix} \, \mathbf{J}^{-1}
\end{aligned}
\end{equation}
Thereby the total stiffness matrix $\mathbf{K}_T$ can be obtained by summation of stress stiffness matrix, small and large displacement stiffness matrices.

\subsection{Stiffness Matrix for Pentagon}
The components of Laplace stretch $\boldsymbol{\mathcal{U}}$ associated with a planar membrane has a  Cholesky factorization expressed in terms of the right Cauchy-Green deformation tensor $\mathbf{C} \defeq \mathbf{F}^{\mathsf{T}} \mathbf{F} = \boldsymbol{\mathcal{U}}^{\mathsf{T}} \boldsymbol{\mathcal{U}}$, which is a symmetric second-order tensor. 
\begin{equation}
\begin{aligned}
{\mathcal{U}}_{11} & = \sqrt{C_{11}} \;\; & 
{\mathcal{U}}_{12} & = C_{12} / {\mathcal{U}_{11}} \\
{\mathcal{U}}_{21} & = 0 &
{\mathcal{U}}_{22} & = \sqrt{C_{22} - ({\mathcal{U}}_{12})^2} 
\end{aligned}
\label{Laplace stretchComponents}
\end{equation} 
where ${C_{11}}$, ${C_{12}}$, ${C_{21}}$ and ${C_{22}}$ are components of the Green deformation matrix $\mathbf{C}$ wherein  $\mathbf{F}$ is the deformation gradient.
\begin{subequations}
	\begin{align}
	\mathbf{C} &= \mathbf{F}^{\mathsf{T}} \mathbf{F} =
	\begin{bmatrix}
	C_{11} & C_{12}  \\
	C_{12} & C_{22} 
	\end{bmatrix} \quad \textrm{,} \quad
	\mathbf{F} =  
	\begin{bmatrix}
	1+\mathrm{\partial u / \partial x} & \mathrm{\partial u / \partial y}  \\
	\mathrm{\partial v / \partial x} & 1+\mathrm{\partial v / \partial y}
	\end{bmatrix}\\
	\intertext{with}
	\mathrm{C_{11}}&= \left(\frac{\mathrm{\partial u}}{\partial x}\right)^2+ \left(\frac{\mathrm{\partial v}}{\partial x}\right)^2 + 2\, \frac{\mathrm{\partial u}}{\partial x}  + 1\\
	\mathrm{C_{12}}&= \frac{\mathrm{\partial u}}{\partial y} + \frac{\mathrm{\partial v}}{\partial x} + \frac{\mathrm{\partial u}}{\partial x} \cdot \frac{\mathrm{\partial u}}{\partial y} + \frac{\mathrm{\partial v}}{\partial x} \cdot \frac{\mathrm{\partial v}}{\partial y}\\
	\mathrm{C_{22}}&= \left(\frac{\mathrm{\partial u}}{\partial y}\right)^2 + \left(\frac{\mathrm{\partial v}}{\partial y}\right)^2 + 2\, \frac{\mathrm{\partial v}}{\partial y} + 1
	\end{align}
\end{subequations}

It is useful to define the total virtual strains Eq.~(\ref{conjugateStrains}) in terms of linear and nonlinear incremental strains as follow
\begin{subequations}
	\begin{align}
	\mathrm  \xi & =\mathrm \xi_{L}+\mathrm  \xi_{N}\\
	\mathrm  \varepsilon & =\mathrm  \varepsilon_{L}+\mathrm  \varepsilon_{N}\\
	\mathrm \gamma & =\mathrm  \gamma_{L}+\mathrm  \gamma_{N}
	\end{align}
	\label{totalvirtualstrain}
\end{subequations}
Taylor series are used here to obtain the linear and nonlinear part of displacements 
\begin{subequations}
	\begin{align}
	\mathrm \xi_{L} & = \frac{1}{2} \, \left(\mathrm{\frac{\partial u}{\partial x}} + \mathrm{\frac{\partial v}{\partial y}}\right)\\
	\mathrm \xi_{N} & = \frac{1}{4} \, \left(- \mathrm{\frac{ \partial v}{\partial y}}\, \mathrm{\frac{ \partial v}{\partial y}} -\mathrm{\frac{\partial u}{\partial x}}\, \mathrm{\frac{\partial u}{\partial x}} - 2 \, \mathrm{\frac{\partial u}{\partial y}}\, \mathrm{\frac{\partial v}{\partial x}}\right)\\
	\mathrm \varepsilon_{L} & = \frac{1}{2} \, \left(\mathrm{\frac{\partial u}{\partial x}} - \mathrm{\frac{\partial v}{\partial y}}\right)\\
	\mathrm \varepsilon_{N} & = \frac{1}{4} \, \left(2 \, \mathrm{\frac{\partial v}{\partial x}}\, \mathrm{\frac{\partial v}{\partial x}} + \mathrm{\frac{ \partial v}{\partial y}}\, \mathrm{\frac{ \partial v}{\partial y}} -\mathrm{\frac{\partial u}{\partial x}}\, \mathrm{\frac{\partial u}{\partial x}} + 2 \, \mathrm{\frac{\partial u}{\partial y}}\, \mathrm{\frac{\partial v}{\partial x}}\right)\\
	\mathrm \gamma_{L} & = \mathrm{\frac{\partial u}{\partial y}} + \mathrm{\frac{\partial v}{\partial x}}\\
	\mathrm \gamma_{N} & = \mathrm{\frac{\partial v}{\partial x}}\, \mathrm{\frac{\partial v}{\partial y}} - 2\, \mathrm{\frac{ \partial u}{\partial x}}\, \mathrm{\frac{ \partial v}{\partial x}} 
	-\mathrm{\frac{\partial u}{\partial x}}\, \mathrm{\frac{\partial u}{\partial y}}
	\end{align}
\end{subequations}
The linear strain displacement matrix $\mathbf{B}_L$ can be obtained by differentiation of displacements expressed through the nodal displacements and shape functions from infinitesimal linear strain vector that take the form of
\begin{equation}
\begin{aligned}
\boldsymbol{\epsilon}_L & = \frac{1}{2} \, \begin{bmatrix}
\partial u / \partial x & 0 & 0 &   \partial v / \partial y \\
\partial u / \partial x & 0 & 0 & - \partial v / \partial y \\
0 & 2 \, \partial u / \partial y & 2 \, \partial v / \partial x & 0 \end{bmatrix} \\
& = \frac{1}{2} \, \begin{bmatrix}
\sum\nolimits_{i=1}^5 N_{i,x} \, u_i & 0 & 0 &   \sum\nolimits_{i=1}^5 N_{i,y} \, v_i \\
\sum\nolimits_{i=1}^5 N_{i,x} \, u_i & 0 & 0 & - \sum\nolimits_{i=1}^5 N_{i,y} \, v_i \\
0 & 2 \, \sum\nolimits_{i=1}^5 N_{i,y} \, u_i & 2 \, \sum\nolimits_{i=1}^5 N_{i,x} \, v_i & 0 \end{bmatrix} \\
& = \begin{bmatrix}
[\mathbf{b}_1], & [\mathbf{b}_2], & [\mathbf{b}_3], & [\mathbf{b}_4], & [\mathbf{b}_5] 
\end{bmatrix}  \begin{Bmatrix} \boldsymbol{\Delta} \end{Bmatrix}  
= [\mathbf{B}_L] \begin{Bmatrix} \boldsymbol{\Delta} \end{Bmatrix} 
\end{aligned}
\end{equation}
wherein 
\begin{subequations}
\begin{align}
\mathbf{b}_i = &
\frac{1}{2} \, \begin{bmatrix}
N_{i,x} & 0 & 0 &   N_{i,y} \\
N_{i,x} & 0 & 0 & - N_{i,y} \\
0 & 2 \,  N_{i,y} & 2 \,  N_{i,x} & 0 \end{bmatrix}\\
\intertext{and}
\mathbf{\Delta}^T  = &
\begin{Bmatrix}
u_1 ,u_1, v_1, v_1, u_2 ,u_2, v_2, v_2, ..., u_n ,u_n, v_n, v_n
\end{Bmatrix}
\end{align}
\end{subequations}


The nonlinear strain terms can be written as 
\begin{equation}
\boldsymbol{\epsilon}_ N = \frac{1}{2} \, \begin{bmatrix}
-\frac{1}{2} \, \partial u / \partial x &  - \, \partial v / \partial x & 0 & -\frac{1}{2} \, \partial v / \partial y \\
-\frac{1}{2} \, \partial u / \partial x &  \partial v / \partial x & \partial v / \partial x & \frac{1}{2} \, \partial v / \partial y \\
-2 \, \partial u / \partial y & 0 & -4 \, \partial u / \partial x  & 2 \, \partial v / \partial x \end{bmatrix} \, \begin{Bmatrix}
\partial u / \partial x\\
\partial u / \partial y\\
\partial v / \partial x\\
\partial v / \partial y
\end{Bmatrix}
= \tfrac{1}{2} \, \mathbf{A} \, \boldsymbol{\theta}
\end{equation}
the derivative of displacement can be related to the nodal parameters via
\begin{equation}
\boldsymbol{\theta} =  \begin{Bmatrix}
\partial u / \partial x\\
\partial u / \partial y\\
\partial v / \partial x\\
\partial v / \partial y
\end{Bmatrix}
= \begin{Bmatrix}
\sum\nolimits_{i=1}^5 N_{i,x} \, u_i\\
\sum\nolimits_{i=1}^5 N_{i,y} \, u_i\\
\sum\nolimits_{i=1}^5 N_{i,x} \, v_i\\
\sum\nolimits_{i=1}^5 N_{i,y} \, v_i
\end{Bmatrix} 
= \begin{bmatrix}
[\mathbf{h}_1], & [\mathbf{h}_2], & [\mathbf{h}_3], & [\mathbf{h}_4], & [\mathbf{h}_5] 
\end{bmatrix}  \begin{Bmatrix} \boldsymbol{\Delta} \end{Bmatrix}  
= [\mathbf{H}] \begin{Bmatrix} \boldsymbol{\Delta} \end{Bmatrix} 
\end{equation}
where 
\begin{equation}
\mathbf{h}_i = \begin{bmatrix}
N_{i,x} &  0 & 0 & 0 \\
0 & N_{i,y} & 0 & 0  \\
0 & 0 & N_{i,x} & 0 \\
0 & 0 & 0  & N_{i,y} \end{bmatrix} 
\end{equation}
hence $\mathbf{B}_N$ become
\begin{equation}
\mathbf{B}_N = \tfrac{1}{2} \, \mathbf{A} \, \mathbf{H} - \mathbf{B}_L = \tfrac{1}{2} \, \begin{bmatrix}
 [\mathbf{b}_1], & [\mathbf{b}_2], & [\mathbf{b}_3], & [\mathbf{b}_4], & [\mathbf{b}_5] 
\end{bmatrix} 
\end{equation}
where
\begin{equation}
\begin{aligned}
\mathbf{b}_i  = &   \begin{bmatrix}
-\frac{1}{2} \, \partial u / \partial x &  - \, \partial v / \partial x & 0 & -\frac{1}{2} \, \partial v / \partial y \\
-\frac{1}{2} \, \partial u / \partial x &  \partial v / \partial x & \partial v / \partial x & \frac{1}{2} \, \partial v / \partial y \\
-2 \, \partial u / \partial y & 0 & -4 \, \partial u / \partial x  & 2 \, \partial v / \partial x \end{bmatrix}  \, \begin{bmatrix}
N_{i,x} &  0 & 0 & 0 \\
0 & N_{i,y} & 0 & 0  \\
0 & 0 & N_{i,x} & 0 \\
0 & 0 & 0  & N_{i,y} \end{bmatrix} \\
 - & \begin{bmatrix}
 N_{i,x} & 0 & 0 &   N_{i,y} \\
 N_{i,x} & 0 & 0 & - N_{i,y} \\
 0 & 2 \,  N_{i,y} & 2 \,  N_{i,x} & 0 \end{bmatrix}
\end{aligned}
\end{equation}

The stress stiffness matrix for a 2-D alveolar septa take the form of
\begin{equation}
\mathbf{K}_{\boldsymbol{\sigma}} = \frac{1}{2} \, \int_{V} \, \mathbf{H}^{\mathsf{T}} \, \boldsymbol{\sigma} \, \mathbf{A}^{'} \, \mathrm{d} V
= \frac{1}{2} \, \int_{\pentagon} \int_{\pentagon} \mathbf{H}^{\mathsf{T}} \, \boldsymbol{\sigma} \, \mathbf{A}^{'} \, |\mathbf{J}|  \, h \,  \mathrm{d} \xi \,  \mathrm{d} \eta =  \frac{1}{2} \, \sum_{i=1}^{n}  \mathbf{H}^{\mathsf{T}} \, \boldsymbol{\sigma} \, \mathbf{A}^{'} \, |\mathbf{J}| \, h \, \mathrm{w}_i
\end{equation}
where $n$ stands for number of Gauss points, $\mathrm{w}_i$ denotes the natural weight of the element demonstrated in Table~\ref{tabQuadrature}, and $\mathbf{A}^{'}$ can be defined as follow
\begin{equation}
\mathrm{d} \mathbf{A} = \mathbf{A}^{'} \, \mathrm{d} \boldsymbol{\Delta} = \begin{bmatrix}
[\mathbf{a}^{'}_1], & [\mathbf{a}^{'}_2], & [\mathbf{a}^{'}_3], & [\mathbf{a}^{'}_4], & [\mathbf{a}^{'}_5] 
\end{bmatrix}  \begin{Bmatrix} \boldsymbol{\Delta} \end{Bmatrix}  
= [\mathbf{A}^{'}] \begin{Bmatrix} \mathrm{d} \boldsymbol{\Delta} \end{Bmatrix}
\end{equation}
where
\begin{equation}
\mathbf{a}^{'}_i = \begin{bmatrix}
-\tfrac{1}{2} \, N_{i,x} &  - N_{i,x} & 0 & -\tfrac{1}{2} \, N_{i,y} \\
-\tfrac{1}{2} \, N_{i,x} &  N_{i,x} &  N_{i,x} &  \tfrac{1}{2} \, N_{i,y} \\
-2 \, N_{i,y} & 0 & -4 \, N_{i,x} & 2 \, N_{i,x} \end{bmatrix} \quad \text{and} \quad  \mathrm{d} \boldsymbol{\Delta} = \begin{Bmatrix}
\mathrm{d} u_i\\
\mathrm{d} u_i\\
\mathrm{d} v_i\\
\mathrm{d} v_i
\end{Bmatrix}
\end{equation}

The small displacement stiffness matrix for pentagon is evaluated numerically as 
\begin{equation}
\mathbf{K}_{L} = \int_{V} \, \mathbf{B}_L^{\mathsf{T}} \, \mathbf{M} \, \mathbf{B}_L \, \mathrm{d} V  = \int_{\pentagon} \int_{\pentagon} \mathbf{B}_L^{\mathsf{T}} \, \mathbf{M} \, \mathbf{B}_L \, |\mathbf{J}| \, h \,  \mathrm{d} \xi \,  \mathrm{d} \eta =  \sum_{i=1}^{n}  \mathbf{B}_L^{\mathsf{T}} \, \mathbf{M} \, \mathbf{B}_L \, |\mathbf{J}|  \, h \, \mathrm{w}_i
\end{equation}

The large displacement stiffness matrix for chord can be presented as follow
\begin{equation}
\mathbf{K}_{N} =  \int_{V} \mathbf{D} \, \mathrm{d} V 
= \int_{\pentagon} \int_{\pentagon} \mathbf{D} \, |\mathbf{J}|\, h \, \mathrm{d} \xi \, \mathrm{d} \eta 
= \sum_{i=1}^{n}  \mathbf{D} \, |\mathbf{J}| \, h \, \mathrm{w}_i
\end{equation}
where $\mathbf{D}$ have the expression 
\begin{equation}
\mathbf{D} = \tfrac{1}{2} \, \mathbf{B}_L^{\mathsf{T}} \, \mathbf{M} \, \mathbf{A}^{'} \, \mathbf{H} \, \boldsymbol{\Delta} + \tfrac{1}{2} \, \mathbf{B}_N^{\mathsf{T}} \, \mathbf{M} \, \mathbf{A}^{'} \, \mathbf{H} \, \boldsymbol{\Delta} + \mathbf{B}_L^{\mathsf{T}} \, \mathbf{M} \, \mathbf{B}_N \, + \mathbf{B}_N^{\mathsf{T}} \, \mathbf{M} \, \mathbf{B}_L \, + \mathbf{B}_N^{\mathsf{T}} \, \mathbf{M} \, \mathbf{B}_N
\end{equation}


