\setcounter{section}{0}
\part{Variational Formulation}
\label{partVariational}

The problem that we have set up to solve is cast in a Lagrangian setting and takes on the general form of a second-order, ordinary, differential equation; specifically,
\begin{subequations}
    \label{FEsysOfEqns}
    \begin{gather}
    \mathbf{M} \, \ddot{\boldsymbol{\Delta}} + 
    \mathbf{C} \, \dot{\boldsymbol{\Delta}} +
    \mathbf{K} \, \boldsymbol{\Delta} = \boldsymbol{F} \\
    \intertext{that under conditions of equilibrium (i.e., whenever $\ddot{\boldsymbol{\Delta}} = \dot{\boldsymbol{\Delta}} = \mathbf{0}$) reduces to}
    \mathbf{K} \, \boldsymbol{\Delta} = \boldsymbol{F}
    \end{gather}
\end{subequations}
where $\mathbf{M}$ is a mass matrix, $\mathbf{C}$ is a tangent stiffness matrix\footnote{\footnotesize 
    In the literature, matrix $\mathbf{C}$ is typically utilized as a damping matrix; however, there are presently no damping mechanisms in our alveolar model.  For example, visco\-elastic effects are often important when modeling biologic tissues.  Nevertheless, they can be neglected here, because the event of a shock wave passing over an alveolus happens so fast that visco\-elastic effects do not have a chance to manifest themselves.  Tissue response will be glassy elastic during such an event.  Here $\mathbf{C}$ provides for an elastic \textit{tangent\/} response, while $\mathbf{K}$ provides for an elastic \textit{secant\/} response. 
},
$\mathbf{K}$ is a secant stiffness matrix, and $\boldsymbol{F}$ is a force vector, while vector $\boldsymbol{\Delta}$ contains the assembled nodal displacements with $\dot{\boldsymbol{\Delta}}$ and $\ddot{\boldsymbol{\Delta}}$ denoting their velocities and accelerations.

Our problem of interest is the dynamic mechanical response of an alveolus, whose geometry is modeled as a dodecahedron.  The shape of an irregular dodecahedron is described by a set of 20 vertices, each experiencing displacements of
\begin{subequations}
    \begin{align}
    \mathbf{u}^{(v)}_i & = \Bigl\{ \begin{matrix} u^{(v)}_i & v^{(v)}_i & w^{(v)}_i \end{matrix} \Bigr\}^{\mathsf{T}} \\
    \intertext{where at the beginning of a solution step $u^{(v)}_i \defeq x^{(v)}_i - x^{(v)}_0$, $v^{(v)}_i \defeq y^{(v)}_i - y^{(v)}_0$ and $w^{(v)}_i \defeq z^{(v)}_i - xz^{(v)}_0$, while at the end of that solution step $u^{(v)}_{i+1} = x^{(v)}_{i+1} - x^{(v)}_0$, $v^{(v)}_{i+1} = y^{(v)}_{i+1} - y^{(v)}_0$ and $w^{(v)}_{i+1} = z^{(v)}_{i+1} - xz^{(v)}_0$, with $(x^{(v)}, y^{(v)}, z^{(v)})$ denoting co-ordinates for vertex $v$ in the co-ordinate frame $( \vec{\mathbfsf{E}}_1 , \vec{\mathbfsf{E}}_2 , \vec{\mathbfsf{E}}_3 )$ of a dodecahedron, as established in \S\ref{reindexing3D}.  The velocities at these vertices are}
    \dot{\mathbf{u}}^{(v)}_i & = \Bigl\{ \begin{matrix} \dot{u}^{(v)}_i & \dot{v}^{(v)}_i & \dot{w}^{(v)}_i \end{matrix} \Bigr\}^{\mathsf{T}} \\
    \intertext{where at the beginning of a solution step $\dot{\delta}_i = \tfrac{1}{2 \mathrm{d}t} (\delta_{i+1} - \delta_{i-1} )$, while at the end of that solution step $\dot{\delta}_{i+1} = \frac{1}{2\mathrm{d}t} ( 3 \delta_{i+1} - 4 \delta_i + \delta_{i-1} )$, with $\delta \in \{ u^{(v)} , v^{(v)} , w^{(v)} \}$. Likewise, their accelerations are}
    \ddot{\mathbf{u}}^{(v)}_i & = \Bigl\{ \begin{matrix} \ddot{u}^{(v)}_i & \ddot{v}^{(v)}_i & \ddot{w}^{(v)}_i \end{matrix} \Bigr\}^{\mathsf{T}}
    \end{align}
\end{subequations}
where at the beginning of a solution step $\ddot{\delta}_i = \frac{1}{(\mathrm{d}t)^2} ( \delta_{i+1} - 2 \delta_i + \delta_{i-1} )$, while at the end of that solution step $\ddot{\delta}_{i+1} = \frac{1}{(\mathrm{d}t)^2} ( 2 \delta_{i+1} - 5 \delta_i + 4 \delta_{i-1} - \delta_{i-2} )$.  An evaluation of these nodal fields requires knowledge of the co-ordinates for each vertex at states $i+1$, $i$, $i-1$ and $i-2$.  All finite difference equations listed above are second-order formul\ae.

Symbol $\boldsymbol{\Delta}$ is used to denote an assemblage of all nodal displacements, while symbol $\mathbf{u}^{(v)}$ is used to denote the nodal displacement of an individual vertex (node) $v$ located within this model, of which there are twenty in our dodecahedral model.  These twenty vertices uniquely establish thirty alveolar chords, twelve alveolar membranes, and the alveolar sac enveloped by them, cf.\ \S\ref{sec:indexingDodecahedra}.

Our problem is not cast as a typical finite element solution, in the sense that we know the nodal displacements $\boldsymbol{\Delta}$, velocities $\dot{\boldsymbol{\Delta}}$ and accelerations $\ddot{\boldsymbol{\Delta}}$ \textit{a~priori}, for which nodal forces $\boldsymbol{F}$ are to be found.  Typically, boundary conditions are known for which displacements are determined in a weak sense, which is the opposite of our situation. Inputs for our model are considered to come from a finite element model of a torso subjected to an impact caused by either a ballistic projectile or a blast wave.  Say, for example, that a location of interest in the lung has been selected.  What will serve as input to our dodecahedral model will be a history of the deformation gradient sequenced in time and taken from that element containing this location of interest.  For example, if 8-noded brick elements are used, then the deformation gradient will be constant over its volume.  An element of size $1~\text{mm}^3$, which is at the resolution capability of current lung imaging technologies, would have one to two hundred alveoli in it.  Consequently, taking a statistical average over many dodecahedral model runs ought to provide a reasonable representation for the parenchymal response at that lung location.  Predictions coming from our dodecahedral model can be compared with predictions coming from a continuum model used to represent the parenchyma of lung in a torso model.  This information from our micro\-scopic alveolar model will allow for refinement and parameterization of a macro\-scopic continuum model for parenchyma.  This is particularly relevant because the spongy nature of lung tissue makes it extremely difficult to perform experiments that are suitable for parameterizing such models.

The assembled nodal forces $\boldsymbol{F}(\boldsymbol{T})$, discussed in \S\ref{secForceVectors}, depend upon stresses $\boldsymbol{T}$ evaluated at the Gauss points, as do the tangent and secant stiffness matrices, i.e., $\mathbf{C}(\boldsymbol{T})$ and $\mathbf{K}(\boldsymbol{T})$, which thereby couples the system of equations that are to be solved.  As such, an iterative solver is proposed.  The mass matrix $\mathbf{M}$ will vary between solution steps, too, but not because the mass matrix of a particular element changes, but rather, because rotations of the local co-ordinate systems for the elements about the global reference frame for the dodecahedron can become large, and as such, affect change in the assembled mass matrix.

The stress that arises from $\mathbf{K} \boldsymbol{\Delta}$ is due to an elastic deformation that begins in some reference state (at an initial time $t_0$) and ends at the current state (at present time $t_i$).  The stress that arises from $\mathbf{C} \dot{\boldsymbol{\Delta}}$ is due to an additional elastic deformation that begins in this current state (at time $t_i$) and ends at some nearby state (at a future time $t_{i+1} = t_i + \mathrm{d}t$).  While an inertial contribution to stress results from $\mathbf{M} \ddot{\boldsymbol{\Delta}}$.

The solution strategy adopted here mimics that of a predictor\slash corrector method used for solving ODEs.  At the beginning of a current solution step, the solution at the end of its previous step takes on the form of
\begin{subequations}
    \label{FEsolver}
    \begin{gather}
    \boldsymbol{F}_i = \mathbf{K}_i \boldsymbol{\Delta}_i + \mathbf{M}_i \ddot{\boldsymbol{\Delta}}_i 
    \label{FEsolverPrev}  \\
    \intertext{where $\boldsymbol{F}_i = \boldsymbol{F} ( \boldsymbol{T}_i )$ and  $\mathbf{K}_i = \mathbf{K} ( \boldsymbol{T}_i )$.  Recall that there is no damping in our model, so there is no $\dot{\boldsymbol{\Delta}}_i$ contribution entering here.  At the beginning of a step the stiffness response arises singularly from a secant modulus.  Meanwhile, the response at the end of the time step is considered to be described by a predictor of the form}
    \boldsymbol{F}^{\,p}_{i+1} =
    \mathbf{K}_i \boldsymbol{\Delta}_i + 
    \mathbf{C}_i \dot{\boldsymbol{\Delta}}_i +
    \mathbf{M}_i \ddot{\boldsymbol{\Delta}}_i \notag \\
    \intertext{where $\boldsymbol{F}^{\,p}_{i+1} = \boldsymbol{F} ( \boldsymbol{T}^{\,p}_{i+1} )$.  Subtracting Eqn.~(\ref{FEsolverPrev}) from the above equation produces}
    \boldsymbol{F}^{\,p}_{i+1} = \boldsymbol{F}_i + \mathbf{C}_i \dot{\boldsymbol{\Delta}}_i 
    \label{FEsolverPredictor} \\
    \intertext{where we observe that the tangent stiffness matrix $\mathbf{C}_i = \mathbf{C} ( \boldsymbol{T}_i )$ is used to extrapolate the known value for $\boldsymbol{F}_i$ into an estimate (prediction) for its next value, viz., $\boldsymbol{F}^{\,p}_{i+1}$.  This step is similar in concept to an ALE finite element formulation \cite{Belytschkoetal00}. At this point in the solution process, one evaluates the mass and secant stiffness matrices according to $\mathbf{M}_{i+1}$ and $\mathbf{K}_{i+1} = \mathbf{K} ( \boldsymbol{T}^{\,p}_{i+1})$ and then corrects the solution via}  \boldsymbol{F}_{i+1} = \mathbf{K}_{i+1} \boldsymbol{\Delta}_{i+1} + \mathbf{M}_{i+1} \ddot{\boldsymbol{\Delta}}_{i+1}
    \label{FEsolverCorrector}
    \end{gather}
\end{subequations}
where $\boldsymbol{F}_{i+1} = \boldsymbol{F} ( \boldsymbol{T}_{i+1} )$, while $\dot{\boldsymbol{\Delta}}_{i+1}$ and $\ddot{\boldsymbol{\Delta}}_{i+1}$ are approximated using backward difference formul\ae, as they are positioned at the end of the solution step.  A reevaluation of $\mathbf{K}_{i+1} = \mathbf{K} ( \boldsymbol{T}_{i+1} )$ now takes place, and Eqn.~(\ref{FEsolverCorrector}) is iterated on until convergence.  In preparation for advancing to the next solution step, one evaluates the tangent stiffness matrix $\mathbf{C}_{i+1} = \mathbf{C} ( \boldsymbol{T}_{i+1} )$.

Equation~(\ref{FEsolver}) is not self starting.  To start, because $\ddot{\boldsymbol{\Delta}}_0 = \boldsymbol{\Delta}_0 = \mathbf{0}$, it follows that 
\begin{subequations}
    \label{FEstartup}
    \begin{gather}
    \boldsymbol{F}^{\,p}_1 = \boldsymbol{F}_0 + 
    \mathbf{C}_0 \dot{\boldsymbol{\Delta}}_0
    \label{FEfirstPredictor} \\
    \intertext{where $\boldsymbol{F}_0 = \boldsymbol{F} ( \boldsymbol{T}_0 )$ denotes a residual force or prestress that must exist in biologic tissues, while $\mathbf{C}_0 = \mathbf{C} ( \boldsymbol{T}_0 )$ and $\boldsymbol{F}^{\,p}_1 = \boldsymbol{F} ( \boldsymbol{T}^{\,p}_1 )$.  Here $\dot{\boldsymbol{\Delta}}_0$ is to be approximated using an Euler forward step.  After evaluating $\mathbf{K}_1 = \mathbf{K} ( \boldsymbol{T}^{\,p}_1 )$, a correction is computed}
    \boldsymbol{F}_1 = \mathbf{K}_1 \boldsymbol{\Delta}_1
    \label{FEfirstCorrector} \\
    \intertext{where $\boldsymbol{F}_1 = \boldsymbol{F} ( \boldsymbol{T}_1 )$.  This allows for an improvement $\mathbf{K}_1 = \mathbf{K} ( \boldsymbol{T}_1 )$ that can be inserted back into the above equation, iterating until convergence.  Upon convergence, one determines the mass matrix $\mathbf{M}_1$ and the tangent stiffness matrix $\mathbf{C}_1 = \mathbf{C} ( \boldsymbol{T}_1 )$ in preparation for advancing to solution step~2.
    \bigskip\newline
    It is during the second solution interval whereat nodal accelerations can first be computed, so that with Eqn.~(\ref{FEfirstCorrector}) appling at the start of this interval, and with the following predictor considered to apply at the end of the interval}
    \boldsymbol{F}^{\,p}_2 = \mathbf{K}_1 \boldsymbol{\Delta}_1 +
    \mathbf{C}_1 \dot{\boldsymbol{\Delta}}_1 + 
    \mathbf{M}_1 \ddot{\boldsymbol{\Delta}}_1 
    \notag \\
    \intertext{then subtracting Eqn.~(\ref{FEfirstCorrector}) from this equation finds the solution advances via}
    \boldsymbol{F}^{\,p}_2 = \boldsymbol{F}_1 + 
    \mathbf{C}_1 \dot{\boldsymbol{\Delta}}_1 +
    \mathbf{M}_1 \ddot{\boldsymbol{\Delta}}_1 
    \label{FEsecondPredictor} \\
    \intertext{where $\boldsymbol{F}^{\,p}_2 = \boldsymbol{F} ( \boldsymbol{T}^{\,p}_2 )$.  At this point there is enough information to estimate the nodal accelerations $\ddot{\boldsymbol{\Delta}}_1$, as displacement data are available for $i+1=2$.  Both $\dot{\boldsymbol{\Delta}}_1$ and $\ddot{\boldsymbol{\Delta}}_1$ are approximated using central difference formul\ae.  Upon evaluating the mass matrix $\mathbf{M}_2$ and the secant stiffness matrix $\mathbf{K}_2 = \mathbf{K} ( \boldsymbol{T}^{\,p}_2 )$, a corrected solution at the end of the step is computed via}
    \boldsymbol{F}_2 = \mathbf{K}_2 \boldsymbol{\Delta}_2 + 
    \mathbf{M}_2 \ddot{\boldsymbol{\Delta}}_2 
    \label{FEsecondCorrector}
    \end{gather}
\end{subequations}
where $\ddot{\boldsymbol{\Delta}}_2 \leftarrow \ddot{\boldsymbol{\Delta}}_1$, because at this juncture there is not enough nodal displacement information to estimate acceleration at the end of this step, while $\dot{\boldsymbol{\Delta}}_2$ is approximated using a backward difference formula.  Equation~(\ref{FEsecondCorrector}) allows for an improvement for $\mathbf{K}_2 = \mathbf{K} ( \boldsymbol{T}_2 )$ that can be inserted back into itself, iterating until convergence.  Upon convergence, one determines the tangent stiffness matrix $\mathbf{C}_2 = \mathbf{C} ( \boldsymbol{T}_1 )$ in preparation for advancing to solution step~3. 

Equation~(\ref{FEsolver}) is used to advance all solution steps from the third step onward.

As a modeling simplification, the alveolar chords, the alveolar membranes, and the alveolar sac are each considered to be described by their own finite element model.  It is assumed that there is no coupling occurring between these three structural groups.  This is an important simplification, because the mass and compliance of these three geometric structures are vastly different, and as such, if not decoupled, these differences in mass and compliance would become problematic sources out of which numerical difficulties would likely arise.

From physics, we know that forces add, and therefore, the nodal forces coming from these three finite-element models can be summed.  These summed forces, positioned at the dodecahedral vertices, are obtained by taking stresses integrated at the various Guass points throughout each geometric model and extrapolating them out to their respective common nodes, i.e., the set of dodecahedral vertices, using the formul\ae\ of \S\ref{sec:Gauss}.  From these three sources for micro\-scopic alveolar force, one can construct a homo\-genized (averaged) state of macro\-scopic stress descriptive of the parenchymal response.

Consequently, we construct three, individual, finite-element models governed by the following three systems of differential equations
\begin{subequations}
    \label{EofMotion}
	\begin{align}
    \boldsymbol{F}_{\mathrm{1D}} & =
    \mathbf{K}_{\mathrm{1D}} \, \boldsymbol{\Delta} +
    \mathbf{C}_{\mathrm{1D}} \, \dot{\boldsymbol{\Delta}} +
    \mathbf{M}_{\mathrm{1D}} \, \ddot{\boldsymbol{\Delta}} \\
    \boldsymbol{F}_{\mathrm{2D}} & =
    \mathbf{K}_{\mathrm{2D}} \, \boldsymbol{\Delta} +
    \mathbf{C}_{\mathrm{2D}} \, \dot{\boldsymbol{\Delta}} + 
    \mathbf{M}_{\mathrm{2D}} \, \ddot{\boldsymbol{\Delta}} \\
    \boldsymbol{F}_{\mathrm{3D}} & = 
    \mathbf{K}_{\mathrm{3D}} \, \boldsymbol{\Delta} +
    \mathbf{C}_{\mathrm{3D}} \, \dot{\boldsymbol{\Delta}} + 
    \mathbf{M}_{\mathrm{3D}} \, \ddot{\boldsymbol{\Delta}}
	\end{align}
\end{subequations}
wherein the assembled nodal displacements $\boldsymbol{\Delta}$, velocities $\dot{\boldsymbol{\Delta}}$ and accelerations $\ddot{\boldsymbol{\Delta}}$ are common fields betwixt them, i.e., each problem is cast as a 3D analysis.  Subscript `$\mbox{}_{\mathrm{1D}}$' associates with alveolar chords that assemble into a 3D space truss, subscript `$\mbox{}_{\mathrm{2D}}$' associates with alveolar membranes that assemble into a 3D tiled balloon-like structure, and subscript `$\mbox{}_{\mathrm{3D}}$' associates with an alveolar sac.  Again, it is thought to be beneficial to split the overall problem space into these three sub-problems due to the vast differences in their structural mass and compliance.

When assembled, vectors $\boldsymbol{F}$, $\boldsymbol{\Delta}$, $\dot{\boldsymbol{\Delta}}$ and $\ddot{\boldsymbol{\Delta}}$ have lengths of 60 for the alveolar chord and alveolar membrane models, and a length of 63 for the alveolar sac model, while matrices $\mathbf{M}$, $\mathbf{C}$ and $\mathbf{K}$ have dimensions of $60 \! \times \! 60$ for the alveolar chord and alveolar membrane models, and dimensions of $63 \! \times \! 63$ for the alveolar sac model.  The model for alveolar volume has an extra node located at the centroid of the dodecahedron, i.e., the co-ordinate origin, which is a node in common betwixt the 60 tetrahedra used to fill the volume of a dodecahedron in this alveolar model. 

The objective of this chapter is to derive the elemental mass matrix, the two stiffness matrices (secant and tangent), plus the forcing functions and necessary boundary conditions, and to then assemble them for analysis. 

\section{Mass Matrices}

The consistent mass matrix of an element, \cite{Archer65} when quantified in the element's co-ordinate frame $( \vec{\mathbfsf{e}}_1 , \vec{\mathbfsf{e}}_2 , \vec{\mathbfsf{e}}_3 )$, is defined as follows:  For 1D elements
\begin{subequations}
    \label{consistentMassMatrix}
    \begin{align}
    \mathbf{M}_{\mathrm{C1D}} & = \int_{L} \rho \, \mathbf{N}^{\mathsf{T}} \mathbf{N} \,
    A \, \mathrm{d} L & M^{\mathrm{C1D}}_{ij} & = 
    \int_L \rho \, N_{1i} N_{1j} \, A \, \mathrm{d} L
    \label{consistentMassMatrix1D} \\
    \intertext{with $i,j = 1, 2, \ldots , n$ where $n$ is the number of nodal points. For 2D elements}
    \mathbf{M}_{\mathrm{C2D}} & = \int_{A} \rho \, \mathbf{N}^{\mathsf{T}} \mathbf{N} \,
    h \, \mathrm{d} A & M^{\mathrm{C2D}}_{ij} & =
    \int_A \rho \sum_{k=1}^2 N_{ki} N_{kj} \, h \, \mathrm{d} A
    \label{consistentMassMatrix2D} \\
    \intertext{with $i,j = 1 , 2, \ldots , 2n$ where $n$ is the number of nodal points.  And for 3D elements}
    \mathbf{M}_{\mathrm{C3D}} & = \int_{V} \rho \, \mathbf{N}^{\mathsf{T}} \mathbf{N} \,
    \mathrm{d} V & M^{\mathrm{C3D}}_{ij} & = 
    \int_V \rho \sum_{k=1}^3 N_{ki} N_{kj} \, \mathrm{d} V
    \label{consistentMassMatrix3D}
    \end{align}
\end{subequations}
with $i, j = 1, 2, \ldots , 3n$ where $n$ is the number of nodal points.  For a rod, $\mathbf{M}_{\mathrm{C1D}}$ is a $2 \! \times \! 2$ matrix; for a pentagon, $\mathbf{M}_{\mathrm{C2D}}$ is a $10 \! \times \! 10$ matrix; and for a tetrahedron, $\mathbf{M}_{\mathrm{C3D}}$ is a $12 \! \times \! 12$ matrix.  In each expression, $\rho$ is mass per unit volume, $\mathbf{N}$ is a matrix of shape functions for the element of interest, $L$ is length and $h$ is height, $A$ is area, and $V$ is volume.  

These mass matrices are said to be consistent in that they are calculated with the same shape functions that are used to create their stiffness matrices.  Consistent mass matrices are symmetric, because $\mathbf{N}^{\mathsf{T}} \mathbf{N}$ is symmetric.  Unfortunately, for all of the elements that we employ, matrices $\mathbf{N}^{\mathsf{T}} \mathbf{N}$ are singular, which is not a desirable feature.

One form of a lumped mass matrix is where the entries from each row of a consistent mass matrix are summed and placed in their respective diagonal entries; specifically: \cite{Reddy93} For 1D elements
\begin{subequations}
    \label{lumpedMassMatrix}
    \begin{align}
    M^{\mathrm{L1D}}_{ii} & = \sum_{j=1}^n M^{\mathrm{C1D}}_{ij} = \int_L \rho N_{1i} \sum_{j=1}^n N_{1j} \, A \, \mathrm{d}L , & 
    M^{\mathrm{L1D}}_{ij} & = 0 \quad i \neq j  
    \label{lumpedMassMatrix1D} \\
    \intertext{with $i = 1, 2, \ldots , n$ where $n$ is the number of nodal points. For 2D elements}
    M^{\mathrm{L2D}}_{ii} & = \sum_{j=1}^{2n} M^{\mathrm{C2D}}_{ij} = \int_A \rho \sum_{k=1}^2 N_{ki} \sum_{j=1}^{2n} N_{kj} \, h \, \mathrm{d} A , & 
    M^{\mathrm{L2D}}_{ij} & = 0 \quad i \neq j  
    \label{lumpedMassMatrix2D} \\
    \intertext{with $i = 1 , 2, \ldots , 2n$ where $n$ is the number of nodal points. And for 3D elements}
    M^{\mathrm{L3D}}_{ii} & = \sum_{j=1}^{3n} M^{\mathrm{C3D}}_{ij} = \int_V \rho \sum_{k=1}^3 N_{ki} \sum_{j=1}^{3n} N_{kj} \, \mathrm{d} V , & 
    M^{\mathrm{L3D}}_{ij} & = 0 \quad i \neq j  
    \label{lumpedMassMatrix3D}
    \end{align}
\end{subequations}
with $i = 1, 2, \ldots , 3n$ where $n$ is the number of nodal points. 

A lumped-consistent (or weighted) mass matrix $\mathbf{M}_{\mathrm{W}}$ can then be created as follows
\begin{equation*}
\mathbf{M}_\mathrm{W}  = (1 - \mu) \, \mathbf{M}_{\mathrm{C}} + \mu \, \mathbf{M}_{\mathrm{L}}
\end{equation*}
wherein $\mu$ is a free scalar parameter for weighting between the consistent and lumped mass matrices.  The reason for mixing $\mathbf{M}_{\mathrm{C}}$ and $\mathbf{M}_{\mathrm{L}}$ is to achieve a non-singular mass matrix by making the resulting matrix diagonally dominant.  In this work, $\mu$ is taken to be a half, i.e., an averaged mass matrix is adopted, which has a nice property of minimizing low frequency dispersion.  Specifically, we select
\begin{subequations}
    \label{LumconsMass}
    \begin{align}
    \mathbf{M}_{\mathrm{1D}} & \defeq \tfrac{1}{2} (\mathbf{M}_{\mathrm{C1D}} + \mathbf{M}_{\mathrm{L1D}}) \\
    \mathbf{M}_{\mathrm{2D}} & \defeq \tfrac{1}{2} (\mathbf{M}_{\mathrm{C2D}} + \mathbf{M}_{\mathrm{L2D}}) \\
    \mathbf{M}_{\mathrm{3D}} & \defeq \tfrac{1}{2} (\mathbf{M}_{\mathrm{C3D}} + \mathbf{M}_{\mathrm{L3D}})
    \end{align}
\end{subequations}
as our means for constructing mass matrices.  Each of these mass matrices is invertible that, for example, is a requirement of the numerical solution strategy presented in \S\ref{sec:solve2ndOrderODE}.

\subsection{Mass Matrix for a Chord}

A two-noded alveolar chord (a pinned beam in finite element terminology) has shape functions $N_i$ that aggregate into a $1 \! \times \! 2$ matrix of shape functions when evaluated in their natural co-ordinate system wherein $-1 \leq \xi \leq 1$, viz.,
\begin{subequations}
    \begin{align}
    \mathbf{N} (\xi) & = \bigl[ \begin{matrix} N_1 & N_2 \end{matrix} \bigr] =
    \bigl[ \begin{matrix}
    \frac{1}{2} \, (1 - \xi) &  \frac{1}{2} \, (1 + \xi)
    \end{matrix} \bigr] \\
    \intertext{whose constituents have gradients}
    N_{1,\xi} & = -\textfrac{1}{2} 
    \quad \text{and} \quad
    N_{2,\xi} = \textfrac{1}{2} \\
    \intertext{from which a symmetric matrix arises to become the backbone for this mass matrix (which happens to be singular), its components being}
    \mathbf{N}^{\mathsf{T}} \mathbf{N} (\xi_i) & = \frac{1}{4} \begin{bmatrix}
    1 - 2\xi_i + \xi_i^2 & 1 - \xi_i^2 \\
    1 - \xi_i^2 & 1 + 2 \xi_i + \xi_i^2    
    \end{bmatrix} 
    \end{align}
\end{subequations}
where $\xi_i$ designates a co-ordinate for the $i^{\mathrm{th}}$ Gauss point associated with a specific Gauss quadrature rule for integration, which in our case comes from Table~\ref{tab:2nodeRod}.

The determinant $| \mathbf{J} |$ of Jacobian matrix $\mathbf{J}$ is used to transform the integrals in Eqns.~(\ref{consistentMassMatrix} \& \ref{lumpedMassMatrix}) from their natural co-ordinates into the co-ordinate system $( \vec{\mathbfsf{e}}_1 , \vec{\mathbfsf{e}}_2 , \vec{\mathbfsf{e}}_3 )$ of a chord, cf.\ Fig.~\ref{figchord}.  Its value is
\begin{equation}
\mathbf{J} \equiv | \mathbf{J} | = \sum\nolimits_{i=1}^2 N_{i,\xi} (\xi) \, x_i = 
-\tfrac{1}{2} \cdot -\tfrac{1}{2} L + \tfrac{1}{2} \cdot \tfrac{1}{2} L = 
\tfrac{1}{2} L
\label{detJac1D}
\end{equation}
given nodal co-ordinates of $x_1 = -\tfrac{1}{2} L$ and $x_2 = \tfrac{1}{2} L$, where $L$ is the length of our alveolar chord.  The Jacobian matrix $\mathbf{J}$ and its determinant $| \mathbf{J} |$ are equivalent in the case of a rod, because this geometric space is one dimensional. 

The consistent mass matrix for a 1D alveolar chord modeled as a two-noded rod, when evaluated in the co-ordinate system of the chord, becomes
\begin{equation} 
\begin{aligned}
\mathbf{M}_{\mathrm{C1D}} & = \int_0^L \rho \, \mathbf{N}^{\mathsf{T}} \mathbf{N} \, A \, \mathrm{d} x  = \int_{-1}^{1} \rho \, \mathbf{N}^{\mathsf{T}} \mathbf{N}\, A \, | \mathbf{J} | \,  \mathrm{d} \xi \\ & 
= | \mathbf{J} | \sum_{i=1}^m  \rho_i A_i  \, \mathbf{N} ^{\mathsf{T}} \mathbf{N}(\xi_i) \, w_i \\ &
= \frac{L}{2} \sum_{i=1}^m \frac{\rho_i A_i w_i}{4} \begin{bmatrix}
1 - 2\xi_i + \xi_i^2 & 1 - \xi_i^2 \\
1 - \xi_i^2 & 1 + 2 \xi_i + \xi_i^2
\end{bmatrix}
\end{aligned}
\label{ConsMassMatrix1DA}
\end{equation}
where $\textbf{N} (\xi_i)$ is a matrix of shape functions evaluated at a node of quadrature $\xi_i$ whose associated weight of quadrature is $w_i$, both evaluated at Gauss point $i$ for a selected Gauss integration rule comprised of $m$ Gauss points.  Table~\ref{tab:2nodeRod} presents values for the co-ordinates $\xi_i$ and weights $w_i$ of quadrature where two Gauss points of integration ($m=2$) are employed for integrating over a length of chord.  

A lumped mass matrix for a 1D alveolar chord, when evaluated in the co-ordinate system of a chord is 
\begin{equation}
    \begin{aligned}
    \mathbf{M}_{\mathrm{L1D}} & = \sum_{\text{rows}} \frac{L}{2} 
    \sum_{i=1}^m \frac{\rho_i A_i w_i}{4} \begin{bmatrix}
    1 - 2\xi_i + \xi_i^2 & 1 - \xi_i^2 \\
    1 - \xi_i^2 & 1 + 2 \xi_i + \xi_i^2
    \end{bmatrix} \\
    & = \frac{L}{2} \sum_{i=1}^m \frac{\rho_i A_i w_i}{2} \begin{bmatrix} 
    1 - \xi_i & 0 \\ 0 & 1 + \xi_i \end{bmatrix} .
    \end{aligned}
    \label{LumconsMassMatrix1DA}
\end{equation}
It is seen that the mass matrix in Eqn.~(\ref{ConsMassMatrix1DA}) is singular at any given Gauss point, whereas the mass matrix in Eqn.~(\ref{LumconsMassMatrix1DA}) has a reciprocal, except whenever $\xi = \pm 1$, which are points not realized in Gaussian quadrature rules. 

A chordal mass matrix that is appropriate for biologic fibers, and that associates with the Gauss quadrature rule listed in Table~\ref{tab:2nodeRod}, has a consistent mass matrix of
\begin{subequations}
    \label{massMatrices1D}
    \begin{align}
    \mathbf{M}_{\mathrm{C1D}} & = \frac{\rho_1 A_1 L}{12} 
    \begin{bmatrix} 2 + \sqrt{3} & 1 \\ 
    1 & 2 - \sqrt{3} \end{bmatrix} + \frac{\rho_2 A_2 L}{12} 
    \begin{bmatrix} 2 - \sqrt{3} & 1 \\ 
    1 & 2 + \sqrt{3} \end{bmatrix} \\
    \intertext{and a lumped mass matrix of}
    \mathbf{M}_{\mathrm{L1D}} & = \frac{\rho_1 A_1 L}{12} 
    \begin{bmatrix} 3 + \sqrt{3} & 0 \\
    0 & 3 - \sqrt{3} \end{bmatrix} + \frac{\rho_2 A_2 L}{12} 
    \begin{bmatrix} 3 - \sqrt{3} & 0 \\ 
    0 & 3 + \sqrt{3} \end{bmatrix} \\
    \intertext{that when averaged become}
    \mathbf{M}_{\mathrm{1D}} & = \frac{\rho_1 A_1 L}{24} 
    \begin{bmatrix} 5 + 2\sqrt{3} & 1 \\ 1 & 5 - 2\sqrt{3} \end{bmatrix} +
    \frac{\rho_2 A_2 L}{24} 
    \begin{bmatrix} 5 - 2\sqrt{3} & 1 \\ 1 & 5 + 2\sqrt{3} \end{bmatrix}
    \end{align}
\end{subequations}
with $\mathbf{M}_{\mathrm{1D}}$ being the 1D mass matrix that we implement.  Because the mass of an alveolar chord does not change when exposed to a traveling shock wave, it follows that $\rho A L = \rho_0 A_0 L_0$, and as such, this mass matrix only needs to be evaluated once. 

In contrast with engineered structures, like steel trusses, where bars have uniform mass density over their lengths, and typically have uniform cross-sectional areas, too, biologic fibers, like alveolar chords, have mass densities $\rho_i$ and cross-sectional areas $A_i$ that vary along their lengths, and hence, they remain properties of the Gauss points and cannot be pulled out in front of the summation, as is usually done.  

The finer details of constructing a mass matrix have been presented above, because the level of complexity is relatively small in the case of a chord.  Much of this finer detail is omitted in the following presentations, since the complexity in these cases is more substantial.

\subsubsection{Assembly of Chordal Mass Matrices}

In our alevolar model comprised of septal chords, there are 20 nodes (vertices) whose numbering scheme and natural co-ordinates (those of a regular dodecahedron) are specified in Table~\ref{TableDodecahedron}.  Connecting these 20 nodes are 30 line segments (septal chords) whose numbering scheme and associated nodal numbers are specified in Table~\ref{Tablechordae}.

In 3D analyses, the components $M^{\mathrm{1D}}_{ij}$ of mass matrix $\mathbf{M}_{\mathrm{1D}}$ from Eqn.~\eqref{massMatrices1D} populate a mass matrix $\boldsymbol{M}_{\mathrm{1D}}^{(e)}$ for element $e$, $e \in \{ 1,2,\ldots,30 \}$, accordingly
\begin{equation}
    \boldsymbol{M}_{\mathrm{1D}}^{(e)} = 
    \left[ \begin{array}{ccc|ccc}
    M_{11}^{\mathrm{1D}} & 0 & 0 & M_{12}^{\mathrm{1D}} & 0 & 0 \\
    0 & 0 & 0 & 0 & 0 & 0 \\
    0 & 0 & 0 & 0 & 0 & 0 \\ \hline
    M_{21}^{\mathrm{1D}} & 0 & 0 & M_{22}^{\mathrm{1D}} & 0 & 0 \\
    0 & 0 & 0 & 0 & 0 & 0 \\
    0 & 0 & 0 & 0 & 0 & 0
    \end{array} \right]
    \label{elementMassMatrix1D}
\end{equation}
so that, e.g., given the formula $\boldsymbol{f} = \mathbf{M} \ddot{\boldsymbol{u}}$, one would have
\begin{displaymath}
    \left\{ \begin{matrix}
    f^{(i)}_1 \\ f^{(i)}_2 \\ f^{(i)}_3 \\ \hline
    f^{(j)}_1 \\ f^{(j)}_2 \\ f^{(j)}_3
    \end{matrix} \right\} = 
    \left[ \begin{array}{ccc|ccc}
    M_{11}^{\mathrm{1D}} & 0 & 0 & M_{12}^{\mathrm{1D}} & 0 & 0 \\
    0 & 0 & 0 & 0 & 0 & 0 \\
    0 & 0 & 0 & 0 & 0 & 0 \\ \hline
    M_{21}^{\mathrm{1D}} & 0 & 0 & M_{22}^{\mathrm{1D}} & 0 & 0 \\
    0 & 0 & 0 & 0 & 0 & 0 \\
    0 & 0 & 0 & 0 & 0 & 0
    \end{array} \right] \left\{ \begin{matrix} 
    \ddot{u}^{(i)}_1 \\ \ddot{u}^{(i)}_2 \\ \ddot{u}^{(i)}_3 \\ \hline
    \ddot{u}^{(j)}_1 \\ \ddot{u}^{(j)}_2 \\ \ddot{u}^{(j)}_3
    \end{matrix} \right\} 
\end{displaymath}
where element $e$ has nodes $i$ and $j$, with all components being evaluated in the co-ordinate system $( \vec{\mathbfsf{e}}_1 , \vec{\mathbfsf{e}}_2 , \vec{\mathbfsf{e}}_3 )^{(e)}$.  Only an axial force is carried by the chord.  No transverse forces are present.  Also, no moments of inertia have been introduced.  Like mass matrix $\mathbf{M}_{\mathrm{1D}}$, Mass matrix $\boldsymbol{M}_{\mathrm{1D}}^{(e)}$ is constant, and therefore only needs to be evaluated once.

To rotate this mass matrix for an element from its co-ordinate system for a chord $( \vec{\mathbfsf{e}}_1 , \vec{\mathbfsf{e}}_2 , \vec{\mathbfsf{e}}_3 )^{(e)}$ into the fixed co-ordinate system for a dodecahedron $( \vec{\mathbfsf{E}}_1 , \vec{\mathbfsf{E}}_2 , \vec{\mathbfsf{E}}_3 )$, where it can be assembled with the mass matrices from the other 29 chordal elements, one must first apply the orthogonal transformation
\begin{equation}
    \mathbf{R}^{(e)}_{\mathrm{1D}} = \left[ \begin{array}{ccc|ccc}
    Q^{(e)}_{11} & Q^{(e)}_{12} & Q^{(e)}_{13} & 0 & 0 & 0 \\
    Q^{(e)}_{21} & Q^{(e)}_{22} & Q^{(e)}_{23} & 0 & 0 & 0 \\
    Q^{(e)}_{31} & Q^{(e)}_{32} & Q^{(e)}_{33} & 0 & 0 & 0 \\ \hline
    0 & 0 & 0 & Q^{(e)}_{11} & Q^{(e)}_{12} & Q^{(e)}_{13} \\
    0 & 0 & 0 & Q^{(e)}_{21} & Q^{(e)}_{22} & Q^{(e)}_{23} \\
    0 & 0 & 0 & Q^{(e)}_{31} & Q^{(e)}_{32} & Q^{(e)}_{33} 
    \end{array} \right]
\end{equation}
so that, accordingly,
\begin{equation}
    \mathbf{M}^{(e)}_{\mathrm{1D}} = \bigl( \mathbf{R}^{(e)}_{\mathrm{1D}} \bigr)^{\mathsf{T}} 
    \boldsymbol{M}^{(e)}_{\mathrm{1D}} \mathbf{R}^{(e)}_{\mathrm{1D}}
\end{equation}
where $\mathbf{M}^{(e)}_{\mathrm{1D}}$ becomes this mass matrix, transformed into a dodecahedral co-ordinate system $( \vec{\mathbfsf{E}}_1 , \vec{\mathbfsf{E}}_2 , \vec{\mathbfsf{E}}_3 )$ with $[ \{ \vec{\mathbfsf{e}}_1 \} \{ \vec{\mathbfsf{e}}_2 \} \{ \vec{\mathbfsf{e}}_3 \} ]^{(e)} = \bigl[ \{ \vec{\mathbfsf{E}}_1 \} \{ \vec{\mathbfsf{E}}_2 \} \{ \vec{\mathbfsf{E}}_3 \} \bigr] \bigl[ \mathbf{Q}^{(e)} ]$, cf.\ Fig.~\ref{figchord}.  Even though $\boldsymbol{M}^{(e)}_{\mathrm{1D}}$ is a constant mass matrix, $\mathbf{M}^{(e)}_{\mathrm{1D}}$ need not be, because $\mathbf{R}^{(e)}_{\mathrm{1D}}$ will typically vary over time in our analysis of alveoli subjected to shock waves.

We can now re-write our example equation for $\boldsymbol{f} = \mathbf{M} \ddot{\boldsymbol{u}}$ as a block matrix equation
\begin{displaymath}
\left\{ \begin{matrix} 
\boldsymbol{f}_i \\ \boldsymbol{f}_j
\end{matrix} \right\} = 
\begin{bmatrix}
\mathbf{M}^{(e)}_{\mathrm{1D}:ii} & \mathbf{M}^{(e)}_{\mathrm{1D}:ij} \\
\mathbf{M}^{(e)}_{\mathrm{1D}:ji} & \mathbf{M}^{(e)}_{\mathrm{1D}:jj}
\end{bmatrix} \left\{ \begin{matrix} 
\ddot{\boldsymbol{u}}_i \\ \ddot{\boldsymbol{u}}_j
\end{matrix} \right\}
\end{displaymath} 
wherein $\boldsymbol{f}_i = f^{(i)}_1 \vec{\mathbfsf{E}}_1 + f^{(i)}_2 \vec{\mathbfsf{E}}_2 + f^{(i)}_3 \vec{\mathbfsf{E}}_3$, etc., where $i$ and $j$ are the nodal numbers for the two nodes that establish this chord.

From this example, following standard procedures, \cite{ClaytonChung18} it becomes apparent how to assemble the overall mass matrix of our space truss when using the geometry of a dodecahedron to model an alveolus.  This mass matrix will be a $20 \! \times \! 20$ block matrix, with each block element being a $3 \! \times \! 3$ matrix.  Entries placed into this block matrix for each truss element are positioned into this matrix according to the element numbering scheme presented in Table~\ref{Tablechordae}.

\subsection{Mass Matrix for a Triangle}

Even though we need to know the shape functions and quadrature rule for triangles in our analysis (they are used to extrapolate Gauss point forces out to the vertices of a dodecahedron in our analysis of alveolar sacs), we do not need to construct their mass matrix for our specific application.

\subsection{Mass Matrix for a Pentagon}

The surface of a dodecahedron is tiled with 12 pentagons, and as such, an analysis to establish a mass matrix for a pentagon becomes the building block needed to be able to assemble a 2D mass matrix $\mathbf{M}_{\mathrm{2D}}$ representing the alveolar membranes that envelope an alveolar sac.

For an alveolar membrane, represented here as an irregular pentagon, the matrix of shape functions $\mathbf{N}(\xi,\eta)$ takes on the general form of
\begin{equation}
\mathbf{N} = 
\begin{bmatrix}
N_1 & 0 & N_2 & 0 & N_3 & 0 & N_4 & 0 & N_5 & 0 \\ 
0 & N_1 & 0 & N_2 & 0 & N_3 & 0 & N_4 & 0 & N_5 
\end{bmatrix} 
\label{shape2D}
\end{equation}
whose product $\mathbf{N}^{\mathsf{T}}\mathbf{N}(\xi,\eta)$ is a symmetric singular matrix with a banded structure
\begin{multline}
     \mathbf{N}^{\mathsf{T}} \mathbf{N} = \\ \footnotesize \begin{bmatrix}
     N_1^2 & 0 & N_1 N_2 & 0 & N_1 N_3 & 0 & N_1 N_4 & 0 & N_1 N_5 & 0 \\
     0 & N_1^2 & 0 & N_1 N_2 & 0 & N_1 N_3 & 0 & N_1 N_4 & 0 & N_1 N_5 \\
     N_1 N_2 & 0 & N_2^2 & 0 & N_2 N_3 & 0 & N_2 N_4 & 0 & N_2 N_5 & 0 \\
     0 & N_1 N_2 & 0 & N_2^2 & 0 & N_2 N_3 & 0 & N_2 N_4 & 0 & N_2 N_5 \\
     N_1 N_3 & 0 & N_2 N_3 & 0 & N_3^2 & 0 & N_3 N_4 & 0 & N_3 N_5 & 0 \\
     0 & N_1 N_3 & 0 & N_2 N_3 & 0 & N_3^2 & 0 & N_3 N_4 & 0 & N_3 N_5 \\
     N_1 N_4 & 0 & N_2 N_4 & 0 & N_3 N_4 & 0 & N_4^2 & 0 & N_4 N_5 & 0 \\
     0 & N_1 N_4 & 0 & N_2 N_4 & 0 & N_3 N_4 & 0 & N_4^2 & 0 & N_4 N_5 \\
     N_1 N_5 & 0 & N_2 N_5 & 0 & N_3 N_5 & 0 & N_4 N_5 & 0 & N_5^2 & 0 \\
     0 & N_1 N_5 & 0 & N_2 N_5 & 0 & N_3 N_5 & 0 & N_4 N_5 & 0 & N_5^2
     \end{bmatrix}
\end{multline}
\normalsize
that when summed along rows produces a diagonal matrix with elements
\begin{equation}
    \sum_{\mathrm{rows}} \mathbf{N}^{\mathsf{T}} \mathbf{N} = 
    \mathrm{diag} \bigl( \begin{matrix}
    N_1 & N_1 & N_2 & N_2 & N_3 & N_3 & N_4 & N_4 & N_5 & N_5
    \end{matrix} \bigr)
\end{equation}
wherein $\mathrm{N}_i$, $i = 1, \ldots, 5$, are the five shape functions that correspond with the five vertices of a pentagon, as established in Eqn.~(\ref{shapeFunctions}).  These shape functions are nonlinear functions of their co-ordinates $(\xi,\eta)$, which is readily apparent in Fig.~\ref{figShapeFuntion}.

A consistent mass matrix $\mathbf{M}_{\mathrm{C2D}}$ is constructed by substituting the above matrix of shape functions into the following expression
\begin{equation}
    \mathbf{M}_{\mathrm{C2D}} = \iint_{\pentagon} \rho \, \mathbf{N}^{\mathsf{T}} \mathbf{N} \,|\mathbf{J}| \, h \, \mathrm{d} \xi \, \mathrm{d} \eta 
    = | \mathbf{J} | \sum_{i=1}^m \rho_i h_i \, \mathbf{N}^{\mathsf{T}} \mathbf{N} 
    (\xi_i , \eta_i) \, w_i
    \label{massintegral2d}
\end{equation}
where $m$ is the number of Gauss points with $( \xi_i , \eta_i )$ and $w_i$ being their respective co-ordinates and weights of quadrature that, in our implementation, are provided by Eqns.~(\ref{pentagonCoordinates} \&\ \ref{pentagonQuadrature}).  As with alveolar chords, alveolar membranes have mass densities $\rho_i$ and heights $h_i$ (thicknesses) that are not uniform across a membrane; hence, they are treated as properties of the Gauss point, and as such, are not pulled out in front of the above summation, as is typically done.

Here $|\mathbf{J}|$ is the Jacobian determinant of a $2 \! \times \! 2$ Jacobian matrix $\mathbf{J}$.  In areal derivations, the Jacobian of a two-dimensional transformation connects the physical ${x, y}$ to the natural ${\xi, \eta}$ co-ordinate systems involved.  Components of this Jacobian matrix are calculated using derivatives of shape functions taken with respect to the local co-ordinates via \cite[pg.~424]{Reddy93}
\begin{equation}
\mathbf{J} = 
\begin{bmatrix}
\partial x / \partial\xi & \partial y / \partial\xi \\
\partial x / \partial\eta & \partial y / \partial\eta 
\end{bmatrix}  
= \begin{bmatrix}
\sum\nolimits_{i=1}^5 N_{i,\xi} (\xi,\eta) \, x_i & \sum\nolimits_{i=1}^5 N_{i,\xi} (\xi,\eta) \, y_i \\
\sum\nolimits_{i=1}^5 N_{i,\eta} (\xi,\eta) \, x_i & \sum\nolimits_{i=1}^5 N_{i,\eta} (\xi,\eta) \, y_i
\end{bmatrix}
\label{jacobian2D}
\end{equation}
where the shape function gradients $N_{i,\xi}$ and $N_{i,\eta}$ are provided by Eqn.~(\ref{shapeFunctionGradients}), with
\begin{equation}
| \mathbf{J} | = \frac{\partial x}{\partial \xi} \frac{\partial y}{\partial \eta} - 
\frac{\partial x}{\partial \eta} \frac{\partial y}{\partial \xi}
\label{jacobianpent}
\end{equation}
establishing the Jacobian determinant.  It is proportional to the area of the pentagon $A_{\pentagon}$ because $A_{\pentagon} = \iint_{\pentagon} \mathrm{d}x \, \mathrm{d}y = \iint_{\pentagon} | \mathbf{J} | \, \mathrm{d}\xi \, \mathrm{d}\eta = | \mathbf{J} | \sum_{i=1}^5 w_i = 2.378 | \mathbf{J} |$ using the quadrature rule for pentagons given in Eqn.~\eqref{pentagonQuadrature}, cf.\ Eqn.~\eqref{regPentagonArea}.

From the consistent mass matrix of Eqn.~(\ref{massintegral2d}), its associated lumped mass matrix is readily computed via Eqn.~(\ref{lumpedMassMatrix2D}) that when averaged with Eqn.~(\ref{massintegral2d}) results in the 2D mass matrix $\mathbf{M}_{\mathrm{2D}}$ that we implement.

\subsubsection{Assembly of Pentagonal Mass Matrices}

In our alevolar model comprised of septal membranes, there are 20 common nodes (vertices) whose numbering scheme and natural co-ordinates (those of a regular dodecahedron) are specified in Table~\ref{TableDodecahedron}.  Subsets of these 20 nodes allow for the construction of 12 pentagons (septal membranes) whose numbering scheme and associated nodal numbers are specified in Table~\ref{TablePentagons}.

In the co-ordinate system of a pentagon $( \vec{\mathbfsf{e}}_1 , \vec{\mathbfsf{e}}_2 , \vec{\mathbfsf{e}}_3  )^{(e)}$, $e \in \{ 1, 2, \ldots, 12 \}$, a pentagon has a mass matrix with a symmetric block structure of
\begin{subequations}
    \begin{align}
    \boldsymbol{M}^{(e)}_{\mathrm{2D}} & = 
    \begin{bmatrix}
    \boldsymbol{M}^{\mathrm{2D}}_{11} & \boldsymbol{M}^{\mathrm{2D}}_{12} & 
    \boldsymbol{M}^{\mathrm{2D}}_{13} & \boldsymbol{M}^{\mathrm{2D}}_{14} & 
    \boldsymbol{M}^{\mathrm{2D}}_{15} \\
    \boldsymbol{M}^{\mathrm{2D}}_{21} & \boldsymbol{M}^{\mathrm{2D}}_{22} & 
    \boldsymbol{M}^{\mathrm{2D}}_{23} & \boldsymbol{M}^{\mathrm{2D}}_{24} & 
    \boldsymbol{M}^{\mathrm{2D}}_{25} \\
    \boldsymbol{M}^{\mathrm{2D}}_{31} & \boldsymbol{M}^{\mathrm{2D}}_{32} & 
    \boldsymbol{M}^{\mathrm{2D}}_{33} & \boldsymbol{M}^{\mathrm{2D}}_{34} & 
    \boldsymbol{M}^{\mathrm{2D}}_{35} \\
    \boldsymbol{M}^{\mathrm{2D}}_{41} & \boldsymbol{M}^{\mathrm{2D}}_{42} & 
    \boldsymbol{M}^{\mathrm{2D}}_{43} & \boldsymbol{M}^{\mathrm{2D}}_{44} & 
    \boldsymbol{M}^{\mathrm{2D}}_{45} \\ 
    \boldsymbol{M}^{\mathrm{2D}}_{51} & \boldsymbol{M}^{\mathrm{2D}}_{52} & 
    \boldsymbol{M}^{\mathrm{2D}}_{53} & \boldsymbol{M}^{\mathrm{2D}}_{54} & 
    \boldsymbol{M}^{\mathrm{2D}}_{55}
    \end{bmatrix} \\
    \intertext{with each element in this matrix being a $3 \! \times \! 3$ matrix with diagonal entries of}
    \boldsymbol{M}^{\mathrm{2D}}_{ij} & = 
    \begin{bmatrix}
    M^{\mathrm{2D}}_{ij} & 0 & 0 \\
    0 & M^{\mathrm{2D}}_{ij} & 0 \\
    0 & 0 & 0
    \end{bmatrix}
    \end{align}
\end{subequations}
whose components have values of
\begin{subequations}
    \begin{align}
    M_{ii}^{\mathrm{2D}} & = 
    \frac{| \mathbf{J}_0 |}{2} \sum_{k=1}^5 \rho_{0k} h_{0k} N_i ( 1 + 
    N_i ) ( \xi_k , \eta_k ) \, w_k & i & = 1,2,\ldots,5 \\
    M_{ij}^{\mathrm{2D}} = M_{ji}^{\mathrm{2D}} & =
    \frac{| \mathbf{J}_0 |}{2} \sum_{k=1}^5 \rho_{0k} h_{0k}
    N_i N_j (\xi_k , \eta_k ) \, w_k & i & \neq j
    \end{align}
\end{subequations}
with co-ordinates $( \xi_i , \eta_i )$ and weights $w_i$ of quadrature being given in Eqns.~(\ref{pentagonCoordinates} \&\ \ref{pentagonQuadrature}), and whose shape functions are defined according to Eqn.~\eqref{shapeFunctions}.  Because the mass of an alveolar membrane is conserved when exposed to a traveling shock wave, it follows that $\rho h | \mathbf{J} | = \rho_0 h_0 | \mathbf{J}_0 |$, and as such, like the mass matrices $\boldsymbol{M}^{(e)}_{\mathrm{1D}}$ and $\boldsymbol{M}^{(e)}_{\mathrm{3D}}$ for chords and tetrahedra, the mass matrix $\boldsymbol{M}^{(e)}_{\mathrm{2D}}$ for each pentagon only needs to be evaluated once. 

To rotate this mass matrix for element $e$, $e \in \{ 1, 2, \ldots, 12 \}$, from its elemental co-ordinate system for the pentagon $( \vec{\mathbfsf{e}}_1 , \vec{\mathbfsf{e}}_2 , \vec{\mathbfsf{e}}_3 )^{(e)}$ into a fixed co-ordinate system for the dodecahedron $( \vec{\mathbfsf{E}}_1 , \vec{\mathbfsf{E}}_2 , \vec{\mathbfsf{E}}_3 )$, where it can be assembled with mass matrices from the other 11 elements, one must apply the orthogonal transformation
\begin{subequations}
    \begin{align}
    \mathbf{R}^{(e)}_{\mathrm{2D}} & = 
    \begin{bmatrix}
    \mathbf{Q}^{(e)} & \mathbf{0} & \mathbf{0} & \mathbf{0} & \mathbf{0} \\
    \mathbf{0} & \mathbf{Q}^{(e)} & \mathbf{0} & \mathbf{0} & \mathbf{0} \\
    \mathbf{0} & \mathbf{0} & \mathbf{Q}^{(e)} & \mathbf{0} & \mathbf{0} \\
    \mathbf{0} & \mathbf{0} & \mathbf{0} & \mathbf{Q}^{(e)} & \mathbf{0} \\
    \mathbf{0} & \mathbf{0} & \mathbf{0} & \mathbf{0} & \mathbf{Q}^{(e)}
    \end{bmatrix}
    \intertext{whose diagonal entries are themselves orthogonal matrices with components}
    \mathbf{Q}^{(e)} & = 
    \begin{bmatrix}
    Q^{(e)}_{11} & Q^{(e)}_{12} & Q^{(e)}_{13} \\
    Q^{(e)}_{21} & Q^{(e)}_{22} & Q^{(e)}_{23} \\
    Q^{(e)}_{31} & Q^{(e)}_{32} & Q^{(e)}_{33}
    \end{bmatrix}
    \intertext{so that, accordingly,}
    \mathbf{M}^{(e)}_{\mathrm{2D}} & = \bigl( \mathbf{R}^{(e)}_{\mathrm{2D}} \bigr)^{\mathsf{T}} 
    \boldsymbol{M}^{(e)}_{\mathrm{2D}} \mathbf{R}^{(e)}_{\mathrm{2D}}
    \end{align}
\end{subequations}
where $\mathbf{M}^{(e)}_{\mathrm{2D}}$ is its mass matrix transformed into the dodecahedral co-ordinate system $( \vec{\mathbfsf{E}}_1 , \vec{\mathbfsf{E}}_2 , \vec{\mathbfsf{E}}_3 )$ according to the map $[ \{ \vec{\mathbfsf{e}}_1 \} \{ \vec{\mathbfsf{e}}_2 \} \{ \vec{\mathbfsf{e}}_3 \} ]^{(e)} = \bigl[ \{ \vec{\mathbfsf{E}}_1 \} \{ \vec{\mathbfsf{E}}_2 \} \{ \vec{\mathbfsf{E}}_3 \} \bigr] \bigl[ \mathbf{Q}^{(e)} ]$, cf.\ Fig.~\ref{figPentagonCoord}.  Even though $\boldsymbol{M}^{(e)}_{\mathrm{2D}}$ is a constant mass matrix, $\mathbf{M}^{(e)}_{\mathrm{2D}}$ need not be, because $\mathbf{R}^{(e)}_{\mathrm{2D}}$ will typically vary over time in our analysis of alveoli subjected to shock waves.

One can now take entries from mass matrix $\mathbf{M}^{(e)}_{\mathrm{2D}}$ for each nodal location in a pentagonal element $e$ and sum them into their appropriate nodal locations in the overall mass matrix $\mathbf{M}_{\mathrm{2D}}$ for the structure, following standard procedures. \cite{ClaytonChung18}

\subsection{Mass Matrix for a Tetrahedron}

The volume of a dodecahedron is filled with 60 tetrahedra, whose centroid (the origin in its natural co-ordinate system) is a common vertex among these 60 tetrahedra. Hence, an analysis to find the mass matrix of a tetrahedron becomes the building block needed to be able to assemble a 3D mass matrix for modeling an alveolar sac.

The matrix of shape functions $\mathbf{N} (\xi, \eta, \zeta)$ for a tetrahedon has a general form of
\begin{equation}
\mathbf{N} =  
\begin{bmatrix*}[r]
N_1 & 0 & 0 & N_2 & 0 & 0 & N_3 & 0 & 0 & N_4 & 0 & 0 \\
0 & N_1 & 0 & 0 & N_2 & 0 & 0 & N_3 & 0 & 0 & N_4 & 0 \\
0 & 0 & N_1 & 0 & 0 & N_2 & 0 & 0 & N_3 & 0 & 0 & N_4
\end{bmatrix*} 
\label{shape3D}
\end{equation}
whose product $\mathbf{N}^{\mathsf{T}} \mathbf{N} ( \xi , \eta , \zeta )$ is a symmetric singular matrix with banded structure
\begin{multline}
\mathbf{N}^{\mathsf{T}} \mathbf{N} = \\
\tiny
\begin{bmatrix}
N_1^2 & 0 & 0 & N_1 N_2 & 0 & 0 & N_1 N_3 & 0 & 0 & N_1 N_4 & 0 & 0 \\
0 & N_1^2 & 0 & 0 & N_1 N_2 & 0 & 0 & N_1 N_3 & 0 & 0 & N_1 N_4 & 0 \\
0 & 0 & N_1^2 & 0 & 0 & N_1 N_2 & 0 & 0 & N_1 N_3 & 0 & 0 & N_1 N_4 \\
N_1 N_2 & 0 & 0 & N_2^2 & 0 & 0 & N_2 N_3 & 0 & 0 & N_2 N_4 & 0 & 0 \\
0 & N_1 N_2 & 0 & 0 & N_2^2 & 0 & 0 & N_2 N_3 & 0 & 0 & N_2 N_4 & 0 \\
0 & 0 & N_1 N_2 & 0 & 0 & N_2^2 & 0 & 0 & N_2 N_3 & 0 & 0 & N_2 N_4 \\
N_1 N_3 & 0 & 0 & N_2 N_3 & 0 & 0 & N_3^2 & 0 & 0 & N_3 N_4 & 0 & 0 \\
0 & N_1 N_3 & 0 & 0 & N_2 N_3 & 0 & 0 & N_3^2 & 0 & 0 & N_3 N_4 & 0 \\
0 & 0 & N_1 N_3 & 0 & 0 & N_2 N_3 & 0 & 0 & N_3^2 & 0 & 0 & N_3 N_4 \\
N_1 N_4 & 0 & 0 & N_2 N_4 & 0 & 0 & N_3 N_4 & 0 & 0 & N_4^2 & 0 & 0 \\
0 & N_1 N_4 & 0 & 0 & N_2 N_4 & 0 & 0 & N_3 N_4 & 0 & 0 & N_4^2 & 0 \\
0 & 0 & N_1 N_4 & 0 & 0 & N_2 N_4 & 0 & 0 & N_3 N_4 & 0 & 0 & N_4^2
\end{bmatrix}
\normalsize
\label{tetNtN}
\end{multline}
that when summed along each row produces a diagonal matrix with elements
\begin{equation}
\sum_{\mathrm{rows}} \mathbf{N}^{\mathsf{T}} \mathbf{N} = 
\footnotesize
\mathrm{diag} \bigl( \begin{matrix}
N_1 & N_1 & N_1 & N_2 & N_2 & N_2 & N_3 & N_3 & N_3 & N_4 & N_4 & N_4
\end{matrix} \bigr)
\label{tetSumNtN}
\end{equation}
in which the $\mathrm{N}_i$, $i = 1, 2, 3, 4$, are the four shape functions corresponding to the four vertices of a tetrahedron that, along with their gradients, are described by
\begin{subequations}
    \label{tetShapeFunctions}
    \begin{align}
    N_1 & = 1 - \xi - \eta - \zeta & N_{1,\xi} & = -1 &
    N_{1,\eta} & = -1 & N_{1,\zeta} & = -1 \\
    N_2 & = \xi & N_{2,\xi} & = 1 & N_{2,\eta} & = 0 & N_{2,\zeta} & = 0 \\
    N_3 & = \eta & N_{3,\xi} & = 0 & N_{3,\eta} & = 1 & N_{3,\zeta} & = 0 \\
    N_4 & = \zeta & N_{4,\xi} & = 0 & N_{4,\eta} & = 0 & N_{4,\zeta} & = 1 
    \end{align}
\end{subequations}
out of which consistent, lumped and weighted mass matrices can be constructed.

Numerical integration is used to obtain a consistent mass matrix for a tetrahedron
\begin{equation}
\begin{aligned}
\mathbf{M}_{\mathrm{C3D}} & = \iiint_V \rho \, \mathbf{N}^{\mathsf{T}} \mathbf{N} \, \mathrm{d} \mathbf{z} \, \mathrm{d} \mathbf{y} \, \mathrm{d} \mathbf{x} \\ & = \int_0^1 \int_0^{1-\xi} \int_0^{1-\xi-\eta} \rho \, \mathbf{N}^{\mathsf{T}} \mathbf{N} \, | \mathbf{J} | \, \mathrm{d} \mathbf{\zeta} \, \mathrm{d} \mathbf{\eta} \, \mathrm{d} \mathbf{\xi} \\ & = \rho \, |\mathbf{J} | \sum_{i=1}^m \mathbf{N}^{\mathsf{T}} \mathbf{N} (\xi_i , \eta_i , \zeta_i ) \, w_i
\end{aligned}
\label{consMass3D}
\end{equation}
where $|\mathbf{J}|$ is the determinant of the Jacobian matrix $\mathbf{J}$, with $m$ being the number of Gauss points used for spatial integration, which in our case is four.  The co-ordinates $( \xi_i , \eta_i , \zeta_i )$ and weights $w_i$ of quadrature used for integrating over the volume of a tetrahedron are found in Table~\ref{tab:4nodedTet}.

The Jacobian is calculated from taking derivatives of the shape functions with respect to their local co-ordinates $(\xi, \eta, \zeta)$, as specified in Eqn.~(\ref{tetShapeFunctions}), which associate with the current global co-ordinates $(x_i, y_i, z_i)$ of its four vertices according to, cf.\ Ref.~\cite[pg.~424]{Reddy93}
\begin{equation}
\begin{aligned}
\mathbf{J}= &
\begin{bmatrix}
\partial x / \partial\xi & \partial y / \partial\xi & \partial z / \partial\xi\\
\partial x / \partial\eta & \partial y / \partial\eta & \partial z / \partial\eta \\
\partial x / \partial\zeta & \partial y / \partial\zeta & \partial z / \partial\zeta 
\end{bmatrix} \\
= & \begin{bmatrix}
\sum\nolimits_{i=1}^4 N_{i,\xi} (\xi,\eta,\zeta) \, x_i & \sum\nolimits_{i=1}^4 N_{i,\xi} (\xi,\eta,\zeta) \, y_i &
\sum\nolimits_{i=1}^4 N_{i,\xi} (\xi,\eta,\zeta) \, z_i\\
\sum\nolimits_{i=1}^4 N_{i,\eta} (\xi,\eta,\zeta) \, x_i & \sum\nolimits_{i=1}^4 N_{i,\eta} (\xi,\eta,\zeta) \, y_i &
\sum\nolimits_{i=1}^4 N_{i,\eta} (\xi,\eta,\zeta) \, z_i\\
\sum\nolimits_{i=1}^4 N_{i,\zeta} (\xi,\eta,\zeta) \, x_i & \sum\nolimits_{i=1}^4 N_{i,\zeta} (\xi,\eta,\zeta) \, y_i &
\sum\nolimits_{i=1}^4 N_{i,\zeta} (\xi,\eta,\zeta) \, z_i
\end{bmatrix} \\
= & \begin{bmatrix}
x_2 - x_1 & y_2 - y_1 & z_2 - z_1 \\
x_3 - x_1 & y_3 - y_1 & z_3 - z_1 \\
x_4 - x_1 & y_4 - y_1 & z_4 - z_1
\end{bmatrix}
\end{aligned}
\tag{\ref{tetJacobian}}
\end{equation}
whose determinant $| \mathbf{J} |$ is proportional to the volume of this element when evaluated in the physical co-ordinate system $( \vec{\mathbfsf{E}}_1 , \vec{\mathbfsf{E}}_2 , \vec{\mathbfsf{E}}_3 )$.  Specifically, $| \mathbf{J} | = 6V_{tet}$ because $V_{tet} = \iiint_{V_{tet}} \mathrm{d} z \, \mathrm{d} y \, \mathrm{d} x = \int_0^1 \int_0^{1-\xi} \int_0^{1-\xi-\eta} | \mathbf{J} | \, \mathrm{d} \zeta \, \mathrm{d} \eta \, \mathrm{d} \xi = | \mathbf{J} | \sum_{i=1}^m w_i = \tfrac{1}{6} | \mathbf{J} |$.  

Averaging this mass matrix based on Eqn.~(\ref{tetNtN}) with its lumped version based on Eqn.~(\ref{tetSumNtN}), according to Eqn.~\eqref{consMass3D}, gives the 3D mass matrix $\mathbf{M}_{\textrm{3D}}$ that we implement.

Because tetrahedra are used to model an alveolar sac, which will either be filled with air or fluid, their mass densities $\rho$ are considered to be uniform over the domains of these elements, which is why $\rho$ can be pulled out in front of the summation in Eqn.~(\ref{consMass3D}).

\subsubsection{Assembly of Tetrahedral Mass Matrices}

In our finite element model for an alveolar sac, there are 21 nodes (20 vertices and the origin) whose numbering scheme and natural co-ordinates are given in Table~\ref{TableDodecahedron}.  Filling this volume are 60 tetrahedra whose numbering scheme and associated nodal numbers are specified according to the following strategy.  Using the element and nodal numbering scheme for pentagons given in Table~\ref{TablePentagons}, begin with pentagon 1 and sequence to pentagon 12.  Associated with any given pentagon are 5 tetrahedra.  Nodes 1 and 4 of these five tetrahedra are the same.  Node 1 is at the centroid of the pentagon, and node 4 is at the origin of the dodecahedron.  Nodes 2 and 3 of the tetrahedron are also nodes of this pentagon, and are sequenced such that when traversing nodes $1 \to 2 \to 3$ of a tetrahedron one undergoes a counter\-clockwise path when viewed looking inward from outside of the dodecahedron.  The first tetrahedron associated with a pentagon shares nodes 2 and 3 of its tetrahedron with nodes 1 and 2 of the pentagon.  The second tetrahedron shares nodes 2 and 3 of its tetrahedron with nodes 2 and 3 of the pentagon, etc.

In the co-ordinate system of a tetrahedron $( \vec{\mathbfsf{e}}_1 , \vec{\mathbfsf{e}}_2 , \vec{\mathbfsf{e}}_3  )^{(e)}$, a tetrahedron has a mass matrix with a symmetric block structure of
\begin{subequations}
    \begin{align}
    \boldsymbol{M}^{(e)}_{\mathrm{3D}} & = 
    \begin{bmatrix}
    \boldsymbol{M}^{\mathrm{3D}}_{11} & \boldsymbol{M}^{\mathrm{3D}}_{12} & 
    \boldsymbol{M}^{\mathrm{3D}}_{13} & \boldsymbol{M}^{\mathrm{3D}}_{14} \\
    \boldsymbol{M}^{\mathrm{3D}}_{21} & \boldsymbol{M}^{\mathrm{3D}}_{22} & 
    \boldsymbol{M}^{\mathrm{3D}}_{23} & \boldsymbol{M}^{\mathrm{3D}}_{24} \\
    \boldsymbol{M}^{\mathrm{3D}}_{31} & \boldsymbol{M}^{\mathrm{3D}}_{32} & 
    \boldsymbol{M}^{\mathrm{3D}}_{33} & \boldsymbol{M}^{\mathrm{3D}}_{34} \\
    \boldsymbol{M}^{\mathrm{3D}}_{41} & \boldsymbol{M}^{\mathrm{3D}}_{42} & 
    \boldsymbol{M}^{\mathrm{3D}}_{43} & \boldsymbol{M}^{\mathrm{3D}}_{44} 
    \end{bmatrix} \\
    \intertext{with each element in this matrix being a $3 \! \times \! 3$ matrix with diagonal entries of}
    \boldsymbol{M}^{\mathrm{3D}}_{ij} & = 
    \begin{bmatrix}
    M^{\mathrm{3D}}_{ij} & 0 & 0 \\
    0 & M^{\mathrm{3D}}_{ij} & 0 \\
    0 & 0 & M^{\mathrm{3D}}_{ij}
    \end{bmatrix}
    \end{align}
\end{subequations}
whose components have values of
\begin{subequations}
    \begin{align}
    M_{ii}^{\mathrm{3D}} & = 
    \frac{\rho_0 | \mathbf{J}_0 |}{2} \sum_{k=1}^4 N_i ( 1 + 
    N_i ) ( \xi_k , \eta_k , \zeta_k ) \, w_k & i & = 1,2,3,4 \\
    M_{ij}^{\mathrm{3D}} = M_{ji}^{\mathrm{3D}} & =
    \frac{\rho_0 | \mathbf{J}_0 |}{2} \sum_{k=1}^4
    N_i N_j (\xi_k , \eta_k , \zeta_k) \, w_k & i & \neq j
    \end{align}
\end{subequations}
with co-ordinates $( \xi_i , \eta_i , \zeta_i )$ and weights $w_i$ of quadrature being given in Table~\ref{tab:4nodedTet}, and whose shape functions are defined according to Eqn.~\eqref{tetShapeFunctions}.  Here we consider that mass is conserved over the volume of each element, and as such, $\rho | \mathbf{J} | = \rho_0 | \mathbf{J}_0 |$.  Specifically, there is insufficient time for the normal transport of air into and out of an alveolar sac to occur through breathing in the presence of a shock wave traversing across the alveolus.   Consequently, the mass matrix $\boldsymbol{M}^{(e)}_{\mathrm{3D}}$ for each element only needs to be evaluated once.

To rotate this mass matrix for element $e$, $e \in \{ 1, 2, \ldots, 60 \}$, from its elemental co-ordinate system for the tetrahedron $( \vec{\mathbfsf{e}}_1 , \vec{\mathbfsf{e}}_2 , \vec{\mathbfsf{e}}_3 )^{(e)}$ into a fixed co-ordinate system for the dodecahedron $( \vec{\mathbfsf{E}}_1 , \vec{\mathbfsf{E}}_2 , \vec{\mathbfsf{E}}_3 )$, where it can be assembled with mass matrices from the other 59 elements, one must apply the orthogonal transformation
\begin{subequations}
    \begin{align}
    \mathbf{R}^{(e)}_{\mathrm{3D}} & = 
    \begin{bmatrix}
    \mathbf{Q}^{(e)} & \mathbf{0} & \mathbf{0} & \mathbf{0} \\
    \mathbf{0} & \mathbf{Q}^{(e)} & \mathbf{0} & \mathbf{0} \\
    \mathbf{0} & \mathbf{0} & \mathbf{Q}^{(e)} & \mathbf{0} \\
    \mathbf{0} & \mathbf{0} & \mathbf{0} & \mathbf{Q}^{(e)}
    \end{bmatrix}
    \intertext{whose diagonal entries are themselves orthogonal matrices with components}
    \mathbf{Q}^{(e)} & = 
    \begin{bmatrix}
    Q^{(e)}_{11} & Q^{(e)}_{12} & Q^{(e)}_{13} \\
    Q^{(e)}_{21} & Q^{(e)}_{22} & Q^{(e)}_{23} \\
    Q^{(e)}_{31} & Q^{(e)}_{32} & Q^{(e)}_{33}
    \end{bmatrix}
    \intertext{so that, accordingly,}
    \mathbf{M}^{(e)}_{\mathrm{3D}} & = \bigl( \mathbf{R}^{(e)}_{\mathrm{3D}} \bigr)^{\mathsf{T}} 
    \boldsymbol{M}^{(e)}_{\mathrm{3D}} \mathbf{R}^{(e)}_{\mathrm{3D}}
    \end{align}
\end{subequations}
where $\mathbf{M}^{(e)}_{\mathrm{3D}}$ is its mass matrix transformed into the dodecahedral co-ordinate system $( \vec{\mathbfsf{E}}_1 , \vec{\mathbfsf{E}}_2 , \vec{\mathbfsf{E}}_3 )$ according to the map $[ \{ \vec{\mathbfsf{e}}_1 \} \{ \vec{\mathbfsf{e}}_2 \} \{ \vec{\mathbfsf{e}}_3 \} ]^{(e)} = \bigl[ \{ \vec{\mathbfsf{E}}_1 \} \{ \vec{\mathbfsf{E}}_2 \} \{ \vec{\mathbfsf{E}}_3 \} \bigr] \bigl[ \mathbf{Q}^{(e)} ]$.  Even though $\boldsymbol{M}^{(e)}_{\mathrm{3D}}$ is a constant mass matrix, $\mathbf{M}^{(e)}_{\mathrm{3D}}$ need not be, because $\mathbf{R}^{(e)}_{\mathrm{3D}}$ will typically vary over time in our analysis of alveoli subjected to shock waves.

One can now take the mass matrix $\mathbf{M}^{(e)}_{\mathrm{3D}}$ for each nodal location in a tetrahedral element $e$ and sum them into their appropriate nodal locations in an overall mass matrix $\mathbf{M}_{\mathrm{3D}}$ for the structure, following standard procedures. \cite{ClaytonChung18}


\section{Constitutive Models for Finite Elements}


In this study, we implement implicit, elastic, material models.  Consequently, their elastic compliance $\boldsymbol{\mathcal{C}}$ and modulus $\boldsymbol{\mathcal{M}}$, where $\boldsymbol{\mathcal{M}} \defeq \boldsymbol{\mathcal{C}}^{-1}$, are taken to be functions of both strain \textit{and\/} stress in a manner that is consistent with thermo\-dynamics, cf.\ Part~\ref{partConstitutive} and Appendix~\ref{appImplicitElasticity}.  Furthermore, the conjugate response between temperature and entropy is not incorporated into our finite element solution strategy, because changes in entropy caused by elastic deformations have been shown to be negligible in our application, cf.\ \S\ref{secCE_verifyCode}.  As such, one can write down the governing constitutive equations for use in finite elements as
\begin{subequations}
    \label{stressStrainRelations}
    \begin{align}
    \boldsymbol{E} & = \boldsymbol{\mathcal{C}}^s (\boldsymbol{E}, \boldsymbol{T}) 
    \cdot ( \boldsymbol{T} - \boldsymbol{T}_0 ) 
    \label{hyperelasticCompliance} \\
    \boldsymbol{T} & = 
    \boldsymbol{T}_0 + \boldsymbol{\mathcal{M}}^s (\boldsymbol{E}, \boldsymbol{T})
    \cdot \boldsymbol{E}
    \label{hyperelasticModulus} \\
    \intertext{where $\boldsymbol{T}_0$ is an initial (residual) stress at zero strain, and where $\boldsymbol{\mathcal{C}}^s$ and $\boldsymbol{\mathcal{M}}^s$ are the secant compliance and secant modulus, respectively, obeying $\boldsymbol{\mathcal{M}}^s = ( \boldsymbol{\mathcal{C}}^s )^{-1}$.  Written symbolically, $\boldsymbol{\mathcal{C}}^s = \boldsymbol{E} / (\boldsymbol{T} - \boldsymbol{T}_0)$ and $\boldsymbol{\mathcal{M}}^s = (\boldsymbol{T} - \boldsymbol{T}_0) / \boldsymbol{E}$.  
    \bigskip
    \newline
    Expressing these constitutive equations in differential form, one can write} 
    \mathrm{d} \boldsymbol{E} & = 
    \boldsymbol{\mathcal{C}}^t (\boldsymbol{E}, \boldsymbol{T}) \cdot
    \mathrm{d} \boldsymbol{T}
    \label{hypoelasticCompliance} \\
    \mathrm{d} \boldsymbol{T} & = 
    \boldsymbol{\mathcal{M}}^t (\boldsymbol{E}, \boldsymbol{T}) \cdot
    \mathrm{d} \boldsymbol{E} 
    \label{hypoelasticModulus} \\
    \intertext{where $\boldsymbol{\mathcal{C}}^t$ and $\boldsymbol{\mathcal{M}}^t$ are the tangent compliance and tangent modulus, respectively, obeying $\boldsymbol{\mathcal{M}}^t = ( \boldsymbol{\mathcal{C}}^t )^{-1}$.  Written symbolically, $\boldsymbol{\mathcal{C}}^t = \mathrm{d} \boldsymbol{E} / \mathrm{d} \boldsymbol{T}$ and $\boldsymbol{\mathcal{M}}^t = \mathrm{d} \boldsymbol{T} / \mathrm{d} \boldsymbol{E}$.  
    \bigskip
    \newline    
    The components from these elastic compliance and moduli relate to one another via}
    \mathcal{C}^t_{ij} & = \left( I_{ik} - \frac{\partial \mathcal{C}^s_{i\ell}}{\partial E_k} \, (T_{\ell} - T_{0\,\ell}) \right)^{\!-1} \left( \mathcal{C}^s_{kj} + \frac{\partial \mathcal{C}^s_{k\ell}}{\partial T_j} (T_{\ell} - T_{0\,\ell}) \right) 
    \label{elasticCompliance} \\
    \mathcal{M}^t_{ij} & = \left( I_{ik} - \frac{\partial \mathcal{M}^s_{i\ell}}{\partial T_k} \, 
    E_{\ell} \right)^{\!-1} \left( \mathcal{M}^s_{kj} + 
    \frac{\partial \mathcal{M}^s_{k\ell}}{\partial E_j} \, E_{\ell} \right)
    \label{elasticModuli} \\
    \intertext{that, because $\boldsymbol{\mathcal{M}}^t = ( \boldsymbol{\mathcal{C}}^t )^{-1}$, enables one to write}
    \mathcal{M}^t_{ij} & = \left( \mathcal{C}^s_{ik} + \frac{\partial \mathcal{C}^s_{i\ell}}{\partial T_k} \, (T_{\ell} - T_{0\,\ell})
    \right)^{\!-1} \left( I_{kj} -
    \frac{\partial \mathcal{C}^s_{k\ell}}{\partial E_j} \, (T_{\ell} - T_{0\,\ell}) \right)
    \label{elasticModuliViaCs}
    \end{align}
\end{subequations}
and therefore we observe that if $\boldsymbol{\mathcal{C}}^s (\boldsymbol{E}, \boldsymbol{T})$ and $\boldsymbol{T}_0$ are known, then $\boldsymbol{\mathcal{M}}^s$, $\boldsymbol{\mathcal{C}}^t$ and $\boldsymbol{\mathcal{M}}^t$ can all be determined in terms of this secant compliance and initial stress.  The tangent compliance will equate with its associated secant compliance only when all of its elastic parameters are constant valued.  A like statement applies for the moduli.  It is the moduli $\boldsymbol{\mathcal{M}}^s$ and $\boldsymbol{\mathcal{M}}^t$ that appear later in our finite element equations~(\ref{EOM}).

In finite element implementations, strain $\boldsymbol{E}$ and stress $\boldsymbol{T}$ are treated as vectors of size $\ell \! \times \! 1$ (Voigt components, typically, but not so in our case) while the compliance $\boldsymbol{\mathcal{C}}^s$ and $\boldsymbol{\mathcal{C}}^t$ and the moduli $\boldsymbol{\mathcal{M}}^s$ and $\boldsymbol{\mathcal{M}}^t$ are each matrices of size $\ell \! \times \! \ell$, where $\ell$ denotes the number of independent stress\slash strain attribute pairs that there are.

Equations~(\ref{hyperelasticCompliance} \&\ \ref{hyperelasticModulus}) represent an implicit version of a Cauchy elastic material. Equations~(\ref{hypoelasticCompliance} \&\ \ref{hypoelasticModulus}) represent an implicit version of a hypo-elastic material.  Thermo\-dynamically admissible compliance and moduli suitable for soft tissue analysis have been derived in Part~\ref{partConstitutive} and Appendix~\ref{appImplicitElasticity}.


\subsection{Moduli for a Chord}

Alveolar chords are comprised of collagen and elastin fibers loaded in parallel. Consequently, they are exposed to the same axial strain of $e = \ln (L / L_0)$ but carry different stresses $s^c$ and $s^e$, where $\mbox{}^c$ is for collagen and $\mbox{}^e$ is for elastin.  The rule of mixtures is used to average their individual responses into a collective chordal response.  Specifically, the chordal, elastic, secant modulus is described by the averaged response
\begin{subequations}
    \label{elasticChordProperties}
    \begin{align}
    E^s & \defeq \phi \, E^c_s + (1 - \phi) E^e_s
    \label{elasticChordModulus} \\
    \intertext{while the chordal, elastic, secant compliance is described by the averaged response}
    C^s & = \frac{C^c_s \, C^e_s}{\phi \, C^e_s + (1-\phi) C^c_s} . 
    \label{elasticChordCompliance} \\
    \intertext{Consequently, the chordal, elastic, secant modulus $\boldsymbol{\mathcal{M}}^s$ appearing in Eqn.~(\ref{hyperelasticModulus}), when described in terms of compliances $C^c_s$ and $C^e_s$, becomes}
    \mathcal{M}^s & = \phi / C^c_s + (1-\phi) / C^e_s .
    \intertext{Given the constitutive equation $s = s_0 + E^s e$, it follows that the stresses average as}
    s_0 & \defeq \phi \, s^c_0 + (1 - \phi)s_0^e \\
    s & \defeq \phi \, s^c + (1 - \phi) s^e \\
    \intertext{because these fibers experience the same strain.  The collagen, fiber, volume fraction $\phi$ that does this partitioning is established by}
    \phi & \defeq A_0^c / (A_0^c + A_0^e)
    \label{volumeFraction}
    \end{align}
\end{subequations}
where the cross-sectional area of a chord is the sum of cross-sectional areas for its collagen $A_0^c$ and elastin $A_0^e$ fibers, here evaluated in a reference state.  These areas are assigned via probability density functions according to Table~\ref{tab:alveolarProp}.  Assuming an isochoric fiber response, i.e., that volume is preserved, it necessarily follows that the homogenization parameter $\phi$ is a constant, i.e., it is independent of deformation.

The secant compliance $C^s$ that we apply to the collagen and elastin fibers in an alveolar chord are derived in Appendix~\ref{appImplicitElasticity}, cf.\ Eqn.~(\ref{secantFiberModulus}).  This model, under isothermal conditions, describes an elastic secant compliance for collagen of
\begin{subequations}
    \label{fiberCompliance}
    \begin{align}
    C_s^c (s^c) & = \frac{e^c_{1_{\max}}}{s^c - s^c_0} \left( 
    1 - \frac{\sqrt{E^c_1 e^c_{1_{\max}}}}{\sqrt{E^c_1 e^c_{1_{\max}} + 
            2 (s^c - s^c_0)}} \right) + \frac{1}{E^c_2} \\
    \intertext{and an elastic secant compliance for elastin of}
    C_s^e (s^e) & = \frac{e^e_{1_{\max}}}{s^e - s^e_0} \left( 
    1 - \frac{\sqrt{E^e_1 e^e_{1_{\max}}}}{\sqrt{E^e_1 e^e_{1_{\max}} + 
            2 (s^e - s^e_0)}} \right) + \frac{1}{E^e_2}
    \end{align}
\end{subequations}
whose inverses, viz., $E^c_s \defeq 1 / C^c_s$ and $E^e_s \defeq 1 / C^e_s$, are their secant moduli, which are defined in accordance with Eqns.~(\ref{hyperelasticCompliance} \&\ \ref{hyperelasticModulus}), and as such, $s^c = s_0^c + E^c_s e$ and $s^e = s_0^e + E^e_s e$.  The material properties associated with collagen fibers are: a soft initial modulus $E^c_1$, a stiff terminal modulus $E^c_2$, and their strain of transition $e^c_{1_{\max}}$, with like material properties describing an elastin fiber, cf.\ Part~\ref{partConstitutive} and Appendix~\ref{appImplicitElasticity}.

Whenever $s^c < s^c_0$, the elastic modulus for collagen is taken to be its modulus at zero strain, i.e., $E^c_s = E^c_1 E^c_2 / ( E^c_1 + E^c_2 )$ so that $C^c_s = (E^c_1 + E^c_2) / E^c_1 E^c_2$, which helps to ensure numeric stability.  Also, whenever a collagen fiber ruptures, $E^c_s \to 0$ and therefore $C^c_s \to \infty$.  Like statements apply to the elastin fiber of an alveolar chord.

The elastic fiber compliance in Eqn.~(\ref{fiberCompliance}) depend only upon stress, not upon strain, and as such the elastic tangent modulus $\boldsymbol{\mathcal{M}}^t$ of Eqn.~\eqref{elasticModuliViaCs}, which is one of two moduli we use in our finite element implementation, reduces in this 1D case to
\begin{subequations}
    \label{fiberTangentModulus}
    \begin{align}
    \mathcal{M}^t & = \left( \mathcal{C}^s + \frac{\partial \, \mathcal{C}^s}
    {\partial s} \, (s - s_0) \right)^{-1} \\
    \intertext{where $C^s$ is given by Eqn.~(\ref{elasticChordCompliance}), whose individual compliance $C^c_s$ and $C^e_s$ are described by Eqn.~(\ref{fiberCompliance}), and whose derivatives are determined to be}
    \mbox{} \qquad \frac{\partial \, \mathcal{C}^s}{\partial s} & = 
    \frac{\partial}{\partial s^c} 
    \frac{\mathcal{C}_s^c (s^c) \, \mathcal{C}_s^e (s^e)}
    {\phi \, \mathcal{C}_s^e (s^e) + (1-\phi) \, \mathcal{C}_s^c (s^c)}
    \left( \frac{\partial s}{\partial s^c} \right)^{\! -1} \notag \\ 
    \mbox{} & \qquad + \frac{\partial}{\partial s^e} 
    \frac{\mathcal{C}_s^c (s^c) \, \mathcal{C}_s^e (s^e)} 
    {\phi \, \mathcal{C}_s^e (s^e) + (1-\phi) \, \mathcal{C}_s^c (s^c)}
    \left( \frac{\partial s}{\partial s^e} \right)^{\! -1} \notag \\ 
    & = \frac{1}{\bigl( \phi C^e_s + (1-\phi) C^c_s \bigr)^2} \left(
    (C^e_s)^2 \, \frac{\partial C^c_s}{\partial s^c} + (C^c_s)^2 \,
    \frac{\mathrm{d} C^e_s}{\mathrm{d} s^e} \right) \\
    \intertext{wherein}
    \frac{\mathrm{d}C^c_s}{\mathrm{d}s^c} & = \frac{e^c_{1_{\max}}}{s^c - s^c_0} 
    \left( \frac{\sqrt{E^c_1 e^c_{1_{\max}}}}
    {\bigl( E^c_1 e^c_{1_{\max}} + 2 (s^c - s^c_0) \bigr)^{3/2}} \right. \notag \\
    \mbox{} & \qquad \left. - 
    \frac{1}{s^c-s^c_0} \left( 1 - \frac{\sqrt{E^c_1 e^c_{1_{\max}}}}
    {\sqrt{E^c_1 e^c_{1_{\max}} + 2 (s^c - s^c_0)}} \right) \right) \\
    \intertext{and}
    \frac{\mathrm{d}C^e_s}{\mathrm{d}s^e} & = \frac{e^e_{1_{\max}}}{s^e - s^e_0} 
    \left( \frac{\sqrt{E^e_1 e^e_{1_{\max}}}}
    {\bigl( E^e_1 e^e_{1_{\max}} + 2 (s^e - s^e_0) \bigr)^{3/2}} \right. \notag \\
    \mbox{} & \qquad \left. - 
    \frac{1}{s^e-s^e_0} \left( 1 - \frac{\sqrt{E^e_1 e^e_{1_{\max}}}}
    {\sqrt{E^e_1 e^e_{1_{\max}} + 2 (s^e - s^e_0)}} \right) \right)
    \end{align}
\end{subequations}
which follow from Eqn.~(\ref{fiberCompliance}).  The complexity here comes from the fact that an alveolar chord is a mixture of collagen and elastin fibers.

\subsection{Moduli for a Pentagon}

The secant response of an isothermal, isochoric, elastic pentagon can be written in terms of Eqn.~(\ref{hyperelasticCompliance}), as established in \S\ref{secConjugatePairs}, whose constitutive behavior is established via an elastic compliance $\boldsymbol{\mathcal{C}}^s$ defined through the matrix equation
\begin{subequations}
    \label{pentagonCEs}
    \begin{align}
    \underbrace{ \left\{ \begin{matrix} 
        \xi \\ \varepsilon \\ \gamma \end{matrix} 
    \right\} }_{\boldsymbol{E}} & =
    \underbrace{ \begin{bmatrix} 
        1/4M^s & 0 & 0 \\ 0 & 3/4M^s & 0 \\ 0 & 0 & 1/G^s 
    \end{bmatrix} }_{\boldsymbol{\mathcal{C}}^s} 
    \underbrace{ \left( \left\{ \begin{matrix}
        s^{\pi} \\ s^{\sigma} \\ s^{\tau}
    \end{matrix} \right\} \right. }_{\boldsymbol{T}} - 
    \underbrace{ \left. \left\{ \begin{matrix}
    s^{\pi}_0 \\ 0 \\ 0
    \end{matrix} \right\} \right) }_{\boldsymbol{T}_0} 
    \label{pentagonSecantCompliance} \\
    \intertext{or in terms of Eqn.~(\ref{hyperelasticModulus}), whose constitutive behavior is established through an elastic modulus $\boldsymbol{\mathcal{M}}^s$ such that}
    \underbrace{ \left\{ \begin{matrix}
    s^{\pi} \\ s^{\sigma} \\ s^{\tau}
    \end{matrix} \right\} }_{\boldsymbol{T}}  & =
    \underbrace{ \left\{ \begin{matrix}
    s^{\pi}_0 \\ 0 \\ 0
    \end{matrix} \right\} }_{\boldsymbol{T}_0} +
    \underbrace{ \begin{bmatrix} 
    4M^s & 0 & 0 \\ 0 & 4M^s/3 & 0 \\ 0 & 0 & G^s 
    \end{bmatrix} }_{\boldsymbol{\mathcal{M}}^s} 
    \underbrace{ \left\{ \begin{matrix} 
    \xi \\ \varepsilon \\ \gamma \end{matrix} 
    \right\} }_{\boldsymbol{E}} 
    \label{pentagonSecantModulus}
    \end{align}
\end{subequations}
which is used in our finite element implementation.  This strain vector $\boldsymbol{E}$ has elements denoting a dilation $\xi = \ln\sqrt{ab/a_0b_0}$, a squeeze (pure shear) $\varepsilon = \ln\sqrt{ab_0/a_0b}$, and a (simple) shear $\gamma = g-g_0$, which in turn are described in terms of two elongations $a$ and $b$ plus an in-plane shear $g$, with their reference values being $a_0$, $b_0$ and $g_0$. These stretch attributes are acquired via a \textbf{QR} decomposition of the deformation gradient $\mathbf{F}$, cf.\ \S\ref{secQR}.    Dilation, squeeze and shear are not coupled in this model.

The stress vector $\boldsymbol{T} = \{ s^{\pi}, s^{\sigma}, s^{\tau} \}^{\mathsf{T}}$ conjugate to strain vector $\boldsymbol{E} = \{ \xi, \varepsilon, \gamma \}^{\mathsf{T}}$ has elements of a surface tension $s^{\pi} = \mathcal{S}_{11} + \mathcal{S}_{22}$, a normal-stress difference $s^{\sigma} = \mathcal{S}_{11} - \mathcal{S}_{22}$, and a shear stress $s^{\tau} = \tfrac{a}{b} \, \mathcal{S}_{12}$.  Only surface tension is considered to have a residual state of stress $s^{\pi}_0$, which is necessary for alveolar stability, and is caused, in part, by the presence of surfactant.  In a reciprocal sense, the stress components are assigned via $\mathcal{S}_{11} = \frac{1}{2} ( s^{\pi} + s^{\sigma} )$, $\mathcal{S}_{22} = \frac{1}{2} ( s^{\pi} - s^{\sigma} )$ and $\mathcal{S}_{12} = \mathcal{S}_{21} = \frac{b}{a} s^{\tau}$ such that $\mathbf{S} = \mathbf{P} \, \boldsymbol{\mathcal{U}}^{-1} \boldsymbol{\mathcal{S}} \, \boldsymbol{\mathcal{U}}^{-\mathsf{T}} \mathbf{P}^{\mathsf{T}}$ with $\mathbf{S}$ being the second Piola-Kirchhoff stress evaluated in the co-ordinate system of a pentagon, while $\boldsymbol{\mathcal{U}}$ is the Laplace stretch, and $\mathbf{P}$ is a re-indexer of co-ordinate labeling needed to ensure invariance under a transformation of Laplace stretch.


The elastic compliance governing an isothermal dilation response is 
\begin{subequations}
    \label{planarCompliance}
    \begin{align}
    \frac{1}{4M^s (s^{\pi})} & = \frac{\xi_{1_{\max}}}{s^{\pi} - s^{\pi}_0} \left( 
    1 - \frac{\sqrt{M_1 \xi_{1_{\max}}}}
    {\sqrt{M_1 \xi_{1_{\max}} + \tfrac{1}{2} ( s^{\pi} - s^{\pi}_0 )}} \right) + 
    \frac{1}{4M_2}
    \intertext{where $M^s (s^{\pi} \! \leq \! s^{\pi}_0) = M_1 M_2 / ( M_1 + M_2 )$. The elastic compliance governing a shear response is }
    \frac{1}{G^s (s^{\tau})} & = 
    \frac{\gamma_{1_{\max}}}{|s^{\tau}|} \left(
    1 - \frac{\sqrt{G_1 \gamma_{1_{\max}}}}
    {\sqrt{G_1 \gamma_{1_{\max}} + 2 | s^{\tau} |}} \right) + 
    \frac{1}{G_2}
    \end{align}
\end{subequations}
where $G^s (s^{\tau} \! = \! 0) = G_1 G_2 / (G_1 + G_2)$.  

Only in the mode of dilation is membrane rupture considered to be possible.  Nevertheless, whenever a rupture event does happen, both moduli vanish, i.e., $M^s \to 0$ and $G^s \to 0$.  The membrane looses structural integrity upon rupture.  The material parameters describing the various modes of a membrane are analogous to those used to describe a fiber.

Like the fiber compliance (Eqn.~\ref{fiberCompliance}) used to model an alveolar chord, the membrane compliance (Eqn.~\ref{planarCompliance}) used to model an alveolar septa has components that depend upon stress, but not upon strain.  Consequently, the tangent modulus of Eqn.~(\ref{elasticModuliViaCs}) required of our finite element implementation has components of
\small
\begin{multline}
    \boldsymbol{\mathcal{M}}^t = \\ \begin{bmatrix}
    \frac{1}{4M^s} + ( s^{\pi} - s^{\pi}_0 ) \frac{\mathrm{d} (1 / 4M^s)}
    {\mathrm{d} s^{\pi}}
    & 0 & 0 \\
    0 & \frac{3}{4M^s} + ( s^{\pi} - s^{\pi}_0 ) \frac{\mathrm{d} (3 / 4M^s)}
    {\mathrm{d} s^{\pi}} & 0 \\
    0 & 0 & \frac{1}{G^s} + s^{\tau} \frac{\mathrm{d} (1/G^s)}{\mathrm{d} s^{\tau}} 
    \end{bmatrix}^{-1}
    \label{pentagonTangentModulus}
\end{multline}
\normalsize
whose entries, taking into account Eqn.~(\ref{planarCompliance}), are determined to be
\begin{subequations}
    \begin{align}
    \frac{1}{4M^s} + ( s^{\pi} - s^{\pi}_0 ) \frac{\mathrm{d}(1/4M^s)}{\mathrm{d} s^{\pi}} & = \frac{\xi_{1_{\max}} \, \sqrt{M_1 \xi_{1_{\max}}}}
    {4 \bigl( M_1 \xi_{1_{\max}} + \frac{1}{2} ( s^{\pi} + s^{\pi}_0 ) \bigr)^{3/2}}
    + \frac{1}{4M_2} \\
    \frac{1}{G^s} + s^{\tau} \, \frac{\mathrm{d}(1/G^s)}{\mathrm{d} s^{\tau}} & = 
    \frac{\gamma_{1_{\max}} \, \sqrt{G_1 \gamma_{1_{\max}}}}
    {( G_1 \gamma_{1_{\max}} + 2 | s^{\tau} | )^{3/2}} + \frac{1}{G_2}
    \end{align}
\end{subequations}
and as such, our implementation becomes quite straightforward.

\subsection{Moduli for a Tetrahedron}

The isothermal response of a volume element located within an alveolar sac will have a secant response governed by
\begin{equation}
    \underbrace{\left\{ \begin{matrix}
        \Pi \\ \sigma_1 \\ \sigma_2 \\ \tau_1 \\ \tau_2 \\ \tau_3
    \end{matrix} \right\}}_{\boldsymbol{T}} = 
    \underbrace{\left\{ \begin{matrix} 
        \Pi_0 \\ 0 \\ 0 \\ 0 \\ 0 \\ 0
    \end{matrix} \right\}}_{\boldsymbol{T}_0} +
    \underbrace{\begin{bmatrix}
        9K & 0 & 0 & 0 & 0 & 0 \\
        0 & 3N & -\frac{3}{2} N & 0 & 0 & 0 \\
        0 & -\frac{3}{2} N & 3N & 0 & 0 & 0 \\
        0 & 0 & 0 & G & 0 & 0 \\
        0 & 0 & 0 & 0 & G & 0 \\
        0 & 0 & 0 & 0 & 0 & G
    \end{bmatrix}}_{\boldsymbol{\mathcal{M}}^s} 
    \underbrace{\left\{ \begin{matrix} 
        \Xi \\ \varepsilon_1 \\ \varepsilon_2 \\ \gamma_1 \\ \gamma_2 \\ \gamma_3
    \end{matrix} \right\}}_{\boldsymbol{E}}
\end{equation}
where, for air, only the bulk modulus $K$ is non-zero; whereas, for blood, the bulk modulus $K$, the squeeze modulus $N$, and the shear modulus $G$ are all non-zero.  Blood is actually a nonlinear visco\-elastic liquid, but for our application of studying pressure and shear waves traveling across a lung, blood will behave glassy elastic.  Blood supports a normal-stress difference, hence $N \neq 0$, and blood supports a shear stress, hence $G \neq 0$.

The strain vector $\boldsymbol{E} = \{ \Xi, \varepsilon_1, \varepsilon_2, \gamma_1, \gamma_2, \gamma_3 \}^{\mathsf{T}}$ has elements that denote a dilatation $\Xi = \ln \sqrt[3]{abc/a_0b_0c_0}$, two separate squeezes (pure shears) $\varepsilon_1 = \ln \sqrt[3]{ab_0/a_0b}$ and $\varepsilon_2 = \ln \sqrt[3]{bc_0/b_0c}$, and three separate (simple) shears $\gamma_1 = \alpha - \alpha_0$, $\gamma_2 = \beta - \beta_0$ and $\gamma_3 = \gamma - \gamma_0$ whose stretch attributes come from a \textbf{QR} decomposition of the deformation gradient $\mathbf{F}$.

The stress vector $\boldsymbol{T} = \{ \Pi, \sigma_1, \sigma_2, \tau_1, \tau_2, \tau_3 \}^{\mathsf{T}}$ conjugate to strain $\boldsymbol{E}$ has elements that comprise a pressure $\Pi = \mathcal{S}_{11} + \mathcal{S}_{22} + \mathcal{S}_{33} = -3P$ where $P$ denotes the common definition for pressure, two separate normal-stress differences $\sigma_1 = \mathcal{S}_{11} - \mathcal{S}_{22}$ and $\sigma_2 = \mathcal{S}_{22} - \mathcal{S}_{33}$, and three separate shear stresses $\tau_1 = \frac{b}{c} \mathcal{S}_{32}$, $\tau_2 = \frac{a}{c} \mathcal{S}_{31}$ and $\tau_3 = \frac{a}{b} \mathcal{S}_{21} - \alpha \tau_2$.  Of these, only pressure has an initial value, viz., $\Pi_0$, which represents atmospheric pressure.

Moduli $K$, $N$ and $G$ are considered to be constants in our modeling of an alveolar sac; therefore, $\boldsymbol{\mathcal{M}}^t = \boldsymbol{\mathcal{M}}^s$ when modeling alveolar volumes.

\section{Stiffness Matrices}

The solution strategy adopted here uses a secant modulus to determine the stress acquired over a past interval in time spanning from $t_0$ to $t_i$, i.e., from an initial to the current time, while a tangent modulus is used to determine an additional stress acquired over a future interval in time spanning from $t_i$ to $t_{i+1} = t_i + \mathrm{d}t$ that is of infinitesimal extent.  Here time $t_i$ denotes time at the beginning of a solution step, where all fields are known, while time $t_{i+1}$ denotes time at the end of a solution step, whereat all dependent fields are to be determined.


\subsection{Strain-Displacement Matrices}

Finite elements introduces a matrix $\mathbf{B}$ that transforms nodal displacements $\boldsymbol{u}^{(e)}$ for an element $e$ into a vector of thermo\-dynamic strains $\boldsymbol{E}$ located at a Gauss point via the mapping
\begin{equation}
\boldsymbol{E} = \mathbf{B} \, \boldsymbol{u}^{(e)}
\label{strain} 
\end{equation}
where $\boldsymbol{E}$ has size $\ell \! \times \! 1$, $\mathbf{B}$ has size $\ell \! \times \! nd$, and $\boldsymbol{u}^{(e)}$ has size $nd \! \times \! 1$.  Here: $d$ is the spatial dimension of an element (viz., $d = 1, 2$ or $3$ that, in our case, associate with a chord, a pentagon, and a tetrahedron, respectively); $\ell$ is the number of conjugate pairs appropriate for an element (viz., $\ell = 1, 3$ or $6$ that, in our case, associate with a chord, a pentagon, and a tetrahedron, respectively); while $n$ is the number of nodes in an element (viz., $n=2, 5$ or $4$ that, in our case, associate with a chord, a pentagon, and a tetrahedron, respectively).

In order to make our computation more systematic, the strain-displacement matrix $\mathbf{B}$ is taken to additively decompose into linear and nonlinear constituents such that
\begin{subequations}
    \label{straindis}
    \begin{align}
    \mathbf{B} & = \mathbf{B}_L + \mathbf{B}_N 
    \label{BLBN} \\
    \intertext{where the entries in $\mathbf{B}_L$ are constants (its strain-displacement relationship is linear in displacement), while the entries in $\mathbf{B}_N$ are functions of displacement (its strain-displacement relationship is nonlinear in displacement).  Hence, in accordance with Eqn.~(\ref{strain}), this decomposition allows definitions for linear and nonlinear strain constituents to be introduced as}
    \boldsymbol{E}_L & \defeq \mathbf{B}_L \, \boldsymbol{u}^{(e)} \\
    \boldsymbol{E}_N & \defeq \mathbf{B}_N \, \boldsymbol{u}^{(e)} \\
    \intertext{which add, i.e., $\boldsymbol{E} = \boldsymbol{E}_L + \boldsymbol{E}_N$.  Their associated derivatives, taken with respect to displacement, produce the formul\ae}
    \mathrm{d} \boldsymbol{E}_L & = \mathbf{B}_L \, \mathrm{d} \boldsymbol{u}^{(e)} 
    \quad \because \quad \mathrm{d} \mathbf{B}_L = \mathbf{0} 
    \label{dEL} \\
    \mathrm{d} \boldsymbol{E}_N & = 
    \mathbf{B}_N \, \mathrm{d} \boldsymbol{u}^{(e)} 
    + \mathrm{d}\mathbf{B}_N \, \boldsymbol{u}^{(e)}
    \label{dEN}
    \end{align}
\end{subequations} 
which obey $\mathrm{d} \boldsymbol{E} = \mathrm{d} \boldsymbol{E}_L + \mathrm{d} \boldsymbol{E}_N$ and $\mathrm{d}\mathbf{B} = \mathrm{d}\mathbf{B}_N$ so that $\mathrm{d} \boldsymbol{E} = \mathbf{B} \, \mathrm{d} \boldsymbol{u}^{(e)} + \mathrm{d} \mathbf{B} \, \boldsymbol{u}^{(e)}$.  This differential equation reduces to the classic result $\mathrm{d} \boldsymbol{E} = \mathbf{B} \, \mathrm{d} \boldsymbol{u}^{(e)}$ found in the finite element literature whenever the total displacements are infinitesimal in extent, under which conditions $\mathbf{B}_N \approx \mathbf{0}$ and $\mathbf{d} \mathbf{B}_N \approx \mathbf{0}$.

It is advantageous for us to re-write this nonlinear strain-displacement relation $\mathbf{B}_N$ as a product between two matrices such that
\begin{subequations}
    \label{decomposeNonlinearB}
    \begin{align}
    \mathbf{B}_N & = \mathbf{A} \, \mathbf{H} \\
    \intertext{where matrix $\mathbf{A}$ has size $\ell \! \times \! d$, while matrix $\mathbf{H}$ has size $d \! \times \! nd$, with $\mathbf{A}$ being comprised of various displacement gradients, and $\mathbf{H}$ being comprised of derivatives of shape functions taken in the element's co-ordinate system, and as such}
    \mathrm{d} \mathbf{B}_N & = \mathrm{d} \mathbf{A} \cdot \mathbf{H} 
    \quad \because \quad \mathrm{d} \mathbf{H} = \mathbf{0} . \\
    \intertext{As a consequence of this definition, at least for the elements of interest to us, it turns out that one can establish another useful relationship, specifically}
    \mathrm{d} \mathbf{A}^{\mathsf{T}} \, \boldsymbol{T} & = 
    \mathbf{S} \, \mathbf{H} \, \mathrm{d} \boldsymbol{u}^{(e)}
    \end{align}
\end{subequations}
wherein $\mathbf{S}$ is a symmetric matrix of size $d \! \times \! d$ whose components come from those of its conjugate stress vector $\boldsymbol{T}$ of size $\ell \! \times \! 1$.  This representation follows whenever one adopts a triangular deconstruction of the deformation gradient $\mathbf{F}$ from which strain $\boldsymbol{E}$ is established, as addressed in \S\ref{partKinematics}.


\subsection{Secant Stiffness Matrix}

For nonlinear elastic materials, like soft tissues, their stress\slash strain response curves generally become stiffer with increasing deformation.  Consequently, the slope of a line segment connecting the origin with its current stress\slash strain state, located somewhere along its response curve, will change with a change in stress and strain, and therefore, its secant modulus will necessarily be a function of stress and\slash or strain.

A variation in the residual energy $R$ of a deformed elastic body is the difference between variations from two energy sources, assuming a simply connected body whose motion maps have sufficient smoothness, etc.  These energies are: a potential energy $U$ that stores an internal strain energy, and a work done $W$ that expends energy though an external loading, specifically
\begin{subequations}
    \label{principle}
    \begin{align}
    \delta{R} &= \delta{U} - \delta{W} \\
    \intertext{such that for an element $e$ one has \cite{Yangetal10}}
    \delta{W} & = \sum\nolimits_e \boldsymbol{F} \cdot \delta\boldsymbol{u}^{(e)} \\
    \delta{U} & = \sum\nolimits_e \int_{V} \, \mathbf{B}^{\mathsf{T}} \,
    \boldsymbol{T} \, \mathrm{d} V \cdot \delta\boldsymbol{u}^{(e)} \\
    \intertext{or alternatively}
    \delta R = \sum\nolimits_e \boldsymbol{R} \cdot \delta \boldsymbol{u}^{(e)} & = 
    \sum\nolimits_e \left( \int_V \mathbf{B}^{\mathsf{T}} \, \boldsymbol{T} \, \mathrm{d}V - 
    \boldsymbol{F} \right) \cdot \delta \boldsymbol{u}^{(e)} 
    \label{residualVariation}
    \end{align}
\end{subequations}
where $\boldsymbol{F}$ and $\boldsymbol{R}$ are vectors denoting the external and residual forces, respectively, while $\boldsymbol{T}$ is a stress conjugate to strain $\boldsymbol{E}$, which are represented here as vector fields, with $\mathbf{B}$ being the well-known strain-displacement matrix found in Eqn.~(\ref{strain}).

In order to satisfy equilibrium, the internal and external forces of Eqn.~\eqref{principle} must be in balance, and therefore, for each element \cite{Elseifi98}
\begin{subequations}
    \label{governingQuasistaticEquations}
    \begin{align}
    \boldsymbol{R} & = \int_{V} \, \mathbf{B}^{\mathsf{T}} \, \boldsymbol{T} \, \mathrm{d} V - \boldsymbol{F} = \mathbf{0}
    \label{residualForce} \\
    \intertext{whose solution is typically achieved through an iterative process.  Substituting the secant constitutive equation \eqref{hyperelasticModulus} along with the strain-displacement relationship of Eqn.~(\ref{strain}) into the above integral allows it to be re-written as}  
    \int_{V} \, \mathbf{B}^{\mathsf{T}} \, \boldsymbol{T} \, \mathrm{d} V & =
    \int_V \mathbf{B}^{\mathsf{T}} \, \boldsymbol{T}_0 \, \mathrm{d} V + 
    \int_V \mathbf{B}^{\mathsf{T}} \, \boldsymbol{\mathcal{M}}^s \, \boldsymbol{E} \, \mathrm{d} V \notag \\
    \mbox{} & = 
    \underbrace{\int_V \mathbf{B}^{\mathsf{T}} \, \boldsymbol{T}_0 \, \mathrm{d} V}_{\boldsymbol{F}_0} + 
    \underbrace{\int_V \mathbf{B}^{\mathsf{T}} \, \boldsymbol{\mathcal{M}}^s \, \mathbf{B} \, \mathrm{d} V}_{\mathbf{K}^s} \, \boldsymbol{u}^{(e)} 
    \label{secantStiffnessMatrix}
    \end{align}
\end{subequations}
where $\mathbf{K}^s$ is a stiffness matrix built around the secant modulus $\boldsymbol{\mathcal{M}}^s$, and $\boldsymbol{F}_0$ is an internal force accounting for an initial residual stress of $\boldsymbol{T}_0$.  Here $\mathbf{B}$ and $\boldsymbol{\mathcal{M}}^s$ are evaluated at current time $t_i$, i.e., at the beginning of an integration step.


\subsection{Tangent Stiffness Matrix}

Motivated by a definition for the tangent stiffness matrix being $\mathbf{C} \defeq \mathrm{d}\boldsymbol{R} / \mathrm{d}\boldsymbol{u}$ that, e.g., would be appropriate for an updated-Lagrangian finite element formulation, \cite{Belytschkoetal00} we differentiate Eqn.~\eqref{residualVariation} to get $\mathrm{d} \delta R = \delta \mathrm{d} R = \mathrm{d} \boldsymbol{R} \cdot \delta \boldsymbol{u}^{(e)}$ where from one gets
\begin{equation}
    \mathrm{d} \boldsymbol{R} = \int_{V} \, \mathrm{d} \mathbf{B}^{\mathsf{T}} \, \boldsymbol{T} \, \mathrm{d} V + \int_{V} \, \mathbf{B}^{\mathsf{T}} \, \mathrm{d} \boldsymbol{T} \, \mathrm{d} V \eqdef \mathbf{C} \, \mathrm{d} \boldsymbol{u}^{(e)} 
    \label{diffresidual}
\end{equation}
which follows because the external force $\boldsymbol{F}$ is considered to be a fixed boundary condition during a variation in its displacements.  This equation establishes that a change in residual force $\mathrm{d}\boldsymbol{R}$ is needed to further deform an elastic body from an equilibrium condition $\boldsymbol{R} = \mathbf{0}$ that exists at current time $t_i$ into another equilibrium state associated with some future moment in time $t_{i+1} = t_i + \mathrm{d}t$.  This differential force depends upon both the stress $\boldsymbol{T}$ at time $t_i$ and its change $\mathrm{d} \boldsymbol{T}$ that occurs when advancing from $t_i$ to $t_{i+1}$.

Substituting constitutive equation~(\ref{hyperelasticModulus}) for $\boldsymbol{T}$ into the first integral of Eqn.~(\ref{diffresidual}), while incorporating Eqn.~\eqref{decomposeNonlinearB}, allows this integral to be re-written as
\begin{equation}
\int_V \mathrm{d} \mathbf{B}^{\mathsf{T}} \, \boldsymbol{T} \, \mathrm{d}V =
\int_V \mathrm{d} \mathbf{B}^{\mathsf{T}} \bigl( \boldsymbol{T}_0 + 
\boldsymbol{\mathcal{M}}^s \boldsymbol{E} \bigr) \, \mathrm{d}V =  
\underbrace{\int_V \mathbf{H}^{\mathsf{T}} \, \mathbf{S} \, \mathbf{H} \, \mathrm{d}V}_{\mathbf{C}^s}  \, \mathrm{d} \boldsymbol{u}^{(e)}
\label{secantStiffness}
\end{equation}
where $\boldsymbol{T}_0 + \boldsymbol{\mathcal{M}}^s \boldsymbol{E} \mapsto \mathbf{S}$, and as such, $\mathbf{C}^s$ is that contribution to the tangent stiffness matrix $\mathbf{C}$ attributed to the secant modulus $\boldsymbol{\mathcal{M}}^s$ appearing in Eqn.~\eqref{hyperelasticModulus}, which is quadratic in $\mathbf{H}$ (not $\mathbf{B}$).

Now, substituting constitutive equation~(\ref{hypoelasticModulus}) for $\mathrm{d} \boldsymbol{T}$ into the second integral in Eqn.~\eqref{diffresidual}, while employing Eqns.~(\ref{dEL} \&\ \ref{dEN}) to describe strain rate $\mathrm{d}\boldsymbol{E}$, this integral can be re-written as 
\begin{subequations}
    \label{tangentStiffness}
    \begin{align}
    \int_V \mathbf{B}^{\mathsf{T}} \, \mathrm{d} \boldsymbol{T} \, \mathrm{d}V & = 
    \int_V \mathbf{B}^{\mathsf{T}} \, \boldsymbol{\mathcal{M}}^t \, \mathrm{d} \boldsymbol{E} \, \mathrm{d}V \\ 
    \mbox{} & = \underbrace{\int_V \mathbf{B}^{\mathsf{T}} \, \boldsymbol{\mathcal{M}}^t \, 
    \mathbf{B} \, \mathrm{d}V}_{\mathbf{C}^t} \mathrm{d}\boldsymbol{u}^{(e)} + 
    \underbrace{\int_V \mathbf{B}^{\mathsf{T}} \, \boldsymbol{\mathcal{M}}^t \, \mathrm{d}
    \mathbf{B} \, \mathrm{d} V}_{\mathbf{K}^t} \, \boldsymbol{u}^{(e)} \\
    \intertext{where the contribution to the secant stiffness can be expressed alternatively as}
    \mathbf{K}^t & = \int_V \mathbf{H}^{\mathsf{T}} \, \mathrm{d} \mathbf{S} \, 
    \mathbf{H} \, \mathrm{d} V 
    \quad \text{given} \quad
    \mathrm{d} \mathbf{S} \defeq \mathbf{A}^{\mathsf{T}} \, 
    \boldsymbol{\mathcal{M}}^t \, \mathrm{d} \mathbf{A}
    \end{align}
\end{subequations}
because of Eqn.~\eqref{decomposeNonlinearB}.


\subsection{Equations of Motion}

Pulling everything together, the equations of motion~\eqref{FEsysOfEqns}, when written for an element, are given by
\begin{subequations}
    \label{EOM}
    \begin{align}
    \mbox{} & \mathbf{M} \, \ddot{\boldsymbol{u}}^{(e)} + 
    \mathbf{C} \, \dot{\boldsymbol{u}}^{(e)} + 
    \mathbf{K} \, \boldsymbol{u}^{(e)} = \boldsymbol{F} & & & \\
    \intertext{which has a tangent stiffness matrix of}
    \mbox{} & \mathbf{C} = \mathbf{C}^s + \mathbf{C}^t & & & \\
    \intertext{a secant stiffness matrix of}
    \mbox{} & \mathbf{K} = \mathbf{K}^s + \mathbf{K}^t & & & \\
    \intertext{and a forcing function of}    
    \mbox{} & \boldsymbol{F} = \boldsymbol{F}_{BC} - \boldsymbol{F}_0 & & & \\
    \intertext{wherein} 
    \mbox{} & \mathbf{C}^s = \int_V \mathbf{H}^{\mathsf{T}} \, 
        \mathbf{S}^s \, \mathbf{H} \, \mathrm{d}V
        & & \text{where} & \boldsymbol{T}_0 + 
        \boldsymbol{\mathcal{M}}^s \boldsymbol{E} \mapsto \mathbf{S}^s \\
    \mbox{} & \mathbf{C}^t = \int_V \mathbf{B}^{\mathsf{T}} \, 
        \boldsymbol{\mathcal{M}}^t \, \mathbf{B} \, \mathrm{d}V & & & \\
    \mbox{} & \mathbf{K}^s = \int_V \mathbf{B}^{\mathsf{T}} \, 
        \boldsymbol{\mathcal{M}}^s \, \mathbf{B} \, \mathrm{d}V & & & \\
    \mbox{} & \mathbf{K}^t = \int_V \mathbf{H}^{\mathsf{T}} \, 
        \mathrm{d}\mathbf{S}^t \, \mathbf{H} \, \mathrm{d}V 
        & & \text{where} &
        \mathrm{d}\mathbf{S}^t = \mathbf{A}^{\mathsf{T}} \, 
        \boldsymbol{\mathcal{M}}^t \, \mathrm{d} \mathbf{A} \\
    \mbox{} & \boldsymbol{F}_0 = \int_V \mathbf{B}^{\mathsf{T}} \, \boldsymbol{T}_0 \,
        \mathrm{d}V & & &
    \end{align} 
\end{subequations}
with $\boldsymbol{F}_{BC}$ being an external force associated with the boundary conditions evaluated at the end of a solution step.  All other fields are evaluated at the beginning of this solution step.  Superscript $\mbox{}^s$ implies that these matrices are evaluated using the secant modulus $\boldsymbol{\mathcal{M}}^s$, while superscript $\mbox{}^t$ implies that these matrices are evaluated using the tangent modulus $\boldsymbol{\mathcal{M}}^t$.  There are contributions from both moduli in both stiffness matrices.

To minimize an accumulation of roundoff error, it is advantageous to compute $\mathbf{K}^s$ as four separate integrals, viz.,
\begin{multline*}
\mathbf{K}^s = \int_V \mathbf{B}_L^{\mathsf{T}} \, 
\boldsymbol{\mathcal{M}}^s \, \mathbf{B}_L \, \mathrm{d}V + 
\int_V \mathbf{B}_L^{\mathsf{T}} \, \boldsymbol{\mathcal{M}}^s \, 
\mathbf{B}_N \, \mathrm{d}V + 
\int_V \mathbf{B}_N^{\mathsf{T}} \, \boldsymbol{\mathcal{M}}^s \, 
\mathbf{B}_L \, \mathrm{d}V + 
\int_V \mathbf{B}_N^{\mathsf{T}} \, \boldsymbol{\mathcal{M}}^s \, 
\mathbf{B}_N \, \mathrm{d}V
\end{multline*}
to compute $\mathbf{C}^t$ as four separate integrals, viz.,
\begin{multline*}
\mathbf{C}^t = \int_V \mathbf{B}_L^{\mathsf{T}} \, 
\boldsymbol{\mathcal{M}}^t \, \mathbf{B}_L \, \mathrm{d}V + 
\int_V \mathbf{B}_L^{\mathsf{T}} \, \boldsymbol{\mathcal{M}}^t \, 
\mathbf{B}_N \, \mathrm{d}V + 
\int_V \mathbf{B}_N^{\mathsf{T}} \, \boldsymbol{\mathcal{M}}^t \, 
\mathbf{B}_L \, \mathrm{d}V + 
\int_V \mathbf{B}_N^{\mathsf{T}} \, \boldsymbol{\mathcal{M}}^t \, 
\mathbf{B}_N \, \mathrm{d}V
\end{multline*}
and to compute $\boldsymbol{F}_0$ as two separate integrals, viz.,
\begin{equation*}
    \boldsymbol{F}_0 = \int_V \mathbf{B}_L^{\mathsf{T}} \, 
        \boldsymbol{T}_0 \, \mathrm{d}V + 
        \int_V \mathbf{B}_N^{\mathsf{T}} \, 
        \boldsymbol{T}_0 \, \mathrm{d}V 
\end{equation*}
where the first integral will only need to be evaluated once, as its argument is constant valued.

The vector and matrices established in Eqn.~\eqref{EOM} pertain to a single finite element.  These arrays are to be assembled using standard techniques \cite{ClaytonChung18} when describing a finite element model.  They will assemble into the form of Eqn.~\eqref{FEsysOfEqns}, at which point these equations of motion can be solved.


\section{Kinematic Matrices of Finite Elements}

To implement our finite element discretization, it is necessary that we know the following matrices for a given element type: the linear strain-displacement matrix $\mathbf{B}_L$, the nonlinear strain-displacement matrix $\mathbf{B}_N$ and its decomposition $\mathbf{B}_N = \mathbf{AH}$, plus the differential rate $\mathrm{d}\mathbf{A}$.  These matrices are acquired in the following sections for a chord, a pentagon, and a dodecahedron where \textbf{QR} kinematics are adopted.

\subsection{Kinematic Matrices for a Chord}

The components of Laplace stretch $\boldsymbol{\mathcal{U}}$ can be obtained from a Cholesky factorization of the right, Cauchy-Green, deformation tensor $\mathbf{C} = \mathbf{F}^{\mathsf{T}} \mathbf{F} = \boldsymbol{\mathcal{U}}^{\mathsf{T}} \boldsymbol{\mathcal{U}}$, \cite{Srinivasa12} which is a symmetric second-order tensor. For a 1D chord, the only possible deformation is a stretch of the chord in its axial direction. Therefore, in this case, the deformation gradient, as-well-as the right Cauchy-Green tensor $\mathbf{C}$, have only one component.  Consequently, the Laplace stretch $\boldsymbol{\mathcal{U}}$ also consists of only one component, which is denoted by $a$. If $u$ is the axial displacement of a chord, then its axial elongation $a$ becomes
\begin{subequations}
    \begin{gather}
    a = {\mathcal{U}}_{11} = \sqrt{C_{11}} \\
    \intertext{with} 
    \mathrm{C_{11}} = 1 + 2 \, \frac{\mathrm{\partial u}}{\partial x} + \left(\frac{\mathrm{\partial u}}{\partial x}\right)^2 
    \quad \text{given} \quad 
    F_{11} = 1 + \frac{\mathrm{\partial u}}{\partial x} .
    \end{gather}
\end{subequations} 
This chord is subjected to an axial strain defined as $e = \ln ( a ) = \ln ( L / L_0 )$, where $L_0$ and $L$ are the initial and current lengths of the chord.  Here we decompose the total strain into its linear and nonlinear components as
\begin{equation}
e = e_{L} + e_{N}
\end{equation} 
as determined by a Taylor expansion of $e = \ln \sqrt{C_{11}}$, which gives
\begin{equation}
    e_{L} = \mathrm{\frac{\partial u}{\partial x}} 
    \quad \text{and} \quad
    e_{N} = -\frac{1}{2} \, \mathrm{\frac{\partial u}{\partial x}}\, \mathrm{\frac{\partial u}{\partial x}} 
\end{equation}
where all contributions have been truncated beyond the quadratic term in this expansion.

The linear strain-displacement matrix $\mathbf{B}_L$ can now be obtained by expressing the linear strain $e_L$ in terms of its nodal displacements, viz.,
\begin{subequations}
    \begin{gather}
    e_L  = \frac{\partial u}{\partial x}  =
    \sum\nolimits_{i=1}^2 N_{i,x} \, u_i = 
    \bigl[ 
    [\mathbf{b}_{L1}] [\mathbf{b}_{L2}]
    \bigr] \bigl\{ \begin{matrix} 
    \mathbf{u}^{(e)} 
    \end{matrix} \bigr\} 
    = \bigl[ \mathbf{B}_L \bigr] \bigl\{ \mathbf{u}^{(e)} \bigr\} \\
    \intertext{wherein} 
    [ \mathbf{b}_{Li} ] = [ N_{i,x} ] = [ N_{i,\xi} ] [ \mathbf{J} ]^{-1}
    \quad \text{and} \quad
    \mathbf{u}^{(e)}  = 
    \bigl\{ \begin{matrix}
    u^{(e)}_1 & u^{(e)}_2
    \end{matrix} \bigr\}^{\mathsf{T}}
    \end{gather}
\end{subequations}
where $N_{i,\xi}$ is the gradient of shape function $N_i$ evaluated in element $(e)$'s natural co-ordinate system, which maps into gradient $N_{i,x}$ evaluated in the element's physical co-ordinate system via its Jacobian matrix $[ \mathbf{J} ]$, with $u^{(e)}_1$ and $u^{(e)}_2$ denoting the two nodal displacements of the chord.  Shape functions $N_i$ and their gradients $N_{i,\xi}$ for a chord are given in Eqn.~(\ref{shapeFnsChord}), whose Jacobian matrix $[ \mathbf{J} ]$ can be found in Eqn.~\eqref{detJac1D}.

We now introduce machinery that is excessive for the chord, but becomes very useful when constructing the nonlinear strain-displacement matrices for a pentagon and a tetrahedron.  Let nonlinear strain $e_N$ be written as a product between some matrix $\mathbf{A}$ and some vector $\boldsymbol{\theta}$; specifically, let
\begin{equation}
e_ N =  \tfrac{1}{2} \,
[-\partial u / \partial x ]
\{\partial u / \partial x\}
= \tfrac{1}{2} \, \mathbf{A} \, \boldsymbol{\theta} 
\end{equation}
where $\mathbf{A} = [ -\partial u / \partial x ]$ whose differential is
\begin{subequations}
	\begin{gather}
	\mathrm{d} \mathbf{A} =  \bigl\{-
	\partial  \mathrm{d} u / \partial x
	\bigr\}
	= \bigl\{ \begin{matrix}
	- \sum\nolimits_{i=1}^2 N_{i,x} \,  \mathrm{d} u_i
	\end{matrix} \bigr\}
	= \bigl[ [\mathbf{l}_1] [\mathbf{l}_2] \bigr] \bigl[ [\mathbf{d}_1] [\mathbf{d}_2] \bigr]^{\mathsf{T}}
	= \mathbf{L} \, \mathbf{D}  \\
	\intertext{wherein}
	[\mathbf{l}_i] = [-N_{i,x}] = [-N_{i,\xi}] [ \mathbf{J}]^{-1} \quad \text{and} \quad  [\mathbf{d}_i] = [\mathrm{d} u_i] .
	\end{gather}
\end{subequations}

Furthermore, we consider that $\boldsymbol{\theta}$ can be expressed in terms of the nodal displacements as
\begin{subequations}
    \begin{gather}
    \boldsymbol{\theta} =  \bigl\{
    \partial u / \partial x
    \bigr\}
    = \bigl\{ \begin{matrix}
    \sum\nolimits_{i=1}^2 N_{i,x} \, u_i
    \end{matrix} \bigr\}
    = \bigl[ [\mathbf{h}_1] [\mathbf{h}_2] \bigr] 
    \bigl\{ \mathbf{u}^{(e)} \bigr\}
    = \mathbf{H} \, \mathbf{u}^{(e)} \\
    \intertext{wherein}
    \mathbf{H} = \bigl[ [\mathbf{h}_1] [\mathbf{h}_2] \bigr]
    \quad \text{with} \quad
    [\mathbf{h}_i] = [N_{i,x}] = [N_{i,\xi}] [ \mathbf{J}]^{-1}
    \end{gather}
\end{subequations}
and we see that, for the chord, there is no difference between $\mathbf{b}_{Li}$ and $\mathbf{h}_i$, which will not be the case in higher-dimensional spaces.  Hence, the nonlinear strain-displacement matrix $\mathbf{B}_N$ can be written as
\begin{equation}
    \mathbf{B}_N = \mathbf{A} \, \mathbf{H}  =  
    \bigl[ [\mathbf{b}_{N1}] [\mathbf{b}_{N2}] \bigr]
\end{equation}
where $\mathbf{b}_{Ni} = [-\partial u / \partial x] [ \mathbf{h}_i ]$.

\subsubsection{Tangent Stiffness Matrix $\mathbf{C}^s$}

The tangent stiffness matrix $\mathbf{C}^s$  associated with $\boldsymbol{T}_0 + 
\boldsymbol{\mathcal{M}}^s \boldsymbol{E} \mapsto \mathbf{S}^s = [ s_0 + E^s e ]$, which is defined in Eqn.~\eqref{elasticChordProperties}, becomes 
\begin{equation}
	\begin{aligned}
		\mathbf{C}^s & = \int_{L} \mathbf{H}^{\mathsf{T}} \,  \mathbf{S}^s \, \mathbf{H} \, A \, \mathrm{d} L
	     = | \mathbf{J}_0 |  \, \sum_{i=1}^{n} \mathbf{H}^{\mathsf{T}} \, \mathbf{S}^s (\xi_i) \, \mathbf{H} \, A_0 (\xi_i)  \, \mathrm{w}_i 
	\end{aligned}
\end{equation}
where an isochoric response is assumed in that $A_0 | \mathbf{J}_0 | = A | \mathbf{J} |$. Here $\xi_i$ and $w_i$ are the co-ordinates and weights of quadrature for Gauss point $i$, and $A_0$ and $A$ are the initial and current cross-sectional areas of the chord with $A_0 (\xi_i)$ being the initial cross-sectional area at Gauss point~$\xi_i$.


\subsubsection{Tangent Stiffness Matrix $\mathbf{C}^t$}

The tangent stiffness matrix $\mathbf{C}^t$, as established in Eqn.~\eqref{tangentStiffness}, becomes
\begin{subequations}
	\begin{align}
		\mathbf{C}^t & = \int_L \mathbf{B}^{\mathsf{T}} \,  \boldsymbol{\mathcal{M}}^t \, \mathbf{B}  \, A \,  \mathrm{d} L
		= |\mathbf{J}_0| \sum_{i=1}^{n}   \mathbf{B}^{\mathsf{T}} \, \boldsymbol{\mathcal{M}}^t (\xi_i) \, \mathbf{B} \, A_0 (\xi_i) \, \mathrm{w}_i 
	\end{align}
\end{subequations} 
where axial stress rate $\mathrm{d} s$ is described by a tangent modulus $\boldsymbol{\mathcal{M}}^t$ from Eqn.~(\ref{fiberTangentModulus}).


\subsubsection{Secant Stiffness Matrix $\mathbf{K}^s$}

The secant stiffness matrix $\mathbf{K}^s$, as established in Eqn.~\eqref{secantStiffnessMatrix}, becomes
\begin{subequations}
	\begin{align}
		\mathbf{K}^s & = \int_L \mathbf{B}^{\mathsf{T}} \,  \boldsymbol{\mathcal{M}}^s \, \mathbf{B}  \, A \,  \mathrm{d} L
		= |\mathbf{J}_0| \sum_{i=1}^{n}  \mathbf{B}^{\mathsf{T}} \, \boldsymbol{\mathcal{M}}^s (\xi_i) \, \mathbf{B} \, A_0 (\xi_i) \, \mathrm{w}_i 
	\end{align}
\end{subequations} 
where axial stress rate $\mathrm{d} s$ is described by a secant modulus $\boldsymbol{\mathcal{M}}^{s}$ from Eqn.~(\ref{elasticChordProperties}).


\subsubsection{Secant Stiffness Matrix $\mathbf{K}^t$}

Likewise, a secant stiffness matrix $\mathbf{K}^t$, also established in Eqn.~\eqref{tangentStiffness}, becomes 
\begin{subequations}
	\begin{align}
		\mathbf{K}^t & = \int_L \mathbf{H}^{\mathsf{T}} \,  \mathrm{d} \mathbf{S}^t \, \mathbf{H}  \, A \,  \mathrm{d} L
		= | \mathbf{J}_0 | \sum_{i=1}^{n}  \mathbf{H}^{\mathsf{T}} \,  \mathrm{d} \mathbf{S}^t (\xi_i) \, \mathbf{H} \, A_0 (\xi_i) \, \mathrm{w}_i
	\end{align}
\end{subequations}
where $\mathrm{d} \mathbf{S}^t \defeq \mathbf{A}^{\mathsf{T}} \, \boldsymbol{\mathcal{M}}^t \, \mathrm{d} \mathbf{A}$.


\subsection{Kinematic Matrices for a Pentagon}

For a planar member, components of Laplace stretch $\boldsymbol{\mathcal{U}}$, obtained from a Cholesky factorization of the right Cauchy-Green tensor $\mathbf{C}\defeq \mathbf{F}^{\mathsf{T}} \mathbf{F} = \boldsymbol{\mathcal{U}}^{\mathsf{T}} \boldsymbol{\mathcal{U}}$, cf.\ Eqn.~\eqref{LaplaceStretch2D}, such that \cite{Freedetal17}
\begin{equation}
\begin{aligned}
{\mathcal{U}}_{11} = a & = \sqrt{C_{11}} \;\; & 
{\mathcal{U}}_{12} = a\:g & = C_{12} / {\mathcal{U}_{11}} \\
{\mathcal{U}}_{21} & = 0 &
{\mathcal{U}}_{22} = b & = \sqrt{C_{22} - ({\mathcal{U}}_{12})^2} 
\end{aligned}
\label{Laplace stretchComponents}
\end{equation} 
where $a$ and $b$ are elongations (stretches) in the 1 and 2~directions, respectively, and $g$ is a magnitude for shear in the 12~plane, while ${C_{11}}$, ${C_{12}} \! = \! {C_{21}}$ and ${C_{22}}$ are components of the right Cauchy-Green tensor $\mathbf{C}$.  Furthermore, components from this Cauchy-Green tensor can be expressed in terms of displacement gradients as
\begin{subequations}
    \begin{align}
    \mathrm{C_{11}} & = 1 + 2\, \frac{\partial u}{\partial x} + \left(\frac{\partial u}{\partial x}\right)^2 + \left(\frac{\partial v}{\partial x}\right)^2 \\
    \mathrm{C_{12}} & = \frac{\partial u}{\partial y} + \frac{\partial v}{\partial x} + \frac{\partial u}{\partial x} \cdot \frac{\partial u}{\partial y} + \frac{\partial v}{\partial x} \cdot \frac{\partial v}{\partial y}\\
    \mathrm{C_{22}} & = 1 + 2\, \frac{\partial v}{\partial y} + \left(\frac{\partial u}{\partial y}\right)^2 + \left(\frac{\partial v}{\partial y}\right)^2
    \intertext{which arise from the deformation gradient}
    \mathbf{F} & =  
    \begin{bmatrix}
    1 + \partial u / \partial x & \partial u / \partial y  \\
    \partial v / \partial x & 1 + \partial v / \partial y
    \end{bmatrix}
    \end{align}
\end{subequations}
where $u$ and $v$ are displacements associated with deformation of a planar member.

Gradients of shape functions are used to construct the above spatial gradients, viz.,
\begin{displaymath}
    \begin{Bmatrix} 
    N_{i,\xi} \\ 
    N_{i,\eta}
    \end{Bmatrix} = \begin{bmatrix}
    \partial x / \partial \xi & \partial y / \partial \xi \\
    \partial x / \partial \eta & \partial y / \partial \eta
    \end{bmatrix} \begin{Bmatrix}
    N_{i,x} \\
    N_{i,y}
    \end{Bmatrix}
\end{displaymath}
whose matrix is the non-singular Jacobian defined in Eqn.~(\ref{jacobian2D}), while $N_{i,\xi}$ and $N_{i,\eta}$ are gradients of the shape functions in their natural co-ordinates, as established in Eqn.~\eqref{shapeFunctionGradients} for pentagons.  These are evaulated at the $i^{\text{th}}$ Gauss point for the quadrature rule used that, in our case, is found in Eqns.~(\ref{pentagonCoordinates} \& \ref{pentagonQuadrature}).  It is necessary to invert this equation for it to become useful for us so that
\begin{subequations}
    \label{pentagonGrads}
    \begin{align}
    \begin{Bmatrix} 
    N_{i,x} \\ 
    N_{i,y}
    \end{Bmatrix} & = \begin{bmatrix}
    \partial x / \partial \xi & \partial y / \partial \xi \\
    \partial x / \partial \eta & \partial y / \partial \eta
    \end{bmatrix}^{-1} \begin{Bmatrix}
    N_{i,\xi} \\
    N_{i,\eta}
    \end{Bmatrix} \\
    \intertext{from which one determines}
    \begin{Bmatrix}
    \partial u / \partial x \\
    \partial u / \partial y \\
    \partial v / \partial x \\
    \partial v / \partial y 
    \end{Bmatrix} & = \begin{Bmatrix}
    \sum\nolimits_{i=1}^5 N_{i,x} u_i \\
    \sum\nolimits_{i=1}^5 N_{i,y} u_i \\
    \sum\nolimits_{i=1}^5 N_{i,x} v_i \\
    \sum\nolimits_{i=1}^5 N_{i,y} v_i
    \end{Bmatrix}
    \end{align}
\end{subequations}
with $N_{i,x}$ and $N_{i,y}$ being employed below.

The thermodynamic strain attributes that we use are defined in Eqn.~\eqref{thermodynamicStrains2D} and can be expressed in terms of the components of Laplace stretch as
\begin{equation}
    \xi = \ln\left( \sqrt{\frac{a}{a_0} \frac{b}{b_0}} \right) ,
    \quad
    \varepsilon = \ln\left(\sqrt{\frac{a}{a_0} \frac{b_0}{b}} \right) ,
    \quad 
    \gamma = g - g_0.
    \label{strainAttributesPentagon}
\end{equation}
Without loss of generality, we consider the membrane to be initially undeformed, which allows us to set $a_0$ and $b_0$ to one, while the initial shear $g_0$ is taken as zero.  To gain computational advantage, we decompose these strain attributes into linear and nonlinear components; specifically, we consider
\begin{equation}
    \xi = \xi_{L} + \xi_{N1} + \xi_{N2} , \quad
    \varepsilon = \varepsilon_{L} + \varepsilon_{N1} + \varepsilon_{N2} , \quad
    \gamma = \gamma_{L} + \gamma_{N1} + \gamma_{N2}.
    \label{totalVirtualStrain}
\end{equation}
Traditionally, finite element constructions decompose strain into a linear component and a non\-linear component.  However, in our case, a further decomposition of the non\-linear strain component into two separate components makes our computation much easier, as will be realized later. 

Decomposition of strain attributes in Eqn.~\eqref{strainAttributesPentagon} is achieved via Taylor expansions that retain terms through second-order.  The linear and nonlinear components of these strain attributes, thus obtained, are given as
\begin{subequations}
	\begin{align}
		\mathrm \xi_{L} & = \frac{1}{2} \, \left(\mathrm{\frac{\partial u}{\partial x}} + \mathrm{\frac{\partial v}{\partial y}}\right)\\
		\mathrm \xi_{N} & = \frac{1}{4} \, \left(- \mathrm{\frac{ \partial v}{\partial y}}\, \mathrm{\frac{ \partial v}{\partial y}} -\mathrm{\frac{\partial u}{\partial x}}\, \mathrm{\frac{\partial u}{\partial x}} - 2 \, \mathrm{\frac{\partial u}{\partial y}}\, \mathrm{\frac{\partial v}{\partial x}}\right)\\
		\mathrm \varepsilon_{L} & = \frac{1}{2} \, \left(\mathrm{\frac{\partial u}{\partial x}} - \mathrm{\frac{\partial v}{\partial y}}\right)\\
		\mathrm \varepsilon_{N} & = \frac{1}{4} \, \left(\mathrm{\frac{ \partial v}{\partial y}}\, \mathrm{\frac{ \partial v}{\partial y}} -\mathrm{\frac{\partial u}{\partial x}}\, \mathrm{\frac{\partial u}{\partial x}} + 2 \, \mathrm{\frac{\partial u}{\partial y}}\, \mathrm{\frac{\partial v}{\partial x}}\right)\\
		\mathrm \gamma_{L} & = \mathrm{\frac{\partial u}{\partial y}} + \mathrm{\frac{\partial v}{\partial x}}\\
		\mathrm \gamma_{N} & = \mathrm{\frac{\partial v}{\partial x}}\, \mathrm{\frac{\partial v}{\partial y}} - 2\, \mathrm{\frac{ \partial u}{\partial x}}\, \mathrm{\frac{ \partial v}{\partial x}} 
		-\mathrm{\frac{\partial u}{\partial x}}\, \mathrm{\frac{\partial u}{\partial y}} 
	\end{align}
\end{subequations}
where the linear components of these strain attributes consist only of first-order derivatives in the displacements, while the nonlinear components contain the second-order terms. (Recall that terms with third-order and higher derivatives in displacement have been truncated.) The nonlinear components of strain have been arranged in such a way that when represented in a set of base vectors, they can be written as the product of a matrix and a vector that contain these derivatives of displacement.  To achieve that, for the dilatation $\xi$ and squeeze $\varepsilon$ strains, the nonlinear components contain squares of derivatives and products between different derivatives of displacement. Note that the nonlinear part of the total shear strain contains only products of different derivatives of displacements, i.e., no square term is present in its expression. 

In terms of the nodal displacements the vector containing the linear strain attributes, $\boldsymbol{E}_L$ can be written as
\begin{subequations}
\begin{align}
    \boldsymbol{E}_L & = \begin{Bmatrix}
    \xi_{L} \\
    \varepsilon_{L} \\
    \gamma_{L} \end{Bmatrix} = 
    \begin{Bmatrix}
    \tfrac{1}{2} \, u_{,x} + \tfrac{1}{2} \, v_{,y} \\
    \tfrac{1}{2} \, u_{,x} - \tfrac{1}{2} \, v_{,y} \\
    u_{,y} + v_{,x} \end{Bmatrix} = 
    \sum_{i=1}^5 \begin{bmatrix}
    \tfrac{1}{2} \, N_{i,x} & \tfrac{1}{2} \, N_{i,y}  \\
    \tfrac{1}{2} \, N_{i,x} & -\tfrac{1}{2} \, N_{i,y} \\ 
    N_{i,y} & N_{i,x}  \end{bmatrix} \, 
    \begin{Bmatrix}
    u_{i} \\
    v_{i} 
    \end{Bmatrix} \notag \\
    & = \bigl[ [\mathbf{b}_{L1}] [\mathbf{b}_{L2}] [\mathbf{b}_{L3}] 
    [\mathbf{b}_{L4}] [\mathbf{b}_{L5}] \bigr]
    \bigl\{ \mathbf{u}^{(e)} \bigr\}  
    = \mathbf{B}_L \, \mathbf{u}^{(e)}  \\
    \intertext{where}
    [\mathbf{b}_{Li}] & = \begin{bmatrix}
    \tfrac{1}{2} \, N_{i,x}  &  \tfrac{1}{2} \, N_{i,y} \\
    \tfrac{1}{2} \, N_{i,x}  & - \tfrac{1}{2} \, N_{i,y} \\
    N_{i,y} & N_{i,x} 
    \end{bmatrix} \\
    \mathbf{u}^{(e)} & = \bigl\{ \begin{matrix}
    u_1 & v_1 & u_2 & v_2 & u_3 & v_3 & u_4 & v_4 & u_5 & v_5 
    \end{matrix} \bigr\}^{\mathsf{T}}
\end{align}
\end{subequations}
for element $e$, whose matrix entries come from Eqn.~\eqref{pentagonGrads}.

Now let nonlinear strain $\boldsymbol{E}_{N1}$ be written as a product between some matrix $\mathbf{A}_1$ and some vector $\boldsymbol{\theta}_1$; specifically, let
\begin{equation}
\begin{aligned}
\mathbf{E}_{N1} & =  \begin{Bmatrix}
\mathrm \xi_{N1} \\
\mathrm \varepsilon_{N1} \\
\mathrm \gamma_{N1} \end{Bmatrix} =
\begin{Bmatrix}
-\tfrac{1}{4} \, u_{,x}^2 -   \tfrac{1}{4} \, v_{,y}^2 \\
-\tfrac{1}{4} \, u_{,x}^2 + \tfrac{1}{4} \, v_{,y}^2 \\
-u_{,y} \, u_{,x} + v_{,x} \, v_{,y} \end{Bmatrix} \notag \\
& = \frac{1}{2} \, \begin{bmatrix}
-\tfrac{1}{2} \, \partial u / \partial x &  - \tfrac{1}{2} \, \partial v / \partial y \\
-\tfrac{1}{2} \, \partial u / \partial x &  \tfrac{1}{2} \, \partial v / \partial y \\
-2 \, \partial u / \partial y  & 2 \, \partial v / \partial x  \end{bmatrix} \, \begin{Bmatrix}
\partial u / \partial x\\
\partial v / \partial y
\end{Bmatrix}\\
& = \tfrac{1}{2} \, \mathbf{A}_1 \, \boldsymbol{\theta}_1
\end{aligned}
\end{equation}
with 
\begin{subequations}
	\begin{align}
	\mathrm{d} \mathbf{A}_1 & =  \begin{bmatrix}
	-\tfrac{1}{2} \, \partial \mathrm{d} u / \partial x &  - \tfrac{1}{2} \, \partial \mathrm{d} v / \partial y \\
	-\tfrac{1}{2} \, \partial \mathrm{d} u / \partial x &  \tfrac{1}{2} \, \partial \mathrm{d} v / \partial y \\
	-2 \, \partial \mathrm{d} u / \partial y  & 2 \, \partial \mathrm{d} v / \partial x  \end{bmatrix} \notag \\
	& = \begin{bmatrix}
	-\tfrac{1}{2} \, \sum\nolimits_{i=1}^5 N_{i,x} \, \mathrm{d} u_i &  - \tfrac{1}{2} \, \sum\nolimits_{i=1}^5 N_{i,y} \, \mathrm{d} v_i \\
	-\tfrac{1}{2} \, \sum\nolimits_{i=1}^5 N_{i,x} \, \mathrm{d} u_i &  \tfrac{1}{2} \, \sum\nolimits_{i=1}^5 N_{i,y} \, \mathrm{d} v_i \\
	-2 \, \sum\nolimits_{i=1}^5 N_{i,y} \, \mathrm{d} u_i  & 2 \, \sum\nolimits_{i=1}^5 N_{i,x} \, \mathrm{d} v_i  \end{bmatrix}\notag \\
	& = \begin{bmatrix}
	[\mathbf{l}_1] [\mathbf{l}_2] [\mathbf{l}_3] [\mathbf{l}_4] [\mathbf{l}_5] 
	\end{bmatrix} \, \begin{bmatrix} [\mathbf{d}_1] [\mathbf{d}_2] [\mathbf{d}_3] [\mathbf{d}_4] [\mathbf{d}_5] \end{bmatrix}^{\mathsf{T}}
	= \mathbf{L}_1 \mathbf{D}_1 \\
	\intertext{wherein}
	[\mathbf{l}_i] & = \begin{bmatrix}
	-\tfrac{1}{2} \, N_{i,x} &  - \tfrac{1}{2} \, N_{i,y} \\
	-\tfrac{1}{2} \, N_{i,x} &  \tfrac{1}{2} \, N_{i,y} \\
	-2 \, N_{i,y} \, & 2 \, N_{i,x} \end{bmatrix}  \quad \text{and} \quad	[\mathbf{d}_i] = \begin{bmatrix}
	\mathrm{d} u_i & 0 \\
	0 &  \mathrm{d} v_i  \end{bmatrix} .
	\end{align}
\end{subequations}
To obtain the nonlinear strain--displacement matrix, we require the nonlinear strain to be expressed in terms of the nodal displacements. This is achieved by expressing the elements of displacement gradient in terms of the nodal displacements by using the shape functions, specifically, the vector $\boldsymbol{\theta}_1$ can be written as
\small
\begin{equation}
\boldsymbol{\theta}_1 =  \begin{Bmatrix}
\partial u / \partial x\\
\partial v / \partial y
\end{Bmatrix}
= \begin{Bmatrix}
\sum\nolimits_{i=1}^5 N_{i,x} \, u_i\\
\sum\nolimits_{i=1}^5 N_{i,y} \, v_i
\end{Bmatrix} 
= \begin{bmatrix}
[\mathbf{h}_1] [\mathbf{h}_2] [\mathbf{h}_3] [\mathbf{h}_4] [\mathbf{h}_5] 
\end{bmatrix} \bigl\{ \begin{matrix} \mathbf{u}^{(e)} \end{matrix} \bigr\} \\ 
= \mathbf{H}_1  \mathbf{u}^{(e)}
\end{equation}
\normalsize
where the components of $\mathbf{H}_1$ contains the derivatives of shape functions with respect to spatial variables, i.e.,
\begin{equation}
[\mathbf{h}_i] = \begin{bmatrix}
N_{i,x} &  0  \\
0 & N_{i,y}  \end{bmatrix} .
\end{equation}
Therefore, the first nonlinear strain--displacement matrix $\mathbf{B}_{N1}$ can be written as
\begin{equation}
\mathbf{B}_{N1} = \mathbf{A}_1 \, \mathbf{H}_1 = \begin{bmatrix}
[\mathbf{b}_{N1}] [\mathbf{b}_{N2}] [\mathbf{b}_{N3}] [\mathbf{b}_{N4}] [\mathbf{b}_{N5}] 
\end{bmatrix} 
\end{equation}
where the components of $\mathbf{B}_{N1}$ are given as
\begin{equation}
[\mathbf{b}_{Ni}] = \begin{bmatrix}
-\tfrac{1}{2} \, \partial u / \partial x &  - \tfrac{1}{2} \, \partial v / \partial y \\
-\tfrac{1}{2} \, \partial u / \partial x &  \tfrac{1}{2} \, \partial v / \partial y \\
-2 \, \partial u / \partial y  & 2 \, \partial v / \partial x  \end{bmatrix}  \, \begin{bmatrix}
N_{i,x} &  0  \\
0 & N_{i,y}  \end{bmatrix}. 
\end{equation}

In a similar manner, the second nonlinear strain terms can be written as
\begin{equation}
\begin{aligned}
\mathbf{E}_{N2} & =  \begin{Bmatrix}
\mathrm \xi_{N2} \\
\mathrm \varepsilon_{N2} \\
\mathrm \gamma_{N2} \end{Bmatrix} =
\begin{Bmatrix}
-\tfrac{1}{2} \, u_{,y} \, v_{,x}  \\
\tfrac{1}{2} \, u_{,y} \, v_{,x} \\
-2 \, u_{,x} \, v_{,x} \end{Bmatrix} \\ 
& = \frac{1}{2} \, \begin{bmatrix}
- \partial v / \partial x &  0 \\
\partial v / \partial x & 0 \\
0 & -4 \, \partial u / \partial x  \end{bmatrix} \, \begin{Bmatrix}
\partial u / \partial y\\
\partial v / \partial x
\end{Bmatrix} \\
& = \tfrac{1}{2} \, \mathbf{A}_2 \, \boldsymbol{\theta}_2
\end{aligned}
\end{equation}
with  
\begin{subequations}
	\begin{align}
	\mathrm{d} \mathbf{A}_2 & =  \begin{bmatrix}
	- \partial \mathrm{d} v / \partial x &  0 \\
	\partial \mathrm{d} v / \partial x & 0 \\
	0 & -4 \, \partial \mathrm{d} u / \partial x  \end{bmatrix} \notag \\
	& = \begin{bmatrix}
	- \sum\nolimits_{i=1}^5 N_{i,x} \, \mathrm{d} v_i &  0 \\
	\sum\nolimits_{i=1}^5 N_{i,x} \, \mathrm{d} v_i &  0 \\
	0  & - 4 \, \sum\nolimits_{i=1}^5 N_{i,x} \, \mathrm{d} u_i  \end{bmatrix} \notag \\
	& = \begin{bmatrix}
	[\mathbf{l}_1] [\mathbf{l}_2] [\mathbf{l}_3] [\mathbf{l}_4] [\mathbf{l}_5] 
	\end{bmatrix}  \, \begin{bmatrix} [\mathbf{d}_1] [\mathbf{d}_2] [\mathbf{d}_3] [\mathbf{d}_4] [\mathbf{d}_5] \end{bmatrix}^{\mathsf{T}}
	= \mathbf{L}_2 \mathbf{D}_2 \\
	\intertext{wherein}
	[\mathbf{l}_i] & = \begin{bmatrix}
	- N_{i,x} &  0 \\
	N_{i,x} & 0 \\
	0  & - 4 \, N_{i,x} \end{bmatrix}  \quad \text{and} \quad	[\mathbf{d}_i] = \begin{bmatrix}
	\mathrm{d} v_i & 0 \\
	0 &  \mathrm{d} u_i  \end{bmatrix} .
	\end{align}
\end{subequations}
The vector $\boldsymbol{\theta}_2$ is expressed in terms of the nodal displacements with the use of shape functions as
\small
\begin{equation}
\boldsymbol{\theta}_2 =  \begin{Bmatrix}
\partial u / \partial y\\
\partial v / \partial x
\end{Bmatrix}
= \begin{Bmatrix}
\sum\nolimits_{i=1}^5 N_{i,y} \, u_i\\
\sum\nolimits_{i=1}^5 N_{i,x} \, v_i
\end{Bmatrix} 
= \begin{bmatrix}
[\mathbf{h}_1] [\mathbf{h}_2] [\mathbf{h}_3] [\mathbf{h}_4] [\mathbf{h}_5] 
\end{bmatrix} \bigl\{ \begin{matrix} \mathbf{u}^{(e)} \end{matrix} \bigr\}
= \mathbf{H}_2  \mathbf{u}^{(e)}  
\end{equation}
\normalsize
where the elements of $\mathbf{H}_2$ are given as
\begin{equation}
[\mathbf{h}_i] = \begin{bmatrix}
N_{i,y} &  0  \\
0 & N_{i,x}  \end{bmatrix}. 
\end{equation}
Hence, the second nonlinear strain--displacement matrix $\mathbf{B}_{N2}$ becomes
\begin{equation}
\mathbf{B}_{N2} = \mathbf{A}_2 \, \mathbf{H}_2 = \begin{bmatrix}
[\mathbf{b}_{N1}] [\mathbf{b}_{N2}] [\mathbf{b}_{N3}] [\mathbf{b}_{N4}] [\mathbf{b}_{N5}] 
\end{bmatrix} 
\end{equation}
where its elements are given as
\begin{equation}
[\mathbf{b}_{Ni}] = \begin{bmatrix}
- \partial v / \partial x &  0 \\
\partial v / \partial x &  0 \\
0 & -4 \, \partial u / \partial x  \end{bmatrix} \, \begin{bmatrix}
N_{i,y} &  0  \\
0 & N_{i,x}  \end{bmatrix}. 
\end{equation}
The total nonlinear strain--displacement matrix is evaluated as the summation of its components $ \mathbf{B}_{N1}$ and $\mathbf{B}_{N2}$. Now, with all the strain--displacement matrices evaluated, we are ready to compute the stiffness matrix for a planar membrane.

To obtain the stiffness matrix for a planar membrane, we need to compute the four constituent strain-displacement matrices $\mathbf{C}^s$, $\mathbf{C}^t$, $\mathbf{K}^s$ and $\mathbf{K}^t$, as mentioned earlier. These strain-displacement matrices are obtained by expressing their corresponding strain components in terms of the five nodal displacements, with help from the shape functions.


\subsubsection{Tangent Stiffness Matrix $\mathbf{C}^s$}

The tangent stiffness matrix $\mathbf{C}^s$, as established in Eqn.~\eqref{secantStiffness}, becomes 
\begin{subequations}
	\begin{gather}
		\mathbf{C}^s = \int_{\pentagon} \mathbf{H}^{\mathsf{T}} \,  \mathbf{S}^s \, \mathbf{H} \, h \, \mathrm{d} A
		=  | \mathbf{J}_0 |  \, \sum_{i=1}^{n} \mathbf{H}^{\mathsf{T}} \, \mathbf{S}^s (\xi_i , \eta_i) \, \mathbf{H} \, h_{0\,i} \, \mathrm{w}_i \\
        \intertext{wherein}
        \mathbf{S}^s = \begin{bmatrix}
        	\mathcal{S}_{11} & \mathcal{S}_{12} \\
        	\mathcal{S}_{21} & \mathcal{S}_{22}
        \end{bmatrix}
    \label{stressMtx2D}
	\end{gather}
\end{subequations}

The stress vector $\boldsymbol{T} = \{ s^{\pi}, s^{\sigma}, s^{\tau} \}^{\mathsf{T}}$ conjugate to strain vector $\boldsymbol{E} = \{ \xi, \varepsilon, \gamma \}^{\mathsf{T}}$ has elements of a surface tension $s^{\pi} = \mathcal{S}_{11} + \mathcal{S}_{22}$, a normal-stress difference $s^{\sigma} = \mathcal{S}_{11} - \mathcal{S}_{22}$, and a shear stress $s^{\tau} = \tfrac{a}{b} \, \mathcal{S}_{12}$.  Only surface tension is considered to have a residual state of stress $s^{\pi}_0$, which is necessary for alveolar stability, and is caused, in part, by the presence of surfactant.  In a reciprocal sense, the stress components are assigned via $\mathcal{S}_{11} = \frac{1}{2} ( s^{\pi} + s^{\sigma} )$, $\mathcal{S}_{22} = \frac{1}{2} ( s^{\pi} - s^{\sigma} )$ and $\mathcal{S}_{12} = \mathcal{S}_{21} = \frac{b}{a} s^{\tau}$ such that $\mathbf{S} = \mathbf{P} \, \boldsymbol{\mathcal{U}}^{-1} \boldsymbol{\mathcal{S}} \, \boldsymbol{\mathcal{U}}^{-\mathsf{T}} \mathbf{P}^{\mathsf{T}}$ with $\mathbf{S}$ being the second Piola-Kirchhoff stress evaluated in the co-ordinate system of a pentagon, while $\boldsymbol{\mathcal{U}}$ is the Laplace stretch, and $\mathbf{P}$ is a re-indexer of co-ordinate labeling needed to ensure invariance under a transformation of Laplace stretch.

where an isochoric response is assumed in that $h_0 | \mathbf{J}_0 | = h | \mathbf{J} |$. Here $h_0$ and $h$ are the initial and current height or thickness of the septal membrane, and $\xi_i$, $\eta_i$ and $w_i$ are the co-ordinates and weights of quadrature for Gauss point~$i$ whereat the membrane has an initial thickness of $h_{0\,i}$. 

\subsubsection{Tangent Stiffness Matrix $\mathbf{C}^t$}

The tangent stiffness matrix $\mathbf{C}^t$, as established in Eqn.~\eqref{tangentStiffness}, becomes 
\begin{equation}
	\begin{aligned}
		\mathbf{C}^t & =\int_{\pentagon} \mathbf{B}^{\mathsf{T}} \,  \boldsymbol{\mathcal{M}}^t \, \mathbf{B} \, |\mathbf{J}|  \, h \,  \mathrm{d} A
		= |\mathbf{J}_0| \sum_{i=1}^{n}   \mathbf{B}^{\mathsf{T}} \, \boldsymbol{\mathcal{M}}^t (\xi_i, \eta_i) \, \mathbf{B} \, h_{0\,i} \, \mathrm{w}_i 
	\end{aligned}
\end{equation} 
where its associated axial stress rate $\mathrm{d} s$ is described by a tangent modulus $\boldsymbol{\mathcal{M}}^{t}$ that for biologic membranes is described by Eqn.~(\ref{pentagonTangentModulus}).

\subsubsection{Secant Stiffness Matrix $\mathbf{K}^s$}

The secant stiffness matrix $\mathbf{K}^s$, as established in Eqn.~\eqref{secantStiffnessMatrix}, becomes
\begin{equation}
	\begin{aligned}
		\mathbf{K}^s & =\int_{\pentagon} \mathbf{B}^{\mathsf{T}} \,  \boldsymbol{\mathcal{M}}^s \, \mathbf{B}  \, h \,  \mathrm{d} A
		= |\mathbf{J}_0| \sum_{i=1}^{n}  \mathbf{B}^{\mathsf{T}} \, \boldsymbol{\mathcal{M}}^s (\xi_i, \eta_i) \, \mathbf{B} \, h_{0\,i} \, \mathrm{w}_i 
	\end{aligned}
\end{equation} 
where its associated axial stress rate $\mathrm{d} s$ is described by a secant modulus $\boldsymbol{\mathcal{M}}^{s}$ that for biologic membranes is described by Eqn.~(\ref{pentagonSecantModulus}).


\subsubsection{Secant Stiffness Matrix $\mathbf{K}^t$}

Likewise, a secant stiffness matrix $\mathbf{K}^t$, also established in Eqn.~\eqref{tangentStiffness}, becomes 
\begin{equation}
	\begin{aligned}
		\mathbf{K}^t & = \int_{\pentagon} \mathbf{H}^{\mathsf{T}} \,  \mathrm{d} \mathbf{S}^t \, \mathbf{H}  \, h \,  \mathrm{d} A
		= | \mathbf{J}_0 | \sum_{i=1}^{n}  \mathbf{H}^{\mathsf{T}} \,  \mathrm{d} \mathbf{S}^t (\xi_i, \eta_i) \, \mathbf{H} \, h_{0\,i} \, \mathrm{w}_i
	\end{aligned}
\end{equation}
where $\mathrm{d} \mathbf{S}^t \defeq \mathbf{A}^{\mathsf{T}} \, \boldsymbol{\mathcal{M}}^t \, \mathrm{d} \mathbf{A}$.

\subsection{Kinematic Matrices for a Tetrahedron}

Let us consider a tetrahedron subjected to the displacements $u$, $v$ and $w$ in three spatial directions respectively. In terms of the displacement gradient, the elements of the deformation gradient can be written as
\begin{equation}
\mathbf{F} =  
\begin{bmatrix}
1 + \mathrm{\partial u / \partial x} & \mathrm{\partial u / \partial y} & \mathrm{\partial u / \partial z} \\
\mathrm{\partial v / \partial x} & 1 + \mathrm{\partial v / \partial y} & \mathrm{\partial v / \partial z} \\
\mathrm{\partial w / \partial x} & \mathrm{\partial w / \partial y} & 1 + \mathrm{\partial w / \partial z}
\end{bmatrix} .
\end{equation}	
Therefore, components of the right Cauchy-Green deformation tensor, defined as $\mathbf{C} \defeq \mathbf{F}^\mathsf{T}\mathbf{F}$, can be expressed as
\begin{subequations}
    \begin{align}	
    \mathrm{C_{11}} & = \left(\frac{\mathrm{\partial u}}{\partial x}\right)^2 + \left(\frac{\mathrm{\partial v}}{\partial x}\right)^2 + \left(\frac{\mathrm{\partial w}}{\partial x}\right)^2 + 2\, \frac{\mathrm{\partial u}}{\partial x}  + 1\\
    \mathrm{C_{22}} & = \left(\frac{\mathrm{\partial u}}{\partial y}\right)^2 + \left(\frac{\mathrm{\partial v}}{\partial y}\right)^2 + \left(\frac{\mathrm{\partial w}}{\partial y}\right)^2 + 2\, \frac{\mathrm{\partial v}}{\partial y} + 1\\
    \mathrm{C_{33}} & = \left(\frac{\mathrm{\partial u}}{\partial z}\right)^2 + \left(\frac{\mathrm{\partial v}}{\partial z}\right)^2 + \left(\frac{\mathrm{\partial w}}{\partial z}\right)^2 + 2\, \frac{\mathrm{\partial w}}{\partial z} + 1 \\
    \mathrm{C_{12}} & = \mathrm{C_{21}} = \frac{\mathrm{\partial u}}{\partial y} + \frac{\mathrm{\partial v}}{\partial x} + \frac{\mathrm{\partial u}}{\partial x} \cdot \frac{\mathrm{\partial u}}{\partial y} + \frac{\mathrm{\partial v}}{\partial x} \cdot \frac{\mathrm{\partial v}}{\partial y} + \frac{\mathrm{\partial w}}{\partial x} \cdot \frac{\mathrm{\partial w}}{\partial y}\\
    \mathrm{C_{13}} & = \mathrm{C_{31}} = \frac{\mathrm{\partial u}}{\partial z} + \frac{\mathrm{\partial w}}{\partial x} + \frac{\mathrm{\partial u}}{\partial x} \cdot \frac{\mathrm{\partial u}}{\partial z} + \frac{\mathrm{\partial v}}{\partial x} \cdot \frac{\mathrm{\partial v}}{\partial z} + \frac{\mathrm{\partial w}}{\partial x} \cdot \frac{\mathrm{\partial w}}{\partial z} \\
    \mathrm{C_{23}} & = \mathrm{C_{32}} = \frac{\mathrm{\partial v}}{\partial z} + \frac{\mathrm{\partial w}}{\partial y} + \frac{\mathrm{\partial u}}{\partial y} \cdot \frac{\mathrm{\partial u}}{\partial z} + \frac{\mathrm{\partial v}}{\partial y} \cdot \frac{\mathrm{\partial v}}{\partial z} + \frac{\mathrm{\partial w}}{\partial y} \cdot \frac{\mathrm{\partial w}}{\partial z}.
    \end{align}
\end{subequations}
The Laplace stretch associated with an alveolar volume is a $3\times3$ upper-triangular matrix whose elements have specific geometric interpretations. The Laplace stretch can be written in matrix form as \cite{FreedSrinivasa15}
\begin{equation}
    \label{LagrangianPhysicalStretch}
    \mathcal{U}_{ij} = \begin{bmatrix}
    a & a \gamma & a \beta \\
    0 & b & b \alpha \\
    0 & 0 & c \end{bmatrix} .
\end{equation}
It is possible to express the components of Laplace stretch $\boldsymbol{\mathcal{U}}$ in terms of the displacement gradient through a Cholesky factorization of the right Cauchy-Green tensor $\mathbf{C} = \boldsymbol{\mathcal{U}}^{\mathsf{T}} \boldsymbol{\mathcal{U}}$. Specifically, the elements of Laplace stretch are obtained as \cite{Srinivasa12}
\begin{equation}
\begin{aligned}
\mathcal{U}_{11} & = \sqrt{C_{11}} & 
\mathcal{U}_{12} & = C_{12} / \mathcal{U}_{11} &
\mathcal{U}_{13} & = C_{13} / \mathcal{U}_{11} \\
\mathcal{U}_{21} & = 0 &
\mathcal{U}_{22} & = \sqrt{C_{22} - \mathcal{U}_{12}^{\,2}} &
\mathcal{U}_{23} & = \bigl( C_{23} - \mathcal{U}_{12\,}\mathcal{U}_{13} \bigr) / \mathcal{U}_{22} \\
\mathcal{U}_{31} & = 0 &
\mathcal{U}_{32} & = 0 & 
\mathcal{U}_{33} & = \sqrt{C_{33} - \mathcal{U}_{13}^{\,2} - \mathcal{U}_{23}^{\,2}}
\end{aligned}
\label{LagrangianLaplaceStretch}
\end{equation}
where ${C_{11}}$, ${C_{12}}$, ${C_{13}}$, $C_{22}$, $C_{23}$ and ${C_{33}}$ are components of the right Cauchy--Green tensor $\mathbf{C}$.

Now in order to obtain the stiffness matrix for an alveolar volume, first we need to we derive the strain attributes and express them in terms of the nodal displacements. The strain attributes are defined in terms of the derived elements of Laplace stretch, as mentioned earlier.

The dilatation $\delta$ for an alveolar volume is
\begin{subequations}
    \begin{align}
    \mathrm  \xi & \defeq \ln \sqrt[3]{\frac{a}{a_0}
        \frac{b}{b_0} \frac{c}{c_0}} \\
    \intertext{whereas the squeeze strains $\varepsilon_i$ and shear strains $\gamma_i$ are defined as}
    \varepsilon_1 & \defeq \ln \sqrt[3]{\frac{a}
        {a_0} \frac{b_0}{b}} & 	\gamma_1 & \defeq \alpha - \alpha_0 \\
    \varepsilon_2 & \defeq \ln \sqrt[3]{\frac{b}
        {b_0} \frac{c_0}{c}} & \gamma_2 & \defeq \beta - \beta_0 
    \end{align}
\end{subequations}
wherein $a_0$, $b_0$ and $c_0$ are their initial elongation ratios, and where $\alpha_0$, $\beta_0$ and $\gamma_0$ are their initial shears. Without loss of generality, we can assume that the initial stretches $a_0$, $b_0$ and $c_0$ are one, whereas the initial shears $\alpha_0$, $\beta_0$ and $\gamma_0$ are zero. There is a third squeeze, too, viz., $\varepsilon_3 = -\varepsilon_1 - \varepsilon_2$, but it is not an independent descriptor of strain.


For computational ease, these strain attributes are additively decomposed into one linear and five nonlinear components. The primary advantage of this decomposition is an emergence of a systematic structure in the strain-displacement matrix, which makes evaluation of the stiffness matrix much easier.  The linear and nonlinear components for the strain attributes are obtained by applying a Taylor series expansion to these strain attributes, and then expressing their constituents in terms of gradients for the displacements with respect to the different spatial variables.  Here only terms up to second-order have been retained. The linear and nonlinear components for the strain attributes, thus obtained, are given by
\begin{subequations}
    \begin{align}
    \mathrm  \xi & = \mathrm \xi_{L} + \mathrm  \xi_{N1} + \mathrm  \xi_{N2} + \mathrm  \xi_{N3} + \mathrm  \xi_{N4} + \mathrm  \xi_{N5}\\
    \mathrm  \varepsilon_{i} & = \mathrm  \varepsilon_{iL} + \mathrm  \varepsilon_{iN1} + \mathrm  \varepsilon_{iN2} + \mathrm  \varepsilon_{iN3} + \mathrm  \varepsilon_{iN4} + \mathrm  \varepsilon_{iN5} \\
    \mathrm \gamma_{i} & = \mathrm  \gamma_{iL} + \mathrm  \gamma_{iN1} + \mathrm  \gamma_{iN2} + \mathrm  \gamma_{iN3} + \mathrm  \gamma_{iN4} + \mathrm  \gamma_{iN5}
    \end{align}
    \label{totalvirtualstrain}
\end{subequations}
where their linear and nonlinear components can be expressed in terms of elements arising from a matrix representation of the displacement gradient as
\begin{subequations}
	\begin{align}
	\mathrm \xi_{L} & = \frac{1}{3} \, \left(\mathrm{\frac{\partial u}{\partial x}} + \mathrm{\frac{\partial v}{\partial y}} + \mathrm{\frac{\partial w}{\partial z}}\right)\\
	\mathrm \xi_{N} & = \frac{1}{6} \, \Big(-\mathrm{\frac{\partial u}{\partial x}}\, \mathrm{\frac{\partial u}{\partial x}} + \mathrm{\frac{\partial u}{\partial z}}\, \mathrm{\frac{\partial u}{\partial z}}  - \mathrm{\frac{ \partial v}{\partial y}}\, \mathrm{\frac{ \partial v}{\partial y}} - \mathrm{\frac{\partial v}{\partial z}}\, \mathrm{\frac{\partial v}{\partial z}} + \mathrm{\frac{\partial w}{\partial x}}\, \mathrm{\frac{\partial w}{\partial x}} - \mathrm{\frac{\partial w}{\partial y}}\, \mathrm{\frac{\partial w}{\partial y}} - \mathrm{\frac{\partial w}{\partial z}}\, \mathrm{\frac{\partial w}{\partial z}} 
	\notag \\
	\mbox{} & \qquad - 2 \, \mathrm{\frac{\partial u}{\partial y}}\, \mathrm{\frac{\partial v}{\partial x}} - 4 \, \mathrm{\frac{\partial v}{\partial z}}\, \mathrm{\frac{\partial w}{\partial y}} \Big)\\
	\mathrm \varepsilon_{1L} & = \frac{1}{3} \, \left(\mathrm{\frac{\partial u}{\partial x}} - \mathrm{\frac{\partial v}{\partial y}}\right)\\
	\mathrm \varepsilon_{1N} & = \frac{1}{6} \, \left(2 \, \mathrm{\frac{\partial v}{\partial x}}\, \mathrm{\frac{\partial v}{\partial x}} + \mathrm{\frac{ \partial v}{\partial y}}\, \mathrm{\frac{ \partial v}{\partial y}} - \mathrm{\frac{\partial u}{\partial x}}\, \mathrm{\frac{\partial u}{\partial x}} +  \mathrm{\frac{\partial w}{\partial x}}\, \mathrm{\frac{\partial w}{\partial x}} -  \mathrm{\frac{\partial w}{\partial y}}\, \mathrm{\frac{\partial w}{\partial y}} + 2 \, \mathrm{\frac{\partial u}{\partial y}}\, \mathrm{\frac{\partial v}{\partial x}}\right)\\
	\mathrm \varepsilon_{2L} & = \frac{1}{3} \, \left(\mathrm{\frac{\partial v}{\partial y}} - \mathrm{\frac{\partial w}{\partial z}}\right)\\
	\mathrm \varepsilon_{2N} & = \frac{1}{6} \, \Big(- \mathrm{\frac{\partial v}{\partial x}}\, \mathrm{\frac{\partial v}{\partial x}} - \mathrm{\frac{ \partial v}{\partial y}}\, \mathrm{\frac{ \partial v}{\partial y}} - \mathrm{\frac{\partial u}{\partial z}}\, \mathrm{\frac{\partial u}{\partial z}} + \mathrm{\frac{\partial v}{\partial z}}\, \mathrm{\frac{\partial v}{\partial z}} + 3 \,  \mathrm{\frac{\partial w}{\partial y}}\, \mathrm{\frac{\partial w}{\partial y}} +  \mathrm{\frac{\partial w}{\partial z}}\, \mathrm{\frac{\partial w}{\partial z}}  \notag \\
	\mbox{} & \qquad - 2 \, \mathrm{\frac{\partial u}{\partial y}}\, \mathrm{\frac{\partial v}{\partial x}} + 4 \, \mathrm{\frac{\partial v}{\partial z}}\, \mathrm{\frac{\partial w}{\partial y}}\Big)\\
	\mathrm \gamma_{1L} & = \mathrm{\frac{\partial v}{\partial z}} + \mathrm{\frac{\partial w}{\partial y}}\\
	\mathrm \gamma_{1N} & = 2 \, \mathrm{\frac{\partial u}{\partial x}}\, \mathrm{\frac{\partial v}{\partial z}} - \mathrm{\frac{ \partial u}{\partial z}}\, \mathrm{\frac{ \partial v}{\partial x}} + 2 \, \mathrm{\frac{\partial u}{\partial x}}\, \mathrm{\frac{\partial w}{\partial y}} - \mathrm{\frac{\partial u}{\partial y}}\, \mathrm{\frac{\partial w}{\partial x}} - \mathrm{\frac{\partial v}{\partial y}}\, \mathrm{\frac{\partial v}{\partial z}}  - 2 \, \mathrm{\frac{\partial v}{\partial y}}\, \mathrm{\frac{\partial w}{\partial y}} \notag \\
	\mbox{} & \qquad + \mathrm{\frac{\partial w}{\partial y}}\, \mathrm{\frac{\partial w}{\partial z}}\\
	\mathrm \gamma_{2L} & = \mathrm{\frac{\partial v}{\partial z}} + \mathrm{\frac{\partial w}{\partial y}}\\
	\mathrm \gamma_{2N} & = \mathrm{\frac{\partial u}{\partial y}}\, \mathrm{\frac{\partial u}{\partial z}} - 2 \, \mathrm{\frac{ \partial u}{\partial x}}\, \mathrm{\frac{ \partial v}{\partial z}} - 2 \, \mathrm{\frac{\partial u}{\partial x}}\, \mathrm{\frac{\partial w}{\partial y}} + \mathrm{\frac{\partial v}{\partial y}}\, \mathrm{\frac{\partial v}{\partial z}} + \mathrm{\frac{\partial w}{\partial y}}\, \mathrm{\frac{\partial w}{\partial z}}\\
	\mathrm \gamma_{3L} & = \mathrm{\frac{\partial u}{\partial y}} + \mathrm{\frac{\partial v}{\partial x}}\\
	\mathrm \gamma_{3N} & = - \mathrm{\frac{\partial u}{\partial x}}\, \mathrm{\frac{\partial u}{\partial y}} - 2 \, \mathrm{\frac{ \partial u}{\partial x}}\, \mathrm{\frac{ \partial v}{\partial x}} + \mathrm{\frac{\partial v}{\partial x}}\, \mathrm{\frac{\partial v}{\partial y}} + \mathrm{\frac{\partial w}{\partial x}}\, \mathrm{\frac{\partial w}{\partial y}} 
	\end{align}
\end{subequations}
The total stiffness matrix can be obtained as a sum of the linear and five nonlinear stiffness matrices. Therefore, we first have to evaluate these components of the stiffness matrices by using the associated strain-displacement matrices. For all these cases, the strain-displacement matrices are obtained by expressing the strains in terms of the nodal displacements with the help of shape functions and their derivatives.

First, the linear strain-displacement matrix $\mathbf{B}_L$ is obtained by expressing the strain attributes in terms of the nodal displacements through derivatives of the shape functions. Specifically, the linear strain-displacement matrix takes the form of
\begin{equation}
\begin{aligned}
\mathbf{E}_L & =  \begin{Bmatrix}
\mathrm \xi_{L} \\
\mathrm \varepsilon_{1L} \\
\mathrm \varepsilon_{2L} \\
\mathrm \gamma_{1L} \\
\mathrm \gamma_{2L} \\
\mathrm \gamma_{3L}\end{Bmatrix} = 
\begin{Bmatrix}
\tfrac{1}{3} \, u_{,x} + \tfrac{1}{3} \, v_{,y} + \tfrac{1}{3} \, w_{,z} \\
\tfrac{1}{3} \, u_{,x} - \tfrac{1}{3} \, v_{,y} \\
\tfrac{1}{3} \, v_{,y} - \tfrac{1}{3} \, w_{,z} \\
v_{,z} +  w_{,y} \\ 
v_{,z} +  w_{,y} \\ 
u_{,y} +  v_{,x} \\\end{Bmatrix} \\
& = \begin{bmatrix}
\tfrac{1}{3} \, \sum\nolimits_{i=1}^4 N_{i,x}  & \tfrac{1}{3} \, \sum\nolimits_{i=1}^4 N_{i,y} & \tfrac{1}{3} \, \sum\nolimits_{i=1}^4 N_{i,z} \\
\tfrac{1}{3} \, \sum\nolimits_{i=1}^4 N_{i,x}  & \tfrac{-1}{3} \, \sum\nolimits_{i=1}^4 N_{i,y} & 0 \\ 
0 & \tfrac{1}{3} \, \sum\nolimits_{i=1}^4 N_{i,y}  &  \tfrac{-1}{3} \, \sum\nolimits_{i=1}^4 N_{i,z} \\  
0 & \sum\nolimits_{i=1}^4 N_{i,z}  &  \sum\nolimits_{i=1}^4 N_{i,y} \\ 
0 & \sum\nolimits_{i=1}^4 N_{i,z}  &  \sum\nolimits_{i=1}^4 N_{i,y} \\ 
\sum\nolimits_{i=1}^4 N_{i,y}  &  \sum\nolimits_{i=1}^4 N_{i,x} & 0 \end{bmatrix} \, \begin{Bmatrix}
u_{i} \\
v_{i} \\
w_{i} \end{Bmatrix} \\
& = \begin{bmatrix}
[\mathbf{b}_{L1}] [\mathbf{b}_{L2}] [\mathbf{b}_{L3}] [\mathbf{b}_{L4}]
\end{bmatrix} \bigl\{ \begin{matrix} \mathbf{u}^{(e)} \end{matrix} \bigr\} 
= \mathbf{B}_L  \mathbf{u}^{(e)}  
\end{aligned}
\end{equation}
wherein each component of $\mathbf{B}_L$ is given by 
\begin{subequations}
    \begin{align}
    [\mathbf{b}_{Li}] = & \begin{bmatrix}
    \tfrac{1}{3} \,  N_{i,x}  & \tfrac{1}{3} \,  N_{i,y} & \tfrac{1}{3} \,  N_{i,z} \\
    \tfrac{1}{3} \,  N_{i,x}  & -\tfrac{1}{3} \,  N_{i,y} & 0 \\ 
    0 & \tfrac{1}{3} \, N_{i,y}  & - \tfrac{1}{3} \,  N_{i,z} \\ 
    0 &  N_{i,z}  &  N_{i,y} \\ 
    0 & N_{i,z}  &   N_{i,y} \\  
    N_{i,y}  &   N_{i,x} & 0 \end{bmatrix}  \\
    \intertext{and the nodal displacement vector for element $e$ is given as}
    \mathbf{u}^{(e)} & = \bigl\{ \begin{matrix}
    u_1 & v_1 & w_1 & u_2 & v_2 & w_2 & u_3 & v_3 & w_3 & u_4 & v_4 & w_4 
    \end{matrix} \bigr\}^{\mathsf{T}}
    \end{align}
\end{subequations}
Note that the linear-strain displacement matrix $\mathbf{B_L}$ consists only of derivatives for the shape functions, and thus, remains the same throughout a deformation process.

Now we derive the nonlinear strain-displacement matrices that will be used to obtain the nonlinear stiffness matrix. The nonlinear components of each strain attribute have been additively decomposed into five components to make our computation easier. Components of each strain attribute are placed into an associated vector resulting in an additive decomposition of the total nonlinear strain $\mathbf{E}_N$. To obtain the nonlinear stiffness matrix corresponding to these nonlinear strain components, the nonlinear strains are written as a product of two quantities: a matrix $\mathbf{A}$ containing various components of the displacement gradient, and a vector $\boldsymbol{\theta}$ that contains the derivatives of displacement with respect to spatial location. The vector $\boldsymbol{\theta}$ essentially represents the slope of the body resulting from the deformation process. The components of the displacement gradient are placed in the matrix $\mathbf{A}$ in such a way so that its product with the slope vector yields the corresponding contribution to the nonlinear strain.

The slope vector $\boldsymbol{\theta}$ can further be expressed in terms of the corresponding nodal displacements by using the derivatives of the shape functions. Thus, the nonlinear strain components $\mathbf{E}_{Ni}, i=1, 2, 3, 4$, can be expressed in terms of the nodal displacements, with the nonlinear strain-displacement matrix $\mathbf{B}_{Ni}$ corresponding to these strain components. These strain-displacement matrices are used to obtain the corresponding nonlinear stiffness matrices in a way described earlier.  Note that, unlike its linear counterpart, the nonlinear strain-displacement matrix varies with the deformation of a body, and hence, the matrices corresponding to it must be updated at each step along a solution path.

Now let us perform the procedure described above on all five nonlinear strain components.

For the first nonlinear strain, $\mathbf{E}_{N1}$ can be written as a product of the matrix $\mathbf{A}_1$ and the slope vector $\boldsymbol{\theta}_1$ as
\begin{equation}
\begin{aligned}
\mathbf{E}_{N1} & =  \begin{Bmatrix}
\mathrm \xi_{1N} \\
\mathrm \varepsilon_{1N} \\
\mathrm \varepsilon_{2N} \\
\mathrm \gamma_{1N} \\
\mathrm \gamma_{2N} \\
\mathrm \gamma_{3N} \end{Bmatrix} =
\begin{Bmatrix}
\tfrac{1}{6} \,  \left( -u_{,x}^2 - v_{,y}^2  -  w_{,z}^2  \right)\\
\tfrac{1}{6} \,  \left( -u_{,x}^2 +  v_{,y}^2 \right) \\
\tfrac{1}{6} \,  \left( - v_{,y}^2 + w_{,z}^2 \right) \\ 
 -v_{,z} \, v_{,y} + w_{,y} \, w_{,z} \\
 v_{,z} \, v_{,y} + w_{,y} \, w_{,z}\\ 
- u_{,x} \, u_{,y} + v_{,x} \, v_{,y}\end{Bmatrix} \\
& = \frac{1}{2} \, \begin{bmatrix}
-\tfrac{1}{3} \, \partial u / \partial x &  - \tfrac{1}{3} \, \partial v / \partial y & -\tfrac{1}{3} \, \partial w / \partial z  \\
-\tfrac{1}{3} \, \partial u / \partial x &  \tfrac{1}{3} \, \partial v / \partial y & 0  \\
0  & -\tfrac{1}{3} \, \partial v / \partial y & \tfrac{1}{3} \, \partial w / \partial z \\
0 &  - 2 \, \partial v / \partial z &  2 \,  \partial w / \partial y  \\
0 &  2 \, \partial v / \partial z  & 2 \,  \partial w / \partial y  \\
- 2 \, \partial u / \partial y &   2 \, \partial v / \partial x & 0   \end{bmatrix} \, \begin{Bmatrix}
\partial u / \partial x\\
\partial v / \partial y \\
\partial w / \partial z
\end{Bmatrix} \\
& = \tfrac{1}{2} \, \mathbf{A}_1 \, \boldsymbol{\theta}_1
\end{aligned}
\end{equation}
with
\begin{subequations}
	\begin{align}
    \mathrm{d} \mathbf{A}_1 & =  \begin{bmatrix}
    	-\tfrac{1}{3} \, \partial  \mathrm{d} u / \partial x &  - \tfrac{1}{3} \, \partial  \mathrm{d} v / \partial y & -\tfrac{1}{3} \, \partial  \mathrm{d} w / \partial z  \\
    	-\tfrac{1}{3} \, \partial  \mathrm{d} u / \partial x &  \tfrac{1}{3} \, \partial  \mathrm{d} v / \partial y & 0  \\
    	0  & -\tfrac{1}{3} \, \partial  \mathrm{d} v / \partial y & \tfrac{1}{3} \, \partial  \mathrm{d} w / \partial z \\
    	0 &  - 2 \, \partial  \mathrm{d} v / \partial z &  2 \,  \partial  \mathrm{d} w / \partial y  \\
    	0 &  2 \, \partial  \mathrm{d} v / \partial z  & 2 \,  \partial  \mathrm{d} w / \partial y  \\
    	- 2 \, \partial  \mathrm{d} u / \partial y &   2 \, \partial  \mathrm{d} v / \partial x & 0   \end{bmatrix} \notag\\
	& = \begin{bmatrix} 
	-\tfrac{1}{3} \, \sum\nolimits_{i=1}^4 N_{i,x} \, \mathrm{d} u_i &  - \tfrac{1}{3} \, \sum\nolimits_{i=1}^4 N_{i,y} \, \mathrm{d} v_i & - \tfrac{1}{3} \, \sum\nolimits_{i=1}^4 N_{i,z} \, \mathrm{d} w_i  \\
	-\tfrac{1}{3} \, \sum\nolimits_{i=1}^4 N_{i,x} \, \mathrm{d} u_i &  \tfrac{1}{3} \, \sum\nolimits_{i=1}^4 N_{i,y} \, \mathrm{d} v_i & 0  \\
	0 & -\tfrac{1}{3} \, \sum\nolimits_{i=1}^4 N_{i,y} \, \mathrm{d} v_i & \tfrac{1}{3} \, \sum\nolimits_{i=1}^4 N_{i,z} \, \mathrm{d} w_i \\
	0 &  - 2 \, \sum\nolimits_{i=1}^4 N_{i,z} \, \mathrm{d} v_i & 2 \,  \sum\nolimits_{i=1}^4 N_{i,y} \, \mathrm{d} w_i  \\
	0 & 2 \, \sum\nolimits_{i=1}^4 N_{i,z} \, \mathrm{d} v_i & 2 \,  \sum\nolimits_{i=1}^4 N_{i,y} \, \mathrm{d} w_i  \\
	- 2 \, \sum\nolimits_{i=1}^4 N_{i,y} \, \mathrm{d} u_i &  2 \, \sum\nolimits_{i=1}^4 N_{i,x} \, \mathrm{d} v_i & 0   \end{bmatrix} \notag \\	
	& = \begin{bmatrix}
	[\mathbf{l}_1] [\mathbf{l}_2] [\mathbf{l}_3] [\mathbf{l}_4] 
	\end{bmatrix}   \, \begin{bmatrix} [\mathbf{d}_1] [\mathbf{d}_2] [\mathbf{d}_3] [\mathbf{d}_4] \end{bmatrix}^{\mathsf{T}}
	= \mathbf{L}_1 \mathbf{D}_1 \\
	\intertext{wherein}
	[\mathbf{l}_i] & = \begin{bmatrix} 
	-\tfrac{1}{3} \, N_{i,x} &  - \tfrac{1}{3} \, N_{i,y} & -\tfrac{1}{3} \, N_{i,z} \\
	-\tfrac{1}{3} \, N_{i,x} &  \tfrac{1}{3} \, N_{i,y} & 0 \\
	0  & -\tfrac{1}{3} \, N_{i,y} & \tfrac{1}{3} \, N_{i,z} \\
    0 &  - 2 \, N_{i,z}  &  2 \, N_{i,y} \\
    0 &  2 \, N_{i,z}  &  2 \, N_{i,y}  \\
	- 2 \,  N_{i,y} &  2 \, N_{i,x} & 0   \end{bmatrix}  \quad \text{and} \quad	[\mathbf{d}_i] = \begin{bmatrix}
	\mathrm{d} u_i & 0  & 0  \\
	0 &  \mathrm{d} v_i   & 0  \\
	0 & 0 & \mathrm{d} w_i \end{bmatrix} .
	\end{align}
\end{subequations}

Now, the derivative of displacement with respect to spatial variables $x, y$ and $z$ can be related to the nodal parameters via
\small
\begin{equation}
\boldsymbol{\theta}_1 =  \begin{Bmatrix}
\partial u / \partial x\\
\partial v / \partial y \\
\partial w / \partial z
\end{Bmatrix}
= \begin{Bmatrix}
\sum\nolimits_{i=1}^4 N_{i,x} \, u_i\\
\sum\nolimits_{i=1}^4 N_{i,y} \, v_i \\
\sum\nolimits_{i=1}^4 N_{i,z} \, w_i
\end{Bmatrix} 
= \begin{bmatrix}
[\mathbf{h}_1] [\mathbf{h}_2] [\mathbf{h}_3] [\mathbf{h}_4]
\end{bmatrix} \bigl\{ \begin{matrix} \mathbf{u}^{(e)} \end{matrix} \bigr\}  
= \mathbf{H}_1 \, \mathbf{u}^{(e)} 
\end{equation}
\normalsize
for element $e$, where 
\begin{equation}
[\mathbf{h}_i] = \begin{bmatrix}
N_{i,x} &  0 & 0  \\
0 & N_{i,y} & 0  \\
0 & 0 & N_{i,z}\end{bmatrix} .
\end{equation}
Hence, the strain-displacement matrix $\mathbf{B}_{N1}$ corresponding to the first nonlinear strain becomes
\begin{equation}
\mathbf{B}_{N1} = \mathbf{A}_1 \, \mathbf{H}_1 = \begin{bmatrix}
[\mathbf{b}_{N1}] [\mathbf{b}_{N2}] [\mathbf{b}_{N3}] [\mathbf{b}_{N4}] 
\end{bmatrix} 
\end{equation}
wherein the components of $\mathbf{B}_{N1}$ are given as
\begin{equation}
[\mathbf{b}_i] = \begin{bmatrix}
	-\tfrac{1}{3} \, \partial u / \partial x &  - \tfrac{1}{3} \, \partial v / \partial y & -\tfrac{1}{3} \, \partial w / \partial z  \\
	-\tfrac{1}{3} \, \partial u / \partial x &  \tfrac{1}{3} \, \partial v / \partial y & 0  \\
	0  & -\tfrac{1}{3} \, \partial v / \partial y & \tfrac{1}{3} \, \partial w / \partial z \\
	0 &  - 2 \, \partial v / \partial z &  2 \,  \partial w / \partial y  \\
	0 &  2 \, \partial v / \partial z  & 2 \,  \partial w / \partial y  \\
	- 2 \, \partial u / \partial y &   2 \, \partial v / \partial x & 0   \end{bmatrix}  \, \begin{bmatrix}
N_{i,x} &  0 & 0  \\
0 & N_{i,y} & 0  \\
0 & 0 & N_{i,z} \end{bmatrix}. 
\end{equation}

In a similar manner, we can obtain the strain-displacement matrices corresponding to the other nonlinear strain components. The second nonlinear strain terms can be written in terms of an $\mathbf{A}_2$ whose slope vector $\boldsymbol{\theta}_2$ is
\begin{equation}
\begin{aligned}
\mathbf{E}_{N2} & =  \begin{Bmatrix}
\mathrm \xi_{L} \\
\mathrm \varepsilon_{1N} \\
\mathrm \varepsilon_{2N} \\
\mathrm \gamma_{1N} \\
\mathrm \gamma_{2N} \\
\mathrm \gamma_{3N} \end{Bmatrix} =
\begin{Bmatrix}
\tfrac{1}{6} \,  \left( v_{,z}^2 - v_{,z}^2 - w_{,y}^2  \right)\\
\tfrac{-1}{6} \, w_{,y}^2 \\
\tfrac{1}{6} \,  \left( - u_{,z}^2  + v_{,z}^2 + 3 \, w_{,y}^2 \right) \\
0 \\
u_{,y} \, u_{,z}  \\ 
w_{,x} \, w_{,y} \end{Bmatrix} \\
& = \frac{1}{2} \, \begin{bmatrix}
 \tfrac{1}{3} \, \partial u / \partial z &  - \tfrac{1}{3} \, \partial v / \partial z & - \tfrac{1}{3} \, \partial w / \partial y  \\
0 & 0 & - \tfrac{1}{3} \, \partial w / \partial y  \\
-\tfrac{1}{3} \, \partial u / \partial z  & \tfrac{1}{3} \, \partial v / \partial z & \partial w / \partial y \\
0 & 0 & 0  \\
2 \, \partial u / \partial y &  0 & 0 \\
0 &  0 &  2 \, \partial w / \partial x  \end{bmatrix} \, \begin{Bmatrix}
\partial u / \partial z\\
\partial v / \partial z \\
\partial w / \partial y
\end{Bmatrix} \\
& = \tfrac{1}{2} \, \mathbf{A}_2 \, \boldsymbol{\theta}_2
\end{aligned}
\end{equation}
with  
\begin{subequations}
	\begin{align}
	\mathrm{d} \mathbf{A}_2 & =  \begin{bmatrix}
		\tfrac{1}{3} \, \partial \mathrm{d} u / \partial z &  - \tfrac{1}{3} \, \partial \mathrm{d} v / \partial z & - \tfrac{1}{3} \, \partial \mathrm{d} w / \partial y  \\
		0 & 0 & - \tfrac{1}{3} \, \partial \mathrm{d} w / \partial y  \\
		-\tfrac{1}{3} \, \partial \mathrm{d} u / \partial z  & \tfrac{1}{3} \, \partial \mathrm{d} v / \partial z & \partial \mathrm{d} w / \partial y \\
		0 & 0 & 0  \\
		2 \, \partial \mathrm{d} u / \partial y &  0 & 0 \\
		0 &  0 &  2 \, \partial \mathrm{d} w / \partial x  \end{bmatrix} \notag\\
	& = \begin{bmatrix}
	\tfrac{1}{3} \, \sum\nolimits_{i=1}^4 N_{i,z} \, \mathrm{d} u_i &  - \tfrac{1}{3} \, \sum\nolimits_{i=1}^4 N_{i,z} \, \mathrm{d} v_i & - \tfrac{1}{3} \, \sum\nolimits_{i=1}^4 N_{i,y} \, \mathrm{d} w_i  \\
	0 &  0 & - \tfrac{1}{3} \, \sum\nolimits_{i=1}^4 N_{i,y} \, \mathrm{d} w_i  \\
	-\tfrac{1}{3} \, \sum\nolimits_{i=1}^4 N_{i,z} \, \mathrm{d} u_i  & \tfrac{1}{3} \, \sum\nolimits_{i=1}^4 N_{i,z} \, \mathrm{d} v_i & \sum\nolimits_{i=1}^4 N_{i,y} \, \mathrm{d} w_i \\
	0 &  0 & 0  \\
	2 \, \sum\nolimits_{i=1}^4 N_{i,y} \, \mathrm{d} u_i &  0 & 0  \\
	0 &  0 &  2 \, \sum\nolimits_{i=1}^4 N_{i,x} \, \mathrm{d} w_i  \end{bmatrix} \notag \\	
	& = \begin{bmatrix}
	[\mathbf{l}_1] [\mathbf{l}_2] [\mathbf{l}_3] [\mathbf{l}_4] 
	\end{bmatrix}  \, \begin{bmatrix} [\mathbf{d}_1] [\mathbf{d}_2] [\mathbf{d}_3] [\mathbf{d}_4] \end{bmatrix}^{\mathsf{T}}
= \mathbf{L}_2 \mathbf{D}_2 \\
	\intertext{wherein}
	[\mathbf{l}_i] & = \begin{bmatrix}
		\tfrac{1}{3} \,  N_{i,z}  &  - \tfrac{1}{3} \,  N_{i,z}  & - \tfrac{1}{3} \,  N_{i,y}   \\
		0 &  0 & - \tfrac{1}{3} \,  N_{i,y}  \\
		-\tfrac{1}{3} \,  N_{i,z}   & \tfrac{1}{3} \,  N_{i,z}  &  N_{i,y}  \\
		0 &  0 & 0  \\
		2 \, N_{i,y} &  0 & 0  \\
		0 &  0 &  2 \,  N_{i,x}   \end{bmatrix}    \quad \text{and} \quad	[\mathbf{d}_i] = \begin{bmatrix}
		\mathrm{d} u_i & 0  & 0  \\
		0 &  \mathrm{d} v_i   & 0  \\
		0 & 0 & \mathrm{d} w_i \end{bmatrix} .
	\end{align}
\end{subequations}

The slope vector can further be expressed in terms of the nodal parameters via
\small
\begin{equation}
\boldsymbol{\theta}_2 =  \begin{Bmatrix}
\partial u / \partial z\\
\partial v / \partial z \\
\partial w / \partial y
\end{Bmatrix}
= \begin{Bmatrix}
\sum\nolimits_{i=1}^4 N_{i,z} \, u_i\\
\sum\nolimits_{i=1}^4 N_{i,z} \, v_i \\
\sum\nolimits_{i=1}^4 N_{i,y} \, w_i
\end{Bmatrix} 
= \begin{bmatrix}
[\mathbf{h}_1] [\mathbf{h}_2] [\mathbf{h}_3] [\mathbf{h}_4] 
\end{bmatrix} \bigl\{ \begin{matrix} \mathbf{u}^{(e)} \end{matrix} \bigr\}
= \mathbf{H}_2 \, \mathbf{u}^{(e)} 
\end{equation}
\normalsize
where 
\begin{equation}
[\mathbf{h}_i] = \begin{bmatrix}
N_{i,z} &  0 & 0  \\
0 & N_{i,z} & 0  \\
0 & 0 & N_{i,y} \end{bmatrix} .
\end{equation}
Hence, the strain-displacement matrix $\mathbf{B}_{N2}$ becomes
\begin{equation}
\mathbf{B}_{N2} = \mathbf{A}_2 \, \mathbf{H}_2 = \begin{bmatrix}
[\mathbf{b}_{N1}] [\mathbf{b}_{N2}] [\mathbf{b}_{N3}] [\mathbf{b}_{N4}]
\end{bmatrix}. 
\end{equation}
The components of this strain-displacement matrix are given as
\begin{equation}
[\mathbf{b}_{Ni}] =  \begin{bmatrix}
	\tfrac{1}{3} \, \partial u / \partial z &  - \tfrac{1}{3} \, \partial v / \partial z & - \tfrac{1}{3} \, \partial w / \partial y  \\
	0 & 0 & - \tfrac{1}{3} \, \partial w / \partial y  \\
	-\tfrac{1}{3} \, \partial u / \partial z  & \tfrac{1}{3} \, \partial v / \partial z & \partial w / \partial y \\
	0 & 0 & 0  \\
	2 \, \partial u / \partial y &  0 & 0 \\
	0 &  0 &  2 \, \partial w / \partial x  \end{bmatrix} \, \begin{bmatrix}
N_{i,z} &  0 & 0  \\
0 & N_{i,z} & 0  \\
0 & 0 & N_{i,y}  \end{bmatrix}. 
\end{equation}

For the third nonlinear strain term, i.e., $\mathbf{E}_{N3}$, can be written as
\begin{equation}
\begin{aligned}
\mathbf{E}_{N3} & =  \begin{Bmatrix}
\mathrm \xi_{L} \\
\mathrm \varepsilon_{1N} \\                    
\mathrm \varepsilon_{2N} \\           
\mathrm \gamma_{1N} \\                
\mathrm \gamma_{2N} \\                
\mathrm \gamma_{3N} \end{Bmatrix} =   
\begin{Bmatrix}                       
\tfrac{1}{6} \,  \left( - 2 \, v_{,x} \, u_{,y} - 4 \, w_{,y} \, v_{,z} +  w_{,x}^2  \right)\\
\tfrac{1}{6} \,  \left( 2 \, v_{,x} \, u_{,y}  +  w_{,x}^2  \right) \\
\tfrac{1}{6} \,  \left( - 2 \, v_{,x} \, u_{,y} + 4 \, w_{,y} \, v_{,z} \right) \\
0 \\
0 \\ 
0 \end{Bmatrix} \\
& = \frac{1}{2} \, \begin{bmatrix}
- \tfrac{2}{3} \, \partial v / \partial x &  - \tfrac{4}{3} \, \partial w / \partial y &  \tfrac{1}{3} \, \partial w / \partial x   \\
\tfrac{2}{3} \, \partial v / \partial x &  0 & \tfrac{1}{3} \, \partial w / \partial x  \\
- \tfrac{2}{3} \, \partial v / \partial x & \tfrac{4}{3} \, \partial w / \partial y & 0 \\
0 &  0 & 0  \\
0 &  0 & 0  \\
0 &  0 & 0  \end{bmatrix} \, \begin{Bmatrix}
\partial u / \partial y\\
\partial v / \partial z \\
\partial w / \partial x
\end{Bmatrix} \\
& = \tfrac{1}{2} \, \mathbf{A}_3 \, \boldsymbol{\theta}_3
\end{aligned}
\end{equation}
with  
\begin{subequations}
	\begin{align}
	\mathrm{d} \mathbf{A}_3 & =  \begin{bmatrix}
		- \tfrac{2}{3} \, \partial \mathrm{d} v / \partial x &  - \tfrac{4}{3} \, \partial \mathrm{d} w / \partial y &  \tfrac{1}{3} \, \partial \mathrm{d} w / \partial x   \\
		\tfrac{2}{3} \, \partial \mathrm{d} v / \partial x &  0 & \tfrac{1}{3} \, \partial \mathrm{d} w / \partial x  \\
		- \tfrac{2}{3} \, \partial \mathrm{d} v / \partial x & \tfrac{4}{3} \, \partial \mathrm{d} w / \partial y & 0 \\
		0 &  0 & 0  \\
		0 &  0 & 0  \\
		0 &  0 & 0  \end{bmatrix} \notag\\
	& = \begin{bmatrix}
		- \tfrac{2}{3} \, \, \sum\nolimits_{i=1}^4 N_{i,x} \, \mathrm{d} v_i &  - \tfrac{4}{3} \, \sum\nolimits_{i=1}^4 N_{i,y} \, \mathrm{d} w_i &  \tfrac{1}{3} \, \sum\nolimits_{i=1}^4 N_{i,x} \, \mathrm{d} w_i  \\
		\tfrac{2}{3} \, \sum\nolimits_{i=1}^4 N_{i,x} \, \mathrm{d} v_i &  0 & \tfrac{1}{3} \, \sum\nolimits_{i=1}^4 N_{i,x} \, \mathrm{d} w_i  \\
		- \tfrac{2}{3} \, \sum\nolimits_{i=1}^4 N_{i,x} \, \mathrm{d} v_i & \tfrac{4}{3} \, \, \sum\nolimits_{i=1}^4 N_{i,y} \, \mathrm{d} w_i & 0 \\
		0 &  0 & 0  \\
		0 &  0 & 0  \\
		0 &  0 & 0  \end{bmatrix} \notag \\	
	& = \begin{bmatrix}
	[\mathbf{l}_1] [\mathbf{l}_2] [\mathbf{l}_3] [\mathbf{l}_4] 
	\end{bmatrix}  \, \begin{bmatrix} [\mathbf{d}_1] [\mathbf{d}_2] [\mathbf{d}_3] [\mathbf{d}_4] \end{bmatrix}^{\mathsf{T}}
= \mathbf{L}_3 \mathbf{D}_3 \\
	\intertext{wherein}
	[\mathbf{l}_i] & = \begin{bmatrix}
		- \tfrac{2}{3} \, N_{i,x}  &  - \tfrac{4}{3} \,  N_{i,y}  &  \tfrac{1}{3} \,  N_{i,x} \\
		\tfrac{2}{3} \,  N_{i,x}  &  0 & \tfrac{1}{3}  N_{i,x}  \\
		- \tfrac{2}{3} \,  N_{i,x}  & \tfrac{4}{3} \, N_{i,y}  & 0 \\
		0 &  0 & 0  \\
		0 &  0 & 0  \\
		0 &  0 & 0  \end{bmatrix}  \quad \text{and} \quad	[\mathbf{d}_i] = \begin{bmatrix}
		\mathrm{d} u_i & 0  & 0  \\
		0 &  \mathrm{d} v_i   & 0  \\
		0 & 0 & \mathrm{d} w_i \end{bmatrix} .
	\end{align}
\end{subequations}

Here the slope vector $\boldsymbol{\theta}_3$ contains derivatives of displacements with respect to spatial variables $y$, $z$, $x$ that relate to the nodal parameters via
\small
\begin{equation}
\boldsymbol{\theta}_3 =  \begin{Bmatrix}
\partial u / \partial y\\
\partial v / \partial z \\
\partial w / \partial x
\end{Bmatrix}
= \begin{Bmatrix}
\sum\nolimits_{i=1}^4 N_{i,y} \, u_i\\
\sum\nolimits_{i=1}^4 N_{i,z} \, v_i \\
\sum\nolimits_{i=1}^4 N_{i,x} \, w_i
\end{Bmatrix} 
= \begin{bmatrix}
[\mathbf{h}_1] [\mathbf{h}_2] [\mathbf{h}_3] [\mathbf{h}_4] 
\end{bmatrix} \bigl\{ \begin{matrix} \mathbf{u}^{(e)} \end{matrix} \bigr\}
= \mathbf{H}_3 \mathbf{u}^{(e)}
\end{equation}
\normalsize
where 
\begin{equation}
[\mathbf{h}_i] = \begin{bmatrix}
N_{i,y} &  0 & 0  \\
0 & N_{i,z} & 0  \\
0 & 0 & N_{i,x} \end{bmatrix}. 
\end{equation}
Therefore, the strain-displacement matrix $\mathbf{B}_{N3}$ becomes
\begin{equation}
\mathbf{B}_{N3} = \mathbf{A}_3 \, \mathbf{H}_3 = \begin{bmatrix}
[\mathbf{b}_{N1}] [\mathbf{b}_{N2}] [\mathbf{b}_{N3}] [\mathbf{b}_{N4}]
\end{bmatrix} 
\end{equation}
whose components are given as
\begin{equation}
\mathbf{b}_{Ni} = \begin{bmatrix}
	- \tfrac{2}{3} \, \partial v / \partial x &  - \tfrac{4}{3} \, \partial w / \partial y &  \tfrac{1}{3} \, \partial w / \partial x   \\
	\tfrac{2}{3} \, \partial v / \partial x &  0 & \tfrac{1}{3} \, \partial w / \partial x  \\
	- \tfrac{2}{3} \, \partial v / \partial x & \tfrac{4}{3} \, \partial w / \partial y & 0 \\
	0 &  0 & 0  \\
	0 &  0 & 0  \\
	0 &  0 & 0  \end{bmatrix} \, \begin{bmatrix}
N_{i,y} &  0 & 0  \\
0 & N_{i,z} & 0  \\
0 & 0 & N_{i,x}  \end{bmatrix}.
\end{equation}

For the fourth nonlinear strain term, $\mathbf{E}_{N4}$ can be written as
\begin{equation}
	\begin{aligned}
		\mathbf{E}_{N4} & =  \begin{Bmatrix}
			\mathrm \xi_{L} \\
			\mathrm \varepsilon_{1N} \\                    
			\mathrm \varepsilon_{2N} \\           
			\mathrm \gamma_{1N} \\                
			\mathrm \gamma_{2N} \\                
			\mathrm \gamma_{3N} \end{Bmatrix} =   
		\begin{Bmatrix}                       
			0\\
			\tfrac{2}{6} \, v_{,x}^2   \\
			- \tfrac{1}{6} \, v_{,x}^2  \\
			2 \, v_{,z} \, u_{,x} + 2 \, u_{,x} \, w_{,y}   \\
			- 2 \, v_{,z} \, u_{,x} - 2 \, u_{,x} \, w_{,y} \\ 
			- 2 \, v_{,x} \, u_{,x}  \end{Bmatrix} \\
		& = \frac{1}{2} \, \begin{bmatrix}
			0 & 0 &  0   \\
			0 &  \tfrac{2}{3} \, \partial v / \partial x & 0  \\
			0 &  - \tfrac{1}{3} \, \partial v / \partial x & 0 \\
			4 \, \partial v / \partial z &  0 & 4 \, \partial u / \partial x  \\
			-4 \,\partial v / \partial z &  0 & -4 \,\partial u / \partial x  \\
			-4 \,\partial v / \partial x &  0 & 0  \end{bmatrix} \, \begin{Bmatrix}
			\partial u / \partial x\\
			\partial v / \partial x \\
			\partial w / \partial y
		\end{Bmatrix} \\
	    &	= \tfrac{1}{2} \, \mathbf{A}_4 \, \boldsymbol{\theta}_4
	\end{aligned}
\end{equation}
with  
\begin{subequations}
	\begin{align}
		\mathrm{d} \mathbf{A}_4 & =  \begin{bmatrix}
			0 & 0 &  0   \\
			0 &  \tfrac{2}{3} \, \partial \mathrm{d} v / \partial x & 0  \\
			0 &  - \tfrac{1}{3} \, \partial \mathrm{d} v / \partial x & 0 \\
			4 \, \partial \mathrm{d} v / \partial z &  0 & 4 \, \partial \mathrm{d} u / \partial x  \\
			-4 \, \partial \mathrm{d} v / \partial z &  0 & -4 \, \partial \mathrm{d} u / \partial x  \\
			-4 \, \partial \mathrm{d} v / \partial x &  0 & 0  \end{bmatrix} \notag\\
		& = \begin{bmatrix}
			0 &  0 &  0  \\
			0 &  \tfrac{2}{3} \, \sum\nolimits_{i=1}^4 N_{i,x} \, \mathrm{d} v_i & 0  \\
			0 &  - \tfrac{1}{3} \, \sum\nolimits_{i=1}^4 N_{i,x} \, \mathrm{d} v_i & 0  \\
			4 \, \sum\nolimits_{i=1}^4 N_{i,z} \, \mathrm{d} v_i &  0 & 4 \, \sum\nolimits_{i=1}^4 N_{i,x} \, \mathrm{d} u_i  \\
			-4 \, \sum\nolimits_{i=1}^4 N_{i,z} \, \mathrm{d} v_i &  0 & - 4 \, \sum\nolimits_{i=1}^4 N_{i,x} \, \mathrm{d} u_i  \\
			- 4 \, \sum\nolimits_{i=1}^4 N_{i,x} \, \mathrm{d} v_i &  0 & 0  \end{bmatrix} \notag \\	
		& = \begin{bmatrix}
			[\mathbf{l}_1] [\mathbf{l}_2] [\mathbf{l}_3] [\mathbf{l}_4] 
		\end{bmatrix}  \, \begin{bmatrix} [\mathbf{d}_1] [\mathbf{d}_2] [\mathbf{d}_3] [\mathbf{d}_4] \end{bmatrix}^{\mathsf{T}}
	= \mathbf{L}_4 \mathbf{D}_4 \\
		\intertext{wherein}
		[\mathbf{l}_i] & = \begin{bmatrix}
			0 &  0 &  0  \\
			0 &  \tfrac{2}{3} \, N_{i,x}  & 0  \\
			0 &  - \tfrac{1}{3} \,  N_{i,x} \,  & 0  \\
			4 \,  N_{i,z}  &  0 & 4 \,  N_{i,x}   \\
			-4 \,  N_{i,z} &  0 & - 4 \,  N_{i,x}  \\
			- 4 \,  N_{i,x} &  0 & 0  \end{bmatrix}   \quad \text{and} \quad	[\mathbf{d}_i] = \begin{bmatrix}
			\mathrm{d} u_i & 0  & 0  \\
			0 &  \mathrm{d} v_i   & 0  \\
			0 & 0 & \mathrm{d} w_i \end{bmatrix} .
	\end{align}
\end{subequations}

Here the slope vector $\boldsymbol{\theta}_4$ contains derivatives of displacements with respect to the spatial variables $x$ and $y$ that relate to the nodal parameters via
\small
\begin{equation}
	\boldsymbol{\theta}_4 =  \begin{Bmatrix}
		\partial u / \partial x\\
		\partial v / \partial x \\
		\partial w / \partial y
	\end{Bmatrix}
	= \begin{Bmatrix}
		\sum\nolimits_{i=1}^4 N_{i,x} \, u_i\\
		\sum\nolimits_{i=1}^4 N_{i,x} \, v_i \\
		\sum\nolimits_{i=1}^4 N_{i,y} \, w_i
	\end{Bmatrix} 
	= \begin{bmatrix}
		[\mathbf{h}_1] [\mathbf{h}_2] [\mathbf{h}_3] [\mathbf{h}_4] 
	\end{bmatrix} \bigl\{ \begin{matrix} \mathbf{u}^{(e)} \end{matrix} \bigr\}
	= \mathbf{H}_4  \mathbf{u}^{(e)} 
\end{equation}
\normalsize
where 
\begin{equation}
	[\mathbf{h}_i] = \begin{bmatrix}
		N_{i,x} &  0 & 0  \\
		0 & N_{i,x} & 0  \\
		0 & 0 & N_{i,y} \end{bmatrix} . 
\end{equation}
Therefore, the strain--displacement matrix $\mathbf{B}_{N4}$ becomes
\begin{equation}
	\mathbf{B}_{N4} = \mathbf{A}_4 \, \mathbf{H}_4 = \begin{bmatrix}
		[\mathbf{b}_{N1}] [\mathbf{b}_{N2}] [\mathbf{b}_{N3}] [\mathbf{b}_{N4}]
	\end{bmatrix} 
\end{equation}
whose components are given as
\begin{equation}
	\mathbf{b}_{Ni} = \begin{bmatrix}
		0 & 0 &  0   \\
		0 &  \tfrac{2}{3} \, \partial v / \partial x & 0  \\
		0 &  - \tfrac{1}{3} \, \partial v / \partial x & 0 \\
		4 \, \partial v / \partial z &  0 & 4 \, \partial u / \partial x  \\
		-4 \,\partial v / \partial z &  0 & -4 \,\partial u / \partial x  \\
		-4 \,\partial v / \partial x &  0 & 0  \end{bmatrix} \, \begin{bmatrix}
		N_{i,x} &  0 & 0  \\
		0 & N_{i,x} & 0  \\
		0 & 0 & N_{i,y}  \end{bmatrix}.
\end{equation}

The last nonlinear strain term, $\mathbf{E}_{N5}$ can be written as
\begin{equation}
	\begin{aligned}
		\mathbf{E}_{N5} & =  \begin{Bmatrix}
			\mathrm \xi_{L} \\
			\mathrm \varepsilon_{1N} \\                    
			\mathrm \varepsilon_{2N} \\           
			\mathrm \gamma_{1N} \\                
			\mathrm \gamma_{2N} \\                
			\mathrm \gamma_{3N} \end{Bmatrix} =   
		\begin{Bmatrix}                       
			0\\
			0   \\
			0  \\
			-2 \, v_{,x} \, u_{,z} - 4 \, w_{,y} \, v_{,y}  - 2 \, v_{,x} \, w_{,x} \\
			0 \\ 
			0 \end{Bmatrix} \\
		& = \frac{1}{2} \, \begin{bmatrix}
			0 & 0 &  0   \\
			0 &  0 & 0  \\
			0 &  0 & 0 \\
			- 2 \, \partial v / \partial x & - 4 \, \partial w / \partial y & - 2 \, \partial v / \partial x  \\
			0 &  0 & 0  \\
			0 &  0 & 0  \end{bmatrix} \, \begin{Bmatrix}
			\partial u / \partial z\\
			\partial v / \partial y \\
			\partial w / \partial x
		\end{Bmatrix} \\
	    &	= \tfrac{1}{2} \, \mathbf{A}_5 \, \boldsymbol{\theta}_5
	\end{aligned}
\end{equation}
with  
\begin{subequations}
	\begin{align}
		\mathrm{d} \mathbf{A}_5 & =  \begin{bmatrix}
			0 & 0 &  0   \\
			0 &  0 & 0  \\
			0 &  0 & 0 \\
			- 2 \, \partial \mathrm{d} v / \partial x & - 4 \, \partial \mathrm{d} w / \partial y & - 2 \, \partial \mathrm{d} v / \partial x  \\
			0 &  0 & 0  \\
			0 &  0 & 0  \end{bmatrix} \notag\\
		& = \begin{bmatrix}
			0 & 0 &  0   \\
			0 &  0 & 0  \\
			0 &  0 & 0 \\
			- 2 \, \sum\nolimits_{i=1}^4 N_{i,x} \, \mathrm{d} v_i  & - 4 \, \sum\nolimits_{i=1}^4 N_{i,y} \, \mathrm{d} w_i  & - 2 \, \sum\nolimits_{i=1}^4 N_{i,x} \, \mathrm{d} v_i   \\
			0 &  0 & 0  \\
			0 &  0 & 0  \end{bmatrix} \notag \\	
		& = \begin{bmatrix}
			[\mathbf{l}_1] [\mathbf{l}_2] [\mathbf{l}_3] [\mathbf{l}_4] 
		\end{bmatrix}  \, \begin{bmatrix} [\mathbf{d}_1] [\mathbf{d}_2] [\mathbf{d}_3] [\mathbf{d}_4] \end{bmatrix}^{\mathsf{T}}
	= \mathbf{L}_5 \mathbf{D}_5 \\
		\intertext{wherein}
		[\mathbf{l}_i] & = \begin{bmatrix}
			0 & 0 &  0   \\
			0 &  0 & 0  \\
			0 &  0 & 0 \\
			- 2 \, N_{i,x}  & - 4 \,  N_{i,y}  & - 2 \, N_{i,x}  \\
			0 &  0 & 0  \\
			0 &  0 & 0  \end{bmatrix}    \quad \text{and} \quad	[\mathbf{d}_i] = \begin{bmatrix}
			\mathrm{d} u_i & 0  & 0  \\
			0 &  \mathrm{d} v_i   & 0  \\
			0 & 0 & \mathrm{d} w_i \end{bmatrix} .
	\end{align}
\end{subequations}

Here the slope vector $\boldsymbol{\theta}_5$ contains derivatives of displacements with respect to the spatial variables $z, y$ and $x$ that relate to the nodal parameters via
\begin{equation}
	\boldsymbol{\theta}_5 =  \begin{Bmatrix}
		\partial u / \partial z\\
		\partial v / \partial y \\
		\partial w / \partial x
	\end{Bmatrix}
	= \begin{Bmatrix}
		\sum\nolimits_{i=1}^4 N_{i,z} \, u_i\\
		\sum\nolimits_{i=1}^4 N_{i,y} \, v_i \\
		\sum\nolimits_{i=1}^4 N_{i,x} \, w_i
	\end{Bmatrix} 
	= \begin{bmatrix}
		[\mathbf{h}_1] [\mathbf{h}_2] [\mathbf{h}_3] [\mathbf{h}_4] 
	\end{bmatrix} \bigl\{ \begin{matrix} \mathbf{u}^{(e)} \end{matrix} \bigr\}
	= \mathbf{H}_5  \mathbf{u}^{(e)}
\end{equation}
where 
\begin{equation}
	[\mathbf{h}_i] = \begin{bmatrix}
		N_{i,z} &  0 & 0  \\
		0 & N_{i,y} & 0  \\
		0 & 0 & N_{i,x} \end{bmatrix}. 
\end{equation}
Therefore, the strain-displacement matrix $\mathbf{B}_{N5}$ becomes
\begin{equation}
	\mathbf{B}_{N5} = \mathbf{A}_5 \, \mathbf{H}_5 = \begin{bmatrix}
		[\mathbf{b}_{N1}] [\mathbf{b}_{N2}] [\mathbf{b}_{N3}] [\mathbf{b}_{N4}]
	\end{bmatrix} 
\end{equation}
whose components are given as
\begin{equation}
	\mathbf{b}_{Ni} = \begin{bmatrix}
		0 & 0 &  0   \\
		0 &  0 & 0  \\
		0 &  0 & 0 \\
		- 2 \, \partial v / \partial x & - 4 \, \partial w / \partial y & - 2 \, \partial v / \partial x  \\
		0 &  0 & 0  \\
		0 &  0 & 0  \end{bmatrix} \, \begin{bmatrix}
		N_{i,z} &  0 & 0  \\
		0 & N_{i,y} & 0  \\
		0 & 0 & N_{i,x}  \end{bmatrix}.
\end{equation}

The total nonlinear strain-displacement matrix $\mathbf{B}_N$ can be obtained as a sum of its five components, i.e., $\mathbf{B}_N = \sum_{i=1}^5 \mathbf{B}_{Ni}$.

With all the strain-displacement matrices evaluated, now we are able to find the total stiffness matrix.


To obtain the stiffness matrix for a tetrahedron, we need to compute the four constituent strain-displacement matrices $\mathbf{C}^s$, $\mathbf{C}^t$, $\mathbf{K}^s$ and $\mathbf{K}^t$, as mentioned earlier. These strain-displacement matrices are obtained by expressing their corresponding strain components in terms of the four nodal displacements, with help from the shape functions.

\subsubsection{Tangent Stiffness Matrix $\mathbf{C}^s$}

The tangent stiffness matrix $\mathbf{C}^s$, as established in Eqn.~\eqref{secantStiffness}, becomes 
\begin{subequations}
	\begin{gather}
		\mathbf{C}^s = \int_{V} \mathbf{H}^{\mathsf{T}} \,  \mathbf{S}^s \, \mathbf{H} \, \mathrm{d} V
		=  | \mathbf{J} |  \, \sum_{i=1}^{n}  \mathbf{H}^{\mathsf{T}} \, \mathbf{S}^s (\xi_i , \eta_i, \zeta_i) \, \mathbf{H} \, \mathrm{w}_i \\
        \intertext{wherein $\xi_i, \eta_i, \zeta_i$ and $w_i$ are the co-ordinates and weights of quadrature at the $i^{\mathrm{th}}$ Gauss point, and 
        }
        \mathbf{S}^s = \begin{bmatrix}
        	\mathcal{S}_{11} & \mathcal{S}_{12} & \mathcal{S}_{13} \\
        	\mathcal{S}_{21} & \mathcal{S}_{22} & \mathcal{S}_{23} \\
        	\mathcal{S}_{31} & \mathcal{S}_{32} & \mathcal{S}_{33}
        	\end{bmatrix}
        \label{stressmtx3D}
	\end{gather}
\end{subequations}


The stress vector $\boldsymbol{T} = \{ \Pi, \sigma_1, \sigma_2, \tau_1, \tau_2, \tau_3 \}^{\mathsf{T}}$ conjugate to strain $\boldsymbol{E}$ has elements that comprise a pressure $\Pi = \mathcal{S}_{11} + \mathcal{S}_{22} + \mathcal{S}_{33} = -3P$ where $P$ denotes the common definition for pressure, two separate normal-stress differences $\sigma_1 = \mathcal{S}_{11} - \mathcal{S}_{22}$ and $\sigma_2 = \mathcal{S}_{22} - \mathcal{S}_{33}$, and three separate shear stresses $\tau_1 = \frac{b}{c} \mathcal{S}_{32}$, $\tau_2 = \frac{a}{c} \mathcal{S}_{31}$ and $\tau_3 = \frac{a}{b} \mathcal{S}_{21} - \alpha \tau_2$.  Of these, only pressure has an initial value, viz., $\Pi_0$, which represents atmospheric pressure. 
In a reciprocal sense, the stress components are assigned via $\mathcal{S}_{11} = \frac{1}{3} ( \pi + 2 \sigma_1 + \sigma_2 )$, $\mathcal{S}_{22} = \frac{1}{3} ( \pi - \sigma_1 + \sigma_2 )$, $\mathcal{S}_{33} = \frac{1}{3} ( \pi - \sigma_1 - 2 \sigma_2 )$, $\mathcal{S}_{12} = \mathcal{S}_{21} = \frac{b}{a} (\tau_3 + \alpha \, \tau_2)$, $\mathcal{S}_{13} = \mathcal{S}_{31} = \frac{c}{a} \, \tau_2$ and $\mathcal{S}_{32} = \mathcal{S}_{23} = \frac{c}{a} \tau_1$ such that $\mathbf{S} = \mathbf{P} \, \boldsymbol{\mathcal{U}}^{-1} \boldsymbol{\mathcal{S}} \, \boldsymbol{\mathcal{U}}^{-\mathsf{T}} \mathbf{P}^{\mathsf{T}}$ with $\mathbf{S}$ being the second Piola-Kirchhoff stress, while $\boldsymbol{\mathcal{U}}$ is the Laplace stretch, and $\mathbf{P}$ is a re-indexer of co-ordinate labeling needed to ensure invariance under a transformation of Laplace stretch.

\subsubsection{Tangent Stiffness Matrix $\mathbf{C}^t$}

The tangent stiffness matrix $\mathbf{C}^t$, as established in Eqn.~\eqref{tangentStiffness}, becomes
\begin{equation}
	\begin{aligned}
		\mathbf{C}^t & =\int_{V} \mathbf{B}^{\mathsf{T}} \,  \boldsymbol{\mathcal{M}}^t \, \mathbf{B}  \,  \mathrm{d} V
		= |\mathbf{J}| \sum_{i=1}^{n} \mathbf{B}^{\mathsf{T}} \, \boldsymbol{\mathcal{M}}^t (\xi_i, \eta_i, \zeta_i) \, \mathbf{B} \, \mathrm{w}_i
	\end{aligned}
\end{equation} 
where axial stress rate $\mathrm{d} s$ is described by a tangent modulus $\boldsymbol{\mathcal{M}}^{t}$.

\subsubsection{Secant Stiffness Matrix $\mathbf{K}^s$}

The secant stiffness matrix $\mathbf{K}^s$, as established in Eqn.~\eqref{secantStiffnessMatrix}, becomes
\begin{equation}
	\begin{aligned}
		\mathbf{K}^s & =\int_{V} \mathbf{B}^{\mathsf{T}} \,  \boldsymbol{\mathcal{M}}^s \, \mathbf{B}  \, \mathrm{d} V
		= |\mathbf{J}| \sum_{i=1}^{n}  \mathbf{B}^{\mathsf{T}} \, \boldsymbol{\mathcal{M}}^s (\xi_i, \eta_i, \zeta_i) \, \mathbf{B} \, \mathrm{w}_i 
	\end{aligned}
\end{equation} 
where axial stress rate $\mathrm{d} s$ is described by a tangent modulus $\boldsymbol{\mathcal{M}}^{s}$.


\subsubsection{Secant Stiffness Matrix $\mathbf{K}^t$}

Likewise, a secant stiffness matrix $\mathbf{K}^t$, also established in Eqn.~\eqref{tangentStiffness}, becomes 
\begin{equation} 
	\begin{aligned}
		\mathbf{K}^t & = \int_{V} \mathbf{H}^{\mathsf{T}} \,  \mathrm{d} \mathbf{S}^t \, \mathbf{H}  \,  \mathrm{d} V
		= | \mathbf{J} | \sum_{i=1}^{n}  \mathbf{H}^{\mathsf{T}} \,  \mathrm{d} \mathbf{S}^t (\xi_i, \eta_i, \zeta_i) \, \mathbf{H} \, \mathrm{w}_i 
	\end{aligned}
\end{equation}
where $\mathrm{d} \mathbf{S}^t \defeq \mathbf{A}^{\mathsf{T}} \, \boldsymbol{\mathcal{M}}^t \, \mathrm{d} \mathbf{A}$.




\section{Force Vector}
\label{secForceVectors}


%In the lung structure, the lung tissue(elastin and collagen) and the alveoli are two sources of force trying to collapse the lung. These inward forces are called the forces of recoil.  Inside of the alveoli, there is a fluid that makes the surface moist(a membrane of fluid is sticking to the surface) that is making an interface with the air, and this air-fluid interface also causes pressure inwards which try to collapse the alveoli called tension. The lung tissue is a third, and the fluid surface tension is $2/3$ of the total recoil. 

%The expansile force is the transpulmonary force, which is the difference between the fluid in the pleural cavity that is called pleural pressure ($-5 cm H_2O$ at rest) and the pressure inside the alveoli that is called alveolar pressure ($0 cm H_2O$ at rest). 

%If the alveolus is filled with air along with a pure water layer(air-fluid interface), the water is going to try to contract and pull the alveolus inverse. That is the tension created by the fluid-air interface. 
%Since the alveolus is squeezing, the air inside it develops pressure on the alveolus.
%This pressure and the recoiling forces and the expansile forces have to become equal for the alveolus to stay stable. 

The principle of stationary potential energy via the Rayleigh-Ritz approach, i.e., Eqn.~(\ref{principle}), determines a basis for finite element stress analysis. The internal strain energy is balanced with the potential energy of applied internal and external loads on the body.

The virtual work done by external forces $\delta{W}$ in Eq.~(\ref{principle}) can be expressed as
\begin{subequations}
\begin{align}
\delta{W} & = \int_{S} \, \mathbf{t} \, \delta \mathbf{u} \, \mathrm{d} S
= \int_{S} \, \mathbf{t} \, \mathbf{N} \, \mathrm{d} \boldsymbol{\Delta} \, \mathrm{d} S = \left( \int_{S} \, \mathbf{N}^{\mathsf{T}} \, \mathbf{t} \, \mathrm{d} S \right) \, \mathrm{d} \boldsymbol{\Delta} 
\intertext{where $\mathrm{d} S$ denotes the surface element and $\mathbf{t}$ is the surface traction vector (per unit surface area) at current time. Hence, the external  $\mathbf{F}_{BC}$ force vectors are}
\mathbf{F}_{BC} & = \int_{S} \, \mathbf{N}^{\mathsf{T}} \, \mathbf{t} \, \mathrm{d} S 
\end{align}
\end{subequations}

To compute $\boldsymbol{F}_0$ as two separate integrals, viz.,
\begin{equation}
	\boldsymbol{F}_0 = \int_V \mathbf{B}_L^{\mathsf{T}} \,	\boldsymbol{T}_0 \, 
	\mathrm{d}V + 	\int_V \mathbf{B}_N^{\mathsf{T}} \,	\boldsymbol{T}_0 \, \mathrm{d}V
\end{equation}

where the first integral only needs to be evaluated once, as the argument is constant valued.


\subsection{Force Vector for a Chord}

Following the procedure described above,the force vector of the 1-D alveolar chord is evaluated numerically in its natural co-ordinate system as
\begin{equation}
\mathbf{F}_{BC} = \int_L \mathbf{N}^{\mathsf{T}} \, \mathbf{t} \, \mathrm{d} L  =  \sum_{i=1}^{n} \mathbf{N}^{\mathsf{T}} \, \mathbf{t} \, | \mathbf{J} | \, \mathrm{w}_i
\end{equation}
where $\mathrm{w}_i$ are the  weighting coefficients of the Gauss integration rule, $\mathbf{N}$ is the shape function matrix for chord, and $\mathbf{t}$ is the traction on the septal chord that is selected so that the traction can be additively decomposed into that carried by the collagen and elastin fibers, i.e., $\mathbf{t} = \mathbf{t}^c + \mathbf{t}^e $ as described in part 4.

The internal force $\boldsymbol{F}_0$ accounting for an initial residual stress of $\boldsymbol{T}_0$ as two separate integrals can be computed as

\begin{subequations}
	\begin{align}
		\boldsymbol{F}_0 & = \int_L \mathbf{B}_L^{\mathsf{T}} \,	\boldsymbol{T}_0 \, A \, \mathrm{d} L + \int_L \mathbf{B}_N^{\mathsf{T}} \,	\boldsymbol{T}_0 \, A \, \mathrm{d} L \\
		& =  \sum_{i=1}^{n} \mathbf{B}_L^{\mathsf{T}} \,	\boldsymbol{T}_0 \, A \, | \mathbf{J} | \, \mathrm{w}_i +  \sum_{i=1}^{n} \mathbf{B}_N^{\mathsf{T}} \,	\boldsymbol{T}_0 \, A \, | \mathbf{J} | \, \mathrm{w}_i
		\intertext{where the first integral will only need to be evaluated once, as the argument is constant valued.
		Here the initial stress $\mathbf{T}_0 = [s_0] \mapsto \mathbf{S}_0 = [s_0]$  contains the initial stress $s_0$ carried by the collagen and elastin fibers; specifically, from the rule of mixtures}
		s_0 & = \bigl( \phi s_0^c + (1-\phi) s_0^e \bigr) 
		\quad \text{where} \quad
		\phi \defeq A^c_0 / ( A^c_0 + A^e_0 ) = A^c_0 / A_0
	\end{align}
\end{subequations}
where $A_0$ and $A$ are the initial and current cross-sectional areas of the chord.
The superscripts $c$ and $e$ designating collagen and elastin.  

\subsection{Force Vector for a Pentagon}

The boundary of a two dimensional pentagon consists of line segments, which can be considered as one-dimensional chords. Hence, the evaluation of the boundary integrals on pentagon amounts to evaluating the line integrals on these boundary lines. Once the interpolation function for a pentagon are evaluated on the boundary of pentagon, we can obtain the corresponding chordal interpolation functions \cite{Reddy93}.
Thus, the force vector $\mathbf{F}_{BC}$  for a pentagon can be obtained by integrating the traction vectors over all sides of pentagon. Specifically, the force on the boundary of the membrane can be obtained as
\begin{equation}
\begin{aligned}
\mathbf{F}_{BC}  = \oint_L \mathbf{N}^{\mathsf{T}} \, \mathbf{t} \, \mathrm{d} L & = \int_{L_{12}} \mathbf{N}^{\mathsf{T}} \, \mathbf{t}_{12} \,|\mathbf{J}| \, \mathrm{d} L + \int_{L_{23}} \mathbf{N}^{\mathsf{T}} \, \mathbf{t}_{23} \,|\mathbf{J}| \, \mathrm{d} L + \int_{L_{34}} \mathbf{N}^{\mathsf{T}} \, \mathbf{t}_{34} \,|\mathbf{J}| \, \mathrm{d} L \\
& + \int_{L_{45}} \mathbf{N}^{\mathsf{T}} \, \mathbf{t}_{45} \, \,|\mathbf{J}| \, \mathrm{d} L + \int_{L_{51}} \mathbf{N}^{\mathsf{T}} \, \mathbf{t}_{51} \,|\mathbf{J}| \, \mathrm{d} L \\
& = \sum_{i=1}^{n} \mathbf{N}^{\mathsf{T}} \, \mathbf{t}_{12} \,|\mathbf{J}| \, \mathrm{w}_i + \sum_{i=1}^{n} \mathbf{N}^{\mathsf{T}} \, \mathbf{t}_{23} \,|\mathbf{J}| \, \mathrm{w}_i + \sum_{i=1}^{n} \mathbf{N}^{\mathsf{T}} \, \mathbf{t}_{34} \,|\mathbf{J}| \, \mathrm{w}_i \\
& + \sum_{i=1}^{n} \mathbf{N}^{\mathsf{T}} \, \mathbf{t}_{45} \, \,|\mathbf{J}| \, \mathrm{w}_i + \sum_{i=1}^{n} \mathbf{N}^{\mathsf{T}} \, \mathbf{t}_{51} \,|\mathbf{J}| \, \mathrm{w}_i
\end{aligned}
\end{equation}
where $\mathbf{N}$ represents the shape function matrix of a chord, but with the matrix dimension of a pentagon, $|\mathbf{J}|$ is the determinant of the Jacobian for a 1-D chord, $\mathrm{w}_i$ denotes the natural weight of the chord, $\mathrm{d} S$ is the arc-length of an infinitesimal line element along the boundary, and $\mathbf{t}$ is the traction vector on each edge of the pentagon defined as
\begin{equation}
\mathbf{t} = \mathbf{S}^{s\mathsf{T}} \cdot \mathbf{n} 
\end{equation}
where $ \mathbf{n} $ is the normal vector to each sides of pentagon on which the traction acts, and $\mathbf{S}^s $ are established in Eqn.~\eqref{stressMtx2D}.

The internal force $\boldsymbol{F}_0$ accounting for an initial residual stress of $\boldsymbol{T}_0$ becomes

\begin{equation}
	\begin{aligned}
		\boldsymbol{F}_0 &= \int_V \mathbf{B}_L^{\mathsf{T}} \,	\boldsymbol{T}_0 \, 
		\mathrm{d}V + 	\int_V \mathbf{B}_N^{\mathsf{T}} \,	\boldsymbol{T}_0 \, \mathrm{d}V \\
		& =  \sum_{i=1}^{n} \mathbf{B}_L^{\mathsf{T}} \,	\boldsymbol{T}_0 \, h \, | \mathbf{J} | \, \mathrm{w}_i +  \sum_{i=1}^{n} \mathbf{B}_N^{\mathsf{T}} \,	\boldsymbol{T}_0 \, h \, | \mathbf{J} | \, \mathrm{w}_i
	\end{aligned}
\end{equation}
Here the initial stress $\mathbf{T}_0 = [s_0] \mapsto \mathbf{S}_0 = [s_0]$  contains the initial stress $s_0$ carried by the septal membrane.

\subsection{Force Vector for a Tetrahedron}

The force vector on the alveolar volume is computed by integrating the traction vector over the four boundary surfaces of the tetrahedron. Here the matrix of shape functions, given in Eq.~(\ref{shape3D}), is used to obtain the force vector for tetrahedron. The force vector $\mathbf{F}_{BC}$ can be specified as
\begin{equation}
\begin{aligned}
\mathbf{F}_{BC} &= \oint_{A} \mathbf{N}^{\mathsf{T}} \, \mathbf{t} \, \mathrm{d} A = \int_{\triangle_{1}} \mathbf{N}^{\mathsf{T}} \, \mathbf{t}_{\triangle_{1}} \, \mathrm{d} A  + \int_{\triangle_{2}}  \mathbf{N}^{\mathsf{T}} \, \mathbf{t}_{\triangle_{2}}  \, \mathrm{d} A \\
& + \int_{\triangle_{3}} \mathbf{N}^{\mathsf{T}} \, \mathbf{t}_{\triangle_{3}} \, \mathrm{d} A + \int_{\triangle_{4}} \mathbf{N}^{\mathsf{T}} \, \mathbf{t}_{\triangle_{4}} \, \mathrm{d} A \\
& = \sum_{i=1}^{n} \mathbf{N}^{\mathsf{T}} \, \mathbf{t}_{\triangle_{1}} \,|\mathbf{J}| \, \mathrm{w}_i  + \sum_{i=1}^{n}  \mathbf{N}^{\mathsf{T}} \, \mathbf{t}_{\triangle_{2}} \,|\mathbf{J}| \, \mathrm{w}_i  \\
& + \sum_{i=1}^{n} \mathbf{N}^{\mathsf{T}} \, \mathbf{t}_{\triangle_{3}} \,|\mathbf{J}| \, \mathrm{w}_i  + \sum_{i=1}^{n}  \mathbf{N}^{\mathsf{T}} \, \mathbf{t}_{\triangle_{4}} \,|\mathbf{J}| \, \mathrm{w}_i 
\end{aligned}
\end{equation}
where $\Delta_i, i=1, 2, 3, 4$ represents the triangular boundary surface of the tetrahedron. Here $ \mathbf{N}$ represents the shape function matrix for these associated triangular boundaries.   $|\mathbf{J}|$ is the determinant of the Jacobian for triangle, $n$ denotes the number of Gauss points, $\mathrm{w}_i$ is the natural weight of the triangle from Table~\ref{tab:3nodeTriangle}, and $\mathbf{t}$ is the surface traction on the triangle surface. $\oint$ denotes the integral over the surface boundary of the tetrahedron.

Note that except for the base of the tetrahedron, the tractions on its other sides have equal and opposite contributions to the total force vector. Therefore, contributions from opposite boundary surfaces of the tetrahedron nullify each other. Hence, in order to obtain the total force vector for a tetrahedron, it is sufficient to only consider the contributions due to the traction on its base. Therefore, the total force vector takes the form of

\begin{equation}
\mathbf{F}_{BC} = \sum_{i=1}^{n} \mathbf{N}^{\mathsf{T}} \, \mathbf{t}_{\triangle_{1}} \,|\mathbf{J}| \, \mathrm{w}_i 
\end{equation}
where $\mathbf{t}_{\triangle_{1}}$ is the traction vector on the surface of triangle defined as
\begin{equation}
	\mathbf{t}_{\triangle_{1}} = \mathbf{S}^{s\mathsf{T}} \cdot \mathbf{n} 
\end{equation}
wherein $ \mathbf{n} $ is the normal vector to each sides of tetrahedron on which the traction acts, and $ \mathbf{S}^s $ has been defined in Eqn.~\eqref{stressmtx3D}.

The internal force $\boldsymbol{F}_0$ accounting for an initial residual stress of $\boldsymbol{T}_0$ becomes

\begin{subequations}
	\begin{align}
		\boldsymbol{F}_0 & = \int_{V} \mathbf{B}^{\mathsf{T}} \,	\boldsymbol{T}_0 \, 
		\mathrm{d} V \\
		& =  \sum_{i=1}^{n} \mathbf{B}_L^{\mathsf{T}} \,	\boldsymbol{T}_0 \, | \mathbf{J} | \, \mathrm{w}_i +  \sum_{i=1}^{n} \mathbf{B}_N^{\mathsf{T}} \,	\boldsymbol{T}_0 \, | \mathbf{J} | \, \mathrm{w}_i
	\end{align}
\end{subequations}
where the first integral will only need to be evaluated once, as the argument is constant valued.
Here the stress vector $\boldsymbol{T}_0 = \{ \Pi, \sigma_1, \sigma_2, \tau_1, \tau_2, \tau_3 \}^{\mathsf{T}}$ conjugate to strain $\boldsymbol{E}$, where $\mathbf{T}_0 = [s_0] \mapsto \mathbf{S}_0 = [s_0]$  contains the pressure $s_0$ inside the tetrahedron.

