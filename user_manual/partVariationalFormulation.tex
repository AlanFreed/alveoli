\setcounter{section}{0}
\part{Variational Formulation}
\label{partVariational}

The problem we have set up to solve takes on the general form of 
\begin{equation}
	\mathbf{M} \ddot{\mathbf{x}} + \mathbf{K} \mathbf{x} = \mathbf{f}(t)
\end{equation}
where $\mathbf{M}$ is a mass matrix, $\mathbf{K}$ is a stiffness matrix, $\mathbf{f}$ is a forcing function, and $\mathbf{x}$ is a displacement vector.  

For our problem of interest, 
\begin{subequations}
\begin{align}
	\mathbf{x} & = \sum_{v=1}^{20} \{ x_v , y_v , z_v \}^{\mathsf{T}} \\
	\intertext{are the co-ordinates of vertex $v$ located in the co-ordinate frame of the dodecahedron $(\boldsymbol{\imath} , \boldsymbol{\jmath} , \vec{\mathbfit{k}} )$ so that vectors $\mathbf{f}$ and $\mathbf{x}$ have length 60 while matrices $\mathbf{M}$ and $\mathbf{K}$ have dimension $60 \times 60$ with}
	\mathbf{M} & = \mathbf{M}_{1D} + \mathbf{M}_{2D} + \mathbf{M}_{3D} \\
	\mathbf{K} & = \mathbf{K}_{1D} + \mathbf{K}_{2D} + \mathbf{K}_{3D} \\
	\mathbf{f} & = \mathbf{f}_{1D} + \mathbf{f}_{2D} + \mathbf{f}_{3D}
\end{align}
\end{subequations}
where subscript `$\mbox{}_{1D}$' applies to the alveolar chords, subscript `$\mbox{}_{2D}$' applies to the alveolar septa, and subscript `$\mbox{}_{3D}$' applies to the alveolar volume.

\section{Mass Matrix}
A consistent mass matrix \cite{Archer65} established in its natural co-ordinate system is defined as
\begin{equation}
	\mathbf{M} = \sum_m \int_{V_m} \rho_m \, \mathbf{N}_m^{\mathsf{T}} \mathbf{N}_m \,
	\mathrm{d} V_m
	\label{consistentMassMatrix}
\end{equation}
wherein $\mathbf{N}_m$ is the shape function matrix used to also construct the stiffness matrix for element $m$.

\subsection{Mass Matrix of Chord}
The determinant of the Jacobian matrix is used for the transformation of integral from the global coordinate system to the natural coordinate system by
\begin{equation}
     |J| = \mathrm{\det} J =  \frac{\partial x }{\partial\xi} = \sum\nolimits_{i=1}^n N_{i,\xi} (\xi) \, x_i
\end{equation}
wherein $N_{i}$ are the shape functions for a two-node alveolar chords in its natural coordinate system which are defined in a matrix form as
\begin{equation}
	\mathbf{N} = \begin{bmatrix}
    \frac{1}{2} \, (1 - \xi) &  \frac{1}{2} \, (1 + \xi)
\end{bmatrix} 
\end{equation}
wherein $\xi$ is abscissae of the Gauss integration rule. 

The consistent mass matrix of the 1-D alveolar chord that is evaluated numerically in its natural coordinate system can be described as
\begin{equation}
    \mathbf{M}_{1C} = \int_{\Gamma} \rho \, \mathbf{N}^{\mathsf{T}} \mathbf{N} \, A \, \mathrm{d} x  = \int_{-1}^{1} \rho \, \mathbf{N}^{\mathsf{T}} \mathbf{N}\, A \, |\mathbf{J}|\,  \mathrm{d} \xi =  \sum_{i=1}^{n}  \rho  \, \mathbf{N}^{\mathsf{T}} \mathbf{N} \, A\, |\mathbf{J}| \, \mathrm{w}_i
\end{equation}
 with $\mathrm{w}_i$ being the  weighting coefficients of the Gauss integration rule, and $A$ being the cross section area of alveolar chord. Table~\ref{tabQuadrature1D} demonstrates the values of $\xi$ and $\mathrm{w}_i$ for $n = 1, 2$, and $3$ Gauss integration points.
\begin{table}
    \centering
    \begin{tabular}{|c|rr|}
        \hline
        node & \centering $\xi$ coordinate \phantom{12}  & 
        weight \phantom{12} \\ \hline
        & \multicolumn{2}{|c|}{Exact for Polynomials of Degree $1^{\phantom{|^|}}$} \\ 
        \hline
        1 & 0.0000000000000 & 2.0000000000000 \\ 
        \hline
        & \multicolumn{2}{|c|}{Exact for Polynomials of Degree $3^{\phantom{|^|}}$} \\ \hline
        1 & -0.577350269189 & 1.000000000000\\
        2 & 0.577350269189 & 1.000000000000\\ 
        \hline
        & \multicolumn{2}{|c|}{Exact for Polynomials of Degree $5^{\phantom{|^|}}$} \\ \hline
        1 & -0.774596669241 & 0.555555555556 \\
        2 & 0.000000000000 & 0.888888888889\\
        3 & 0.774596669241 & 0.555555555556\\ 
        \hline
    \end{tabular}
    \caption{Generalized, Gaussian, quadrature, weights and nodes for integrating over a alveolar chord in its natural coordinate system.}
    \label{tabQuadrature1D}
\end{table}

The row-sum techniques is considered to construct the lumped mass matrix for 1-D alveolar chord, that is the sum of the elements of each row of the consistent mass matrix is used as the diagonal element \cite{Reddy93}:
\begin{equation}
{M}_{1L_{ii}} = \sum_{j=1}^n \int_{\Gamma} \rho \, N_i \, N_j \, A \, \mathrm{d} x  = \int_{-1}^{1} \rho \, N_i\, A \, |\mathbf{J}|\,  \mathrm{d} \xi =  \sum_{i=1}^n  \rho  \, N_i\, A\, |\mathbf{J}| \, \mathrm{w}_i
\end{equation}
wherein $\sum_{j=1}^n N_j = 1$. 

The lumped-consistent weighted mass matrix $\mathbf{M}_{1LC} $ of the 1-D alveolar chord is defined as 
\begin{equation}
\mathbf{M}_{1LC}  = (1 - \mu) \, \mathbf{M}_{1C} + \mu \, \mathbf{M}_{1L}
\label{LumconsMass1D}
\end{equation}
wherein $\mu$ is a free scalar parameter that is considered to be $\mu = 1/2$ to minimize low frequency dispersion.


For instance, the consistent mass matrix for alveolar chords with $1$ Gauss integration point that is approximated by the weighted sum of function at the center of chord becomes
\begin{equation}
\mathbf{M}_{1C}  = \frac{\rho \, A \, L}{4}\begin{bmatrix}
1 & 1 \\
1 & 1
\end{bmatrix} 
	\label{ConsMassMatrix1D}
\end{equation}
where $L$ is the length of alveolar chord. The row-sum techniques, gives the lumped mass matrix
\begin{equation}
\mathbf{M}_{1L}  = \frac{\rho \, A \, L}{2}\begin{bmatrix}
1 & 0 \\
0 & 1
\end{bmatrix} 
\label{LumMassMatrix1D}
\end{equation}
and the lumped-consistent weighted mass matrix is constructed as follow 
\begin{equation}
\mathbf{M}_{1LC}  = \frac{\rho \, A \, L}{8}\begin{bmatrix}
3 & 1 \\
1 & 3
\end{bmatrix} 
\label{LumconsMassMatrix1D}
\end{equation}


\subsection{Mass Matrix of Pentagon}
For the alveolar septa, the matrix of shape functions $\mathbf{N}$ is arranged as
 \begin{equation}
	\mathbf{N} = 
	\begin{bmatrix}
	N_1 & 0 & N_2 & 0 & N_3 & 0 & N_4 & 0 & N_5 & 0 \\ 0 & N_1 & 0 & N_2 & 0 & N_3 & 0 & N_4 & 0 & N_5 
\end{bmatrix} 
	\label{shape2D}
\end{equation}
in which $\mathrm{N}_i (i = 1, 2, 3, 4, 5)$ are five shape functions corresponding to the five vertices of the pentagon that are defined in Eq.~(\ref{shapeFunctions}).
The consistent mass matrix $\mathbf{M}_{2C}$ can also be obtained by substituting the above shape function matrix into 
\begin{equation}
    \mathbf{M}_{2C} = \int_{V} \rho \, \mathbf{N}^{\mathsf{T}} \mathbf{N} \, \mathrm{d} V = \int_{\pentagon} \int_{\pentagon} \rho \, \mathbf{N}^{\mathsf{T}} \mathbf{N} \,|\mathbf{J}| \, h \, \mathrm{d} \xi \, \mathrm{d} \eta
    \label{massintegral2d}
\end{equation}
wherein $h$ being membrane thickness, and $|\mathbf{J}|$ being the determinant of the Jacobian matrix for pentagon. 
In quadrilateral derivations,  the Jacobian of two-dimensional transformations that connect the ${x, y}$ to ${\xi, \eta}$ coordinate systems is needed. The components of Jacobian matrix are calculated using derivatives of shape functions with respect to the local and the current global coordinates at the $i^{\mathrm{th}}$ vertex via
\begin{equation}
[\mathbf{J}] = 
\begin{bmatrix}
\partial x / \partial\xi & \partial y / \partial\xi \\
\partial x / \partial\eta & \partial y / \partial\eta 
\end{bmatrix}  
= \begin{bmatrix}
\sum\nolimits_{i=1}^5 N_{i,\xi} (\xi,\eta) \, x_i & \sum\nolimits_{i=1}^5 N_{i,\xi} (\xi,\eta) \, y_i \\
\sum\nolimits_{i=1}^5 N_{i,\eta} (\xi,\eta) \, x_i & \sum\nolimits_{i=1}^5 N_{i,\eta} (\xi,\eta) \, y_i
\end{bmatrix}
\label{jacobianpent}
\end{equation}
The numerical integration of Eq.~(\ref{massintegral2d}) result in 
\begin{equation}
    \mathbf{M}_{2C} = \sum_{p=1}^{n} \rho \, \mathbf{N}^{\mathsf{T}} \mathbf{N} \,|\mathbf{J}| \, h \, w_i
\end{equation}
where $n$ stands for number of Gauss points, and $\mathrm{w}_i$ denotes the natural weight of the element.
There are three mass matrix for each pentagon based upon three Gauss, quadrature rules in Table~\ref{tabQuadrature} which are integrating polynomials of order 3 and 5, respectively. 

The row-sum techniques is considered to construct the Lumped mass matrix for pentagon, that makes the diagonal parameters as follow
\begin{equation}
{M}_{2L_{ii}} = \sum_{j=1}^n \int_{V}  \rho \, N_i \, N_j \, A \, \mathrm{d} V  = \int_{\pentagon} \int_{\pentagon} \rho \, N_i\, A \, |\mathbf{J}|\,  \mathrm{d} \xi \, \mathrm{d} \eta =  \sum_{i=1}^n  \rho  \, N_i\, h\, |\mathbf{J}| \, \mathrm{w}_i
\label{LumMass2D}
\end{equation}
wherein $\sum_{j=1}^n N_j = 1$. 

The lumped-consistent weighted mass matrix $\mathbf{M}_{2LC} $ for the 2-D alveolar septa by choosing $\mu = 1/2$ is defined as 
\begin{equation}
\mathbf{M}_{2LC}  = (1 - \mu) \, \mathbf{M}_{2C} + \mu \, \mathbf{M}_{2L} = \frac{1}{2} \, (\mathbf{M}_{2C} + \mathbf{M}_{2L})
\label{LumconsMass2D}
\end{equation}
For instance,the lumped-consistent mass matrix of a pentagon  with $1$ Gauss integration point at the center of pentagon that is constructed by averaging consistent mass matrix and the lumped mass matrix becomes 
\begin{equation}
\mathbf{M}_{2LC}  = \rho \, h
\begin{bmatrix}
0.28532 & 0 & 0.04755 & 0 & 0.04755 & 0 & 0.04755 & 0 \\ 0.04755 & 0 \\
0 & 0.28532 & 0 & 0.04755 & 0 & 0.04755 & 0 & 0.04755\\ 0 & 0.04755
\\
0.04755 & 0 & 0.28532 & 0 & 0.04755 & 0 & 0.04755 & 0 \\ 0.04755 & 0\\
0 & 0.04755 & 0 & 0.28532 & 0 & 0.04755 & 0 & 0.04755\\ 0 & 0.04755 \\
0.04755 & 0 & 0.04755 & 0 & 0.28532 & 0 & 0.04755 & 0 \\ 0.04755 & 0 \\
0 & 0.04755 & 0 & 0.04755 & 0 & 0.28532 & 0 & 0.04755\\ 0 & 0.04755 \\
0.04755 & 0 & 0.04755 & 0 & 0.04755 & 0 & 0.28532 & 0\\ 0.04755 & 0\\
0 & 0.04755 & 0 & 0.04755 & 0 & 0.04755 & 0 & 0.28532\\ 0 & 0.04755\\
0.04755 & 0 & 0.04755 & 0 & 0.04755 & 0 & 0.04755 & 0 \\ 0.28532 & 0 & 0.04755\\
0 & 0.04755 & 0 & 0.04755 & 0 & 0.04755 & 0 & 0.04755\\ 0 & 0.28532\\
\end{bmatrix} 
\label{LumconsMassMatrix2D}
\end{equation}


\subsection{Mass Matrix of Tetrahedron}

The dodecahedron has 60 individual tetrahedral whereas its origin being the common vertex of all tetrahedrons. Hence, the analysis to find the mass matrix of a tetrahedron is used to reach the mass matrix of whole alveolar volume.

The matrix of shape functions $\mathbf{N}$ for a tetrahedon has the form of
\begin{equation}
	\mathbf{N} =  
\begin{bmatrix*}[r]
	N_1 & 0 & 0 & N_2 & 0 & 0 & N_3 & 0 & 0 & N_4 & 0 & 0 \\
	0 & N_1 & 0 & 0 & N_2 & 0 & 0 & N_3 & 0 & 0 & N_4 & 0 \\
	0 & 0 & N_1 & 0 & 0 & N_2 & 0 & 0 & N_3 & 0 & 0 & N_4
\end{bmatrix*} 
	\label{shape3D}
\end{equation}
in which $\mathrm{N}_i (i = 1, 2, 3, 4)$ are four shape functions corresponding to the four vertices of the tetrahedron that are defined as follow
\begin{subequations}
\begin{align}
	N_1 & = 1 - \xi - \eta - \zeta \\
	N_2 & = \xi \\
	N_3 & = \eta \\
	N_4 & = \zeta
\end{align}
\end{subequations}
The numerical integration is used to obtain the mass matrix of tetrahedron via
\begin{equation}
    \mathbf{M}_{3D} = \sum_{p=1}^{n_p} \rho  \mathbf{N}^{\mathsf{T}} \mathbf{N} \,  \,|\mathbf{J}| \, w_p
\end{equation}
wherein $|\mathbf{J}|$ being the determinant of the Jacobian matrix in a tetrahedron that are calculated using derivatives of shape functions with respect to the local coordinates $(\xi, \eta, \zeta)$, and the current global coordinates $(x_i, y_i, z_i)$ at the $i^{\mathrm{th}}$ vertex via
\begin{equation}
\begin{aligned}
\left[\mathbf{J}\right]= &
\begin{bmatrix}
\partial x / \partial\xi & \partial y / \partial\xi & \partial z / \partial\xi\\
\partial x / \partial\eta & \partial y / \partial\eta & \partial z / \partial\eta \\
\partial x / \partial\zeta & \partial y / \partial\zeta & \partial z / \partial\zeta 
\end{bmatrix}\\
  = & \begin{bmatrix}
\sum\nolimits_{i=1}^4 N_{i,\xi} (\xi,\eta,\zeta) \, x_i & \sum\nolimits_{i=1}^4 N_{i,\xi} (\xi,\eta,\zeta) \, y_i &
\sum\nolimits_{i=1}^4 N_{i,\xi} (\xi,\eta,\zeta) \, z_i\\
\sum\nolimits_{i=1}^4 N_{i,\eta} (\xi,\eta,\zeta) \, x_i & \sum\nolimits_{i=1}^4 N_{i,\eta} (\xi,\eta,\zeta) \, y_i &
\sum\nolimits_{i=1}^4 N_{i,\eta} (\xi,\eta,\zeta) \, z_i\\
\sum\nolimits_{i=1}^4 N_{i,\zeta} (\xi,\eta,\zeta) \, x_i & \sum\nolimits_{i=1}^4 N_{i,\zeta} (\xi,\eta,\zeta) \, y_i &
\sum\nolimits_{i=1}^4 N_{i,\zeta} (\xi,\eta,\zeta) \, z_i
\end{bmatrix}
\end{aligned}
\label{jacobiantet}
\end{equation}

There are three mass matrix for each tetrahedron based upon three Gauss, quadrature rules in Table~\ref{tabQuadraturetetra} which are integrating polynomials of order 1, 2 and 3, respectively. 

\begin{table}
    \centering
    \begin{tabular}{|c|rrrr|}
        \hline
        node & \centering $\xi$ coordinate \phantom{1234}  & 
        $\eta$ coordinate \phantom{1234} & 
        $\zeta$ coordinate \phantom{1234} & weight \phantom{12345} \\ \hline        
        & \multicolumn{4}{|c|}{Exact for Polynomials of Degree $1^{\phantom{|^|}}$} \\ 
        \hline
        1 & 1/4 & 1/4 & 1/4 & 1/6 \\ 
        \hline
        & \multicolumn{4}{|c|}{Exact for Polynomials of Degree $2^{\phantom{|^|}}$} \\ \hline
        1 & (5 - $\sqrt{5}$)/20 & (5 - $\sqrt{5}$)/20 & (5 - $\sqrt{5}$)/20 & 1/24\\
        2 & (5 - $\sqrt{5}$)/20 & (5 - $\sqrt{5}$)/20 & (5 + 3 $\sqrt{5}$)/20 & 1/24\\
        3 & (5 - $\sqrt{5}$)/20 & (5 + 3 $\sqrt{5}$)/20 & (5 - $\sqrt{5}$)/20 & 1/24\\ 
        4 & (5 + 3 $\sqrt{5}$)/20 & (5 - $\sqrt{5}$)/20 & (5 - $\sqrt{5}$)/20 & 1/24\\ 
        \hline
        & \multicolumn{4}{|c|}{Exact for Polynomials of Degree $3^{\phantom{|^|}}$} \\ \hline
        1 & 1/4 & 1/4 & 1/4 & -2/15 \\
        2 & 1/6 & 1/6 & 1/6 & 3/40\\
        3 & 1/6 & 1/6 & 1/2 & 3/40\\ 
        4 & 1/6 & 1/2 & 1/6 & 3/40 \\
        5 & 1/2 & 1/6 & 1/6 & 3/40 \\
        \hline
    \end{tabular}
    \caption{Generalized, Gaussian, quadrature, weights and nodes  for integrating over a tetrahedron in its natural coordinate system.}
    \label{tabQuadraturetetra}
\end{table}


The row-sum techniques is considered to construct the lumped mass matrix for pentagon, that makes the diagonal parameters as follow
\begin{equation}
{M}_{3L_{ii}} = \sum_{j=1}^n \int_{V}  \rho \, N_i \, N_j \, A \, \mathrm{d} V = \sum_{i=1}^n  \rho  \, N_i\, |\mathbf{J}| \, \mathrm{w}_i
\label{LumMass3D}
\end{equation}

wherein $\sum_{j=1}^n N_j = 1$. 

The lumped-consistent weighted mass matrix $\mathbf{M}_{3LC} $ for the 3-D alveolar volume by choosing $\mu = 1/2$ becomes 
\begin{equation}
\mathbf{M}_{3LC}  = (1 - \mu) \, \mathbf{M}_{3C} + \mu \, \mathbf{M}_{3L} = \frac{1}{2} \, (\mathbf{M}_{3C} + \mathbf{M}_{3L})
\label{LumconsMass3D}
\end{equation}

