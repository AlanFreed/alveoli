\setcounter{section}{0}
\part{Variational Formulation}
\label{partVariational}

The problem we have set up to solve takes on the general form of a second-order, hyperbolic, ordinary, differential equation; specifically,
\begin{equation}
\mathbf{M} \ddot{\mathbf{u}} + \mathbf{K} \mathbf{u} = \mathbf{f}(t)
\end{equation}
where $\mathbf{M}$ is a mass matrix, $\mathbf{K}$ is a stiffness matrix, $\mathbf{f}$ is a forcing function dependent upon time $t$, and $\mathbf{u}$ is a displacement vector with $\ddot{\mathbf{u}}$ denoting acceleration.  At present, a possible contribution arising from dissipation is not included, e.g., a $\mathbf{C} \dot{\mathbf{u}}$ term accounting for visco\-elastic effects.  The numerical solution strategy employed to solve this ODE is discussed in \S\ref{sec:solve2ndOrderODE}, which requires $\mathbf{M}$ to be an invertible matrix.

A dodecahedron used to model an alveolar sac is our problem of interest.  The shape of an irregular dodecahedron is described by a set of 20 vertices, each experiencing a displacement
\begin{equation}
\mathbf{u}_i = \{ u_{xi} , u_{yi} , u_{zi} \}^{\mathsf{T}} \defeq
\{ x_i - x_{0i} , y_i - y_{0i} , z_i - z_{0i} \}^{\mathsf{T}} ,
\quad i = 1, 2, \ldots , 20
\end{equation}
that contains co-ordinate differences at a dodecahedral vertex evaluated in the physical co-ordinate frame of this dodecahedron $( \vec{\mathbfsf{E}}_1 , \vec{\mathbfsf{E}}_2 , \vec{\mathbfsf{E}}_3 )$, as established in \S\ref{reindexing3D}, with each vertex $i$ having reference co-ordinates denoted as $\{ x_{0i} , y_{0i} , z_{0i} \}^{\mathsf{T}}$ and current co-ordinates denoted as $\{ x_i , y_i , z_i \}^{\mathsf{T}}$.  The 20 vertices of a dodecahedron uniquely establish its 30 alveolar chords, its 12 alveolar septa, and the alveolar sac that these chords and septa envelop, cf.\ \S\ref{sec:indexingDodecahedra}.

When assembled, vectors $\mathbf{f}$, $\mathbf{u}$ and $\ddot{\mathbf{u}}$ have lengths of 60 for the alveolar chord and septa models, and a length of 63 for the alveolar sac model, while matrices $\mathbf{M}$ and $\mathbf{K}$ have dimensions of $60 \! \times \! 60$ for the alveolar chord and septa models, and dimensions of $63 \! \times \! 63$ for the alveolar sac model.  The model for alveolar volume has an extra node located at the centroid of the dodecahedron, which is a node in common betwixt the 60 tetrahedra needed to fill the volume of a dodecahedron in our modeling of an alveolus. 

As a modeling simplification, the alveolar chords, the alveolar septa, and the alveolar sac are each modeled separately.  It is the nodal forces, i.e., forces at the vertices, resulting from these three, separate, finite-element models that are then summed.  These are obtained by interpolating the stresses integrated at each model's Guass points out to these common nodes, which belong to the set of dodecahedral vertices.  These three sources for vertex force can be added, from which an uniform (homo\-genized) state of stress can then be calculated for return.  

Consequently, we are constructing three, individual, finite-element models governed by the following systems of hyperbolic differential equations
\begin{subequations}
	\begin{align}
	\mathbf{M}_{1D} \ddot{\mathbf{u}} + \mathbf{K}_{1D} \mathbf{u} & 
	= \mathbf{f}_{1D}(t) \\
	\mathbf{M}_{2D} \ddot{\mathbf{u}} + \mathbf{K}_{2D} \mathbf{u} & 
	= \mathbf{f}_{2D}(t) \\
	\mathbf{M}_{3D} \ddot{\mathbf{u}} + \mathbf{K}_{3D} \mathbf{u} & 
	= \mathbf{f}_{3D}(t)
	\end{align}
\end{subequations}
where subscript `$\mbox{}_{1D}$' designates alveolar chords, subscript `$\mbox{}_{2D}$' designates alveolar septa, and subscript `$\mbox{}_{3D}$' designates an alveolar sac.  It is thought to be necessary to split the overall problem into these three subproblems due to the vast differences in their compliance moduli.

\section{Mass Matrix}

A consistent mass matrix \cite{Archer65} for an element, established in its natural co-ordinate system, is defined as
\begin{equation}
\mathbf{M}_{C} = \int_{V} \rho \, \mathbf{N}^{\mathsf{T}} \mathbf{N} \,
\mathrm{d} V
\label{consistentMassMatrix}
\end{equation}
wherein $\mathbf{N}$ is the shape function matrix for an element of interest whose mass density is $\rho$.  This mass matrix is said to be consistent in that it is calculated with the same shape functions that are used to create the stiffness matrix.  Consistent mass matrices are symmetric.  Unfortunately, they may be singular, too, which is not a desirable feature for our chosen numerical approach.

To construct a non-singular mass matrix, we introduce a lumped mass matrix.  Specifically, a sum of all column elements for each row of a consistent mass matrix establishes the diagonal elements of a lumped mass matrix \cite{Reddy93} in that
\begin{equation}
\mathbf{M}_L = \sum_{i=1}^n M_{Li} 
\quad \text{where} \quad
M_{Li} = \sum_{j=1}^n \int_{V} \rho \, N_i \, N_j \, \mathrm{d} V 
\label{lumpedMassMatrix}
\end{equation}
wherein $\sum_{j=1}^n N_j = 1$, with $i$ being a Gauss point from the $n$ Gauss points in an element.

A lumped-consistent (or weighted) mass matrix $\mathbf{M}_W $ can be created as follows
\begin{equation}
\mathbf{M}_W  = (1 - \mu) \, \mathbf{M}_{C} + \mu \, \mathbf{M}_{L}
\end{equation}
wherein $\mu$ is a free scalar parameter for weighting the consistent and lumped mass matrices.  The reason for mixing $\mathbf{M}_C$ and $\mathbf{M}_L$ is to ensure a nonsingular mass matrix.  In this work, $\mu$ is taken to be a half, i.e., an averaged mass matrix is adopted, which has a nice property of minimizing low frequency dispersion; specifically, 
\begin{equation}
\mathbf{M}_A  \defeq \tfrac{1}{2} \, (\mathbf{M}_{C} + \mathbf{M}_{L})
\label{LumconsMass}
\end{equation}
is how we shall construct all of our mass matrices.  The averaged mass matrix $\mathbf{M}_A$ is invertible---a requirement of the selected numerical solution strategy presented in \S\ref{sec:solve2ndOrderODE}.

\subsection{Mass Matrices for a Chord}

A two-noded alveolar chord has shape functions $N_i$ that, when evaluated in its natural co-ordinate system wherein $-1 \leq \xi \leq 1$, aggregate into a matrix of shape functions
\begin{multline}
\mathbf{N} = \begin{bmatrix} N_1 & N_2 \end{bmatrix} =
\begin{bmatrix}
\frac{1}{2} \, (1 - \xi) &  \frac{1}{2} \, (1 + \xi)
\end{bmatrix} \\
\text{with} \quad
N_{1,\xi} = -\tfrac{1}{2} 
\quad \text{and} \quad
N_{2,\xi} = \tfrac{1}{2}
\end{multline}
where $\xi$ is an abscissa associated with a specific Gauss quadrature rule, cf.\ Table~\ref{tabQuadrature1D}.

The Jacobian matrix $\mathbf{J}$ for a one-dimensional chord, which is also its Jacobian determinant $| \mathbf{J} |$, is used to transform the integrals in Eqns.~(\ref{consistentMassMatrix} \& \ref{lumpedMassMatrix}) from their global co-ordinate system into their natural co-ordinate system.  It has a value of
\begin{equation}
\mathbf{J} \equiv | \mathbf{J} | = \sum\nolimits_{i=1}^2 N_{i,\xi} (\xi) \, x_i = 
-\tfrac{1}{2} \cdot -\tfrac{1}{2} L + \tfrac{1}{2} \cdot \tfrac{1}{2} L = 
\tfrac{1}{2} L
\label{detJac1D}
\end{equation}
given nodal co-ordinates of $x_1 = -\tfrac{1}{2} L$ and $x_2 = \tfrac{1}{2} L$, where $L$ is the length of our alveolar chord.  The Jacobian matrix and its Jacobian determinant are equal in the case of a rod, because the space is one dimensional. 

The consistent mass matrix for a 1D alveolar chord modeled as a two-noded rod, when evaluated in its natural co-ordinate system, becomes
\begin{equation} 
\begin{aligned}
\mathbf{M}_{1C} & = \int_0^L \rho \, \mathbf{N}^{\mathsf{T}} \mathbf{N} \, A \, \mathrm{d} x  = \int_{-1}^{1} \rho \, \mathbf{N}^{\mathsf{T}} \mathbf{N}\, A \, | \mathbf{J} | \,  \mathrm{d} \xi \\ & 
= \sum_{i=1}^n  \rho  \, \mathbf{N} ^{\mathsf{T}} (\xi_i) \, \mathbf{N}(\xi_i) \, A \, | \mathbf{J} | \, w_i \\ &
= \sum_{i=1}^n \frac{\rho A L w_i}{8} \begin{bmatrix}
1 - 2\xi_i + \xi_i^2 & 1 - \xi_i^2 \\
1 - \xi_i^2 & 1 + 2 \xi_i + \xi_i^2
\end{bmatrix}
\end{aligned}
\label{ConsMassMatrix1DA}
\end{equation}
with $\textbf{N} (\xi_i)$ and $w_i$ being the shape functions and weights of quadrature for a selected Gauss integration rule evaluated at its Gauss point $i$, while $A$ is the cross-sectional area of our alveolar chord. Table~\ref{tabQuadrature1D} presents values for the absciss\ae\ $\xi_i$ and weights $w_i$ of integration for schemes where there are $n = 1, 2$ or $3$ nodes of integration along an alveolar chord.  

The lumped mass matrix for a 1D alveolar chord in its natural co-ordinate system is 
\begin{multline}
\mathbf{M}_{1L} = \sum_{\text{rows}} \sum_{i=1}^n \frac{\rho A L w_i}{8} \begin{bmatrix}
1 - 2\xi_i + \xi_i^2 & 1 - \xi_i^2 \\
1 - \xi_i^2 & 1 + 2 \xi_i + \xi_i^2
\end{bmatrix} \\
= \sum_{i=1}^n \frac{\rho A L w_i}{4} \begin{bmatrix} 
1 - \xi_i & 0 \\ 0 & 1 + \xi_i \end{bmatrix}
\label{LumconsMassMatrix1DA}
\end{multline}
where it is seen that the mass matrix in Eqn.~(\ref{ConsMassMatrix1DA}) is singular at any given Gauss point, whereas the mass matrix in Eqn.~(\ref{LumconsMassMatrix1DA}) has a reciprocal, except when $\xi = \pm 1$. 

Chordal mass matrices associated with the Gauss quadratures listed in Table~\ref{tabQuadrature1D} become: the consistent mass matrices for a chord are
\begin{subequations}
	\label{massMatrices1D}
	\begin{align}
	\mathbf{M}_{1C} & = \frac{\rho A L}{4} 
	\begin{bmatrix} 1 & 1 \\ 1 & 1 \end{bmatrix} & 
	\mathbf{M}_{1C} & = \frac{\rho A L}{6} 
	\begin{bmatrix} 2 & 1 \\ 1 & 2 \end{bmatrix} & 
	\mathbf{M}_{1C} & = \frac{\rho A L}{6} 
	\begin{bmatrix} 2 & 1 \\ 1 & 2 \end{bmatrix} 
	\label{ConsMassMatrix1D} \\
	\intertext{the lumped mass matrices for a chord are}
	\mathbf{M}_{1L} & = \frac{\rho A L}{2} 
	\begin{bmatrix} 1 & 0 \\ 0 & 1 \end{bmatrix} & 
	\mathbf{M}_{1L} & = \frac{\rho A L}{2} 
	\begin{bmatrix} 1 & 0 \\ 0 & 1 \end{bmatrix} & 
	\mathbf{M}_{1L} & = \frac{\rho A L}{2} 
	\begin{bmatrix} 1 & 0 \\ 0 & 1 \end{bmatrix} 
	\label{LumMassMatrix1D} \\
	\intertext{that when averaged become}
	\mathbf{M}_{1A} & = \frac{\rho A L}{8} 
	\begin{bmatrix} 3 & 1 \\ 1 & 3 \end{bmatrix} & 
	\mathbf{M}_{1A} & = \frac{\rho A L}{12} 
	\begin{bmatrix} 5 & 1 \\ 1 & 5 \end{bmatrix} & 
	\mathbf{M}_{1A} & = \frac{\rho A L}{12} 
	\begin{bmatrix} 5 & 1 \\ 1 & 5 \end{bmatrix}
	\label{avgMassMatrix1D}
	\end{align}
\end{subequations}
which are the mass matrices that we employ.  This collection of mass matrices pertain to one Gauss point (left column of mass matrices), two Gauss points (center column of mass matrices), and three Gauss points (right column of mass matrices).  For this element, there is no difference between the mass matrices for two and three Gauss points of integration.  In the derivation of these matrices, it is assumed that the cross-sectional area and mass density  are both uniform along the length of a septal chord.  Furthermore, because septal chord volume $V$ is considered to be preserved, i.e., $V = V_0$ where $V_0 = A_0 L_0$ and $V = A L$, it follows that these mass matrices are constant; hence, they only need to be inverted once.

\subsubsection{Assemble Mass Matrices for the Septal Chords}

\newpage
\subsection{Mass Matrices for a Pentagon}

For an alveolar septa represented as an irregular pentagon, its matrix of shape functions $\mathbf{N}$ takes on the form of
\begin{equation}
\mathbf{N} = 
\begin{bmatrix}
N_1 & 0 & N_2 & 0 & N_3 & 0 & N_4 & 0 & N_5 & 0 \\ 0 & N_1 & 0 & N_2 & 0 & N_3 & 0 & N_4 & 0 & N_5 
\end{bmatrix} 
\label{shape2D}
\end{equation}
where $\mathrm{N}_i$, $i = 1, \ldots, 5$, are the five shape functions that correspond with the five vertices of a pentagon, as established in Eqn.~(\ref{shapeFunctions}).

A consistent mass matrix $\mathbf{M}_{2C}$ is constructed by substituting the above matrix of shape functions into the following expression
\begin{equation}
\mathbf{M}_{2C} = \iint_{\pentagon} \rho \, \mathbf{N}^{\mathsf{T}} \mathbf{N} \,|\mathbf{J}| \, h \, \mathrm{d} \xi \, \mathrm{d} \eta
\label{massintegral2d}
\end{equation}
with $h$ denoting membrane thickness, and where $|\mathbf{J}|$ specifies the determinant of its $2 \! \times \! 2$ Jacobian matrix $\mathbf{J}$.  In areal derivations, the Jacobian of a two-dimensional transformation connects the physical ${x, y}$ to the natural ${\xi, \eta}$ co-ordinate systems involved.  Components of this Jacobian matrix are calculated using derivatives of shape functions taken with respect to the local co-ordinates at the $i^{\mathrm{th}}$ vertex via
\begin{equation}
\mathbf{J} = 
\begin{bmatrix}
\partial x / \partial\xi & \partial y / \partial\xi \\
\partial x / \partial\eta & \partial y / \partial\eta 
\end{bmatrix}  
= \begin{bmatrix}
\sum\nolimits_{i=1}^5 N_{i,\xi} (\xi,\eta) \, x_i & \sum\nolimits_{i=1}^5 N_{i,\xi} (\xi,\eta) \, y_i \\
\sum\nolimits_{i=1}^5 N_{i,\eta} (\xi,\eta) \, x_i & \sum\nolimits_{i=1}^5 N_{i,\eta} (\xi,\eta) \, y_i
\end{bmatrix}
\end{equation}
where the shape function gradients $N_{i,\xi}$ and $N_{i,\eta}$ have been established in Eqn.~(\ref{shapeFunctionGradients}), with
\begin{equation}
\mathbf{J}^{-1}  = \frac{1}{|\mathbf{J}|} \,
\begin{bmatrix}
\partial y / \partial\eta & - \partial y / \partial\xi \\
- \partial x / \partial\eta & \partial x / \partial\xi
\end{bmatrix} 
\quad \text{wherein} \quad 
| \mathbf{J} | = \frac{\partial x}{\partial \xi} \frac{\partial y}{\partial \eta} - 
\frac{\partial x}{\partial \eta} \frac{\partial y}{\partial \xi}
\label{jacobianpent}
\end{equation}
providing its inverse and Jacobian determinant.  A numerical integration of Eqn.~(\ref{massintegral2d}) results in 
\begin{equation}
\mathbf{M}_{2C} = \rho h \sum_{i=1}^{n} \mathbf{N}^{\mathsf{T}} ( \xi_i , \eta_i ) \, \mathbf{N} ( \xi_i , \eta_i ) \,|\mathbf{J} ( \xi_i , \eta_i ) | \, w_i
\end{equation}
where co-ordinates $\xi_i$ and $\eta_i$ locate a Gauss point $i$ within the pentagon whose associate weight of integration is $w_i$, which can be found in Table~\ref{tabQuadrature}.  Here it is assumed that the mass density $\rho$ and membrane thickness $h$ are uniform across the alveolar septa.

A lumped mass matrix for a pentagon gets its diagonal elements as follows
\begin{equation}
{M}_{2Li} = \sum_{j=1}^n M_{2Cij}
\label{LumMass2D}
\end{equation}
because $\sum_{j=1}^n N_j = 1$. 

For instance, the averaged lumped-consistent mass matrix for a pentagon with $1$ Gauss point of integration located at its centroid is constructed by averaging its consistent and lumped mass matrices resulting in
\begin{equation}
\mathbf{M}_{2A}  = \frac{\rho \, h A}{100}
\begin{bmatrix}
6 & 0 & 1 & 0 & 1 & 0 & 1 & 0 & 1 & 0 \\
0 & 6 & 0 & 1 & 0 & 1 & 0 & 1 & 0 & 1 \\
1 & 0 & 6 & 0 & 1 & 0 & 1 & 0 & 1 & 0 \\
0 & 1 & 0 & 6 & 0 & 1 & 0 & 1 & 0 & 1 \\
1 & 0 & 1 & 0 & 6 & 0 & 1 & 0 & 1 & 0 \\
0 & 1 & 0 & 1 & 0 & 6 & 0 & 1 & 0 & 1 \\
1 & 0 & 1 & 0 & 1 & 0 & 6 & 0 & 1 & 0 \\
0 & 1 & 0 & 1 & 0 & 1 & 0 & 6 & 0 & 1 \\
1 & 0 & 1 & 0 & 1 & 0 & 1 & 0 & 6 & 0 \\
0 & 1 & 0 & 1 & 0 & 1 & 0 & 1 & 0 & 6 
\end{bmatrix} 
\label{LumconsMassMatrix2D}
\end{equation}
where $h$ and $A$ are the thickness and area of the membrane.  It just so happens that, in this case, the mass matrix turns out nice.  Because septal volume $V$ is considered to be preserved, i.e., $V = V_0$ where $V_0 = h_0 A_0$ and $V = h A$, it follows that this mass matrix is constant; hence, its inverse need only be determined once.

Three mass matrices have been constructed for a pentagon based upon the three, Gauss, quadrature rules presented in Table~\ref{tabQuadrature}, which integrate polynomials of orders 1, 3 and 5 exactly, respectively. 

\subsubsection{Assemble Mass Matrices for the Alveolar Septa}


\subsection{Mass Matrices for a Tetrahedron}

A dodecahedron has 60 individual tetrahedra contained within it, whose centroid is a common vertex to all 60 of these tetrahedra. Hence, an analysis to find the mass matrix of a tetrahedron is used as the building block needed to assemble the mass matrix for an alveolar sac.

The matrix of shape functions $\mathbf{N}$ for a tetrahedon has the form of
\begin{equation}
\mathbf{N} =  
\begin{bmatrix*}[r]
N_1 & 0 & 0 & N_2 & 0 & 0 & N_3 & 0 & 0 & N_4 & 0 & 0 \\
0 & N_1 & 0 & 0 & N_2 & 0 & 0 & N_3 & 0 & 0 & N_4 & 0 \\
0 & 0 & N_1 & 0 & 0 & N_2 & 0 & 0 & N_3 & 0 & 0 & N_4
\end{bmatrix*} 
\label{shape3D}
\end{equation}
in which $\mathrm{N}_i$, $i = 1, 2, 3, 4$, are the four shape functions corresponding to the four vertices of a tetrahedron that are defined as follows
\begin{subequations}
	\begin{align}
	N_1 & = 1 - \xi - \eta - \zeta \\
	N_2 & = \xi \\
	N_3 & = \eta \\
	N_4 & = \zeta .
	\end{align}
\end{subequations}
Numerical integration is used to obtain the mass matrix of a tetrahedron via
\begin{equation}
\begin{aligned}
\mathbf{M}_{3C} & = \iiint_V \rho \, \mathbf{N}^{\mathsf{T}} \mathbf{N} \, \mathrm{d} \mathbf{x} \, \mathrm{d} \mathbf{y} \, \mathrm{d} \mathbf{z} = \int_0^1 \int_0^1 \int_0^1 \rho \, \mathbf{N}^{\mathsf{T}} \mathbf{N} \, | \mathbf{J} | \, \mathrm{d} \mathbf{\xi} \, \mathrm{d} \mathbf{\eta} \, \mathrm{d} \mathbf{\zeta} \\ & = \rho \sum_{i=1}^n \mathbf{N}^{\mathsf{T}} ( \xi_i , \eta_i , \zeta_i ) \, \mathbf{N} (\xi_i , \eta_i , \zeta_i ) \, |\mathbf{J} (\xi_i , \eta_i , \zeta_i )| \, w_i
\end{aligned}
\end{equation}
with $|\mathbf{J}|$ being the determinant of the Jacobian matrix, and $n$ denoting the number of Gauss points used for integration.  The Jacobian is calculated from derivatives of the shape functions taken with respect to its local co-ordinates $(\xi, \eta, \zeta)$ that associate with their current global co-ordinates $(x_i, y_i, z_i)$ according to
\begin{equation}
\begin{aligned}
\mathbf{J}= &
\begin{bmatrix}
\partial x / \partial\xi & \partial y / \partial\xi & \partial z / \partial\xi\\
\partial x / \partial\eta & \partial y / \partial\eta & \partial z / \partial\eta \\
\partial x / \partial\zeta & \partial y / \partial\zeta & \partial z / \partial\zeta 
\end{bmatrix} = \begin{bmatrix} 
x_{,\xi} & y_{,\xi} & z_{,\xi} \\
x_{,\eta} & y_{\eta} & z_{,\eta} \\
x_{,\zeta} & y_{,\zeta} & z_{,\zeta}
\end{bmatrix} \\
= & \begin{bmatrix}
\sum\nolimits_{i=1}^4 N_{i,\xi} (\xi,\eta,\zeta) \, x_i & \sum\nolimits_{i=1}^4 N_{i,\xi} (\xi,\eta,\zeta) \, y_i &
\sum\nolimits_{i=1}^4 N_{i,\xi} (\xi,\eta,\zeta) \, z_i\\
\sum\nolimits_{i=1}^4 N_{i,\eta} (\xi,\eta,\zeta) \, x_i & \sum\nolimits_{i=1}^4 N_{i,\eta} (\xi,\eta,\zeta) \, y_i &
\sum\nolimits_{i=1}^4 N_{i,\eta} (\xi,\eta,\zeta) \, z_i\\
\sum\nolimits_{i=1}^4 N_{i,\zeta} (\xi,\eta,\zeta) \, x_i & \sum\nolimits_{i=1}^4 N_{i,\zeta} (\xi,\eta,\zeta) \, y_i &
\sum\nolimits_{i=1}^4 N_{i,\zeta} (\xi,\eta,\zeta) \, z_i
\end{bmatrix} 
\end{aligned}
\label{jacobiantet}
\end{equation}
with
\begin{equation}
\mathbf{J}^{-1} = \frac{1}{|\mathbf{J}|} \,
\begin{bmatrix}
y_{,\eta} \, z_{,\zeta} - z_{,\eta} \, y_{,\zeta}  & z_{,\xi} \, y_{,\zeta} - y_{,\xi} \, z_{,\zeta} & y_{,\xi} \, z_{,\eta} - y_{,\eta} \, z_{,\xi} \\
z_{,\eta} \, x_{,\zeta} - x_{,\eta} \, z_{,\zeta}  &  x_{,\xi} \, z_{,\zeta} - z_{,\xi} \, x_{,\zeta} & z_{,\xi} \, x_{,\eta} - x_{,\xi} \, z_{,\eta} \\
x_{,\eta} \, y_{,\zeta} - y_{,\eta} \, x_{,\zeta}  & y_{,\xi} \, x_{,\zeta} - x_{,\xi} \, y_{,\zeta} & y_{,\eta} \, x_{,\xi} - y_{,\xi} \, x_{,\eta} \\
\end{bmatrix} 
\label{detjacobiantet}
\end{equation}
establishing its inverse.

For instance, the averaged lumped-consistent mass matrix for a tetrahedron with 1 Gauss point of integration located at its centroid is
\begin{equation}
\mathbf{M}_{3A} = \frac{\rho \, V}{96} \begin{bmatrix}
5 & 0 & 0 & 1 & 0 & 0 & 1 & 0 & 0 & 1 & 0 & 0 \\
0 & 5 & 0 & 0 & 1 & 0 & 0 & 1 & 0 & 0 & 1 & 0 \\
0 & 0 & 5 & 0 & 0 & 1 & 0 & 0 & 1 & 0 & 0 & 1 \\
1 & 0 & 0 & 5 & 0 & 0 & 1 & 0 & 0 & 1 & 0 & 0 \\
0 & 1 & 0 & 0 & 5 & 0 & 0 & 1 & 0 & 0 & 1 & 0 \\
0 & 0 & 1 & 0 & 0 & 5 & 0 & 0 & 1 & 0 & 0 & 1 \\
1 & 0 & 0 & 1 & 0 & 0 & 5 & 0 & 0 & 1 & 0 & 0 \\
0 & 1 & 0 & 0 & 1 & 0 & 0 & 5 & 0 & 0 & 1 & 0 \\
0 & 0 & 1 & 0 & 0 & 1 & 0 & 0 & 5 & 0 & 0 & 1 \\
1 & 0 & 0 & 1 & 0 & 0 & 1 & 0 & 0 & 5 & 0 & 0 \\
0 & 1 & 0 & 0 & 1 & 0 & 0 & 1 & 0 & 0 & 5 & 0 \\
0 & 0 & 1 & 0 & 0 & 1 & 0 & 0 & 1 & 0 & 0 & 5 
\end{bmatrix}
\end{equation}
where the volume $V$ varies with deformation because air is compressible.  The other cases are not as simple.

Three mass matrices for a tetrahedron are implemented based upon the three, Gauss, quadrature rules presented in Table~\ref{tabQuadraturetetra}.  These quadratures integrate polynomials of order 1, 2 and 3, respectively. 

\subsubsection{Assemble Tetrahedral Mass Matrices to Get Mass Matrix for an Alveolar Sac}


\section{Stiffness Matrix}
\label{secStiffnessMatrices}

For a nonlinear elastic material, like a soft tissue, the stress-strain response generally becomes stiffer with increasing deformation.  Consequently, the slope along their stress-strain curve changes with strain and, therefore, its instantaneous stiffness changes, too.  Therefore, tangent stiffness matrices are required for our finite element simulations. 

The total potential energy of a deformed body can be expressed as the difference between a variation in the potential energy of deformation $\delta{U}$ and a variation in the potential energy of the external loading $\delta{W}$ \cite{Yangetal10}
\begin{subequations}
	\label{principle}
	\begin{align}
	\delta{R} &= \delta{U} - \delta{W}\\
	\intertext{with}
	\delta{W} & = \mathbf{F} \, \mathrm{d} \boldsymbol{\Delta}\\
	\delta{U} & = \int_{V} \, \bar{\mathbf{B}}^{\mathsf{T}} \, \mathbf{T} \, \mathrm{d} V \, \mathrm{d} \boldsymbol{\Delta}
	\end{align}
\end{subequations}
where $\mathbf{F}$ is a vector of external forces, $\mathbf{T}$ are the conjugate stresses to our strains, and $\bar{\mathbf{B}}$ is a matrix relation between strain $\mathbf{E}$ and nodal displacements $\boldsymbol{\Delta}$ such that
\begin{equation}
\mathbf{E} = \bar{\mathbf{B}} \,  \boldsymbol{\Delta} \quad \text{with} \quad \mathrm{d} \mathbf{E} = \bar{\mathbf{B}} \, \mathrm{d} \boldsymbol{\Delta}
\label{strain} 
\end{equation}
with displacement fields being interpolated as
\begin{equation}
\bar{\mathbf{ u}} = 
\begin{Bmatrix}
u \\
v     
\end{Bmatrix}
=
\begin{Bmatrix}
\sum_{i=1}^{n} u_i \, N_i \\
\sum_{i=1}^{n} v_i \, N_i     
\end{Bmatrix}
= \mathbf{N} \, \boldsymbol{\Delta} .
\end{equation} 
\textcolor{red}{In order to make our computation more systematic, the total strain-displacement matrix is additively decomposed into its linear and nonlinear components as}
\begin{equation}
\bar{\mathbf{B}} = \mathbf{B}_L + \mathbf{B}_N
\label{straindis}.
\end{equation} 
\textcolor{red}{Here $\mathbf{B}_L$ is obtained from a linear analysis, i.e., $\mathbf{E}_L = \mathbf{B}_L \, \boldsymbol{\Delta}$. Note that the linear strain-displacement matrix $\mathbf{B}_L$ does not vary with the displacement whereas its nonlinear counterpart $\mathbf{B}_N$ is a function of the displacement vector $ \mathbf{u}$ associated with the nonlinear part of strain, $\mathbf{E}_N$. This fact has a greater consequence in obtaining stiffness matrices as this implies that $\mathrm{d}\mathbf{B}_L=0$ which allows one to write that $\mathrm{d}\bar{\mathbf{B}}=\mathrm{d}\mathbf{B}_N$.}

\textcolor{red}{The nonlinear part of strain $\mathbf{E}_N$ is composed of products of different elements of the displace gradient matrix. Thus, $\mathbf{E}_N$ can be written as the product of two matrices $\mathbf{A}$ and $\boldsymbol{\theta}$, i.e.,} 
\begin{equation}
\mathbf{E}_N = \tfrac{1}{2} \, {\mathbf{A}} \,  \boldsymbol{\theta}
\label{nonlinstrain}
\end{equation}
\textcolor{red}{where ${\mathbf{A}}$ is a matrix consisting of different elements of the displacement gradient, and $\boldsymbol{\theta}$ is a vector containing the derivatives of displacements. The elements of displacement gradient are placed in the matrix $\mathbf{A}$ in a way such that its matrix multiplication with the vector $\boldsymbol{\theta}$ provides the required nonlinear strain matrix $\mathbf{E}_N$. To obtain the stiffness matrix, one needs to express the strain matrices in terms of nodal displacements. To achieve that, the vector $\boldsymbol{\theta}$ is further expressed as $\boldsymbol{\theta} = \mathbf{H} \, \boldsymbol{\Delta} $, or in rate form as $\mathrm{d} \boldsymbol{\theta} = \mathbf{H} \, \mathrm{d} \boldsymbol{\Delta}$, where $\mathbf{H}$ is the derivative of shape functions taken with respect to the global co-ordinate system. Following the same argument as before, one can conclude that $\mathrm{d}\mathbf{E}_L$ is also zero. Thus, the derivative of strain takes on the form of}
\begin{equation}
\mathrm{d} \mathbf{E} = \mathrm{d} (\mathbf{E}_L + \mathbf{E}_N) = \mathrm{d} \mathbf{E}_N = \tfrac{1}{2} \, (\mathrm{d} \mathbf{A} \, \boldsymbol{\theta} + \mathbf{A} \, \mathrm{d} \boldsymbol{\theta}) = \mathbf{A} \, \mathbf{H} \, \mathrm{d} \boldsymbol{\Delta} .
\label{dstrain1}
\end{equation} 
\textcolor{red}{The last of Eqn.~\eqref{dstrain1} is derived using the property $\mathrm{d}\mathbf{A}\boldsymbol{\theta}=\mathbf{A}\mathrm{d}\boldsymbol{\theta}$ which follows from the definitions of $\mathbf{A}$ and $\boldsymbol{\theta}$.}

\textcolor{red}{Since $\mathrm{d}\mathbf{E}_L=\mathrm{d}\mathbf{B}_L=0$, Eqn.~\eqref{strain} reduces to $\mathrm{d}\mathbf{E}_N=\mathbf{B}_N\mathrm{d}\boldsymbol{\Delta}$. Now, expressing $\mathrm{d}\mathbf{E}_N$ in terms of incremental nodal displacements in Eqn.~\eqref{dstrain1}, one can easily write that}
\begin{equation}
\mathbf{B}_N = \mathbf{A} \, \mathbf{H} \quad \text{with} \quad \mathrm{d} \mathbf{B}_N = \mathrm{d} \mathbf{A} \, \mathbf{H}.
\label{nonlinStrainDisp}
\end{equation} 

\textcolor{red}{With the relationships between the strain; strain-displacement matrices and the displacements established, now we are ready to determine the stiffness matrix. In order to satisfy equilibrium, the internal and external forces must be balanced~\cite{Elseifi98}. The difference between these two forces is called the residual force. If $\mathbf{R}$ denotes the residual force vector, then for equilibrium, the following condition must be satisfied:}
\begin{equation}
\mathbf{R} = \int_{V} \, \bar{\mathbf{B}}^{\mathsf{T}} \, \mathbf{T} \, \mathrm{d} V - \mathbf{F} = 0
\label{residual}.
\end{equation}
\textcolor{red}{The total tangent stiffness matrix is defined as $\mathbf{K}_T\defeq \mathrm{d}\mathbf{R}/\mathrm{d}\boldsymbol{\Delta}$. Now derivative of Eqn.~\eqref{residual} yields}
\begin{equation}
\mathrm{d} \mathbf{R} = \int_{V} \, \mathrm{d} \bar{\mathbf{B}}^{\mathsf{T}} \, \mathbf{T} \, \mathrm{d} V + \int_{V} \, \bar{\mathbf{B}}^{\mathsf{T}} \, \mathrm{d} \mathbf{T} \, \mathrm{d} V 
\label{diffresidual}
\end{equation}
where $\mathrm{d} \mathbf{T} = \mathbf{M} \, \mathrm{d} \mathbf{E} = \mathbf{M} \,  \bar{\mathbf{B}} \, \mathrm{d} \boldsymbol{\Delta}$, with $\mathbf{M}$ being the matrix of elastic tangent moduli. 

Substituting the definition of $\mathrm{d} \mathbf{T}$ into Eq.~(\ref{diffresidual}), we obtain 
\begin{subequations}
	\begin{align}
	\mathrm{d} \mathbf{R} & = \int_{V} \, \mathrm{d} \mathbf{B}_N^{\mathsf{T}} \, \mathbf{T} \, \mathrm{d} V + \int_{V} \, \bar{\mathbf{B}}^{\mathsf{T}} \, \mathbf{M} \, \bar{\mathbf{B}} \, \mathrm{d} V \, \mathrm{d} \boldsymbol{\Delta} \\
	& = \int_{V} \,  (\mathbf{H}^{\mathsf{T}} \, \mathrm{d} \mathbf{A}^{\mathsf{T}} \mathbf{T}) \, \mathrm{d} V + \int_{V} \, \left(\mathbf{B}_L + \mathbf{B}_N \right)^{\mathsf{T}} \, \mathbf{M} \,  \left(\mathbf{B}_L + \mathbf{B}_N \right) \, \mathrm{d} V \, \mathrm{d} \boldsymbol{\Delta} \\ 
	\intertext{\textcolor{red}{Notice that the first term in $\mathrm{d}\mathbf{R}$ does not include variation of the stress and hence, is associated with the initial stress applied on the body. Now we need to express the variation of residual force $\mathrm{d}\mathbf{R}$ in terms of the varation of nodal displacements to compute the stiffness matrices.}} 
	& = \mathbf{K}_\mathbf{S}  \, \mathrm{d} \boldsymbol{\Delta} + \underbrace{\int_{V} \, \mathbf{B}_L^{\mathsf{T}} \, \mathbf{M} \, \mathbf{B}_L \, \mathrm{d} V}_{\mathbf{K}_L} \, \mathrm{d} \boldsymbol{\Delta}\\ 
	& + \underbrace{\int_{V} \,  \left(\mathbf{B}_L^{\mathsf{T}} \, \mathbf{M} \, \mathbf{B}_N + \mathbf{B}_N^{\mathsf{T}} \, \mathbf{M} \, \mathbf{B}_L + \mathbf{B}_N^{\mathsf{T}} \, \mathbf{M} \, \mathbf{B}_N \, \right) \, \mathrm{d} V}_{\mathbf{K}_N} \, \mathrm{d} \boldsymbol{\Delta}\\
	& = \left(\mathbf{K}_\mathbf{S} + \mathbf{K}_L + \mathbf{K}_N\right)\mathrm{d} \boldsymbol{\Delta} = \mathbf{K}_T \, \mathrm{d} \boldsymbol{\Delta}
	\end{align} 
	\label{stiffnessMatrices}
\end{subequations}
wherein $\mathbf{K}_\mathbf{S}$ is the stiffness matrix associated with the initial stress, $\mathbf{K}_L$ is the conventional small displacement stiffness matrix\footnote{\textcolor{red}{When expressed in terms of stress and strain, the small displacement stiffness matrix $\mathbf{K}_L$ is nothing but the secant modulus, whereas the total stiffness matrix $\mathbf{K}_T$ is the tangent modulus.}}, and $\mathbf{K}_N$ is the large displacement stiffness matrix. $\mathbf{K}_T$ denotes the total tangent-stiffness matrix relating the residual force vector with nodal displacements. 

\subsection{Stiffness Matrix for Chord}

\textcolor{red}{The components of Laplace stretch $\boldsymbol{\mathcal{U}}$ can be obtained from a Cholesky factorization of the right Cauchy-Green tensor $\mathbf{C} = \mathbf{F}^{\mathsf{T}} \mathbf{F} = \boldsymbol{\mathcal{U}}^{\mathsf{T}} \boldsymbol{\mathcal{U}}$, which is a symmetric second-order tensor. For a 1-D chord, the only possible deformation is a stretch of the chord in axial direction. Therefore, in this case, the deformation gradient as well as the right Cauchy-Green tensor $\mathbf{C}$ has only one component. Consequently, the Laplace stretch $\boldsymbol{\mathcal{U}}$ also consists of only one component which is denoted by $a$. If $u$ denotes the axial displacement of the chord, then the axial stretch $a$ can be written as}
\begin{equation}
a = {\mathcal{U}}_{11} = \sqrt{C_{11}}  \quad \text{with} \quad \mathrm{C_{11}} = \left(\frac{\mathrm{\partial u}}{\partial x}\right)^2 + 2\, \frac{\mathrm{\partial u}}{\partial x}  + 1
\end{equation} 
\textcolor{red}{The chord is subjected to an axial strain defined as $e = \ln ( a / a_0 ) \equiv \ln ( L / L_0 )$ where $L$ is the length of the chord, and $L_0$ and $a_0$ are initial length of the chord and the initial stretch respectively. Here, without loss of generality, we assume that $a_0=1$. As before, we decompose the total strain into its linear and nonlinear components as}
\begin{equation}
e = e_{L} + e_{N}. 
\end{equation}
\textcolor{red}{The linear and nonlinear strain components can be obtained from a Taylor series expansion of $e = \ln \sqrt{C_{11}}$. Here, terms only up to second order have been considered. The strain components, thus obtained, are given as} 
\begin{subequations}
	\begin{align}
	e_{L} & = \mathrm{\frac{\partial u}{\partial x}} \\
	e_{N} & = -\frac{1}{2} \, \mathrm{\frac{\partial u}{\partial x}}\, \mathrm{\frac{\partial u}{\partial x}}
	\end{align}
\end{subequations}

\textcolor{red}{The linear strain-displacement matrix $\mathbf{B}_L$ can now be obtained by expressing the linear strain $e_L$ in terms of the nodal displacements, viz.,}
\begin{subequations}
	\begin{align}
	e_L  = 
	\partial u / \partial x   = &
	\sum\nolimits_{i=1}^2 N_{i,x} \, u_i = \begin{bmatrix}
	[\mathbf{b}_{L1}], & [\mathbf{b}_{L2}]
	\end{bmatrix}  \begin{Bmatrix} \boldsymbol{\Delta} \end{Bmatrix}  
	= [\mathbf{B}_L] \begin{Bmatrix} \boldsymbol{\Delta} \end{Bmatrix}\\
	\intertext{with} \mathbf{b}_{Li}  & = 
	N_{i,x} \quad \text{and} \quad
	\mathbf{\Delta}^T  = 
	\begin{Bmatrix}
	u_1 ,u_2
	\end{Bmatrix} 
	\end{align}
\end{subequations}
\textcolor{red}{where $u_1$ and $u_2$ are the nodal displacements of the chord.}

\textcolor{red}{As discussed earlier, the nonlinear strain $e_N$ can be written as a product of a matrix $\mathbf{A}$ and $\boldsymbol{\theta}$. For a 1-D chord, both $\mathbf{A}$ and $\boldsymbol{\theta}$ consist of a single element, i.e.,} 
\begin{equation}
e_ N =  \tfrac{1}{2} \,
[-\partial u / \partial x ]
\{\partial u / \partial x\}
= \tfrac{1}{2} \, \mathbf{A} \, \boldsymbol{\theta}.
\end{equation}
\textcolor{red}{Furthermore, $\boldsymbol{\theta}$ can be expressed in terms of the nodal displacements as}
\begin{equation}
\begin{aligned}
&\boldsymbol{\theta} =  \begin{Bmatrix}
\partial u / \partial x
\end{Bmatrix}
= \begin{Bmatrix}
\sum\nolimits_{i=1}^2 N_{i,x} \, u_i
\end{Bmatrix} 
= \begin{bmatrix}
[\mathbf{h}_1], & [\mathbf{h}_2]
\end{bmatrix}  \begin{Bmatrix} \boldsymbol{\Delta} \end{Bmatrix}  
= [\mathbf{H}] \begin{Bmatrix} \boldsymbol{\Delta} \end{Bmatrix}\\
&\text{with} \quad \mathbf{h}_i = N_{i,x}
\end{aligned}.  
\end{equation}
\textcolor{red}{Hence, in view of Eqn.~\eqref{nonlinStrainDisp}, the nonlinear strain-displacement matrix $\mathbf{B}_N$ can be written as}
\begin{equation}
\mathbf{B}_N = \mathbf{A} \, \mathbf{H}  =  \begin{bmatrix}
[\mathbf{b}_{N1}], & [\mathbf{b}_{N2}] 
\end{bmatrix} 
\end{equation}
where
\begin{equation}
\mathbf{b}_{Ni}  =   
[-\partial u / \partial x] \cdot
N_{i,x} 
\end{equation}

\textcolor{red}{Now we derive the stiffness matrix $\mathbf{K}_\mathbf{S}$ associated with initial stress as defined in Eqn.~\eqref{stiffnessMatrices}. Note that when expressed in terms of the nodal displacements, $\mathrm{d}\mathbf{A}$ can be written as $\mathrm{d}\mathbf{A}=\mathbf{H}\:\mathrm{d}\boldsymbol{\Delta}$. Using this property, one can obtain $\mathbf{K_S}$ as}
\begin{equation}
\mathbf{K}_\mathbf{S} = \int_{\Gamma} \, \mathbf{H}^{\mathsf{T}} \, \mathrm{d} \mathbf{A}^{\mathsf{T}} \mathbf{T} \, A \, \mathrm{d} x 
= \int_{-1}^{1} \mathbf{H}^{\mathsf{T}} \,  \mathbf{T} \, \mathbf{H} \, \mathbf{J}  \, A \,  \mathrm{d} \xi
= \sum_{i=1}^{n}  \mathbf{H}^{\mathsf{T}} \, \mathbf{T} \, \mathbf{H}  \, \mathbf{J} \, A \, \mathrm{w}_i
\end{equation}
\textcolor{red}{wherein $\mathbf{T}$ is the initial stress carried by collagen and elastin fibers, as discussed in Part~4. Note that the term $\mathbf{K_S}$ does not contain any material property. This is due to the fact that the force component $\mathbf{K_S}\:\mathrm{d}\boldsymbol{\Delta}$ exhibits the influence of the initial stress on the axial displacement of the chord. As before, $\mathbf{H}$ can be expressed as follows}
\begin{equation}
\begin{aligned}
\mathbf{H} = \begin{bmatrix}
N_{1,x} &  N_{2,x}
\end{bmatrix} = \begin{bmatrix}
N_{1,\xi} & N_{2,\xi} \, 
\end{bmatrix} \, \mathbf{J}^{-1}
\end{aligned}
\end{equation}
where the Jacobian $\mathbf{J}$ is defined in Eqn.~\ref{detJac1D}.

The small displacement stiffness matrix for a 1-D alveolar chord, that is transformed from global co-ordinate system to the natural coordiante system by using the determinant of the Jacobian matrix, (i.e., Eq.~\eqref{detJac1D}) can be evaluated numerically as 
\begin{equation}
\mathbf{K}_{L} = \int_{\Gamma} \, \mathbf{B}_L^{\mathsf{T}} \, \mathbf{M} \, \mathbf{B}_L \, A \, \mathrm{d} x  = \int_{-1}^{1} \mathbf{B}_L^{\mathsf{T}} \, \mathbf{M} \, \mathbf{B}_L \, \mathbf{J}  \, A \,  \mathrm{d} \xi =  \sum_{i=1}^{n}  \mathbf{B}_L^{\mathsf{T}} \, \mathbf{M} \, \mathbf{B}_L \, \mathbf{J} \, A \, \mathrm{w}_i
\end{equation}
with $\mathrm{w}_i$ being the  weighting coefficients of the Gauss integration rule, and $A$ being the cross section area of alveolar chord. The values of $\xi$ and $\mathrm{w}_i$ for $n = 1, 2$, and $3$ Gauss integration points are demonstrated in Table~\ref{tabQuadrature1D}.

In calculation of the stiffness matrix for an element, we need the linear strain-displacement matrix $\mathbf{B}_L$. $\mathbf{B}_L$ contains the derivatives of the shape functions with respect to the global co-ordinate system and is given as 
\begin{equation}
\mathbf{B}_L = \begin{bmatrix}
N_{1,x} &   N_{2,x}
\end{bmatrix} = \begin{bmatrix}
N_{1,\xi} &   N_{2,\xi} \, 
\end{bmatrix} \, \mathbf{J}^{-1}
\end{equation}
wherein $\xi$ is abscissae of the Gauss integration rule. 

The large displacement stiffness matrix for chord can be written as
\begin{equation}
\mathbf{K}_{N} = \int_{-1}^{1} \mathbf{D} \, \mathbf{J}  \, A \,  \mathrm{d} \xi 
= \sum_{i=1}^{n} \mathbf{D} \, \mathbf{J} \, A \, \mathrm{w}_i
\end{equation}
where $\mathbf{D}$ takes on the form of
\begin{equation}
\mathbf{D} = \mathbf{B}_L^{\mathsf{T}} \, \mathbf{M} \, \mathbf{B}_N \, + \mathbf{B}_N^{\mathsf{T}} \, \mathbf{M} \, \mathbf{B}_L \, + \mathbf{B}_N^{\mathsf{T}} \, \mathbf{M} \, \mathbf{B}_N 
\end{equation}
wherein $\mathbf{B}_N$ can be expressed as 
\begin{equation}
\mathbf{B}_{N} =  \begin{bmatrix}
\partial \mathbf{u} / \partial \mathbf{x} 
\end{bmatrix} \, \begin{bmatrix}
N_{1,x} &  N_{2,x}
\end{bmatrix} = \begin{bmatrix}
\sum\nolimits_{=1}^n N_{i,x} \, u_i
\end{bmatrix} \, \begin{bmatrix}
N_{1,\xi} &   N_{2,\xi} \, 
\end{bmatrix} \, \mathbf{J}^{-1} 
\end{equation}
Thereby the total stiffness matrix $\mathbf{K}_T$ can be obtained by summation of stress stiffness matrix, small and large displacement stiffness matrices.

\subsection{Stiffness Matrix for Pentagon}
\textcolor{red}{For a planar member, the components of Laplace stretch $\boldsymbol{\mathcal{U}}$, obtained from a Cholesky factorization of the right Cauchy-Green tensor $\mathbf{C}\defeq \mathbf{F}^{\mathsf{T}} \mathbf{F} = \boldsymbol{\mathcal{U}}^{\mathsf{T}} \boldsymbol{\mathcal{U}}$, can be written as}
\begin{equation}
\begin{aligned}
{\mathcal{U}}_{11} = a & = \sqrt{C_{11}} \;\; & 
{\mathcal{U}}_{12} = a\:g & = C_{12} / {\mathcal{U}_{11}} \\
{\mathcal{U}}_{21} & = 0 &
{\mathcal{U}}_{22} = b & = \sqrt{C_{22} - ({\mathcal{U}}_{12})^2} 
\end{aligned}
\label{Laplace stretchComponents}
\end{equation} 
\textcolor{red}{where ${C_{11}}$, ${C_{12}}$=${C_{21}}$ and ${C_{22}}$ are the components of the right Cauchy-Green tensor $\mathbf{C}$. Furthermore, these components of Cauchy-Green tensor can be expressed in terms of the components of the deformation gradient $\mathbf{F}$ as}
\begin{subequations}
	\begin{align}
	\mathrm{C_{11}}&= \left(\frac{\mathrm{\partial u}}{\partial x}\right)^2+ \left(\frac{\mathrm{\partial v}}{\partial x}\right)^2 + 2\, \frac{\mathrm{\partial u}}{\partial x}  + 1\\
	\mathrm{C_{12}}&= \frac{\mathrm{\partial u}}{\partial y} + \frac{\mathrm{\partial v}}{\partial x} + \frac{\mathrm{\partial u}}{\partial x} \cdot \frac{\mathrm{\partial u}}{\partial y} + \frac{\mathrm{\partial v}}{\partial x} \cdot \frac{\mathrm{\partial v}}{\partial y}\\
	\mathrm{C_{22}}&= \left(\frac{\mathrm{\partial u}}{\partial y}\right)^2 + \left(\frac{\mathrm{\partial v}}{\partial y}\right)^2 + 2\, \frac{\mathrm{\partial v}}{\partial y} + 1
	\intertext{wherein}
	\mathbf{F} &=  
	\begin{bmatrix}
	1+\mathrm{\partial u / \partial x} & \mathrm{\partial u / \partial y}  \\
	\mathrm{\partial v / \partial x} & 1+\mathrm{\partial v / \partial y}
	\end{bmatrix}
	\end{align}
\end{subequations}
\textcolor{red}{Here, $u$ and $v$ are the displacements associated with the deformation of the planar member.}

\textcolor{red}{Now, the thermodynamic strain attributes, as defined earlier, can be expressed in terms of the components of Laplace stretch as}
\begin{equation}
\xi=\ln\left(\sqrt{\dfrac{a\:b}{a_0\:b_0}}\right);\quad\varepsilon=\ln\left(\sqrt{\dfrac{a\:b_0}{b\:a_0}}\right);\quad \gamma=g-g_0.
\end{equation}
\textcolor{red}{Without loss of generality, we can assume the membrane to be initially undeformed which allows us to assume $a_0$ and $b_0$ to be $1$ whereas the initial shear $g_0$ is taken as zero. To gain computational advantage, we decompose the total strain increments, thus obtained, into their linear and two nonlinear components as}
\begin{subequations}
	\begin{align}
	\mathrm  \xi & = \mathrm \xi_{L} + \mathrm  \xi_{N1} + \mathrm  \xi_{N2}\\
	\mathrm  \varepsilon & = \mathrm  \varepsilon_{L} + \mathrm  \varepsilon_{N1} + \mathrm  \varepsilon_{N2}\\
	\mathrm \gamma & = \mathrm  \gamma_{L} + \mathrm  \gamma_{N1} + \mathrm  \gamma_{N2}.
	\end{align}
	\label{totalvirtualstrain}
\end{subequations}
\textcolor{red}{Traditionally, the total strain increment is decomposed into a linear and a nonlinear components. However, in our case, further decomposition of the nonlinear strain component makes our computation much easier as will be realized later. The decomposition of total strain increment is achieved by using Taylor series expansion up to second-order terms. The linear and nonlinear components of total strain increments, thus obtained, are given as}
\begin{subequations}
	\begin{align}
	\mathrm \xi_{L} & = \frac{1}{2} \, \left(\mathrm{\frac{\partial u}{\partial x}} + \mathrm{\frac{\partial v}{\partial y}}\right)\\
	\mathrm \xi_{N1} & = \frac{1}{4} \, \left(- \mathrm{\frac{ \partial v}{\partial y}}\, \mathrm{\frac{ \partial v}{\partial y}} -\mathrm{\frac{\partial u}{\partial x}}\, \mathrm{\frac{\partial u}{\partial x}}\right)\\
	\mathrm{\xi_{N2}} & = - \frac{1}{2} \, \mathrm{\frac{\partial u}{\partial y}}\, \mathrm{\frac{\partial v}{\partial x}}\\
	\mathrm \varepsilon_{L} & = \frac{1}{2} \, \left(\mathrm{\frac{\partial u}{\partial x}} - \mathrm{\frac{\partial v}{\partial y}}\right)\\
	\mathrm \varepsilon_{N1} & = \frac{1}{4} \, \left(\mathrm{\frac{ \partial v}{\partial y}}\, \mathrm{\frac{ \partial v}{\partial y}} -\mathrm{\frac{\partial u}{\partial x}}\, \mathrm{\frac{\partial u}{\partial x}} \right)\\
	\mathrm{\varepsilon_{N2}} & = \frac{1}{2} \left(\mathrm{\frac{\partial v}{\partial x}}\, \mathrm{\frac{\partial v}{\partial x}} + \, \mathrm{\frac{\partial u}{\partial y}}\, \mathrm{\frac{\partial v}{\partial x}} \right)\\
	\mathrm \gamma_{L} & = \mathrm{\frac{\partial u}{\partial y}} + \mathrm{\frac{\partial v}{\partial x}}\\
	\mathrm \gamma_{N1} & = \mathrm{\frac{\partial v}{\partial x}}\, \mathrm{\frac{\partial v}{\partial y}}-\mathrm{\frac{\partial u}{\partial x}}\, \mathrm{\frac{\partial u}{\partial y}}\\
	\mathrm{\gamma_{N2}} & =  - 2\, \mathrm{\frac{ \partial u}{\partial x}}\, \mathrm{\frac{ \partial v}{\partial x}}.  
	\end{align}
\end{subequations}
\textcolor{red}{The linear components of strain attributes consist of only first-order derivatives of the displacements whereas the nonlinear components contain the second-order terms. The nonlinear components of strains are arranged in a way such that when represented in a set of bases, they can be written as a product of matrix and a vector containing the derivatives of displacements. To achieve that, for the dilatation and squeeze strains, the first nonlinear components consist of the squares of derivatives whereas the second nonlinear components are made up of products between different derivatives of the displacements, e.g., the component $\xi_{N1}$ contains terms of the form $\left(\dfrac{\partial u}{\partial x}\right)^2$ whereas its other counterpart, $\xi_{N2}$ is made up of a cross term like $\dfrac{\partial u}{\partial y}\dfrac{\partial v}{\partial x}$. Note that the nonlinear part of the total shear strain contains only products of different derivatives of displacements, i.e., no square term is present in its expression. Therefore, we club together the terms containing the products of derivatives of the \textit{same} displacement with respect to different spatial variables, e.g., $\dfrac{\partial v}{\partial x}\,\dfrac{\partial v}{\partial y}$ in $\gamma_{N1}$ whereas $\gamma_{N2}$ comprises the rest of the terms. This organization allows one to write the linear and the two nonlinear components of \textit{all} the strain attributes in a matrix form. Moreover, this procedure helps us to write the strain--displacement matrix corresponding to the linear and two nonlinear components of strain and thus, provides additional computational advantage.}

\textcolor{red}{To obtain the stiffness matrix for the planar membrane, first we need to compute the strain--displacement matrices as mentioned earlier. The strain--displacement matrices are obtained by expressing the corresponding strain components in terms of the five nodal displacements with the help of shape functions.}

\textcolor{red}{In terms of the nodal displacements the vector containing the linear strain attributes, $\mathbf{E}_L$ can be written as}
\begin{equation}
\begin{aligned}
\mathbf{E}_L & =  \begin{Bmatrix}
\mathrm \xi_{L} \\
\mathrm \varepsilon_{L} \\
\mathrm \gamma_{L} \end{Bmatrix} = 
\begin{Bmatrix}
\tfrac{1}{2} \, u_{,x} +   \tfrac{1}{2} \, v_{,y} \\
\tfrac{1}{2} \, u_{,x} - \tfrac{1}{2} \, v_{,y} \\
u_{,y} + v_{,x} \end{Bmatrix} = \begin{bmatrix}
\tfrac{1}{2} \, \sum\nolimits_{i=1}^5 N_{i,x}  & \tfrac{1}{2} \, \sum\nolimits_{i=1}^5 N_{i,y}  \\
\tfrac{1}{2} \, \sum\nolimits_{i=1}^5 N_{i,x}  & -\tfrac{1}{2} \, \sum\nolimits_{i=1}^5 N_{i,y}  \\ \sum\nolimits_{i=1}^5 N_{i,y}  & \sum\nolimits_{i=1}^5 N_{i,x}  \end{bmatrix} \, \begin{Bmatrix}
u_{i} \\
v_{i} \end{Bmatrix} \\
& = \begin{bmatrix}
[\mathbf{b}_{L1}], & [\mathbf{b}_{L2}], & [\mathbf{b}_{L3}], & [\mathbf{b}_{L4}], & [\mathbf{b}_{L5}] 
\end{bmatrix}  \begin{Bmatrix} \boldsymbol{\Delta} \end{Bmatrix}  
= [\mathbf{B}_L] \begin{Bmatrix} \boldsymbol{\Delta} \end{Bmatrix} 
\end{aligned}
\end{equation}
\textcolor{red}{wherein the components of the linear strain--displacement matrix can be written in terms of the derivatives of the shape function as}
\begin{subequations}
	\begin{align}
	\mathbf{b}_{Li} = & \begin{bmatrix}
	\tfrac{1}{2} \, N_{i,x}  &  \tfrac{1}{2} \, N_{i,y} \\
	\tfrac{1}{2} \, N_{i,x}  & - \tfrac{1}{2} \, N_{i,y} \\
	N_{i,y} & N_{i,x}  \end{bmatrix}\\
	\intertext{and}
	\mathbf{\Delta}^T  = &
	\begin{Bmatrix}
	u_1, v_1, u_2, v_2, ...,u_n, v_n
	\end{Bmatrix}.
	\end{align}
\end{subequations}
\textcolor{red}{The two nonlinear parts of strain are expressed as products of a matrix, $\mathbf{A}$ containing the elements of the displacement gradient and a vector, $\boldsymbol{\theta}$ that contains the derivatives of displacements. Following this procedure, the first nonlinear strain can be written as}
\begin{equation}
\begin{aligned}
\mathbf{E}_{N1} & =  \begin{Bmatrix}
\mathrm \xi_{N1} \\
\mathrm \varepsilon_{N1} \\
\mathrm \gamma_{N1} \end{Bmatrix} =
\begin{Bmatrix}
-\tfrac{1}{4} \, u_{,x}^2 -   \tfrac{1}{4} \, v_{,y}^2 \\
-\tfrac{1}{4} \, u_{,x}^2 + \tfrac{1}{4} \, v_{,y}^2 \\
-u_{,y} \, u_{,x} + v_{,x} \, v_{,y} \end{Bmatrix} = \frac{1}{2} \, \begin{bmatrix}
-\tfrac{1}{2} \, \partial u / \partial x &  - \tfrac{1}{2} \, \partial v / \partial y \\
-\tfrac{1}{2} \, \partial u / \partial x &  \tfrac{1}{2} \, \partial v / \partial y \\
-2 \, \partial u / \partial y  & 2 \, \partial v / \partial x  \end{bmatrix} \, \begin{Bmatrix}
\partial u / \partial x\\
\partial v / \partial y
\end{Bmatrix}
= \tfrac{1}{2} \, [\mathbf{A}_{N1}] \, [\boldsymbol{\theta}_{N1}]
\end{aligned}
\end{equation}
\textcolor{red}{To obtain the nonlinear strain--displacement matrix, we require the nonlinear strain to be expressed in terms of the nodal displacements. This is achieved by expressing the elements of displacement gradient in terms of the nodal displacements by using the shape functions, specifically, the vector $\boldsymbol{\theta}_{N1}$ can be written as}
\begin{equation}
[\boldsymbol{\theta}_{N1}] =  \begin{Bmatrix}
\partial u / \partial x\\
\partial v / \partial y
\end{Bmatrix}
= \begin{Bmatrix}
\sum\nolimits_{i=1}^5 N_{i,x} \, u_i\\
\sum\nolimits_{i=1}^5 N_{i,y} \, v_i
\end{Bmatrix} 
= \begin{bmatrix}
[\mathbf{h}_{N1}], & [\mathbf{h}_{N2}], & [\mathbf{h}_{N3}], & [\mathbf{h}_{N4}], & [\mathbf{h}_{N5}] 
\end{bmatrix}  \begin{Bmatrix} \boldsymbol{\Delta} \end{Bmatrix}  
= [\mathbf{H}_{N1}] \begin{Bmatrix} \boldsymbol{\Delta} \end{Bmatrix} 
\end{equation}
\textcolor{red}{where the components of $\mathbf{H}_{N1}$ contains the derivatives of shape functions with respect to spatial variables, i.e.,}
\begin{equation}
\mathbf{h}_{Ni} = \begin{bmatrix}
N_{i,x} &  0  \\
0 & N_{i,y}  \end{bmatrix} 
\end{equation}
\textcolor{red}{Therefore, the first nonlinear strain--displacement matrix $\mathbf{B}_{N1}$ can be written as}
\begin{equation}
\mathbf{B}_{N1} = [\mathbf{A}_{N1}] \, [\mathbf{H}_{N1}] = \begin{bmatrix}
[\mathbf{b}_{N1}], & [\mathbf{b}_{N2}], & [\mathbf{b}_{N3}], & [\mathbf{b}_{N4}], & [\mathbf{b}_{N5}] 
\end{bmatrix} 
\end{equation}
\textcolor{red}{where the components of $\mathbf{B}_{N1}$ are given as}
\begin{equation}
\mathbf{b}_{Ni} = \begin{bmatrix}
-\tfrac{1}{2} \, \partial u / \partial x &  - \tfrac{1}{2} \, \partial v / \partial y \\
-\tfrac{1}{2} \, \partial u / \partial x &  \tfrac{1}{2} \, \partial v / \partial y \\
-2 \, \partial u / \partial y  & 2 \, \partial v / \partial x  \end{bmatrix}  \, \begin{bmatrix}
N_{i,x} &  0  \\
0 & N_{i,y}  \end{bmatrix}. 
\end{equation}
\textcolor{red}{In a similar manner, the second nonlinear strain terms can be written as} 
\begin{equation}
\begin{aligned}
\mathbf{E}_{N2} & =  \begin{Bmatrix}
\mathrm \xi_{N2} \\
\mathrm \varepsilon_{N2} \\
\mathrm \gamma_{N2} \end{Bmatrix} =
\begin{Bmatrix}
-\tfrac{1}{2} \, u_{,y} \, v_{,x}  \\
\tfrac{1}{2} \, u_{,y} \, v_{,x}  + \tfrac{1}{2} \, v_{,x}^2 \\
-2 \, u_{,x} \, v_{,x} \end{Bmatrix} = \frac{1}{2} \, \begin{bmatrix}
- \partial v / \partial x &  0 \\
\partial v / \partial x &  \partial v / \partial x \\
0 & -4 \, \partial u / \partial x  \end{bmatrix} \, \begin{Bmatrix}
\partial u / \partial y\\
\partial v / \partial x
\end{Bmatrix}
= \tfrac{1}{2} \, [\mathbf{A}_{N2}] \, [\boldsymbol{\theta}_{N2}]
\end{aligned}.
\end{equation}
\textcolor{red}{The vector $\boldsymbol{\theta}_{N2}$ is expressed in terms of the nodal displacements with the use of shape functions as}
\begin{equation}
[\boldsymbol{\theta}_{N2}] =  \begin{Bmatrix}
\partial u / \partial y\\
\partial v / \partial x
\end{Bmatrix}
= \begin{Bmatrix}
\sum\nolimits_{i=1}^5 N_{i,y} \, u_i\\
\sum\nolimits_{i=1}^5 N_{i,x} \, v_i
\end{Bmatrix} 
= \begin{bmatrix}
[\mathbf{h}_{N1}], & [\mathbf{h}_{N2}], & [\mathbf{h}_{N3}], & [\mathbf{h}_{N4}], & [\mathbf{h}_{N5}] 
\end{bmatrix}  \begin{Bmatrix} \boldsymbol{\Delta} \end{Bmatrix}  
= [\mathbf{H}_2] \begin{Bmatrix} \boldsymbol{\Delta} \end{Bmatrix} 
\end{equation}
\textcolor{red}{where the elements of $\mathbf{H}_{N2}$ are given as}
\begin{equation}
\mathbf{h}_{Ni} = \begin{bmatrix}
N_{i,y} &  0  \\
0 & N_{i,x}  \end{bmatrix}. 
\end{equation}
\textcolor{red}{Hence, the second nonlinear strain--displacement matrix $\mathbf{B}_{N2}$ becomes}
\begin{equation}
\mathbf{B}_{N2} = [\mathbf{A}_{N2}] \, [\mathbf{H}_{N2}] = \begin{bmatrix}
[\mathbf{b}_{N1}], & [\mathbf{b}_{N2}], & [\mathbf{b}_{N3}], & [\mathbf{b}_{N4}], & [\mathbf{b}_{N5}] 
\end{bmatrix} 
\end{equation}
\textcolor{red}{where its elements are given as}
\begin{equation}
\mathbf{b}_{Ni} = \begin{bmatrix}
- \partial v / \partial x &  0 \\
\partial v / \partial x &  \partial v / \partial x \\
0 & -4 \, \partial u / \partial x  \end{bmatrix} \, \begin{bmatrix}
N_{i,y} &  0  \\
0 & N_{i,x}  \end{bmatrix}. 
\end{equation}
\textcolor{red}{The total nonlinear strain--displacement matrix is evaluated as the summation of its components $ \mathbf{B}_{N1}$ and $\mathbf{B}_{N2}$. Now, with all the strain--displacement matrices evaluated, we are ready to compute the stiffness matrix for a planar membrane.}

\textcolor{red}{The total stiffness matrix is additively decomposed into three components: the stiffness matrix associated with initial stress $\mathbf{K_S}$, the linear stiffness matrix $\mathbf{K}_L$ and the nonlinear stiffness matrix $\mathbf{K}_N$. The initial stress stiffness matrix  for a 2-D alveolar septa is given as}
\begin{equation}
\mathbf{K}_\mathbf{S}\,\mathrm{d}\boldsymbol{\Delta} = \int_{\pentagon} \int_{\pentagon}  \mathrm{d} \mathbf{B}_N^{\mathsf{T}} \, \mathbf{T} \, |\mathbf{J}|  \, h \,  \mathrm{d} \xi \,  \mathrm{d} \eta \label{stressstif}
\end{equation}
\textcolor{red}{From the expression of $\mathbf{K_{\mathbf{S}}}$, it is easily understood that we need to express $\mathrm{d}\mathbf{B}_N$ in terms of the incremental nodal displacements $\mathrm{d}\boldsymbol{\Delta}$. Using Eqn.~\eqref{nonlinStrainDisp}, we get}
\begin{equation}
\mathrm{d}[\mathbf{B}_N]^{\mathsf{T}} = [\mathbf{H}_{N1}^{\mathsf{T}}] \mathrm{d} [\mathbf{A}_{N1}^{\mathsf{T}}] + [\mathbf{H}_{N2}^{\mathsf{T}}] \mathrm{d} [\mathbf{A}_{N2}^{\mathsf{T}}] 
\end{equation}
\textcolor{red}{which on substitution into Eq.~(\ref{stressstif}) yields} 
\begin{equation}
\mathbf{K_\mathbf{S}} \mathrm{d} \boldsymbol{\Delta} = \int_A \left(
[\mathbf{H}_1^{\mathsf{T}}] \mathrm{d} [\mathbf{A}_1^{\mathsf{T}}] + [\mathbf{H}_2^{\mathsf{T}}] \mathrm{d} [\mathbf{A}_2^{\mathsf{T}}]\right) \, \begin{Bmatrix} s^{\pi} \\ s^{\sigma} \\  s^{\tau} \end{Bmatrix} \, \mathrm{d} A
\label{initstiff}
\end{equation}
\textcolor{red}{Furthermore, from the definition of $\mathbf{A}$, one can observe that}
\begin{equation}
\mathrm{d} [\mathbf{A}_1^{\mathsf{T}}] \, \begin{Bmatrix} s^{\pi} \\ s^{\sigma} \\  s^{\tau} \end{Bmatrix} = \begin{bmatrix}s^{\pi} & s^{\tau} \\ 
s^{\tau} & s^{\sigma} \end{bmatrix} \, [\mathbf{H}_1] \, \mathrm{d} \boldsymbol{\Delta} \quad \text{and} \quad \mathrm{d} [\mathbf{A}_2^{\mathsf{T}}] \, \begin{Bmatrix} s^{\pi} \\ s^{\sigma} \\  s^{\tau} \end{Bmatrix} = \begin{bmatrix} s^{\pi} & s^{\tau} \\ 
s^{\tau} & s^{\sigma} \end{bmatrix} \, [\mathbf{H}_2] \, \mathrm{d} \boldsymbol{\Delta}
\end{equation}.
\textcolor{red}{Using the above relation in Eqn.~\eqref{initstiff}, finally we obtain the stiffness matrix associated with initial stress as}
\begin{equation}
\begin{aligned}
\mathbf{K_{\mathbf{S}}} & =  \int_A \, [\mathbf{H}_1]^{\mathsf{T}}  \begin{bmatrix} s^{\pi} & s^{\tau} \\ 
s^{\tau} & s^{\sigma}
\end{bmatrix} \, [\mathbf{H}_1] \, \mathrm{d} A + \int_A \, [\mathbf{H}_2]^{\mathsf{T}}  \begin{bmatrix} s^{\pi} & s^{\tau} \\ 
s^{\tau} & s^{\sigma} 
\end{bmatrix} \, [\mathbf{H}_2] \, \mathrm{d} A\\
& =  \sum_{i=1}^{n} \sum_{j=1}^{n} \left( [\mathbf{H}_1]^{\mathsf{T}}  \begin{bmatrix} s^{\pi} & s^{\tau} \\ 
s^{\tau} & s^{\sigma}
\end{bmatrix} \, [\mathbf{H}_1] \, |\mathbf{J}| \, h  + [\mathbf{H}_2]^{\mathsf{T}}  \begin{bmatrix} s^{\pi} & s^{\tau} \\ 
s^{\tau} & s^{\sigma}
\end{bmatrix} \, [\mathbf{H}_2] \, |\mathbf{J}| \, h \right) \,  w_i \, w_j
\end{aligned}
\end{equation}
\textcolor{red}{where $n$ denotes the number of Gauss points, $\mathrm{w}_i$ and $\mathrm{w}_j$ denote the natural weight of the element demonstrated in Table~\ref{tabQuadrature}.}

\textcolor{red}{The linear and nonlinear stiffness matrices are rather easy to calculate from their definitions provided in Eqn.~\eqref{stiffnessMatrices}. For the sake of completeness their expressions are provided below.}

\textcolor{red}{The small/linear displacement stiffness matrix for the membrane is evaluated numerically as} 
\begin{equation}
\mathbf{K}_{L} = \int_{\pentagon} \int_{\pentagon} \mathbf{B}_L^{\mathsf{T}} \, \mathbf{M} \, \mathbf{B}_L \, |\mathbf{J}| \, h \,  \mathrm{d} \xi \,  \mathrm{d} \eta =  \sum_{i=1}^{n} \sum_{j=1}^{n} \mathbf{B}_L^{\mathsf{T}} \, \mathbf{M} \, \mathbf{B}_L \, |\mathbf{J}|  \, h \, \mathrm{w}_i \mathrm{w}_j
\end{equation}
\textcolor{red}{whereas the large/nonlinear displacement stiffness matrix takes the form of}
\begin{equation}
\mathbf{K}_{N} = \int_{\pentagon} \int_{\pentagon} \mathbf{D} \, |\mathbf{J}|\, h \, \mathrm{d} \xi \, \mathrm{d} \eta 
= \sum_{i=1}^{n} \sum_{j=1}^{n} \mathbf{D} \, |\mathbf{J}| \, h \, \mathrm{w}_i \mathrm{w}_j
\end{equation}
\textcolor{red}{where $\mathbf{D}$ contains the products of linear and nonlinear strain--displacement matrices and is given as} 
\begin{equation}
\mathbf{D} = \mathbf{B}_L^{\mathsf{T}} \, \mathbf{M} \, \mathbf{B}_N \, + \mathbf{B}_N^{\mathsf{T}} \, \mathbf{M} \, \mathbf{B}_L \, + \mathbf{B}_N^{\mathsf{T}} \, \mathbf{M} \, \mathbf{B}_N.
\end{equation}
\textcolor{red}{Once the number of Gauss points and their corresponding weights have been selected, one can numerically evaluate all the components of stiffness matrices, addition of which yields the total stiffness matrix for the pentagonal membrane.}


\subsection{Stiffness Matrix for a tetrahedron}
\textcolor{red}{Let us consider a tetrahedron subjected to the displacements $u$, $v$ and $w$ in three spatial directions respectively. In terms of the displacement gradient, the elements of the deformation gradient can be written as}
\begin{equation}
\mathbf{F} =  
\begin{bmatrix}
1 + \mathrm{\partial u / \partial x} & \mathrm{\partial u / \partial y} & \mathrm{\partial u / \partial z} \\
\mathrm{\partial v / \partial x} & 1 + \mathrm{\partial v / \partial y} & \mathrm{\partial v / \partial z} \\
\mathrm{\partial w / \partial x} & \mathrm{\partial w / \partial y} & 1 + \mathrm{\partial w / \partial z}
\end{bmatrix}\\
\end{equation}	
\textcolor{red}{Therefore, the components of the right Cauchy--Green tensor, defined as $\mathbf{C}\defeq \mathbf{F}^\mathsf{T}\mathbf{F}$, can be expressed as}
\begin{subequations}
	\begin{align}	
	\mathrm{C_{11}} & = \left(\frac{\mathrm{\partial u}}{\partial x}\right)^2 + \left(\frac{\mathrm{\partial v}}{\partial x}\right)^2 + \left(\frac{\mathrm{\partial w}}{\partial x}\right)^2 + 2\, \frac{\mathrm{\partial u}}{\partial x}  + 1\\
	\mathrm{C_{22}} & = \left(\frac{\mathrm{\partial u}}{\partial y}\right)^2 + \left(\frac{\mathrm{\partial v}}{\partial y}\right)^2 + \left(\frac{\mathrm{\partial w}}{\partial y}\right)^2 + 2\, \frac{\mathrm{\partial v}}{\partial y} + 1\\
	\mathrm{C_{33}} & = \left(\frac{\mathrm{\partial u}}{\partial z}\right)^2 + \left(\frac{\mathrm{\partial v}}{\partial z}\right)^2 + \left(\frac{\mathrm{\partial w}}{\partial z}\right)^2 + 2\, \frac{\mathrm{\partial w}}{\partial z} + 1 \\
	\mathrm{C_{12}} & = \mathrm{C_{21}} = \frac{\mathrm{\partial u}}{\partial y} + \frac{\mathrm{\partial v}}{\partial x} + \frac{\mathrm{\partial u}}{\partial x} \cdot \frac{\mathrm{\partial u}}{\partial y} + \frac{\mathrm{\partial v}}{\partial x} \cdot \frac{\mathrm{\partial v}}{\partial y} + \frac{\mathrm{\partial w}}{\partial x} \cdot \frac{\mathrm{\partial w}}{\partial y}\\
	\mathrm{C_{13}} & = \mathrm{C_{31}} = \frac{\mathrm{\partial u}}{\partial z} + \frac{\mathrm{\partial w}}{\partial x} + \frac{\mathrm{\partial u}}{\partial x} \cdot \frac{\mathrm{\partial u}}{\partial z} + \frac{\mathrm{\partial v}}{\partial x} \cdot \frac{\mathrm{\partial v}}{\partial z} + \frac{\mathrm{\partial w}}{\partial x} \cdot \frac{\mathrm{\partial w}}{\partial z} \\
	\mathrm{C_{23}} & = \mathrm{C_{32}} = \frac{\mathrm{\partial v}}{\partial z} + \frac{\mathrm{\partial w}}{\partial y} + \frac{\mathrm{\partial u}}{\partial y} \cdot \frac{\mathrm{\partial u}}{\partial z} + \frac{\mathrm{\partial v}}{\partial y} \cdot \frac{\mathrm{\partial v}}{\partial z} + \frac{\mathrm{\partial w}}{\partial y} \cdot \frac{\mathrm{\partial w}}{\partial z}.
	\end{align}
\end{subequations}
\textcolor{red}{The Laplace stretch associated with an alveolar volume is a $3\times3$ upper-triangular matrix whose elements have specific geometric interpretations. This upper-triangular Laplace stretch can be written in matrix form as \cite{FreedSrinivasa15}}
\begin{subequations}
	\label{LagrangianPhysicalStretch}
	\begin{align}
	\mathcal{U}_{ij} & = \begin{bmatrix}
	a & a \gamma & a \beta \\
	0 & b & b \alpha \\
	0 & 0 & c \end{bmatrix} 
	\end{align}
\end{subequations}
\textcolor{red}{It is possible to express the components of Laplace stretch $\boldsymbol{\mathcal{U}}$ in terms of the displacement gradient through a Cholesky factorization of the right Cauchy-Green tensor $\mathbf{C} \defeq \boldsymbol{\mathcal{U}}^{\mathsf{T}} \boldsymbol{\mathcal{U}}$. Specifically, the elements of Laplace stretch are obtained as} 
\begin{equation}
\begin{aligned}
\mathcal{U}_{11} & = \sqrt{C_{11}} & 
\mathcal{U}_{12} & = C_{12} / \mathcal{U}_{11} &
\mathcal{U}_{13} & = C_{13} / \mathcal{U}_{11} \\
\mathcal{U}_{21} & = 0 &
\mathcal{U}_{22} & = \sqrt{C_{22} - \mathcal{U}_{12}^{\,2}} &
\mathcal{U}_{23} & = \bigl( C_{23} - \mathcal{U}_{12\,}\mathcal{U}_{13} \bigr) / \mathcal{U}_{22} \\
\mathcal{U}_{31} & = 0 &
\mathcal{U}_{32} & = 0 & 
\mathcal{U}_{33} & = \sqrt{C_{33} - \mathcal{U}_{13}^{\,2} - \mathcal{U}_{23}^{\,2}}
\end{aligned}
\label{LagrangianLaplaceStretch}
\end{equation}
\textcolor{red}{where ${C_{11}}$, ${C_{12}}$, ${C_{13}}$, $C_{22}$, $C_{23}$ and ${C_{33}}$ are components of the right Cauchy--Green tensor $\mathbf{C}$.}

\textcolor{red}{Now in order to obtain the stiffness matrix for an alveolar volume, first we need to we derive the strain attributes and express them in terms of the nodal displacements. The strain attributes are defined in terms of the derived elements of Laplace stretch as mentioned earlier.}

\textcolor{red}{The dilatation $\delta$ for an alveolar volume is} 
\begin{subequations}
	\begin{align}
	\mathrm  \xi & \defeq \ln \sqrt[3]{\frac{a}{a_0}
		\frac{b}{b_0} \frac{c}{c_0}} \\
	\intertext{whereas the squeezes $\varepsilon_i$ and the shear strains $\gamma_i$ are defined as}
	\varepsilon_1 & \defeq \ln \sqrt[3]{\frac{a}
		{a_0} \frac{b_0}{b}} & 	\gamma_1 & \defeq \alpha - \alpha_0 \\
	\varepsilon_2 & \defeq \ln \sqrt[3]{\frac{b}
		{b_0} \frac{c_0}{c}} & \gamma_2 & \defeq \beta - \beta_0 \\
	\varepsilon_3 & \defeq \ln \sqrt[3]{\frac{c}
		{c_0} \frac{a_0}{a}} & 	\gamma_3 & \defeq \gamma - \gamma_0 
	\end{align}
\end{subequations}
\textcolor{red}{wherein $a_0$, $b_0$ and $c_0$ are their initial elongation ratios, and where $\alpha_0$, $\beta_0$ and $\gamma_0$ are their initial shears. Without loss of generality, we can assume that the initial stretches $a_0$, $b_0$ and $c_0$ are one whereas the initial shears $\alpha_0$, $\beta_0$ and $\gamma_0$ are zero.}


\textcolor{red}{For computational ease, these strain attributes are additively decomposed into a linear and three nonlinear components. The primary advantage of this decomposition is the emergence of a systematic structure of the strain--displacement matrix which makes the evaluation of the stiffness matrix much easier. The linear and nonlinear components of the strain attributes are obtained by using a Taylor series expansion of the strain attributes and expressing their constituents in terms of the gradients of the displacements with respect to different spatial variables. Here only terms up to second-order have been considered. The linear and nonlinear components of the strain attributes, thus obtained, are given as}
\begin{subequations}
	\begin{align}
	\mathrm  \xi & = \mathrm \xi_{L} + \mathrm  \xi_{N1} + \mathrm  \xi_{N2} + \mathrm  \xi_{N3}\\
	\mathrm  \varepsilon_{i} & = \mathrm  \varepsilon_{iL} + \mathrm  \varepsilon_{iN1} + \mathrm  \varepsilon_{iN2} + \mathrm  \varepsilon_{iN3} \\
	\mathrm \gamma_{i} & = \mathrm  \gamma_{iL} + \mathrm  \gamma_{iN1} + \mathrm  \gamma_{iN2} + \mathrm  \gamma_{iN3}
	\end{align}
	\label{totalvirtualstrain}
\end{subequations}
where their linear and nonlinear components can be expressed in terms of elements of the displacement gradient matrix as

dilatation, $\xi$:
\begin{subequations}
	\begin{align}
	\mathrm \xi_{L} & = \frac{1}{3} \, \left(\mathrm{\frac{\partial u}{\partial x}} + \mathrm{\frac{\partial v}{\partial y}} + \mathrm{\frac{\partial w}{\partial z}}\right)\\
	\mathrm \xi_{N1} & = \frac{1}{6} \Big(-\mathrm{\frac{\partial u}{\partial x}}\, \mathrm{\frac{\partial u}{\partial x}}- 2 \, \mathrm{\frac{\partial u}{\partial y}}\, \mathrm{\frac{\partial v}{\partial x}} + \mathrm{\frac{\partial w}{\partial x}}\, \mathrm{\frac{\partial w}{\partial x}}\Big)\\
	\mathrm \xi_{N2} & = \frac{1}{6}\Big(- \mathrm{\frac{ \partial v}{\partial y}}\, \mathrm{\frac{ \partial v}{\partial y}} - \mathrm{\frac{\partial w}{\partial y}}\, \mathrm{\frac{\partial w}{\partial y}}\Big)\\
	\mathrm \xi_{N3} & = \frac{1}{6}\Big(\mathrm{\frac{\partial u}{\partial z}}\, \mathrm{\frac{\partial u}{\partial z}} - \mathrm{\frac{\partial v}{\partial z}}\, \mathrm{\frac{\partial v}{\partial z}} - \mathrm{\frac{\partial w}{\partial z}}\, \mathrm{\frac{\partial w}{\partial z}}  - 4 \, \mathrm{\frac{\partial v}{\partial z}}\, \mathrm{\frac{\partial w}{\partial y}}\Big)
	\end{align}
\end{subequations}

normal strain, $\varepsilon_1$:
\begin{subequations}
	\begin{align}
	\mathrm \varepsilon_{1L} & = \frac{1}{3} \, \left(\mathrm{\frac{\partial u}{\partial x}} - \mathrm{\frac{\partial v}{\partial y}}\right)\\
	\mathrm \varepsilon_{1N1} & = \frac{1}{6}\Big( - \mathrm{\frac{\partial u}{\partial x}}\, \mathrm{\frac{\partial u}{\partial x}}2 \, + \mathrm{\frac{\partial v}{\partial x}}\, \mathrm{\frac{\partial v}{\partial x}} +  \mathrm{\frac{\partial w}{\partial x}}\, \mathrm{\frac{\partial w}{\partial x}}\Big)\\
	\mathrm \varepsilon_{1N2} & = \frac{1}{6}\Big(2 \, \mathrm{\frac{\partial u}{\partial y}}\, \mathrm{\frac{\partial v}{\partial x}}+ 2\,\mathrm{\frac{ \partial v}{\partial y}}-  \mathrm{\frac{\partial w}{\partial y}}\, \mathrm{\frac{\partial w}{\partial y}}\Big)\\
	\mathrm \varepsilon_{1N3} & = 0
	\end{align}
\end{subequations}

normal strain, $\varepsilon_2$:
\begin{subequations}
	\begin{align}
	\mathrm \varepsilon_{2L} & = \frac{1}{3} \, \left(\mathrm{\frac{\partial v}{\partial y}} - \mathrm{\frac{\partial w}{\partial z}}\right)\\
	\mathrm \varepsilon_{2N1} & = \frac{1}{6} \, \Big(- \mathrm{\frac{\partial v}{\partial x}}\, \mathrm{\frac{\partial v}{\partial x}}\Big)\\
	\mathrm \varepsilon_{2N2} & = \frac{1}{6} \, \Big( - 2 \, \mathrm{\frac{\partial u}{\partial y}}\, \mathrm{\frac{\partial v}{\partial x}} - \mathrm{\frac{ \partial v}{\partial y}}\, \mathrm{\frac{ \partial v}{\partial y}}+ 3 \,  \mathrm{\frac{\partial w}{\partial y}}\, \mathrm{\frac{\partial w}{\partial y}}\Big)\\
	\mathrm \varepsilon_{2N3} & = \frac{1}{6} \, \Big(	 - \mathrm{\frac{\partial u}{\partial z}}\, \mathrm{\frac{\partial u}{\partial z}} + \mathrm{\frac{\partial v}{\partial z}}\, \mathrm{\frac{\partial v}{\partial z}}  +  \mathrm{\frac{\partial w}{\partial z}}\, \mathrm{\frac{\partial w}{\partial z}} + 4 \, \mathrm{\frac{\partial v}{\partial z}}\, \mathrm{\frac{\partial w}{\partial y}}\Big)
	\end{align}
\end{subequations}

normal strain, $\varepsilon_3$:
\begin{subequations}
	\begin{align}
	\mathrm \varepsilon_{3L} & = \frac{1}{3} \, \left(-\mathrm{\frac{\partial u}{\partial x}} + \mathrm{\frac{\partial w}{\partial z}}\right)\\
	\mathrm \varepsilon_{3N1} & = \frac{1}{6} \, \Big(\mathrm{\frac{ \partial u}{\partial x}}\, \mathrm{\frac{ \partial u}{\partial x}}	- \mathrm{\frac{\partial v}{\partial x}}\, \mathrm{\frac{\partial v}{\partial x}}- \mathrm{\frac{\partial w}{\partial x}}\, \mathrm{\frac{\partial w}{\partial x}}\Big)\\
	\mathrm \varepsilon_{3N2} & = \frac{1}{6} \, \Big( - 2 \,  \mathrm{\frac{\partial w}{\partial y}}\, \mathrm{\frac{\partial w}{\partial y}}\Big)\\
	\mathrm \varepsilon_{3N3} & = \frac{1}{6} \, \Big(\mathrm{\frac{\partial u}{\partial z}}\, \mathrm{\frac{\partial u}{\partial z}} - \mathrm{\frac{\partial v}{\partial z}}\, \mathrm{\frac{\partial v}{\partial z}} -  \mathrm{\frac{\partial w}{\partial z}}\, \mathrm{\frac{\partial w}{\partial z}} - 4 \, \mathrm{\frac{\partial v}{\partial z}}\, \mathrm{\frac{\partial w}{\partial y}}\Big)
	\end{align}
\end{subequations}

shear strain, $\gamma_1$:
\begin{subequations}
	\begin{align}
	\mathrm \gamma_{1L} & = \mathrm{\frac{\partial v}{\partial z}} + \mathrm{\frac{\partial w}{\partial y}}\\
	\mathrm \gamma_{1N1} & = 2 \, \mathrm{\frac{\partial u}{\partial x}}\, \mathrm{\frac{\partial v}{\partial z}}	 - \mathrm{\frac{ \partial u}{\partial z}}\, \mathrm{\frac{ \partial v}{\partial x}}-   \mathrm{\frac{\partial v}{\partial x}}\, \mathrm{\frac{\partial w}{\partial x}}\\
	\mathrm \gamma_{1N2} & = - \mathrm{\frac{\partial u}{\partial y}}\, \mathrm{\frac{\partial w}{\partial x}} - 2 \, \mathrm{\frac{\partial v}{\partial y}}\, \mathrm{\frac{\partial w}{\partial y}} + 2 \mathrm{\frac{\partial w}{\partial y}}\, \mathrm{\frac{\partial u}{\partial x}}\\
	\mathrm \gamma_{1N3} & = - \mathrm{\frac{\partial v}{\partial y}}\, \mathrm{\frac{\partial v}{\partial z}} + \dfrac{\partial w}{\partial y} \dfrac{\partial w}{\partial z}
	\end{align}
\end{subequations}

shear strain, $\gamma_2$:
\begin{subequations}
	\begin{align}
	\mathrm \gamma_{2L} & = \mathrm{\frac{\partial v}{\partial z}} + \mathrm{\frac{\partial w}{\partial y}}\\
	\mathrm \gamma_{2N1} & =  - 2 \, \mathrm{\frac{ \partial u}{\partial x}}\, \mathrm{\frac{ \partial v}{\partial z}}\\
	\mathrm \gamma_{2N2} & = \mathrm{\frac{\partial u}{\partial y}}\, \mathrm{\frac{\partial u}{\partial z}} - 2 \, \mathrm{\frac{\partial u}{\partial x}}\, \mathrm{\frac{\partial w}{\partial y}}\\
	\mathrm \gamma_{2N3} & = \mathrm{\frac{\partial v}{\partial y}}\, \mathrm{\frac{\partial v}{\partial z}} + \mathrm{\frac{\partial w}{\partial y}}\, \mathrm{\frac{\partial w}{\partial z}}
	\end{align}
\end{subequations}

shear strain, $\gamma_3$:
\begin{subequations}
	\begin{align}
	\mathrm \gamma_{3L} & = \mathrm{\frac{\partial u}{\partial y}} + \mathrm{\frac{\partial v}{\partial x}}\\
	\mathrm \gamma_{3N1} & = - \mathrm{\frac{\partial u}{\partial x}}\, \mathrm{\frac{\partial u}{\partial y}} - 2 \, \mathrm{\frac{ \partial u}{\partial x}}\, \mathrm{\frac{ \partial v}{\partial x}}\\
	\mathrm \gamma_{3N2} & = \mathrm{\frac{\partial v}{\partial x}}\, \mathrm{\frac{\partial v}{\partial y}} + \mathrm{\frac{\partial w}{\partial x}}\, \mathrm{\frac{\partial w}{\partial y}}\\
	\mathrm \gamma_{3N3} & = 0
	\end{align}.
\end{subequations}

\textcolor{red}{The total stiffness matrix can be obtained as a sum of the linear and three nonlinear stiffness matrices. Therefore, we first have to evaluate these components of the stiffness matrices by using the associated strain--displacement matrices. For all these cases, the strain--displacement matrices are obtained by expressing the strains in terms of the nodal displacements with the help of shape functions and their derivatives.}

\textcolor{red}{First, the linear strain--displacement matrix $\mathbf{B}_L$ is obtained by expressing the strain attributes in terms of the nodal displacements through derivatives of the shape functions. Specifically, the linear strain--displacement matrix takes the form of}
\begin{equation}
\begin{aligned}
\mathbf{E}_L & =  \begin{Bmatrix}
\mathrm \xi_{L} \\
\mathrm \varepsilon_{1L} \\
\mathrm \varepsilon_{2L} \\
\mathrm \varepsilon_{3L} \\
\mathrm \gamma_{1L} \\
\mathrm \gamma_{2L} \\
\mathrm \gamma_{3L}\end{Bmatrix} = 
\begin{Bmatrix}
\tfrac{1}{3} \, u_{,x} + \tfrac{1}{3} \, v_{,y} + \tfrac{1}{3} \, w_{,z} \\
\tfrac{1}{3} \, u_{,x} - \tfrac{1}{3} \, v_{,y} \\
\tfrac{1}{3} \, v_{,y} - \tfrac{1}{3} \, w_{,z} \\
- \tfrac{1}{3} \, u_{,x} + \tfrac{1}{3} \, w_{,z} \\ 
v_{,z} +  w_{,y} \\ 
v_{,z} +  w_{,y} \\ 
u_{,y} +  v_{,x} \\\end{Bmatrix} = \begin{bmatrix}
\tfrac{1}{3} \, \sum\nolimits_{i=1}^4 N_{i,x}  & \tfrac{1}{3} \, \sum\nolimits_{i=1}^4 N_{i,y} & \tfrac{1}{3} \, \sum\nolimits_{i=1}^4 N_{i,z} \\
\tfrac{1}{3} \, \sum\nolimits_{i=1}^4 N_{i,x}  & \tfrac{-1}{3} \, \sum\nolimits_{i=1}^4 N_{i,y} & 0 \\ 0 & \tfrac{1}{3} \, \sum\nolimits_{i=1}^4 N_{i,y}  &  \tfrac{-1}{3} \, \sum\nolimits_{i=1}^4 N_{i,z} \\  \tfrac{-1}{3} \, \sum\nolimits_{i=1}^4 N_{i,x}  & 0 & \tfrac{1}{3} \, \sum\nolimits_{i=1}^4 N_{i,z} \\ 0 & \sum\nolimits_{i=1}^4 N_{i,z}  &  \sum\nolimits_{i=1}^4 N_{i,y} \\ 0 & \sum\nolimits_{i=1}^4 N_{i,z}  &  \sum\nolimits_{i=1}^4 N_{i,y} \\ \sum\nolimits_{i=1}^4 N_{i,y}  &  \sum\nolimits_{i=1}^4 N_{i,x} & 0 \end{bmatrix} \, \begin{Bmatrix}
u_{i} \\
v_{i} \\
w_{i} \end{Bmatrix} \\
& = \begin{bmatrix}
[\mathbf{b}_{L1}], & [\mathbf{b}_{L2}], & [\mathbf{b}_{L3}], & [\mathbf{b}_{L4}]
\end{bmatrix}  \begin{Bmatrix} \boldsymbol{\Delta} \end{Bmatrix}  
= [\mathbf{B}_L] \begin{Bmatrix} \boldsymbol{\Delta} \end{Bmatrix} 
\end{aligned}
\end{equation}
\textcolor{red}{wherein each component of $\mathbf{B}_L$ is given by} 
\begin{subequations}
	\begin{align}
	\mathbf{b}_{Li} = & \begin{bmatrix}
	\tfrac{1}{3} \,  N_{i,x}  & \tfrac{1}{3} \,  N_{i,y} & \tfrac{1}{3} \,  N_{i,z} \\
	\tfrac{1}{3} \,  N_{i,x}  & -\tfrac{1}{3} \,  N_{i,y} & 0 \\ 0 & \tfrac{1}{3} \, N_{i,y}  & - \tfrac{1}{3} \,  N_{i,z} \\ - \tfrac{1}{3} \,  N_{i,x}  & 0 & \tfrac{1}{3} \,  N_{i,z} \\ 0 &  N_{i,z}  &  N_{i,y} \\ 0 & N_{i,z}  &   N_{i,y} \\  N_{i,y}  &   N_{i,x} & 0 \end{bmatrix}  \\
	\intertext{and the nodal displacement vector is given as}
	\mathbf{\Delta}^T  = &
	\begin{Bmatrix}
	u_1, v_1, w_1, u_2, v_2, w_2 ...,u_n, v_n, w_n
	\end{Bmatrix}
	\end{align}
\end{subequations}
\textcolor{red}{Note that the linear--strain displacement matrix $\mathbf{B_L}$ consists only of the derivatives of the shape functions and thus, remains the same throughout the deformation process.}

\textcolor{red}{Now we derive the nonlinear strain--displacement matrices that will be used to obtain the nonlinear stiffness matrix. The nonlinear components of each strain attribute have been additively decomposed into three components to make our computation easier. Components of each strain attribute are placed into an associated vector resulting in an additive decomposition of the total nonlinear strain $\mathbf{E}_N$. To obtain the nonlinear stiffness matrix corresponding to these nonlinear strain components, the nonlinear strains are written as a product of two quantities: a matrix $\mathbf{A}_N$ containing various components of the displacement gradient and a vector $\boldsymbol{\theta}_N$ that contains the derivatives of displacement with respect to spatial derivatives. The vector $\boldsymbol{\theta}_N$ essentially represents the slope of the body resulting from the deformation process. The components of the displacement gradient are placed in the matrix $\mathbf{A}$ in such a way so that its product with the slope vector yields the corresponding nonlinear strain component. Note that the nonlinear strain is decomposed into three components in a way such that their corresponding slope vectors $\boldsymbol{\theta}_{N1}, \boldsymbol{\theta}_{N2}$ and $\boldsymbol{\theta}_{N3}$ consist of derivatives of displacements with respect to spatial variables $x, y$ and $z$ respectively.}

\textcolor{red}{The slope vector $\boldsymbol{\theta}_N$ can further be expressed in terms of the corresponding nodal displacements by using the derivatives of the shape functions. Thus, the nonlinear strain components $\mathbf{E}_{Ni}, i=1, 2, 3$ can be expressed in terms of the nodal displacements and the nonlinear strain--displacement matrix $\mathbf{B}_{Ni}$ corresponding to these strain components are obtained. These strain--displacement matrices are used to obtain the corresponding nonlinear stiffness matrices in a way described earlier Note that, unless its linear counterpart, the nonlinear strain--displacement matrix varies with the deformation of the body and hence, the matrices corresponding to it must be updated at each time step.}

\textcolor{red}{Now let us perform the procedure described above on all three nonlinear strain components.}

\textcolor{red}{For the first nonlinear strain, $\mathbf{E}_{N1}$ can be written as a product of the matrix $\mathbf{A}_{N1}$ and the slope vector $\boldsymbol{\theta}_{N1}$ as} 
\begin{equation}
\begin{aligned}
\mathbf{E}_{N1} & =  \begin{Bmatrix}
\mathrm \xi_{1N} \\
\mathrm \varepsilon_{1N} \\
\mathrm \varepsilon_{2N} \\
\mathrm \varepsilon_{3N} \\
\mathrm \gamma_{1N} \\
\mathrm \gamma_{2N} \\
\mathrm \gamma_{3N} \end{Bmatrix} =
\begin{Bmatrix}
\tfrac{1}{6} \,  \left( -u_{,x}^2 - 2 \, u_{,y} \, v_{,x}  +  w_{,x}^2  \right)\\
\tfrac{1}{6} \,  \left( -u_{,x}^2 + 2 \, v_{,x}^2 +  w_{,x}^2 \right) \\
\tfrac{1}{6} \,  \left( - v_{,x}^2 \right) \\
\tfrac{1}{6} \,  \left( u_{,x}^2 - v_{,x}^2 -  w_{,x}^2  \right) \\ 
2 \, u_{,x} \, v_{,z} - u_{,z} \, v_{,x} - v_{,x} \, w_{,x} \\
- 2 \, u_{,x} \, v_{,z} \\ 
- u_{,x} \, u_{,y} - 2 \, u_{,x} \, v_{,x}\end{Bmatrix} \\
& = \frac{1}{2} \, \begin{bmatrix}
-\tfrac{1}{3} \, \partial u / \partial x &  - \tfrac{2}{3} \, \partial u / \partial y & \tfrac{1}{3} \, \partial w / \partial x  \\
-\tfrac{1}{3} \, \partial u / \partial x &  \tfrac{2}{3} \, \partial v / \partial x & \tfrac{1}{3} \, \partial w / \partial x  \\
0  & -\tfrac{1}{3} \, \partial v / \partial x & 0 \\
\tfrac{1}{3} \, \partial u / \partial x &  - \tfrac{1}{3} \, \partial v / \partial x & - \tfrac{1}{3} \, \partial w / \partial x  \\
4 \, \partial v / \partial z &  - 2 \, \partial u / \partial z & - 2 \,  \partial v / \partial x  \\
- 4 \, \partial v / \partial z &  0 & 0  \\
- 2 \, \partial u / \partial y &  - 4 \, \partial u / \partial x & 0   \end{bmatrix} \, \begin{Bmatrix}
\partial u / \partial x\\
\partial v / \partial x \\
\partial w / \partial x
\end{Bmatrix}
= \tfrac{1}{2} \, [\mathbf{A}_{N1}] \, [\boldsymbol{\theta}_{N1}]
\end{aligned}
\end{equation}

\textcolor{red}{Now, the derivative of displacement with respect to spatial variable $x$\footnote{Recall that the slope vector corresponding to the first nonlinear term contains derivatives of displacements only with respect to the spatial variable $x$.} can be related to the nodal parameters via}
\begin{equation}
[\boldsymbol{\theta}_{N1}] =  \begin{Bmatrix}
\partial u / \partial x\\
\partial v / \partial x \\
\partial w / \partial x
\end{Bmatrix}
= \begin{Bmatrix}
\sum\nolimits_{i=1}^5 N_{i,x} \, u_i\\
\sum\nolimits_{i=1}^5 N_{i,x} \, v_i \\
\sum\nolimits_{i=1}^5 N_{i,x} \, w_i
\end{Bmatrix} 
= \begin{bmatrix}
[\mathbf{h}_{N1}], & [\mathbf{h}_{N2}], & [\mathbf{h}_{N3}], & [\mathbf{h}_{N4}]
\end{bmatrix}  \begin{Bmatrix} \boldsymbol{\Delta} \end{Bmatrix}  
= [\mathbf{H}_{N1}] \begin{Bmatrix} \boldsymbol{\Delta} \end{Bmatrix} 
\end{equation}
where 
\begin{equation}
\mathbf{h}_{Ni} = \begin{bmatrix}
N_{i,x} &  0 & 0  \\
0 & N_{i,x} & 0  \\
0 & 0 & N_{i,x}\end{bmatrix} 
\end{equation}
\textcolor{red}{Hence, the strain--displacement matrix $\mathbf{B}_{N1}$ corresponding to the first nonlinear strain becomes}
\begin{equation}
\mathbf{B}_{N1} = [\mathbf{A}_{N1}] \, [\mathbf{H}_{N1}] = \begin{bmatrix}
[\mathbf{b}_{N1}], & [\mathbf{b}_{N2}], & [\mathbf{b}_{N3}], & [\mathbf{b}_{N4}] 
\end{bmatrix} 
\end{equation}
wherein the components of $\mathbf{B}_{N1}$ are given as
\begin{equation}
\mathbf{b}_i = \begin{bmatrix}
-\tfrac{1}{3} \, \partial u / \partial x &  - \tfrac{2}{3} \, \partial u / \partial y & \tfrac{1}{3} \, \partial w / \partial x  \\
-\tfrac{1}{3} \, \partial u / \partial x &  \tfrac{2}{3} \, \partial v / \partial x & \tfrac{1}{3} \, \partial w / \partial x  \\
0  & -\tfrac{1}{3} \, \partial v / \partial x & 0 \\
\tfrac{1}{3} \, \partial u / \partial x &  - \tfrac{1}{3} \, \partial v / \partial x & - \tfrac{1}{3} \, \partial w / \partial x  \\
4 \, \partial v / \partial z &  - 2 \, \partial u / \partial z & - 2 \,  \partial v / \partial x  \\
- 4 \, \partial v / \partial z &  0 & 0  \\
- 2 \, \partial u / \partial y &   - 4 \, \partial u / \partial x & 0   \end{bmatrix}  \, \begin{bmatrix}
N_{i,x} &  0 & 0  \\
0 & N_{i,x} & 0  \\
0 & 0 & N_{i,x} \end{bmatrix}. 
\end{equation}

\textcolor{red}{In a similar manner, we can obtain the strain--displacement matrices corresponding to the other nonlinear strain components. The second nonlinear strain terms can be written in terms of $\mathbf{A}_N$ and the slope vector as} 
\begin{equation}
\begin{aligned}
\mathbf{E}_{N2} & =  \begin{Bmatrix}
\mathrm \xi_{L} \\
\mathrm \varepsilon_{1N} \\
\mathrm \varepsilon_{2N} \\
\mathrm \varepsilon_{3N} \\
\mathrm \gamma_{1N} \\
\mathrm \gamma_{2N} \\
\mathrm \gamma_{3N} \end{Bmatrix} =
\begin{Bmatrix}
\tfrac{1}{6} \,  \left( - v_{,y}^2 - w_{,y}^2  \right)\\
\tfrac{1}{6} \,  \left( 2 \, u_{,y} \, v_{,x} + 2 v_{,y}^2 -  w_{,y}^2 \right) \\
\tfrac{1}{6} \,  \left( - 2 \, u_{,y} \, v_{,x} - v_{,y}^2 + 3 \, w_{,y}^2 \right) \\
\tfrac{1}{6} \,  \left(-2 \, w_{,y}^2 \right) \\ 
- u_{,y} \, w_{,x} - 2 \, v_{,y} \, w_{,y} + 2 w_{,y} \, u_{,x} \\
u_{,y} \, u_{,z} - 2 \, u_{,x} \, w_{,y} \\ 
v_{,x} \, v_{,y}  + w_{,x} \, w_{,y} \end{Bmatrix} \\
& = \frac{1}{2} \, \begin{bmatrix}
0 &  - \tfrac{1}{3} \, \partial v / \partial y & \tfrac{1}{3} \, \partial w / \partial y  \\
\tfrac{2}{3} \, \partial v / \partial x &  \tfrac{2}{3} \, \partial v / \partial y & - \tfrac{1}{3} \, \partial w / \partial y  \\
-\tfrac{2}{3} \, \partial v / \partial x  & -\tfrac{1}{3} \, \partial v / \partial y & \partial w / \partial y \\
0 & 0 & - \tfrac{2}{3} \, \partial w / \partial y   \\
- 2 \, \partial w / \partial x &  - 4 \, \partial w / \partial y & 4 \,  \partial u / \partial x  \\
2 \, \partial u / \partial z &  0 & - 4 \, \partial u / \partial x  \\
0 &  2 \, \partial v / \partial x &  2 \, \partial w / \partial x  \end{bmatrix} \, \begin{Bmatrix}
\partial u / \partial y\\
\partial v / \partial y \\
\partial w / \partial y
\end{Bmatrix}
= \tfrac{1}{2} \, [\mathbf{A}_{N2}] \, [\boldsymbol{\theta}_{N2}]
\end{aligned}
\end{equation}
\textcolor{red}{The slope vector can further be expressed in terms of the nodal parameters via}
\begin{equation}
[\boldsymbol{\theta}_{N2}] =  \begin{Bmatrix}
\partial u / \partial y\\
\partial v / \partial y \\
\partial w / \partial y
\end{Bmatrix}
= \begin{Bmatrix}
\sum\nolimits_{i=1}^5 N_{i,y} \, u_i\\
\sum\nolimits_{i=1}^5 N_{i,y} \, v_i \\
\sum\nolimits_{i=1}^5 N_{i,y} \, w_i
\end{Bmatrix} 
= \begin{bmatrix}
[\mathbf{h}_{N1}], & [\mathbf{h}_{N2}], & [\mathbf{h}_{N3}], & [\mathbf{h}_{N4}] 
\end{bmatrix}  \begin{Bmatrix} \boldsymbol{\Delta} \end{Bmatrix}  
= [\mathbf{H}_{N2}] \begin{Bmatrix} \boldsymbol{\Delta} \end{Bmatrix} 
\end{equation}
where 
\begin{equation}
\mathbf{h}_{Ni} = \begin{bmatrix}
N_{i,y} &  0 & 0  \\
0 & N_{i,y} & 0  \\
0 & 0 & N_{i,y} \end{bmatrix} 
\end{equation}
\textcolor{red}{Hence, the strain--displacement matrix $\mathbf{B}_{N2}$ becomes}
\begin{equation}
\mathbf{B}_{N2} = [\mathbf{A}_{N2}] \, [\mathbf{H}_{N2}] = \begin{bmatrix}
[\mathbf{b}_{N1}], & [\mathbf{b}_{N2}], & [\mathbf{b}_{N3}], & [\mathbf{b}_{N4}]
\end{bmatrix}. 
\end{equation}
\textcolor{red}{The components of the strain--displacement matrix are given as}
\begin{equation}
\mathbf{b}_{Ni} = \begin{bmatrix}
0 &  - \tfrac{1}{3} \, \partial v / \partial y & \tfrac{1}{3} \, \partial w / \partial y  \\
\tfrac{2}{3} \, \partial v / \partial x &  \tfrac{2}{3} \, \partial v / \partial y & - \tfrac{1}{3} \, \partial w / \partial y  \\
-\tfrac{2}{3} \, \partial v / \partial x  & -\tfrac{1}{3} \, \partial v / \partial y & \partial w / \partial y \\
0 & 0 & - \tfrac{2}{3} \, \partial w / \partial y   \\
- 2 \, \partial w / \partial x &  - 4 \, \partial w / \partial y & 4 \,  \partial u / \partial x  \\
2 \, \partial u / \partial z &  0 & - 4 \, \partial u / \partial x  \\
0 &  2 \, \partial v / \partial x &  2 \, \partial w / \partial x  \end{bmatrix} \, \begin{bmatrix}
N_{i,y} &  0 & 0  \\
0 & N_{i,y} & 0  \\
0 & 0 & N_{i,y}  \end{bmatrix}. 
\end{equation}

\textcolor{red}{For the third nonlinear strain term, $\mathbf{E}_{N3}$ can be written as} 
\begin{equation}
\begin{aligned}
\mathbf{E}_{N3} & =  \begin{Bmatrix}
\mathrm \xi_{L} \\
\mathrm \varepsilon_{1N} \\
\mathrm \varepsilon_{2N} \\
\mathrm \varepsilon_{3N} \\
\mathrm \gamma_{1N} \\
\mathrm \gamma_{2N} \\
\mathrm \gamma_{3N} \end{Bmatrix} =
\begin{Bmatrix}
\tfrac{1}{6} \,  \left( u_{,z}^2 - v_{,z}^2 - 4 \, v_{,z} \, w_{,y} - w_{,z}^2   \right)\\
0 \\
\tfrac{1}{6} \,  \left( - u_{,z}^2 + v_{,z}^2 + 4 \, v_{,z} \, w_{,y} + w_{,z}^2  \right) \\
\tfrac{1}{6} \,  \left( u_{,z}^2 - v_{,z}^2 - w_{,z}^2 - 4 \, v_{,z} \, w_{,y}\right) \\ 
- v_{,y} \, v_{,z} + w_{,y} \, w_{,z} \\
v_{,y} \, v_{,z} + w_{,y} \, w_{,z} \\ 
0 \end{Bmatrix} \\
& = \frac{1}{2} \, \begin{bmatrix}
\tfrac{1}{3} \, \partial u / \partial z &  - \tfrac{1}{3} \, \partial v / \partial z - \tfrac{4}{3} \, \partial w / \partial y & - \tfrac{1}{3} \, \partial w / \partial z   \\
0 &  0 & 0  \\
- \tfrac{1}{3} \, \partial u / \partial z &  \tfrac{1}{3} \, \partial v / \partial z + \tfrac{4}{3} \, \partial w / \partial y & \tfrac{1}{3} \, \partial w / \partial z \\
\tfrac{1}{3} \, \partial u / \partial z &  - \tfrac{1}{3} \, \partial v / \partial z - \tfrac{4}{3} \, \partial w / \partial y & - \tfrac{1}{3} \, \partial w / \partial z \\
0 &  -2 \, \partial v / \partial y & 2\, \partial w/\partial y \, \partial w / \partial z \\
0 &  2 \, \partial v / \partial y & 2 \, \partial w / \partial y  \\
0 &  0 &  0  \end{bmatrix} \, \begin{Bmatrix}
\partial u / \partial z\\
\partial v / \partial z \\
\partial w / \partial z
\end{Bmatrix}
= \tfrac{1}{2} \, [\mathbf{A}_{N3}] \, [\boldsymbol{\theta}_{N3}]
\end{aligned}
\end{equation}
\textcolor{red}{Here the slope vector $\boldsymbol{\theta}_{N3}$ contains the derivatives of displacements with respect to the spatial variable $z$ and is related to the nodal parameters via}
\begin{equation}
[\boldsymbol{\theta}_{N3}] =  \begin{Bmatrix}
\partial u / \partial z\\
\partial v / \partial z \\
\partial w / \partial z
\end{Bmatrix}
= \begin{Bmatrix}
\sum\nolimits_{i=1}^5 N_{i,z} \, u_i\\
\sum\nolimits_{i=1}^5 N_{i,z} \, v_i \\
\sum\nolimits_{i=1}^5 N_{i,z} \, w_i
\end{Bmatrix} 
= \begin{bmatrix}
[\mathbf{h}_{N1}], & [\mathbf{h}_{N2}], & [\mathbf{h}_{N3}], & [\mathbf{h}_{N4}] 
\end{bmatrix}  \begin{Bmatrix} \boldsymbol{\Delta} \end{Bmatrix}  
= [\mathbf{H}_{N3}] \begin{Bmatrix} \boldsymbol{\Delta} \end{Bmatrix} 
\end{equation}
where 
\begin{equation}
\mathbf{h}_{Ni} = \begin{bmatrix}
N_{i,z} &  0 & 0  \\
0 & N_{i,z} & 0  \\
0 & 0 & N_{i,z} \end{bmatrix}. 
\end{equation}
\textcolor{red}{Therefore, the strain--displacement matrix $\mathbf{B}_{N3}$ becomes}
\begin{equation}
\mathbf{B}_{N3} = [\mathbf{A}_{N3}] \, [\mathbf{H}_{N3}] = \begin{bmatrix}
[\mathbf{b}_{N1}], & [\mathbf{b}_{N2}], & [\mathbf{b}_{N3}], & [\mathbf{b}_{N4}]
\end{bmatrix} 
\end{equation}
\textcolor{red}{whose components are given as}
\begin{equation}
\mathbf{b}_{Ni} = \begin{bmatrix}
\tfrac{1}{3} \, \partial u / \partial z &  - \tfrac{1}{3} \, \partial v / \partial z - \tfrac{4}{3} \, \partial w / \partial y & - \tfrac{1}{3} \, \partial w / \partial z   \\
0 &  0 & 0  \\
- \tfrac{1}{3} \, \partial u / \partial z &  \tfrac{1}{3} \, \partial v / \partial z + \tfrac{4}{3} \, \partial w / \partial y & \tfrac{1}{3} \, \partial w / \partial z \\
\tfrac{1}{3} \, \partial u / \partial z &  - \tfrac{1}{3} \, \partial v / \partial z - \tfrac{4}{3} \, \partial w / \partial y & - \tfrac{1}{3} \, \partial w / \partial z \\
0 &  -2 \, \partial v / \partial y & 2\, \partial w/\partial y  \\
0 &  2 \, \partial v / \partial y & 2 \, \partial w / \partial y  \\
0 &  0 &  0 \end{bmatrix} \, \begin{bmatrix}
N_{i,z} &  0 & 0  \\
0 & N_{i,z} & 0  \\
0 & 0 & N_{i,z}  \end{bmatrix}.
\end{equation}
\textcolor{red}{The total nonlinear strain--displacement matrix $[\overline{\mathbf{B}}]$ can be obtained as a sum of its three components $\mathbf{B}_{Ni}, i=1, 2, 3$.}

\textcolor{red}{With all the strain--displacement matrices evaluated, now we are able to find the total stiffness matrix.}

\textcolor{red}{First, the stress stiffness matrix for a 3-D alveolar volume can be evaluated as~(cf.~\S~\ref{secStiffnessMatrices})}
\begin{equation}
\mathbf{K}_\mathbf{S} = \int_{V}  \mathrm{d} \mathbf{B}_N^{\mathsf{T}} \, \mathbf{T} \, |\mathbf{J}|  \, \mathrm{d} \xi \,  \mathrm{d} \eta \, \mathrm{d} \zeta.
\label{stressstifvol}
\end{equation}
\textcolor{red}{Now by taking the variation of Eq.~(\ref{straindis}) and substituting the definition of the components of nonlinear strain--displacement matrices $\mathbf{B}_{Ni}$, the total nonlinear strain displacement matrix can be written as}
\begin{equation}
\mathrm{d}[\mathbf{B}_N]^{\mathsf{T}} = [\mathbf{H}_1^{\mathsf{T}}] \mathrm{d} [\mathbf{A}_1^{\mathsf{T}}] + [\mathbf{H}_2^{\mathsf{T}}] \mathrm{d} [\mathbf{A}_2^{\mathsf{T}}] + [\mathbf{H}_3^{\mathsf{T}}] \mathrm{d} [\mathbf{A}_3^{\mathsf{T}}].
\end{equation}
which on substituting into Eq.~(\ref{stressstifvol}) gives 
\begin{equation}
\mathbf{K_\mathbf{S}} \mathrm{d} \boldsymbol{\Delta} = \int_A \left(
[\mathbf{H}_1^{\mathsf{T}}] \mathrm{d} [\mathbf{A}_1^{\mathsf{T}}] + [\mathbf{H}_2^{\mathsf{T}}] \mathrm{d} [\mathbf{A}_2^{\mathsf{T}}] + [\mathbf{H}_3^{\mathsf{T}}] \mathrm{d} [\mathbf{A}_3^{\mathsf{T}}] \right) \, \begin{Bmatrix} s^{\pi} \\ s^{\sigma_1} \\ s^{\sigma_2}  \\ s^{\tau_1} \\ s^{\tau_2} \\ s^{\tau_3} \end{Bmatrix} \, \mathrm{d} A.
\label{eq:Ks_3d}
\end{equation}
\textcolor{red}{However, using the definition of the matrix $\mathbf{A}$, one can easily obtain} 
\begin{equation}
\begin{aligned}
& \mathrm{d} [\mathbf{A}_1^{\mathsf{T}}] \, \begin{Bmatrix} s^{\pi} \\ s^{\sigma_1} \\ s^{\sigma_2} \\ s^{\tau_1} \\ s^{\tau_2} \\ s^{\tau_3}  \end{Bmatrix} = 
\begin{bmatrix}s^{\pi} & s^{\tau_1} & s^{\tau_3}\\ 
s^{\tau_1} & s^{\sigma_1} & s^{\tau_2} \\ 
s^{\tau_3} & s^{\tau_2} & s^{\sigma_2} \end{bmatrix} \, [\mathbf{H}_1] \, \mathrm{d} \boldsymbol{\Delta} \\
& \mathrm{d} [\mathbf{A}_2^{\mathsf{T}}] \, \begin{Bmatrix}s^{\pi} \\ s^{\sigma_1} \\ s^{\sigma_2}  \\ s^{\tau_1} \\ s^{\tau_2} \\ s^{\tau_3}  \end{Bmatrix} = 
\begin{bmatrix} s^{\pi} & s^{\tau_1} & s^{\tau_3}\\ 
s^{\tau_1} & s^{\sigma_1} & s^{\tau_2} \\ 
s^{\tau_3} & s^{\tau_2} & s^{\sigma_2}  \end{bmatrix} \, [\mathbf{H}_2] \, \mathrm{d} \boldsymbol{\Delta}\\
& \mathrm{d} [\mathbf{A}_3^{\mathsf{T}}] \, \begin{Bmatrix}s^{\pi} \\ s^{\sigma_1} \\ s^{\sigma_2}  \\ s^{\tau_1} \\ s^{\tau_2} \\ s^{\tau_3}  \end{Bmatrix} = 
\begin{bmatrix} s^{\pi} & s^{\tau_1} & s^{\tau_3}\\ 
s^{\tau_1} & s^{\sigma_1} & s^{\tau_2} \\ 
s^{\tau_3} & s^{\tau_2} & s^{\sigma_3}  \end{bmatrix} \, [\mathbf{H}_3] \, \mathrm{d} \boldsymbol{\Delta}
\end{aligned}.
\end{equation}
\textcolor{red}{Substituting these properties into eqn.~\eqref{eq:Ks_3d}, the stiffness matrix due to initial stress takes on the form}
\begin{equation}
\begin{aligned}
{\mathbf{K_{\mathbf{S}}}}\,\mathrm{d}\boldsymbol{\Delta}=\int_A \:\left([\mathbf{H}^\mathsf{T}_1]\,\begin{bmatrix}s^{\pi} & s^{\tau_1} & s^{\tau_3}\\ 
s^{\tau_1} & s^{\sigma_1} & s^{\tau_2} \\ 
s^{\tau_3} & s^{\tau_2} & s^{\sigma_2} \end{bmatrix} \, [\mathbf{H}_1]+ 
[\mathbf{H}^\mathsf{T}_2]\,\begin{bmatrix}s^{\pi} & s^{\tau_1} & s^{\tau_3}\\ 
s^{\tau_1} & s^{\sigma_1} & s^{\tau_2} \\ 
s^{\tau_3} & s^{\tau_2} & s^{\sigma_2} \end{bmatrix} \, [\mathbf{H}_2]+
[\mathbf{H}^\mathsf{T}_3]\,\begin{bmatrix}s^{\pi} & s^{\tau_1} & s^{\tau_3}\\ 
s^{\tau_1} & s^{\sigma_1} & s^{\tau_2} \\ 
s^{\tau_3} & s^{\tau_2} & s^{\sigma_2} \end{bmatrix} \, [\mathbf{H}_3]\right)\:\mathrm{d}\boldsymbol{\Delta}
\end{aligned}.
\end{equation}
\textcolor{red}{Using numerical integration techniques, we finally obtain the stiffness matrix associated with the initial stress as} 
\begin{equation}
\begin{aligned}
\mathbf{K_{\mathbf{S}}} & = \sum_{i=1}^{n}  \sum_{j=1}^{n}  \sum_{k=1}^{n} [\mathbf{H}_1]^{\mathsf{T}}  \begin{bmatrix} s^{\pi} & s^{\tau_1} & s^{\tau_3}\\ 
s^{\tau_1} & s^{\sigma_1} & s^{\tau_2} \\ 
s^{\tau_3} & s^{\tau_2} & s^{\sigma_2}
\end{bmatrix} \, [\mathbf{H}_1] \, |\mathbf{J}|  \mathrm{w}_i \mathrm{w}_j \mathrm{w}_k \\
& + \sum_{i=1}^{n}  \sum_{j=1}^{n}  \sum_{k=1}^{n} [\mathbf{H}_2]^{\mathsf{T}}  \begin{bmatrix} s^{\pi} & s^{\tau_1} & s^{\tau_3}\\ 
s^{\tau_1} & s^{\sigma_1} & s^{\tau_2} \\ 
s^{\tau_3} & s^{\tau_2} & s^{\sigma_2}
\end{bmatrix} \, [\mathbf{H}_2] \, |\mathbf{J}|  \, \mathrm{w}_i \mathrm{w}_j \mathrm{w}_k \\
& + \sum_{i=1}^{n}  \sum_{j=1}^{n}  \sum_{k=1}^{n} [\mathbf{H}_3]^{\mathsf{T}}  \begin{bmatrix} s^{\pi} & s^{\tau_1} & s^{\tau_3}\\ 
s^{\tau_1} & s^{\sigma_1} & s^{\tau_2} \\ 
s^{\tau_3} & s^{\tau_2} & s^{\sigma_2}
\end{bmatrix} \, [\mathbf{H}_3] \, |\mathbf{J}|  \, \mathrm{w}_i \mathrm{w}_j \mathrm{w}_k
\end{aligned}
\end{equation}
\textcolor{red}{where $n$ denotes the number of Gauss points as mentioned in  tab.~\ref{tabQuadraturetetra}.}

Since the linear strain--displacement matrix has been evaluated, now it is possible to compute the  stiffness matrix corresponding to small displacement as 
\begin{equation}
\mathbf{K}_{L} = \int_{V} \mathbf{B}_L^{\mathsf{T}} \, \mathbf{M} \, \mathbf{B}_L \, |\mathbf{J}| \,  \mathrm{d} \xi \,  \mathrm{d} \eta \,  \mathrm{d} \zeta =  \sum_{i=1}^{n}  \sum_{j=1}^{n}  \sum_{k=1}^{n}  \mathbf{B}_L^{\mathsf{T}} \, \mathbf{M} \, \mathbf{B}_L \, |\mathbf{J}|  \, \mathrm{w}_i \mathrm{w}_j \mathrm{w}_k.
\end{equation}

The large/nonlinear displacement stiffness matrix for the tetrahedron can be numerically evaluated as
\begin{equation}
\mathbf{K}_{N} = \int_{V} \mathbf{D} \, |\mathbf{J}|\, \mathrm{d} \xi \, \mathrm{d} \eta \, \mathrm{d} \zeta
= \sum_{i=1}^{n}  \sum_{j=1}^{n}  \sum_{k=1}^{n}  \mathbf{D} \, |\mathbf{J}| \, \mathrm{w}_i \mathrm{w}_j \mathrm{w}_k
\end{equation}
where $\mathbf{D}$ has the form of 
\begin{equation}
\mathbf{D} = \mathbf{B}_L^{\mathsf{T}} \, \mathbf{M} \, \mathbf{B}_N \, + \mathbf{B}_N^{\mathsf{T}} \, \mathbf{M} \, \mathbf{B}_L \, + \mathbf{B}_N^{\mathsf{T}} \, \mathbf{M} \, \mathbf{B}_N.
\end{equation}

\section{Force Vector}
%In the lung structure, the lung tissue(elastin and collagen) and the alveoli are two sources of force trying to collapse the lung. These inward forces are called the forces of recoil.  Inside of the alveoli, there is a fluid that makes the surface moist(a membrane of fluid is sticking to the surface) that is making an interface with the air, and this air-fluid interface also causes pressure inwards which try to collapse the alveoli called tension. The lung tissue is a third, and the fluid surface tension is $2/3$ of the total recoil. 

%The expansile force is the transpulmonary force, which is the difference between the fluid in the pleural cavity that is called pleural pressure ($-5 cm H_2O$ at rest) and the pressure inside the alveoli that is called alveolar pressure ($0 cm H_2O$ at rest). 

%If the alveolus is filled with air along with a pure water layer(air-fluid interface), the water is going to try to contract and pull the alveolus inverse. That is the tension created by the fluid-air interface. 
%Since the alveolus is squeezing, the air inside it develops pressure on the alveolus.
%This pressure and the recoiling forces and the expansile forces have to become equal for the alveolus to stay stable. 

The principle of stationary potential energy with the Rayleigh-Ritz approach, i.e., Eqn.~(\ref{principle}), determines the basis of finite element stress analysis. The internal strain energy is balanced with the potential energy of applied internal and external loads on the body.

The virtual work done by external forces $\delta{W}$ in Eq.~(\ref{principle}) can be expressed as
\begin{equation}
\begin{aligned}
\delta{W} = \int_{S} \, \mathbf{t} \, \delta \mathbf{u} \, \mathrm{d} S
= \int_{S} \, \mathbf{t} \, \mathbf{N} \, \mathrm{d} \boldsymbol{\Delta} \, \mathrm{d} S
= \left( \int_{S} \, \mathbf{N}^{\mathsf{T}} \, \mathbf{t} \, \mathrm{d} S \right) \, \mathrm{d} \boldsymbol{\Delta}
\end{aligned}
\end{equation}
where $\mathrm{d} S$ denotes the surface element and $\mathbf{t}$ is the surface traction vector (per unit surface area) at current time. Hence, the external FE force vectors are
\begin{equation}
\mathbf{F} = \int_{S} \, \mathbf{N}^{\mathsf{T}} \, \mathbf{t} \, \mathrm{d} S 
\end{equation}
where $ \mathbf{F} $ is the deformation gradient.


\subsection{Force Vector for a Chord}

Following the procedure described above,the force vector of the 1-D alveolar chord is evaluated numerically in its natural co-ordinate system as
\begin{equation}
\mathbf{F}_{1D} = \int_{\Gamma} \mathbf{N}^{\mathsf{T}} \, \mathbf{t} \, \mathrm{d} x  = \int_{-1}^{1} \mathbf{N}^{\mathsf{T}}\, \mathbf{t} \, \mathbf{J}\,  \mathrm{d} \xi =  \sum_{i=1}^{n} \mathbf{N}^{\mathsf{T}} \, \mathbf{t} \, \mathbf{J} \, \mathrm{w}_i
\end{equation}
where $\mathrm{w}_i$ are the  weighting coefficients of the Gauss integration rule, $\mathbf{N}$ is the shape function matrix for chord, and $\mathbf{t}$ is the traction on the septal chord that is selected so that the traction can be additively decomposed into that carried by the collagen and elastin fibers, i.e., $\mathbf{t} = \mathbf{t}^c + \mathbf{t}^e $ as described in part 4.

Table~\ref{tabQuadrature1D} provides the values of $\xi$ and $\mathrm{w}_i$ for $n = 1, 2$, and $3$ Gauss integration points.

\subsection{Force Vector for a Pentagon}
\textcolor{red}{The boundary of a two dimensional pentagon consists of line segments, which can be considered as one-dimensional chords. Hence, the evaluation of the boundary integrals on pentagon amounts to evaluating the line integrals on these boundary lines. Once the interpolation function for a pentagon are evaluated on the boundary of pentagon, we can obtain the corresponding chordal interpolation functions \cite{Reddy93}.
	Thus, the force vector $\mathbf{F}_{2D}$  for a pentagon can be obtained by integrating the traction vectors over all sides of pentagon. Specifically, the force on the boundary of the membrane can be obtained as}
\begin{equation}
\begin{aligned}
\mathbf{F}_{2D}  = \oint_{\Gamma} \mathbf{N}^{\mathsf{T}} \, \mathbf{t} \, \mathrm{d} S & = \int_{\Gamma_{12}} \mathbf{N}^{\mathsf{T}} \, \mathbf{t}_{12} \,|\mathbf{J}| \, \mathrm{d} \xi + \int_{\Gamma_{23}} \mathbf{N}^{\mathsf{T}} \, \mathbf{t}_{23} \,|\mathbf{J}| \, \mathrm{d} \xi + \int_{\Gamma_{34}} \mathbf{N}^{\mathsf{T}} \, \mathbf{t}_{34} \,|\mathbf{J}| \, \mathrm{d} \xi \\
& + \int_{\Gamma_{45}} \mathbf{N}^{\mathsf{T}} \, \mathbf{t}_{45} \, \,|\mathbf{J}| \, \mathrm{d} \xi + \int_{\Gamma_{51}} \mathbf{N}^{\mathsf{T}} \, \mathbf{t}_{51} \,|\mathbf{J}| \, \mathrm{d} \xi \\
& = \sum_{i=1}^{n} \mathbf{N}^{\mathsf{T}} \, \mathbf{t}_{12} \,|\mathbf{J}| \, \mathrm{w}_i + \sum_{i=1}^{n} \mathbf{N}^{\mathsf{T}} \, \mathbf{t}_{23} \,|\mathbf{J}| \, \mathrm{w}_i + \sum_{i=1}^{n} \mathbf{N}^{\mathsf{T}} \, \mathbf{t}_{34} \,|\mathbf{J}| \, \mathrm{w}_i \\
& + \sum_{i=1}^{n} \mathbf{N}^{\mathsf{T}} \, \mathbf{t}_{45} \, \,|\mathbf{J}| \, \mathrm{w}_i + \sum_{i=1}^{n} \mathbf{N}^{\mathsf{T}} \, \mathbf{t}_{51} \,|\mathbf{J}| \, \mathrm{w}_i
\end{aligned}
\end{equation}
where $\mathbf{N}$ represents the shape function matrix of a chord, but with the matrix dimension of a pentagon, $|\mathbf{J}|$ is the determinant of the Jacobian for a 1-D chord, $\mathrm{w}_i$ denotes the natural weight of the chord, $\mathrm{d} S$ is the arc-length of an infinitesimal line element along the boundary, and $\mathbf{t}$ is the traction vector on each edge of the pentagon defined as
\begin{equation}
\mathbf{t} = \boldsymbol{\sigma}^{\mathsf{T}} \cdot \mathbf{n} 
\end{equation}
where $ \mathbf{n} $ is the normal vector to each sides of pentagon on which the traction acts and $ \boldsymbol{\sigma} $ is the Cauchy stress as established in part 4.

\subsection{Force Vector for a Tetrahedron}

\textcolor{red}{The force vector on the alveolar volume is computed by integrating the traction vector over the four boundary surfaces of the tetrahedron. Here the matrix of shape functions, given in Eq.~(\ref{shape3D}), is used to obtain the force vector for tetrahedron. The force vector $\mathbf{F}_{3D}$ can be specified as}
\begin{equation}
\begin{aligned}
\mathbf{F}_{3D} &= \oint_{A} \mathbf{N}^{\mathsf{T}} \, \mathbf{t} \, \mathrm{d} A = \int_{\triangle_{1}} \int_{\triangle_{1}} \mathbf{N}^{\mathsf{T}} \, \mathbf{t}_{\triangle_{1}} \,|\mathbf{J}| \, \mathrm{d} \xi \, \mathrm{d} \eta  + \int_{\triangle_{2}} \int_{\triangle_{2}} \mathbf{N}^{\mathsf{T}} \, \mathbf{t}_{\triangle_{2}} \,|\mathbf{J}| \, \mathrm{d} \xi \, \mathrm{d} \eta \\
& + \int_{\triangle_{3}} \int_{\triangle_{3}} \mathbf{N}^{\mathsf{T}} \, \mathbf{t}_{\triangle_{3}} \,|\mathbf{J}| \, \mathrm{d} \xi \, \mathrm{d} \eta + \int_{\triangle_{4}} \int_{\triangle_{4}} \mathbf{N}^{\mathsf{T}} \, \mathbf{t}_{\triangle_{4}} \,|\mathbf{J}| \, \mathrm{d} \xi \, \mathrm{d} \eta \\
& = \sum_{i=1}^{n} \sum_{j=1}^{n} \mathbf{N}^{\mathsf{T}} \, \mathbf{t}_{\triangle_{1}} \,|\mathbf{J}| \, \mathrm{w}_i \mathrm{w}_j + \sum_{i=1}^{n} \sum_{j=1}^{n} \mathbf{N}^{\mathsf{T}} \, \mathbf{t}_{\triangle_{2}} \,|\mathbf{J}| \, \mathrm{w}_i \mathrm{w}_j \\
& + \sum_{i=1}^{n} \sum_{j=1}^{n} \mathbf{N}^{\mathsf{T}} \, \mathbf{t}_{\triangle_{3}} \,|\mathbf{J}| \, \mathrm{w}_i \mathrm{w}_j + \sum_{i=1}^{n} \sum_{j=1}^{n} \mathbf{N}^{\mathsf{T}} \, \mathbf{t}_{\triangle_{4}} \,|\mathbf{J}| \, \mathrm{w}_i \mathrm{w}_j
\end{aligned}
\end{equation}
\textcolor{red}{where $\Delta_i, i=1, 2, 3, 4$ represents the triangular boundary surface of the tetrahedron. Here $ \mathbf{N}$ represents the shape function matrix for these associated triangular boundaries.   $|\mathbf{J}|$ is the determinant of the Jacobian for triangle, $n$ denotes the number of Gauss points, $\mathrm{w}_i$ and $\mathrm{w}_j$ are the natural weight of the triangle from Table~\ref{tabGaussPointsTriangle}, and $\mathbf{t}$ is the surface traction on the triangle surface. $\oiint$ denotes the integral over the surface boundary of the tetrahedron.} 

\textcolor{red}{Note that except for the base of the tetrahedron, the tractions on its other sides have equal and opposite contributions to the total force vector. Therefore, contributions from opposite boundary surfaces of the tetrahedron nullify each other. Hence, in order to obtain the total force vector for a tetrahedron, it is sufficient to only consider the contributions due to the tractions on its base. Therefore, the total force vector takes the form of}

\begin{equation}
\mathbf{F}_{3D} = \sum_{i=1}^{n} \sum_{j=1}^{n} \mathbf{N}^{\mathsf{T}} \, \mathbf{t}_{\triangle_{1}} \,|\mathbf{J}| \, \mathrm{w}_i \mathrm{w}_j 
\end{equation}



\begin{table}
	\centering
	\begin{tabular}{|c|rrr|}
		\hline
		node & $\xi$ co-ordinate   & 
		$\eta$ co-ordinate & weight \\   \hline        
		& \multicolumn{3}{|c|}{Exact for Polynomials of Degree $1^{\phantom{|^|}}$} \\ 
		\hline
		1 & 1/3 & 1/3 & 1.0 \\ 
		\hline
		& \multicolumn{3}{|c|}{Exact for Polynomials of Degree $2^{\phantom{|^|}}$} \\ 
		\hline
		1 & 2/3 & 1/6 & 1/3\\
		2 & 1/6 & 1/6 & 1/3\\
		3 & 1/6 & 2/3 & 1/3\\ 
		\hline
		& \multicolumn{3}{|c|}{Exact for Polynomials of Degree $3^{\phantom{|^|}}$} \\ \hline
		1 & 1/3 & 1/3 & -27/48 \\
		2 & 3/5 & 1/5 & 25/48 \\
		3 & 1/5 & 1/5 & 25/48 \\ 
		4 & 1/5 & 3/5 & 25/48 \\
		\hline
	\end{tabular}
	\caption{Generalized, Gaussian, weights and nodes for integrating over a triangle in its natural co-ordinate system.}
	\label{tabGaussPointsTriangle}
\end{table}

