\documentclass{arlticle}
\usepackage{verbatimbox}
\def\arlticle{\textsf{\small arlticle}}
\def\boxhandler{\textsf{\small ARLboxhandler}}
\def\subfig{\textsf{\footnotesize subfig}}
\def\subcaption{\textsf{\footnotesize subcaption}}
\begin{document}
\baresection{\large Common Mistakes in Formatting Your Technical Reports Using
  the \textsf{arlticle} Document Class}
Here are tips that authors should keep in mind when using the \arlticle{}
  document class.
ARL TechPubs has noted that the following formatting mistakes arise again and
  again among the ARL user base.
It would make your life and their life easier if you reminded yourself of these
  guidelines when formulating your document draft.

\section{No Report Title Hyphenation}
Left to itself, hyphenation will naturally occur in long report titles, on both the 
  cover and the title page.
The standard is to avoid hyphenation here and page 3 of the \arlticle {} documentation
  tells you how.
Manually introduce \verb|\\|\, line breaks, as needed, when specifying the
  argument to \verb|\arltitle|.\,
The document class is smart enough to remove these manual breaks when regurgitating 
  the title as item~4 on the SF-298.

\section{``List of Symbols, Acronyms, and Abbreviations''}

\subsubsection*{Location in Document}
This is the official name of the list containing such things, and \textit{it is
  to appear immediately before the Distribution List} at the \textit{end} of the
  report, not in the report's front matter.
The proper invocation to provide such an \textit{unnumbered} section, as
  described on pp.~9,\,11 of the \arlticle{} documentation, is 

\begin{verbbox}[\footnotesize]
\baresection{List of Symbols, Acronyms, and Abbreviations}
\end{verbbox}
\theverbbox

\subsubsection*{Format}
The \arlticle{} class does not provide a list making environment for creating
  such a list, so use the standard \LaTeX{} list-making tools as you see fit.
However, note that all such entries in the list should be provided in medium-%
  weight, roman, upright-shape font (no bold, italic, etc.)

\section{Figure and Table Caption Format}

\subsubsection*{Case}

Table and figure captions are to be in \textit{sentence case}, meaning only the
  first letter of caption sentences should be capitalized.  
Exceptions, to also be capitalized, would include proper nouns.

\subsubsection*{Periods}

Captions do not end in periods; however, periods are used 
  when the caption requires more than one sentence.

\subsubsection*{Text Format}

Captions should be bold, centered, and in \verb|\footnotesize|.
The \arlticle{} document class provides the tools to automatically
  get things right, \textit{if you use them}.
In particular the class provides, as described on pp.~9,\,11 of the documentation,
  the \verb|ARLfigure| and \verb|ARLtable| environments, that properly preset
  the formatting of captions and figure/table font sizes/formats.%
\begin{verbbox}[\footnotesize]
\captionsetup{font={bf,footnotesize}}
\end{verbbox}
\footnote{ARL user James Ramsey noted that use of the \subfig{} package can 
  interfere with this caption style.  The solution is to invoke \theverbbox{} 
  in the preamble.  Alternately, consider using the \subcaption{} package.}
Alternately, if you use the \boxhandler{} package, with its \verb|\bxfigure|
  and \verb|\bxtable| macros, pp.~8--11 of the \arlticle{} documentation shows
  how that will likewise produce correctly formatted captions.

\section{List of Figures/Tables Caption Indent}

The Lists of Figures and Tables are set up to provide proper indent of the caption
  when the number of figures/tables is 99 or less \textit{and} when no 
  figures/tables appear in appendices.
If either of these conditions is violated, extra left margin needs to be added to
  the caption indent of theses lists.

\begin{sloppypar}
The ARL report stencil provides the placeholders, which are the lengths
\verb|\cftfignumwidth| and \verb|\cfttabnumwidth|.
In particular, those placeholders are
\end{sloppypar}
\begin{verbbox}[\footnotesize]
  \addtolength{\cftfignumwidth}{0.0ex}
  \addtolength{\cfttabnumwidth}{0.0ex}
\end{verbbox}
\theverbbox\\
which can be modified to increase the space allocated for the figure and table
  numbers in the lists.
A supplement of \texttt{1ex} should be sufficient.

\section{Up to Date Mandatory Distribution List(s)}

We all know that the lab (and some directorates) require mandatory distribution
  lists to be included at the end of a report, in addition to the distribution
  desired by the author.
The stencil has been set up so that these lists are stored externally in
  separate files and \verb|\input| into your document.
The author is responsible for pointing to these \texttt{dls} files at the
  beginning of the report stencil, in the manner of

\begin{verbbox}[\footnotesize]
  \def\MandatoryDL{ARL-02-01-19(Unlimited).dls}
  \def\LocalMandatoryDL{blank.dls}
  \def\UserDL{ARLstencil.dls}
\end{verbbox}
\theverbbox\\
Note here that \texttt{blank.dls} applies when there is no directorate-specific
  distribution to be applied. 
Also, the \texttt{Unlimited} version of the mandatory ARL list applies to
  distribution A reports; there is a comparable version for limited-distribution
  documents.

The issue that occasionally arises is that the authors are pointing to 
  out-of-date mandatory lists.
These lists are occasionally updated, and the authors must ensure that
  their mandatory list definitions point to the current versions of these
  lists.
The most up-to-date list files are provided in Steven's \LaTeX{} share drive
  location.

\hfill -Steven B Segletes (\today)
\end{document}