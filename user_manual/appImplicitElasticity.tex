\appendix{Implicit Elasticity}
\label{appImplicitElasticity}

Both explicit (i.e., Green \cite{Green41} elastic) and implicit (i.e., Rajagopal \cite{Rajagopal03} elastic) material models are put forward in this appendix for one's consideration when choosing a material model to represent biologic fibers and membranes.  We discuss thermo\-elastic fibers first, and then thermo\-elastic membranes.  We have no need to address thermo\-elastic volumes for our application beyond what has been presented in \S\ref{sec:IdealGasLaw}.  We employ Gibbs free-energy potential $\mathcal{G}$ instead of the internal-energy potential $\mathcal{U}$.  These potentials relate to one another through a well-known Legendre transformation.  A Gibbs energy approach implies that a change in the intensive variables (thermo\-dynamic forces) will cause a response in the extensive variables (thermo\-dynamic displacements), which is the exact opposite of an internal energy approach.  Cause and effect are correct in a Gibbs formulation.

\subsection{Alveolar Chords as Green (Explicit) Thermoelastic Fibers}

For a 1D fiber with a mass density of $\rho$ per unit length, the conjugate fields are: temperature $\theta$ and entropy $\eta$, plus force $F$ and strain $e \defeq \ln (L/L_0)$ with $L_0$ and $L$ denoting the initial and current fiber lengths.  A Green elastic fiber adopts a Gibbs free-energy function with an explicit dependence upon state, viz., $\mathcal{G} (\theta , F)$ such that $\mathrm{d} \mathcal{G} = -\eta \, \mathrm{d} \theta - \tfrac{1}{\rho} e \, \mathrm{d}F$ (cf.~Eqn.~\ref{thermoelastic1Dlaw}), from which one derives its governing constitutive equations, they being
\begin{subequations}
    \label{fiberConstitutiveTheory}
    \begin{align}
\eta & = -\partial_{\theta\,} \mathcal{G} (\theta , F)
\quad \text{and} \quad
e = -\rho \, \partial_{F\,} \mathcal{G} (\theta, F)
\label{entropyStrain1D} \\
\intertext{that, when differentiated, can be rearranged into the following hypo-elastic equation}
\left\{ \begin{matrix}
\mathrm{d} \eta \\ \mathrm{d} e 
\end{matrix} \right\} & = -\begin{bmatrix}
\partial_{\theta\theta\,} \mathcal{G} & \partial_{\theta F\,} \mathcal{G} \\
\rho \, \partial_{F\theta\,} \mathcal{G} & \rho \, \partial_{FF\,} \mathcal{G}
\end{bmatrix} 
\left\{ \begin{matrix}
\mathrm{d} \theta \\ \mathrm{d} F
\end{matrix} \right\}
= \begin{bmatrix}
C / \theta & \alpha / \rho \\
\alpha & 1 / E
\end{bmatrix}
\left\{ \begin{matrix}
\mathrm{d} \theta \\ \mathrm{d} F
\end{matrix} \right\}
\label{thermoelasticCE1D} \\
    \intertext{whose thermo-physical material properties are, in general, functions of state with}
    C & \defeq \theta \, \partial_{\theta\,} \eta |_F = 
    -\theta \, \partial_{\theta\theta\,} \mathcal{G} (\theta , F)
    \label{specificHeat} \\
    \alpha & \defeq L^{-1} \, \partial_{\theta\,} L |_F = \partial_{\theta\,} e |_F =
    -\rho \, \partial_{F\theta\,} \mathcal{G} (\theta , F) \equiv
    -\rho \, \partial_{\theta F\,} \mathcal{G} (\theta , F)
    \label{thermalExpansion} \\
    1 / E & \defeq L^{-1} \, \partial_{F\,} L |_{\theta} = \partial_{F\,} e |_{\theta} =
    -\rho \, \partial_{FF\,} \mathcal{G} (\theta , F)
    \label{compliance}
    \end{align}
\end{subequations}
where the elastic compliance $1/E = L^{-1} \, \partial_{F\,} L |_{\theta } = \partial_{F} \ln (L / L_0) |_{\theta } = \partial_{F\,} e |_{\theta }$ has units of reciprocal force evaluated at a constant temperature, while the thermal expansion coefficient $\alpha = L^{-1} \, \partial_{\theta\,} L |_F = \partial_{\theta} \ln (L/L_0) |_F = \partial_{\theta\,} e |_F$ has units of reciprocal temperature evaluated at a constant force.  The mass density $\rho$ is one-dimensional in this presentation, i.e., it has units of mass per unit length of fiber; likewise, modulus $E$ has units of force, not stress.

\subsubsection{A Hookean Fiber}

A thermoelastic Hookean fiber is Green elastic with a Gibbs free energy given by
\begin{equation}
    \mathcal{G} (\theta , F) = -C \left( \theta \ln \left( \frac{\theta}{\theta_0} \right) - 
    (\theta - \theta_0) \right) - 
    \frac{F}{\rho} \left( \alpha ( \theta - \theta_0 ) + \frac{F}{2E} \right)
    \label{GreenEnergy}
\end{equation}
which is a function of temperature $\theta$ and force $F$, normalized so $\mathcal{G} (\theta_0 , 0) = 0$.  This energy function is consistent with the thermo-physical material properties put forward in Eqn.~(\ref{specificHeat}--\ref{compliance}).  In this model the material properties $\rho$, $C$, $\alpha$ and $E$ are all considered to be of constant value across state space.

\subsection{Alveolar Chords as Rajagopal (Implicit) Thermoelastic Fibers}

In 2003, Rajagopal \cite{Rajagopal03} introduced the idea of an implicit elastic solid.  In 2016, Freed \&\ Rajagopal \cite{FreedRajagopal16} constructed an elastic fiber model that convolves an explicit energy with an implicit energy.  In their approach they decomposed strain $e \defeq \ln (L / L_0)$ into a sum of two strains, viz., $e = e_1 + e_2$ where $e_1 \defeq \ln (L_1 / L_0)$ and $e_2 \defeq \ln (L / L_1)$.  Length $L_0$ is an initial fiber length, viz., its length whenever $F = 0$.  Length $L_1$ can be thought of as the fiber's length caused solely by an unraveling of molecular configuration (e.g., an unraveling of collagen crimp) under an applied load of $F$.  The state associated with length $L_1$ is non-physical in that one cannot unravel molecules without also stretching their bonds to some extent.  Final length $L$ is the actual fiber length under an applied load $F$ caused by both a reconfiguration and a stretching of its molecular network.  Here we present their ideas in terms of a Gibbs free-energy function.\footnote{%
    Freed \& Rajagopal \cite{FreedRajagopal16} originally used a Helmholtz free-energy function.
}

Let the Gibbs free energy be described by a function of the form\footnote{
    One might be tempted to consider an implicit energy function of the form $\mathcal{G} = \mathcal{G}_1 (\theta ,  e_1 , F ) + \mathcal{G}_2 (\theta , F)$, but this would lead to a non-symmetric susceptibility matrix.  Consequently, it would not satisfy Maxwell's thermo\-dynamic constraint, a.k.a.\ Sylvester's condition for integrability of a Pfaffian form.  Hence, it would be inadmissible as a Gibbs potential.
}
\begin{equation}
\mathcal{G} (\theta , e , F) \defeq \mathcal{G}_1 ( e_1 , F ) + \mathcal{G}_2 ( \theta , F )
\quad \text{where} \quad
\mathrm{d} \mathcal{G} = -\eta \, \mathrm{d} \theta - 
\tfrac{1}{\rho} e \, \mathrm{d} F
\label{GibbsFreeEnergy}
\end{equation}
with $\mathcal{G}_1$ being an implicit potential (a configuration energy) and $\mathcal{G}_2$ being an explicit potential (a strain energy).  This energy function leads to the same constitutive equation displayed in Eqn.~(\ref{thermoelasticCE1D}), but whose material properties (\ref{specificHeat}--\ref{compliance}) are now interpreted according to the following formul\ae
\begin{subequations}
    \label{physicalFields1Dfiber}
    \begin{align}
    C & \defeq \theta \, \partial_{\theta\,} \eta |_F = 
    -\theta \, \partial_{\theta\theta\,} \mathcal{G}_2(\theta , F) =
    -\theta \, \partial_{\theta\theta\,} \mathcal{G} ( \theta , e_1 , F )
    \label{specificHeat1D} \\
    \alpha & \defeq \partial_{\theta\,} e |_F = 
    -\rho \, \partial_{F\theta\,} \mathcal{G}_2(\theta , F) =
    -\rho \, \partial_{F\theta\,} \mathcal{G} (\theta , e_1 , F)
    \label{thermalExpansion1D} \\
    1/E & \defeq \partial_{F\,} e |_{\theta} = -
    \bigl( \rho \, \partial_{e_1} \mathcal{G}_1 ( e_1, F ) \bigr)^{-1} 
    \bigl( e + \rho \, \partial_{F\,} \mathcal{G} (\theta , e_1 , F) \bigr) -
    \rho \, \partial_{FF\,} \mathcal{G}_2(\theta , F)
    \label{compliance1D}
    \end{align}
\end{subequations}
where mass density $\rho$ is a mass per unit length of line.  The elastic compliance $1/E$ is found to be a sum of two compliances, independent of the functional forms for $\mathcal{G}_1 ( e_1 , F )$ and $\mathcal{G}_2 ( \theta , F )$.  One compliance is explicit in origin, i.e., $e_2 = -\rho \, \partial_{F\,} \mathcal{G}_2$ with rate $\mathrm{d} e_2 = -\rho \, \partial_{F\theta\,} \mathcal{G}_2 \, \mathrm{d} \theta - \rho \, \partial_{FF\,} \mathcal{G}_2 \, \mathrm{d}F$.  The other compliance is implicit in origin, viz., $\mathrm{d} e_1 = - ( \rho \, \partial_{e_1 \,} \mathcal{G}_1 )^{-1} ( e_1 + \rho \, \partial_{F\,} \mathcal{G} ) \mathrm{d}F$.  Also, $\partial_{F\theta\,} \mathcal{G} = \partial_{\theta F\,} \mathcal{G}$ because of Maxwell's thermo\-dynamic constraint.

The material properties of Eqn.~(\ref{physicalFields1Dfiber}) apply to matrix equation~(\ref{thermoelasticCE1D}), just as those for a Hookean material (Eqn.~\ref{specificHeat}--\ref{compliance}) do.  The specific heat $C$ and thermal expansion coefficient $\alpha$ have the same interpretations for both the explicit and implicit fiber theories.  It is with respect to their compliances through which they differ.

\medskip\noindent
\textbf{Derivation}: 
Because Gibbs free energy is a state function, its differential describes a Pfaffian form so that the left-hand side of the thermo\-dynamic expression $\mathrm{d} \mathcal{G} = -\eta \, \mathrm{d} \theta - \tfrac{1}{\rho} e \, \mathrm{d}F$ becomes $\mathrm{d} \mathcal{G} = \partial_{e_1} \mathcal{G}_1 \, \mathrm{d} e_1 + \partial_{F\,} \mathcal{G}_1 \, \mathrm{d}F + \partial_{\theta\,} \mathcal{G}_2 \, \mathrm{d} \theta + \partial_{F\,} \mathcal{G}_2 \, \mathrm{d}F$. Recalling that $e = e_1 + e_2$, the explicit (hyperelastic like) terms combine to produce constitutive equations
\begin{displaymath}
\eta = -\partial_{\theta\,} \mathcal{G}_2 (\theta , F) 
\quad \text{and} \quad
e_2 = -\rho \, \partial_{F\,} \mathcal{G}_2 (\theta , F)
\end{displaymath} 
with the remaining implicit terms producing a differential constitutive equation
\begin{displaymath}
\rho \, \partial_{e_1} \mathcal{G}_1 ( e_1 , F ) \, \mathrm{d} e_1 = 
-\bigl( e_1 + \rho \, \partial_{F\,} \mathcal{G}_1 ( e_1 , F )
\bigr) \mathrm{d}F .
\end{displaymath}
Differentiating the constitutive equation for entropy with respect to state leads directly to the expressions for specific heat $C$ and thermal expansion $\alpha$ stated in Eqns.~(\ref{specificHeat1D} \& \ref{thermalExpansion1D}).  Recalling that the strains add, i.e., $e = e_1 + e_2$, and therefore so do their rates, viz., $\mathrm{d} e = \mathrm{d} e_1 + \mathrm{d} e_2$, a consequence of them being logarithmic in construction, then upon rearranging the implicit constitutive equation to solve for $\mathrm{d} e_1$, while differentiating the explicit constitutive equation for $e_2$, and finally adding these strain increments to get $\mathrm{d} e$, one obtains the elastic compliance function stated in Eqn.~(\ref{compliance1D}) whose reciprocal appears in the matrix equation~(\ref{Helmholtz1D}).  \hfill $\qed$

\subsubsection{Biologic Fibers}
\label{secBioFiber}

The fiber model of Freed \&\ Rajagopal \cite{FreedRajagopal16} imposes a strain-limiting constraint $e_t$ onto the internal strain $e_1$ by considering a Gibbs free-energy function of the form
\begin{subequations}
    \label{RajagopaleanFiber}
    \begin{align}
    \mathcal{G}_1 ( e_1 , F ) & = - \frac{1}{\rho} \Bigl(
    e_t ( E_1 e_1 - F ) + 2 e_1 F \Bigr)
    \label{FreedEnergy} \\
    \mathcal{G}_2(\theta , F) & = -C \left( \theta \ln \left( \frac{\theta}{\theta_0} \right) - 
    (\theta - \theta_0) \right) - 
    \frac{F}{\rho} \left( \alpha ( \theta - \theta_0 ) + \frac{F}{2E_2} \right)
    \label{HookeanEnergy} \\
    \intertext{that, collectively, depend upon tempreature $\theta$, force $F$, and an internal strain $e_1$, whose energy is normalized so that $\mathcal{G}_1(0, 0) = 0$ and $\mathcal{G}_2(\theta_0,0)=0$.  In fact, the explicit contribution to the free energy adopted here is Hookean, cf.\ Eqn.~(\ref{GreenEnergy}).  The resulting constitutive responses for entropy $\eta$ and strain $e=\ln (L/L_0)$ are therefore described by the following matrix differential equation}
    \left\{ \begin{matrix}
    \mathrm{d} \eta \\ \mathrm{d} e 
    \end{matrix} \right\} & = \begin{bmatrix}
    C / \theta & \alpha / \rho  \\
    \alpha & 1 / E
    \end{bmatrix}
    \left\{ \begin{matrix}
    \mathrm{d} \theta \\ \mathrm{d} F
    \end{matrix} \right\} .
    \label{FungCE} \\
    \intertext{This is a Gibbs version of Eqn.~(\ref{Helmholtz1D}), which is expressed in terms of Helmholtz's conjugate variables.  Here the elastic tangent compliance is described by}
    \frac{1}{E(\theta , e , F)} & = 
    \frac{e_t - e_1}{E_1 e_t + 2F} + \frac{1}{E_2} 
    \quad \text{wherein} \quad
    e_1 = e - \alpha (\theta - \theta_0) - \frac{F}{E_2}
    \label{FRcompliance}
    \end{align}
\end{subequations}
with an initial tangent modulus $E(\theta_0, 0, 0)$ of $E_1 E_2 / (E_1 + E_2)$ ($\approx E_1$ whenever $E_2 \gg E_1 > 0$) while the terminal tangent modulus $E(e_1 \! = \! e_t)$ is $E_2$.  A transition strain occurs at $e_t$ $(> 0)$, which establishes a limiting state for internal strain $e_1$, i.e., $0 \leq e_1 \leq e_t$.  This is a strain whereat the fiber's molecular configuration becomes completely unraveled.  The 2 in term $2 e_1 F$ of Eqn.~(\ref{FreedEnergy}) leads to the desired numerator, viz., $e_t - e_1$, for the implicit contribution to compliance in Eqn.~(\ref{FRcompliance}).  This fiber model has been found to be superior to other models commonly employed in the literature for modeling biologic fibers \cite{AkintundeMiller18,Robbinsetal20}.

Here $\rho$ is a mass per unit length of fiber, $C$ is a specific heat per unit mass measured at a constant force, and $\alpha$ is a coefficient of linear thermal expansion evaluated at a constant force, all of which have the same physical interpretation as their Hookean counterparts.  Only their elastic compliances are interpreted differently.

The sum of implicit and explicit fiber compliances, as established in Eqn.~(\ref{FRcompliance}), was originally a conjecture by Freed \& Rajagopal \cite{FreedRajagopal16}; whereas, here it is a derived consequence from thermo\-dynamics provided that $\mathcal{G} ( \theta , e , F) = \mathcal{G}_1 ( e_1 , F ) + \mathcal{G}_2 ( \theta , F )$ and given that $e = e_1 + e_2$ with $e_2 = - \rho \, \partial_{F\,} \mathcal{G}_2$.  This follows because here a Gibbs free-energy is used, whereas Freed \& Rajagopal employed a Helmholtz free-energy.

Biologic fibers, per our application, are long and slender.  Consequently, they will buckle under compression.  Buckling is not accounted for in our modeling of alveolar chords.  Rather, it is assumed that the compliant response at the origin, with a modulus of $E_1 E_2 / ( E_1 + E_2 )$, continues into compression, thereby ensuring some measure of numeric stability in our software.

The above methodology would allow us to construct a suite of thermo\-dynamically admissible elastic compliance functions, but we will only have need for the fiber model put forward in Eqn.~(\ref{RajagopaleanFiber}).

\subsubsection{Viscoelastic Biologic Fibers}

Freed \& Rajagopal \cite{FreedRajagopal16a} have shown that realistic visco\-elastic responses for biologic fibers can be constructed by keeping the implicit contribution of the compliance modulus $1/E$ elastic, while only extending its explicit contribution into the visco\-elastic domain.  This finding is significant!  It allows one to model the visco\-elastic response of non-linear biologic fibers by employing a \textit{linear\/} theory for visco\-elasticity.  Effectively, elastic modulus $E_2$ in Eqn.~(\ref{FRcompliance}) becomes a visco\-elastic function of state.  This is a topic for future work.

\subsection{Alveolar Septa as Green (Explicit) Thermoelastic Membranes}

We observed in \S\ref{secNonuniform2D} that an alveolar membrane has a response comprised of uniform and non-uniform contributions and that, for our application, these two contributions are not coupled.  Consequently, their internal energies add $\mathcal{G} (\theta , \pi , \sigma , \tau ) = \mathcal{G}_u (\theta , \pi ) + \mathcal{G}_n (\sigma , \tau)$, with $\mathcal{G}_u$ being the uniform contribution of $\mathcal{G}$ and $\mathcal{G}_n$ being the non-uniform contribution of $\mathcal{G}$.

\subsubsection{Uniform Response}

From thermo\-dynamics, Eqn.~(\ref{thermoelastic2Dlaw}), comes $\mathrm{d}\hspace{1pt}\mathcal{U}_u = \theta \, \mathrm{d} \eta + \tfrac{1}{\rho} T \, \mathrm{d}A / \! A = \theta \, \mathrm{d} \eta + \tfrac{1}{\rho} \pi \, \mathrm{d}\xi$ where $\pi = 2T$ and $\mathrm{d}\xi = \tfrac{1}{2} A^{-1} \, \mathrm{d}A$, and whose mass density $\rho$ has units of mass per unit area of surface.  Writing this in its Gibbs format $\mathrm{d} \mathcal{G}_u = -\eta \, \mathrm{d} \theta - \tfrac{1}{\rho} \xi \, \mathrm{d} \pi$ via a Legendre transformation, one gets a constitutive equation appropriate for describing the uniform response of a thermo\-elastic planar membrane, namely
\begin{subequations}
    \label{uniformMembraneModel}
    \begin{align}
\left\{ \begin{matrix}
\mathrm{d} \eta \\ \mathrm{d} \xi
\end{matrix} \right\} & = -\begin{bmatrix}
\partial_{\theta\theta\,} \mathcal{G}_u & \partial_{\theta\pi\,} \mathcal{G}_u \\
\rho \, \partial_{\pi\theta\,} \mathcal{G}_u & \rho \, \partial_{\pi\pi\,} \mathcal{G}_u
\end{bmatrix} 
\left\{ \begin{matrix}
\mathrm{d} \theta \\ \mathrm{d} \pi
\end{matrix} \right\} = \begin{bmatrix}
C / \theta & \alpha / \rho \\ \alpha & 1 / 4 M
\end{bmatrix} \left\{ \begin{matrix}
\mathrm{d} \theta \\ \mathrm{d} \pi
\end{matrix} \right\}
\label{thermoelasticCE2D} \\
\intertext{with material properties defined accordingly}
    C & \defeq \theta \, \partial_{\theta\,} \eta |_{\pi} = -\theta \, \partial_{\theta\theta\,} \mathcal{G}_u \\
    \alpha & \defeq L^{-1} \, \partial_{\theta\,} L |_T = \tfrac{1}{2} A^{-1} \, \partial_{\theta\,} A |_T = \partial_{\theta\,} \xi |_{\pi} = -\rho \, \partial_{\pi\theta\,} \mathcal{G}_u = -\rho \, \partial_{\theta\pi\,} \mathcal{G}_u \\
    1/M & \defeq A^{-1} \, \partial_{T\,} A |_{\theta} = 2 \, \partial_{T\,} \xi |_{\theta} = 4 \, \partial_{\pi\,} \xi |_{\theta} = -4\rho \, \partial_{\pi\pi\,} \mathcal{G}_u
    \end{align}
\end{subequations}
where $C$ is the specific heat, $\alpha$ is the coefficient of thermal expansion, and $M$ is the modulus of dilation (uniform expansion).  Here $\eta$ is entropy, $\theta$ is temperature, $T \defeq \tfrac{1}{2} ( \sigma_{11} + \sigma_{22})$ is surface tension (an invariant in 2D whose intensive variable is $\pi \defeq \sigma_{11} + \sigma_{22} = 2T$ with $\sigma_{ij}$ being components of Cauchy stress quantified in 2D), and $A$ is area (whose extensive variable is $\xi \defeq \ln \sqrt{A / \! A_0}$, which denotes dilation, i.e., that strain describing an uniform areal expansion).  

\subsubsection{Non-Uniform Response}

From $\mathrm{d} \mathcal{G} = \mathrm{d} \mathcal{G}_u + \mathrm{d} \mathcal{G}_n$ with $\mathrm{d} \mathcal{G}_u = -\eta \, \mathrm{d} \theta - \tfrac{1}{\rho} \xi \, \mathrm{d} \pi$ comes $\mathrm{d} \mathcal{G}_n = -\tfrac{1}{\rho} ( \varepsilon \, \mathrm{d} \sigma + \gamma \, \mathrm{d} \tau )$ out of which come the constitutive equations governing the non-uniform response of a Green elastic membrane, viz., $\varepsilon = -\rho \, \partial_{\sigma\,} \mathcal{G}_n$ and $\gamma = -\rho \, \partial_{\tau\,} \mathcal{G}_n$, that, assuming they are continuous and differentiable functions of state, can be expressed as the matrix differential equation
\begin{subequations}
    \begin{align}
    \left\{ \begin{matrix}
    \mathrm{d} \varepsilon \\ \mathrm{d} \gamma
    \end{matrix} \right\}  & = -\rho \begin{bmatrix}
    \partial_{\sigma\sigma\,} \mathcal{G}_n & 
    \partial_{\sigma\tau\,} \mathcal{G}_n \\
    \partial_{\tau\sigma\,} \mathcal{G}_n &
    \partial_{\tau\tau\,} \mathcal{G}_n
    \end{bmatrix}
    \left\{ \begin{matrix}  
    \mathrm{d} \sigma \\ \mathrm{d} \tau
    \end{matrix} \right\} = \begin{bmatrix}
    1/2 N & 0 \\ 
    0 & 1 / G
    \end{bmatrix} \left\{ \begin{matrix}
    \mathrm{d} \sigma \\ \mathrm{d} \tau
    \end{matrix} \right\} \\
    \intertext{where $\partial_{\sigma\tau\,} \mathcal{G}_n = \partial_{\tau\sigma\,} \mathcal{G}_n = 0$ because the modes of squeeze and shear are taken to be decoupled.  The associated material properties are established via}
    \frac{1}{N} & \defeq \Gamma^{-1} \, \partial_{\sigma_{11} - \sigma_{22}} \Gamma |_g = 2 \, \partial_{\sigma\,} \varepsilon |_{\gamma} = -2 \rho \, \partial_{\sigma\sigma\,} \mathcal{G}_n 
    \label{squeezeModulus} \\
    \frac{1}{G} & \defeq \Gamma^{-1} \, \partial_{\sigma_{21}} g |_{\Gamma} = \partial_{\tau\,} \gamma |_{\varepsilon} = -\rho \, \partial_{\tau\tau\,} \mathcal{G}_n 
    \label{shearModulus}
    \end{align}
\end{subequations}
where the conjugate stresses are defined as $\sigma \defeq \sigma_{11} - \sigma_{22}$ and $\tau \defeq \Gamma \sigma_{21}$ with $\Gamma \defeq a/b$ being the stretch of squeeze from which it follows that $\Gamma^{-1} \mathrm{d} \Gamma = 2 \, \mathrm{d} \varepsilon$ because the strain of squeeze is given by $\varepsilon = \ln \sqrt{ \Gamma / \Gamma_0 }$.  The squeeze compliance $1/N = 2 \, \mathrm{d} \varepsilon / \mathrm{d} \sigma |_{\gamma}$ is evaluated at a constant shear $\gamma$, while the shear compliance $1/G = \mathrm{d} \gamma / \mathrm{d} \tau |_{\varepsilon}$ is evaluated at a constant squeeze $\varepsilon$.

\subsubsection{A Hookean Membrane}

Uniform response in a Green elastic membrane is described by a Gibbs free-energy function of the form
\begin{subequations}
\label{membraneHookeanEnergies}
\begin{align}
\mathcal{G}_u (\theta , \pi) & = -C \left( \theta \ln \left( \frac{\theta}{\theta_0} \right) -
(\theta - \theta_0) \right) - 
\frac{\pi}{2\rho} \left( 2 \alpha ( \theta - \theta_0 ) + \frac{\pi}{4M} \right)
\label{Gibbs2DHookean} \\
   \intertext{whose non-uniform response satisfies a Gibbs free-energy function of the form}
\mathcal{G}_n (\sigma , \tau) & = \mathcal{G}_n (-\sigma , \tau) = \mathcal{G}_n (\sigma , -\tau ) = \mathcal{G}_n ( -\sigma , -\tau ) = \frac{1}{2\rho} \Bigl( \frac{\sigma^2}{2N} + \frac{\tau^2}{G} \Bigr) \\
\intertext{with negative arguments ensuring necessary symmetries in response, which are pertinent for all terms in a non-uniform energy function.  Form these formul\ae\ come the following system of differential equations}
\left\{ \begin{matrix}
\mathrm{d} \eta \\ \mathrm{d} \pi \\
\mathrm{d} \sigma \\ \mathrm{d} \tau
\end{matrix} \right\} & = \begin{bmatrix}
C / \theta - 4 \alpha^2 M / \rho & 4 \alpha M / \rho & 0 & 0 \\
-4 \alpha M & 4 M & 0 & 0 \\
0 & 0 & 2 N & 0 \\
0 & 0 & 0 & G 
\end{bmatrix} \left\{ \begin{matrix}
\mathrm{d} \theta \\ \mathrm{d} \xi \\
\mathrm{d} \varepsilon \\ \mathrm{d} \gamma
\end{matrix} \right\}
\label{elasticMembrane}
\end{align}
\end{subequations}
that describe a thermo\-elastic Hookean membrane, expressed in terms of Helm\-holtz state variables, whose material properties $\rho$, $C$, $\alpha$, $M$, $N$ and $G$ are each constant valued across state space, cf.\ Eqn.(\ref{HelmholtzMembraneODEs}).  It is in terms of the Helmholtz conjugate pairs that our alveolar modeling is most adept to, even though cause and effect are incorrect.

\subsection{Alveolar Septa as Rajagopal (Implicit) Thermoelastic Membranes}

We employ implicit elasticity here to derive a constitutive theory suitable for describing biologic membranes.

\subsubsection{Biologic Membranes Under Uniform Motions}

Like the implicit elastic fiber introduced in Eqn.~(\ref{RajagopaleanFiber}), the uniform response of an implicit elastic membrane with a strain-limiting dilation can be modeled using a Gibbs free energy $\mathcal{G}_u (\theta , \xi , \pi ) \defeq \mathcal{G}_1 (\xi_1 ,\pi) + \mathcal{G}_2 (\theta , \pi)$ where our definition for dilation $\xi \defeq \ln \sqrt{A / \! A_0}$ decomposes into a sum of two dilations: $\xi_1 \defeq \ln \sqrt{A_1 / \! A_0}$ and $\xi_2 \defeq \ln \sqrt{A / \! A_1}$ so that $\xi = \xi_1 + \xi_2$, with like interpretations as those from their 1D fiber counterparts, viz., $e$, $e_1$ and $e_2$; specifically, its material properties are
\begin{subequations}
    \label{physicalFields2Dmembrane}
    \begin{align}
    C & \defeq \theta \, \partial_{\theta\,} \eta |_{\pi} = 
    -\theta \, \partial_{\theta\theta\,} \mathcal{G}_2 (\theta , \pi) =
    -\theta \, \partial_{\theta\theta\,} \mathcal{G}_u  (\theta , \xi_1 , \pi)
    \label{specificHeat2Dmembrane} \\
    \alpha & \defeq \partial_{\theta\,} \xi |_{\pi} = 
    -\rho \, \partial_{\pi\theta\,} \mathcal{G}_2 (\theta , \pi) = 
    -\rho \, \partial_{\pi\theta\,} \mathcal{G}_u (\theta , \xi_1 , \pi)
    \label{thermalExpansion2Dmembrane} \\
    1/4M & \defeq \partial_{\pi\,} \xi |_{\theta} = -
    \bigl( \rho \, \partial_{\xi_1} \mathcal{G}_1 ( \xi_1, \pi ) \bigr)^{-1} 
    \bigl( \xi + \rho \, \partial_{\pi\,} \mathcal{G}_u (\theta , \xi_1 , \pi ) \bigr) -
    \rho \, \partial_{\pi\pi\,} \mathcal{G}_2(\theta , \pi)
    \label{compliance2Dmembrane}
    \end{align}
\end{subequations}
where $\partial_{\pi\theta\,} \mathcal{G}_u = \partial_{\theta\pi\,} \mathcal{G}_u$ because of Maxwell's constraint.  Their derivations are analogous to those for the implicit fiber derived above.    These material properties pertain to the matrix constitutive equations found in Eqns.~(\ref{HelmholtzMembraneODEs} \& \ref{elasticMembrane}).

Like our model for a biologic fiber, we consider a Gibbs free-energy function for describing the uniform response of a biologic membrane that takes on the form of
\begin{subequations} 
    \label{uniformMembrane}
    \begin{align}
    \mathcal{G}_1 (\xi_1 , \pi) & = - \frac{1}{\rho} 
    \Bigl( \xi_t (4M_1 \xi_1 - \pi ) + 2 \xi_1 \pi \Bigr) \\
    \mathcal{G}_2 (\theta , \pi) & = -C \left( \theta \ln \left( \frac{\theta}{\theta_0} \right) -
    (\theta - \theta_0) \right) - \frac{\pi}{2\rho} \left( 
    2 \alpha ( \theta - \theta_0 ) + \frac{\pi}{4M_2} \right) \\
    \intertext{whose resulting elastic compliance, as established through Eqn.~(\ref{compliance2Dmembrane}), is}
    \frac{1}{M(\theta, \xi, \pi)} & = 
    \frac{\xi_t - \xi_1}{M_1 \xi_t + \pi / 2} + \frac{1}{M_2} 
    \quad \text{wherein} \quad 
    \xi_1 = \xi - \alpha (\theta - \theta_0) - \frac{\pi}{4M_2}
    \label{membraneCompliance}
    \end{align}
\end{subequations}
with $\xi_t > 0$ being an upper bound on strain $\xi_1$, i.e., $0 \leq \xi_1 \leq \xi_t$.  Such a membrane has an initial tangent stiffness $M(\theta_0, 0, 0)$ of $M_1 M_2 / ( M_1 + M_2 )$ ($\approx M_1$ whenever $M_2 \gg M_1 > 0$) and it has a terminal tangent stiffness $M(\xi_1 \! = \! \xi_t)$ of $M_2$.

Membranes will wrinkle under states of negative surface tension (or dilation).  In alveolar mechanics, surfactant helps to prevent this, and a possible ensuing alveolar collapse.  Wrinkling is not accounted for in our modeling of alveolar septa.  Rather, like fibers, membranes are assumed to support compression with a soft modulus of $M_1 M_2 / ( M_1 + M_2 )$, which associates with the compliant response found at the origin (zero surface tension and zero dilation).  This is done to help ensure numeric stability in our software.

The difference between a Green and Rajagopal thermo\-elastic membrane under\-going a dilation is in their definitions for elastic compliance.  There is no difference in their properties for specific heat or thermal expansion.  The above model has been successfully applied to a visceral pleura membrane \cite{Freedetal17}.

\subsection{Biologic Membranes Under Non-Uniform Motions}

We seek an energetic construction that is consistent with the Freed \& Rajagopal \cite{FreedRajagopal16} fiber model, but which is applicable to the non-uniform responses that planar membranes can support.  A Rajagopal elastic solid is implicit. Therefore, we choose a Gibbs free-energy function for governing non-uniform behavior that looks like
\begin{equation}
\mathcal{G}_n ( \varepsilon_1 , \gamma_1 , \sigma , \tau ) = \mathcal{G}_1 ( \varepsilon_1 , \sigma ) + \mathcal{G}_2 ( \sigma ) + \mathcal{G}_3 ( \gamma_1 , \tau ) + \mathcal{G}_4 ( \tau )
\label{nonuniformEnergy}
\end{equation}
which depend upon three squeeze strains $\varepsilon \defeq \ln \sqrt{\Gamma \! / \Gamma_0}$, $\varepsilon_1 \defeq \ln \sqrt{ \Gamma_1 / \Gamma_0}$ and $\varepsilon_2 \defeq \ln \sqrt{ \Gamma \! / \Gamma_1}$, and three shear strains $\gamma \defeq g - g_0$, $\gamma_1 \defeq g_1 - g_0$ and $\gamma_2 \defeq g - g_1$, both of which are additive in the sense that $\varepsilon = \varepsilon_1 + \varepsilon_2$ and $\gamma = \gamma_1 + \gamma_2$, and as such, so are their differential rates of change $\mathrm{d} \varepsilon = \mathrm{d} \varepsilon_1 + \mathrm{d} \varepsilon_2$ and $\mathrm{d} \gamma = \mathrm{d} \gamma_1 + \mathrm{d} \gamma_2$.  Strains $\varepsilon_1$ and $\gamma_1$ may be thought of as describing an unraveling of molecular configuration, analogous to $e_1$ in the fiber model of Eqn.~(\ref{RajagopaleanFiber}) and $\xi_1$ in the uniform membrane model of Eqn.~(\ref{membraneCompliance}).  No coupling between squeeze and shear is assumed in this energy function.  Energies $\mathcal{G}_1$ and $\mathcal{G}_3$ are Rajagopal elastic (implicit), while energies $\mathcal{G}_2$ and $\mathcal{G}_4$ are Green elastic (explicit).

From the thermo\-dynamic expression $-\rho \, \mathrm{d} \mathcal{G}_n = \varepsilon \, \mathrm{d} \sigma + \gamma \, \mathrm{d} \tau$, the non-uniform Gibbs free energy $\mathcal{G}_n$, when expressed in the form of Eqn.~(\ref{nonuniformEnergy}), and given the definitions for squeeze $1/N$ and shear $1/G$ compliances put forward in Eqns.~(\ref{squeezeModulus} \& \ref{shearModulus}), one determines a squeeze compliance that is described by
\begin{subequations}
    \label{nonuniformCompliances}
    \begin{align}
    \frac{1}{2N} & \defeq \frac{\mathrm{d} \varepsilon}{\mathrm{d} \sigma} = - \bigl( \rho \, \partial_{\varepsilon_1} \mathcal{G}_1 \bigr)^{-1} \bigl( \varepsilon + \rho \, \partial_{\sigma} ( \mathcal{G}_1 + \mathcal{G}_2 ) \bigr) - \rho \, \partial_{\sigma\sigma\,} \mathcal{G}_2
    \label{squeezeCompliance} \\
    \intertext{and a shear compliance that is described by}
    \frac{1}{G} & \defeq \frac{\mathrm{d} \gamma}{\mathrm{d} \tau} = - \bigl( \rho \, \partial_{\gamma_1} \mathcal{G}_3 \bigr)^{-1} \bigl( \gamma + \rho \, \partial_{\tau} ( \mathcal{G}_3 + \mathcal{G}_4 ) \bigr) - \rho \, \partial_{\tau\tau\,} \mathcal{G}_4
    \label{shearCompliance}
    \end{align}
\end{subequations}
whose mathematical structure is similar to that of the fiber model presented in Eqn.~(\ref{RajagopaleanFiber}).  The first collection of terms on the right-hand side of both formul\ae\ is Rajagopal elastic; the second is Green elastic.  

\medskip\noindent
\textbf{Derivation}: The First and Second Laws of Thermo\-dynamics, as they pertain to non-uniform contributions of stress power, have energetic components described in Eqn.~(\ref{nonuniformEnergy}) so that $\rho \, \mathrm{d} \mathcal{G}_n = \rho \, \partial_{\varepsilon_1} \mathcal{G}_1 ( \varepsilon_1 , \sigma ) \, \mathrm{d} \varepsilon_1 + \rho \, \partial_{\sigma\,} \mathcal{G}_1 ( \varepsilon_1 , \sigma ) \, \mathrm{d} \sigma + \rho \, \partial_{\sigma} \mathcal{G}_2 ( \sigma ) \, \mathrm{d} \sigma + \rho \, \partial_{\gamma_1} \mathcal{G}_3 ( \gamma_1 , \tau ) \, \mathrm{d} \gamma_1 + \rho \, \partial_{\tau\,} \mathcal{G}_3 ( \gamma_1 , \tau ) \, \mathrm{d} \tau + \rho \, \partial_{\tau\,} \mathcal{G}_4 ( \tau ) \, \mathrm{d} \tau$ that associate with the conjugate pairings $-\varepsilon_1 \, \mathrm{d} \sigma - \varepsilon_2 \, \mathrm{d} \sigma - \gamma_1 \, \mathrm{d} \tau - \gamma_2 \, \mathrm{d} \tau$ because of the prescribed additivity in strains.  These follow from a Legendre transformation of the internal energy.  Gathering like terms result in a pair of Green elastic formul\ae\ that describe two of the internal strains
\begin{displaymath}
\varepsilon_2 = -\rho \, \partial_{\sigma\,} \mathcal{G}_2 ( \sigma ) 
\quad \text{and} \quad
\gamma_2 = -\rho \, \partial_{\tau\,} \mathcal{G}_4 ( \tau )
\end{displaymath}
and two Rajagopal elastic formul\ae\ whose ODEs describe the other internal strains
\begin{align*}
    \mathrm{d} \varepsilon_1 & = - \bigl( \rho \, \partial_{\varepsilon_1} \mathcal{G}_1 ( \varepsilon_1 , \sigma ) \bigr)^{-1} \bigl( \varepsilon_1 + \rho \, \partial_{\sigma\,} \mathcal{G}_1 ( \varepsilon_1 , \sigma ) \bigr) \mathrm{d} \sigma \\
    \mathrm{d} \gamma_1 & = -\bigl( \rho \, \partial_{\gamma_1} \mathcal{G}_3 ( \gamma_1 , \tau ) \bigr)^{-1} \bigl( \gamma_1 + \rho \, \partial_{\tau\,} \mathcal{G}_3 ( \gamma_1 , \tau ) \bigr) \mathrm{d} \tau
\end{align*}
that when combined as rates become the constitutive formul\ae\ in Eqn.~(\ref{nonuniformCompliances}). \hfill $\qed$

We now seek a specific free-energy function for Eqn.~(\ref{nonuniformEnergy}) that produces compliances $1/N$ and $1/G$ with like mathematical structure to Eqn.~(\ref{membraneCompliance}) for dilation, viz., $1/M$; specifically, we consider Gibbs free-energy functions of the form
\begin{subequations}
    \label{nonuniformComplianceEnergies}
    \begin{align}
    -\rho \, \mathcal{G}_1 ( \varepsilon_1 , \sigma ) & = \mathrm{sgn} ( \varepsilon_1 ) \, \varepsilon_t \bigl( 2 N_1 \varepsilon_1 - \sigma \bigr) + 2 \varepsilon_1 \sigma &
    -\rho \, \mathcal{G}_2 ( \sigma ) & = \sigma^2 / 4 N_2
    \label{squeezeEnergy} \\
    -\rho \, \mathcal{G}_3 ( \gamma_1 , \tau ) & = \mathrm{sgn} ( \gamma_1 ) \, \gamma_t \bigl( G_1 \gamma_1 - \tau \bigr) + 2 \gamma_1 \tau &
    -\rho \, \mathcal{G}_4 ( \tau ) & = \tau^2 / 2 G_2 
    \label{shearEnergy}
    \end{align}
\end{subequations}
which have the same mathematical structure as the energies for the biologic fiber (\ref{RajagopaleanFiber}) and uniform membrane (\ref{uniformMembrane}), less their temperature dependence.  The sign functions, viz., $\mathrm{sgn}( \varepsilon_1 )$ and $\mathrm{sgn} ( \gamma_1 )$, account for the fact that squeeze and shear strains can be of either sign, but the Gibbs energy must remain negative.  In effect, the sign functions flip the limiting state between tension and compression, i.e., they change the signs of $\varepsilon_t$ and $\gamma_t$, respectively.  When substituted into Eqn.~(\ref{nonuniformCompliances}), these energy functions produce the following thermo\-elastic compliances
\begin{subequations}
    \label{nonuniformComplianceFns}
    \begin{align}
    \frac{1}{N(\varepsilon , \sigma)} & = \frac{ \mathrm{sgn} (\varepsilon_1) \, \varepsilon_t - \varepsilon_1}{N_1 \, \mathrm{sgn} (\varepsilon_1) \, \varepsilon_t + \sigma} + \frac{1}{N_2} &
    \varepsilon_1 & = \varepsilon - \frac{\sigma}{2N_2}
    \label{squeezeCompliance2D} \\
    \frac{1}{G(\gamma , \tau)} & = \frac{ \mathrm{sgn} (\gamma_1) \, \gamma_t - \gamma_1}{G_1 \, \mathrm{sgn} (\gamma_1) \, \gamma_t + 2 \tau} + \frac{1}{G_2} & 
    \gamma_1 & = \gamma - \frac{\tau}{G_2}
    \label{shearCompliance2D}
    \end{align}
\end{subequations}
which provide the tangent operators that we will use to describe the non-uniform behavior of a biologic membrane.

Like our other biologic models, the squeeze compliance $1/N$ is described by three material properties: an asymptotic modulus at the reference state of $N_1 N_2 / (N_1 + N_2)$ ($\approx N_1$ whenever $N_2 \gg N_1 > 0$) where $N_1$ may be thought of as the stiffness of an unstretched molecular network, and a terminal modulus $N_2$ designating a stiffness after its molecular network has been stretched out at a limiting state of configurational squeeze $\varepsilon_t$.  The shear compliance $1/G$ is also described by three material properties: an asymptotic modulus at the reference state of $G_1 G_2 / ( G_1 + G_2 )$ ($\approx G_1$ whenever $G_2 \gg G_1 > 0$), a terminal modulus $G_2$, and a limiting state of configurational shear $\gamma_t$.  

In soft biological tissues, the shear moduli $G_1$ and $G_2$ will be several orders in magnitude smaller than their respective squeeze moduli $N_1$ and $N_2$.  Classical theories cannot make such a distinction.